% Código extraído e adaptado de: https://tex.stackexchange.com/questions/20267/how-to-construct-a-confusion-matrix-in-latex
\usetikzlibrary{matrix,calc,positioning,fadings}

%The matrix in numbers
%Horizontal target class
%Vertical output class

\def\myConfMat{{
{2,10,15,12,2,1},
{1,8,23,16,2,0},
{1,22,70,53,0,0},
{0,2,43,202,15,0},
{0,0,2,80,39,0},
{0,1,3,18,25,18}
}}


\def\classNames{{"0","1","2","3","4","5"}} %class names. Adapt at will

\def\numClasses{6} %number of classes. Could be automatic, but you can change it for tests.

\def\myScale{1.5} % 1.5 is a good scale. Values under 1 may need smaller fonts!
\begin{tikzpicture}[
    scale = \myScale,
    %font={\scriptsize}, %for smaller scales, even \tiny may be useful
    ]

\tikzset{vertical label/.style={rotate=90,anchor=east}}   % usable styles for below
\tikzset{diagonal label/.style={rotate=45,anchor=north east}}

\foreach \y in {1,...,\numClasses} %loop vertical starting on top
{
    % Add class name on the left
    \node [anchor=east] at (0.4,-\y) {\pgfmathparse{\classNames[\y-1]}\pgfmathresult};

    \foreach \x in {1,...,\numClasses}  %loop horizontal starting on left
    {
%---- Start of automatic calculation of totSamples for the line ------------
    \def\totSamples{0}
    \foreach \ll in {1,...,\numClasses}
    {
        \pgfmathparse{\myConfMat[\x-1][\ll-1]}   %fetch next element
        \xdef\totSamples{\totSamples+\pgfmathresult} %accumulate it with previous sum
        %must use \xdef fro global effect otherwise lost in foreach loop!
    }
    \pgfmathparse{\totSamples} \xdef\totSamples{\pgfmathresult}  % put the final sum in variable
%---- End of automatic calculation of totSamples ----------------
    \begin{scope}[shift={(\x,-\y)}]
        \def\mVal{\myConfMat[\x-1][\y-1]} % The value at index y,x (-1 because of zero indexing)
        \pgfmathtruncatemacro{\r}{\mVal}   %
        \ifthenelse{\equal{\totSamples}{0.0}}{\pgfmathtruncatemacro{\p}{0}}{\pgfmathtruncatemacro{\p}{round(\r/\totSamples*100)}}
        \coordinate (C) at (0,0);
        \ifthenelse{\p<50}{\def\txtcol{black}}{\def\txtcol{white}} %decide text color for contrast
        \node[
            draw,                 %draw lines
            text=\txtcol,         %text color (automatic for better contrast)
            align=center,         %align text inside cells (also for wrapping)
            fill=blue!\p,        %intensity of fill (can change base color)
            minimum size=\myScale*10mm,    %cell size to fit the scale and integer dimensions (in cm)
            inner sep=0,          %remove all inner gaps to save space in small scales
            ] (C) {\r\\\p\%};     %text to put in cell (adapt at will)
        %Now if last vertical class add its label at the bottom
        \ifthenelse{\y=\numClasses}{
        \node [] at ($(C)-(0,0.75)$) % can use vertical or diagonal label as option
        {\pgfmathparse{\classNames[\x-1]}\pgfmathresult};}{}
    \end{scope}
    }
}

% Add colormap on right with gradient node
\coordinate (colormap) at (\numClasses+1,-\numClasses+2.5);
\node[
    draw,
    shading=axis,
    top color=blue!100,
    bottom color=blue!0,
    minimum height=\myScale*\numClasses*10mm,
    minimum width=5mm,
    inner sep=0] (a) at (colormap) {};

% Add 100% on top of colormap
\node [above,inner sep=0] at ($ (a.north east) + (\myScale*3mm, -\myScale*2mm) $) {100\%};

% Add 0% on bottom of colormap
\node [below,inner sep=0] at ($ (a.south east) + (\myScale*3mm, \myScale*2mm) $) {0\%};


%Now add x and y labels on suitable coordinates
\coordinate (yaxis) at (-0.3,0.5-\numClasses/2);  %must adapt if class labels are wider!
\coordinate (xaxis) at (0.5+\numClasses/2, -\numClasses-1.25); %id. for non horizontal labels!
\node [vertical label] at (yaxis) {Nota Inferida};
\node []               at (xaxis) {Nota Real};
\end{tikzpicture}

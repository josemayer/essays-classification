\usetikzlibrary{decorations.pathreplacing, calc, arrows.meta, shapes, positioning}

\begin{tikzpicture}[
    remember picture,
    newline/.style={draw, circle, dotted, gray, inner sep=0.4pt},
]
    \definecolor{snsblue}{RGB}{213,228,236}
    \definecolor{snsred}{RGB}{247,210,205}
    \definecolor{snsyellow}{RGB}{247,225,200}
    \definecolor{snsgreen}{RGB}{223,236,201}

    \node[rectangle, fill=snsblue, rounded corners, text width=5cm] (int1) {\tiny\baselineskip=3pt No jogo eletrônico “League of Legends”, Viktor é um ciborgue que mora na  cidade de Piltover, uma das mais tecnológicas e sofisticadas de seu entorno. Na trama  desse personagem, as condições do meio em que vivia o fizeram substituir partes de seu  corpo, que lhe garantiam a condição de humano, por estruturas mecanizadas, que lhe forneciam maior eficiência nas atividades realizadas. Fora de tal microcosmo ficcional, é  notório que o âmbito do trabalho brasileiro vem adotando o mesmo postulado de Viktor  ao promover a automação em massa da produção, o que gera impactos nos índices de emprego da população. Desse modo, é fato que a automatização da indústria suscita uma problemática cada vez mais crescente no país, uma vez que tanto diminui o grau de pleno emprego na sociedade quanto seleciona cognitivamente os profissionais do futuro.\par};

    \node[rectangle, fill=snsred, rounded corners, text width=5cm, below of=int1, yshift=-1em] (dev1) {\tiny\baselineskip=3pt Em primeiro lugar, convém ressaltar que a exigência de maior eficiência na cadeia produtiva faz com que as empresas optem pela utilização do maquinário na mão de obra, diminuindo os empregos disponíveis à população. No meio rural, por exemplo, a mudança abrupta do modelo extensivo de produção ficou conhecida como Revolução Verde e ocorreu em meados da década de 70. O novo sistema instaurado no campo, a partir desse período, passou a utilizar o maquinário pesado para as atividades que, antes, eram realizadas por trabalhadores assalariados. Assim, do mesmo modo que no âmbito rural, uma crise decorrente da falta de empregos passa a vigorar no meio urbano, já que a indústria adota, de modo crescente, novas tecnologias que possam reduzir a demanda de contratados.\par};

    \node[rectangle, fill=snsyellow, rounded corners, text width=5cm, below of=dev1, yshift=-2em] (dev2) {\tiny\baselineskip=3pt Além disso, para lidar com o novo modelo, é evidente que há uma maior exigência de mão de obra qualificada no mercado. Para o filósofo Jürgen Habermas, isso é fruto da racionalidade instrumental do modelo capitalista de produção, que condiciona a mão de obra com base na lucratividade que ela fornecerá ao empregador. O problema é que boa parte da população não tem o integral acesso ao ensino superior ou técnico, fazendo essa parcela ser excluída do perfil adequado à atual indústria. Logo, à medida que a demanda por profissionais qualificados aumenta, o desemprego nas classes com menos acesso à educação segue a mesma tendência, inviabilizando uma maior equidade na realidade brasileira.\par};

    \node[rectangle, fill=snsgreen, rounded corners, text width=5cm, below of=dev2, yshift=-3em] (conc1) {\tiny\baselineskip=3pt Sendo assim, medidas devem ser tomadas para reverter o quadro promovido pela automação produtiva no país. Para tanto, cabe à Secretaria do Trabalho do Ministério da Economia, por meio da criação de um fundo monetário específico, a elaboração de uma política de fomento ao emprego que possa conciliar não só incentivos fiscais às empresas, mas também a promoção de cursos técnicos de manuseamento de equipamentos e outros aparatos surgidos com a automação industrial aos trabalhadores já empregados nesses ramos. Isso deve ser realizado com o apoio das secretarias a nível estadual, que podem aplicar a política tanto nas capitais quanto no interior das unidades federativas, objetivando, com a ação, promover a diminuição do desemprego e dos impasses decorrentes da automatização aos trabalhadores. Somente assim, será possível reverter a crise causada desde a Revolução Verde no mundo do trabalho e, além disso, atender as demandas, como as de Viktor de “League of Legends”, da indústria do Brasil no século XXI.\par};

    \draw [-, thick] ($ (int1.north west) + (-0.5em,-0.5em) $) -- ($ (int1.north west) + (-0.5em, 0.5em) $) -- ($ (int1.north east) + (0.5em, 0.5em) $) -- ($ (int1.north east) + (0.5em, -0.5em) $);
    \draw [-, thick] ($ (conc1.south west) + (-0.5em,0.5em) $) -- ($ (conc1.south west) + (-0.5em,-0.5em) $) -- ($ (conc1.south east) + (0.5em,-0.5em) $) -- ($ (conc1.south east) + (0.5em, 0.5em) $);

    \node [rectangle, fill=snsblue, rounded corners, text width=7cm, right of=dev1, xshift=16em, yshift=-3.5em] (merged) {\tiny\baselineskip=3pt No jogo eletrônico “League of Legends”, Viktor é um ciborgue que mora na  cidade de Piltover, uma das mais tecnológicas e sofisticadas de seu entorno. Na trama  desse personagem, as condições do meio em que vivia o fizeram substituir partes de seu  corpo, que lhe garantiam a condição de humano, por estruturas mecanizadas, que lhe forneciam maior eficiência nas atividades realizadas. Fora de tal microcosmo ficcional, é  notório que o âmbito do trabalho brasileiro vem adotando o mesmo postulado de Viktor  ao promover a automação em massa da produção, o que gera impactos nos índices de emprego da população. Desse modo, é fato que a automatização da indústria suscita uma problemática cada vez mais crescente no país, uma vez que tanto diminui o grau de pleno emprego na sociedade quanto seleciona cognitivamente os profissionais do futuro. \subnode{sub1}{\textcolor{gray}{$\backslash$n}} \par\vspace{1.5pt}
        \baselineskip=3pt Em primeiro lugar, convém ressaltar que a exigência de maior eficiência na cadeia produtiva faz com que as empresas optem pela utilização do maquinário na mão de obra, diminuindo os empregos disponíveis à população. No meio rural, por exemplo, a mudança abrupta do modelo extensivo de produção ficou conhecida como Revolução Verde e ocorreu em meados da década de 70. O novo sistema instaurado no campo, a partir desse período, passou a utilizar o maquinário pesado para as atividades que, antes, eram realizadas por trabalhadores assalariados. Assim, do mesmo modo que no âmbito rural, uma crise decorrente da falta de empregos passa a vigorar no meio urbano, já que a indústria adota, de modo crescente, novas tecnologias que possam reduzir a demanda de contratados. \subnode{sub2}{\textcolor{gray}{$\backslash$n}} \par\vspace{1.5pt}
        \baselineskip=3pt Além disso, para lidar com o novo modelo, é evidente que há uma maior exigência de mão de obra qualificada no mercado. Para o filósofo Jurgën Habermas, isso é fruto da racionalidade instrumental do modelo capitalista de produção, que condiciona a mão de obra com base na lucratividade que ela fornecerá ao empregador. O problema é que boa parte da população não tem o integral acesso ao ensino superior ou técnico, fazendo essa parcela ser excluída do perfil adequado à atual indústria. Logo, à medida que a demanda por profissionais qualificados aumenta, o desemprego nas classes com menos acesso à educação segue a mesma tendência, inviabilizando uma maior equidade na realidade brasileira. \subnode{sub3}{\textcolor{gray}{$\backslash$n}} \par\vspace{1.5pt}
        \baselineskip=3pt Sendo assim, medidas devem ser tomadas para reverter o quadro promovido pela automação produtiva no país. Para tanto, cabe à Secretaria do Trabalho do Ministério da Economia, por meio da criação de um fundo monetário específico, a elaboração de uma política de fomento ao emprego que possa conciliar não só incentivos fiscais às empresas, mas também a promoção de cursos técnicos de manuseamento de equipamentos e outros aparatos surgidos com a automação industrial aos trabalhadores já empregados nesses ramos. Isso deve ser realizado com o apoio das secretarias a nível estadual, que podem aplicar a política tanto nas capitais quanto no interior das unidades federativas, objetivando, com a ação, promover a diminuição do desemprego e dos impasses decorrentes da automatização aos trabalhadores. Somente assim, será possível reverter a crise causada desde a Revolução Verde no mundo do trabalho e, além disso, atender as demandas, como as de Viktor de “League of Legends”, da indústria do Brasil no século XXI. \subnode{sub4}{\textcolor{gray}{$\backslash$n}} \par};

\node[newline] at (sub1) {\tiny $\backslash$n};
\node[newline] at (sub2) {\tiny $\backslash$n};
\node[newline] at (sub3) {\tiny $\backslash$n};
\node[newline] at (sub4) {\tiny $\backslash$n};

\draw[-latex, line width=0.7em, orange!25] ($ (dev2.east) + (0em,2em) $) -- ($ (merged.west) + (0,1.2em) $);

\end{tikzpicture}

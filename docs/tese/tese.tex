% Arquivo LaTeX de exemplo de dissertação/tese a ser apresentada à CPG do IME-USP
%
% Criação: Jesús P. Mena-Chalco
% Revisão: Fabio Kon e Paulo Feofiloff
% Adaptação para UTF8, biblatex e outras melhorias: Nelson Lago
%
% Except where otherwise indicated, these files are distributed under
% the MIT Licence. The example text, which includes the tutorial and
% examples as well as the explanatory comments in the source, are
% available under the Creative Commons Attribution International
% Licence, v4.0 (CC-BY 4.0) - https://creativecommons.org/licenses/by/4.0/


%%%%%%%%%%%%%%%%%%%%%%%%%%%%%%%%%%%%%%%%%%%%%%%%%%%%%%%%%%%%%%%%%%%%%%%%%%%%%%%%
%%%%%%%%%%%%%%%%%%%%%%%%%%%%%%% PREÂMBULO LaTeX %%%%%%%%%%%%%%%%%%%%%%%%%%%%%%%%
%%%%%%%%%%%%%%%%%%%%%%%%%%%%%%%%%%%%%%%%%%%%%%%%%%%%%%%%%%%%%%%%%%%%%%%%%%%%%%%%

% A opção twoside (frente-e-verso) significa que a aparência das páginas pares
% e ímpares pode ser diferente. Por exemplo, as margens podem ser diferentes ou
% os números de página podem aparecer à direita ou à esquerda alternadamente.
% Mas nada impede que você crie um documento "só frente" e, ao imprimir, faça
% a impressão frente-e-verso.
%
% Aqui também definimos a língua padrão do documento
% (a última da lista) e línguas adicionais.
%\documentclass[12pt,twoside,brazilian,english]{book}
\documentclass[12pt,twoside,english,brazilian]{book}

% Ao invés de definir o tamanho das margens, vamos definir os tamanhos do
% texto, do cabeçalho e do rodapé, e deixamos a package geometry calcular
% o tamanho das margens em função do tamanho do papel. Assim, obtemos o
% mesmo resultado impresso, mas com margens diferentes, se o tamanho do
% papel for diferente.
\usepackage[a4paper]{geometry}

\geometry{
  textwidth=152mm,
  hmarginratio=12:17, % 24:34 -> com papel A4, 24mm + 152mm + 34mm = 210mm
  textheight=237mm,
  vmarginratio=8:7, % 32:28 -> com papel A4, 32mm + 237mm + 28mm = 297mm
  headsep=11mm, % distância entre a base do cabeçalho e o texto
  headheight=21mm, % qualquer medida grande o suficiente, p.ex., top - headsep
  footskip=10mm,
  marginpar=20mm,
  marginparsep=5mm,
}

% Vários pacotes e opções de configuração genéricos; para personalizar o
% resultado, modifique estes arquivos.
\input{extras/basics}
\input{extras/languages}
\input{extras/fonts}
\input{extras/floats}
\input{extras/imeusp-thesis} % capa, páginas preliminares e alguns detalhes
\input{extras/imeusp-formatting}
\input{extras/index}
\input{extras/bibconfig}
\input{extras/hyperlinks}
%\nocolorlinks % para impressão em P&B
\input{extras/source-code}
\input{extras/utils}
\input{extras/sectionsconfig}


% Diretórios onde estão as figuras; com isso, não é preciso colocar o caminho
% completo em \includegraphics (e nem a extensão).
\graphicspath{{figuras/},{logos/}}

% Comandos rápidos para mudar de língua:
% \en -> muda para o inglês
% \br -> muda para o português
% \texten{blah} -> o texto "blah" é em inglês
% \textbr{blah} -> o texto "blah" é em português
\babeltags{br = brazilian, en = english}

% Bibliografia
\usepackage[
  style=extras/plainnat-ime, % variante de autor-data, similar a plainnat
  backend=biber, % biber é o recomendado para usar com biblatex
  %style=alphabetic, % similar a alpha
  %style=numeric, % comum em artigos
  %style=authoryear-comp, % autor-data "padrão" do biblatex
  %style=apa, % variante de autor-data, muito usado
  %style=abnt,
]{biblatex}


%%%%%%%%%%%%%%%%%%%%%%%%%%%%%%%%%%%%%%%%%%%%%%%%%%%%%%%%%%%%%%%%%%%%%%%%%%%%%%%%
%%%%%%%%%%%%%%%%%%%%%%%%%%%%%%%%%% METADADOS %%%%%%%%%%%%%%%%%%%%%%%%%%%%%%%%%%%
%%%%%%%%%%%%%%%%%%%%%%%%%%%%%%%%%%%%%%%%%%%%%%%%%%%%%%%%%%%%%%%%%%%%%%%%%%%%%%%%

% O arquivo com os dados bibliográficos para biblatex; você pode usar
% este comando mais de uma vez para acrescentar múltiplos arquivos
\addbibresource{bibliografia.bib}

% Este comando permite acrescentar itens à lista de referências sem incluir
% uma referência de fato no texto (pode ser usado em qualquer lugar do texto)
%\nocite{bronevetsky02,schmidt03:MSc, FSF:GNU-GPL, CORBA:spec, MenaChalco08}
% Com este comando, todos os itens do arquivo .bib são incluídos na lista
% de referências
% \nocite{*}

% É possível definir como determinadas palavras podem (ou não) ser
% hifenizadas; no entanto, a hifenização automática geralmente funciona bem
\babelhyphenation{documentclass latexmk soft-ware clsguide} % todas as línguas
\babelhyphenation[brazilian]{Fu-la-no}
\babelhyphenation[english]{what-ever}

% Estes comandos definem o título e autoria do trabalho e devem sempre ser
% definidos, pois além de serem utilizados para criar a capa, também são
% armazenados nos metadados do PDF.
\title{
    % Obrigatório nas duas línguas
    titlept={Avaliação Automática de Redações Dissertativo-Argumentativas no Modelo do ENEM},
    titleen={Automatic Evaluation of Argumentative Essays in the ENEM Model},
    % Opcional, mas se houver deve existir nas duas línguas
    subtitlept={},
    subtitleen={},
}

\author{José Lucas Silva Mayer}

% Para TCCs, este comando define o supervisor
\orientador{Prof. Dr. Denis Deratani Mauá}

% Se não houver, remova; se houver mais de um, basta
% repetir o comando quantas vezes forem necessárias
\coorientador{Igor Cataneo Silveira}

% A página de rosto da versão para depósito (ou seja, a versão final
% antes da defesa) deve ser diferente da página de rosto da versão
% definitiva (ou seja, a versão final após a incorporação das sugestões
% da banca).
\defesa{
  nivel=tcc, % mestrado, doutorado ou tcc
  % É a versão para defesa ou a versão definitiva?
  %definitiva,
  local={São Paulo},
  data=2023-30-11, % YYYY-MM-DD
  % A licença do seu trabalho. Use CC-BY, CC-BY-NC, CC-BY-ND, CC-BY-SA,
  % CC-BY-NC-SA ou CC-BY-NC-ND para escolher a licença Creative Commons
  % correspondente (o sistema insere automaticamente o texto da licença).
  % Se quiser estabelecer regras diferentes para o uso de seu trabalho,
  % converse com seu orientador e coloque o texto da licença aqui, mas
  % observe que apenas TCCs sob alguma licença Creative Commons serão
  % acrescentados ao BDTA. Se você tem alguma intenção de publicar o
  % trabalho comercialmente no futuro, sugerimos a licença CC-BY-NC-ND.
  direitos={CC-BY}, % Creative Commons Attribution 4.0 International License
  %direitos={CC-BY-NC-ND}, % Creative Commons Attribution / NonCommercial /
                           % NoDerivatives 4.0 International License
  %direitos={Autorizo a reprodução e divulgação total ou parcial
  %          deste trabalho, por qualquer meio convencional ou
  %          eletrônico, para fins de estudo e pesquisa, desde que
  %          citada a fonte.},
  %direitos={I authorize the complete or partial reproduction and disclosure
  %          of this work by any conventional or electronic means for study
  %          and research purposes, provided that the source is acknowledged.}
  % Para gerar a ficha catalográfica, acesse https://fc.ime.usp.br/,
  % preencha o formulário e escolha a opção "Gerar Código LaTeX".
  % Basta copiar e colar o resultado aqui.
  fichacatalografica={},
}

%%%%%%%%%%%%%%%%%%%%%%%%%%%%%%%%%%%%%%%%%%%%%%%%%%%%%%%%%%%%%%%%%%%%%%%%%%%%%%%%
%%%%%%%%%%%%%%%%%%%%%%% AQUI COMEÇA O CONTEÚDO DE FATO %%%%%%%%%%%%%%%%%%%%%%%%%
%%%%%%%%%%%%%%%%%%%%%%%%%%%%%%%%%%%%%%%%%%%%%%%%%%%%%%%%%%%%%%%%%%%%%%%%%%%%%%%%

\begin{document}

%%%%%%%%%%%%%%%%%%%%%%%%%%% CAPA E PÁGINAS INICIAIS %%%%%%%%%%%%%%%%%%%%%%%%%%%%

% Aqui começa o conteúdo inicial que aparece antes do capítulo 1, ou seja,
% página de rosto, resumo, sumário etc. O comando frontmatter faz números
% de página aparecem em algarismos romanos ao invés de arábicos e
% desabilita a contagem de capítulos.
\frontmatter

\pagestyle{plain}

\onehalfspacing % Espaçamento 1,5 na capa e páginas iniciais

\maketitle % capa e folha de rosto

%%%%%%%%%%%%%%%% DEDICATÓRIA, AGRADECIMENTOS, RESUMO/ABSTRACT %%%%%%%%%%%%%%%%%%

\begin{dedicatoria}
\textit{In memoriam} de Luiz Augusto Paulino Jorge, que se aventurou comigo nas
primeiras jornadas da minha vida pela computação.
\end{dedicatoria}

% Reinicia o contador de páginas (a próxima página recebe o número "i") para
% que a página da dedicatória não seja contada.
\pagenumbering{roman}

% Agradecimentos:
% Se o candidato não quer fazer agradecimentos, deve simplesmente eliminar
% esta página. A epígrafe, obviamente, é opcional; é possível colocar
% epígrafes em todos os capítulos. O comando "\chapter*" faz esta seção
% não ser incluída no sumário.
\chapter*{Agradecimentos}
\epigrafe{Love is the one thing we're capable of perceiving that transcends dimensions of time and space}{Dr. Amelia Brand, \textit{Interstellar}}

Agradeço, de ínicio, à minha família, por ser minha base em toda a minha jornada acadêmica. Em especial, agradeço à minha mãe, Alexandra D. Silva Mayer, ao meu pai, Alaor Mayer, e ao meu irmão, Luan Silva Mayer, que sempre acreditaram no meu potencial e me deram todo o apoio e suporte necessário até aqui. Agradeço, também, às minhas avós Cícera Donizete da Silva e Elfrides Gasparini Mayer, por serem meu amparo emocional desde cedo.

Aos meus amigos do ensino fundamental e da minha cidade natal, em especial ao Jadeir Kelvin, ao Pedro Zauchin e ao Ulisses Borges, agradeço imensamente o apoio durante os anos da graduação. Foram incontáveis os dias em que vocês me ouviram desabafar sobre matérias difíceis ou questões emocionais da fase adulta. Sou muito grato pela presença de vocês em minha vida, e sei que o nosso amor não tem distância.

Aos amigos que tive o privilégio de conhecer na graduação: vocês tornaram o processo muito mais leve. Agradeço imensamente pelos sofrimentos compartilhados, pelos almoços e jantas na universidade, pelas festas, pelos jogos, pelos papos furados no saguão do CCSL, entre outros. Em especial, agradeço ao Max Cabrajac, um dos amigos mais legais que tive o privilégio de encontrar na vida, com quem sei que posso contar para qualquer coisa. Desbravar a Ciência da Computação com você foi muito mais legal.

Agradeço, também, aos meus professores, que sempre foram muito solícitos e dedicados em ensinar. Vocês foram muito importantes para o meu processo de maturação acadêmica e com certeza têm um papel crucial no avanço da ciência no país. Agradeço, em especial, ao meu orientador Denis Deratani Mauá, e ao meu coorientador Igor Cataneo Silveira, que estiveram disponíveis ao longo de todo o ano não só para tirar minhas dúvidas a respeito deste projeto, mas também para oferecer suporte e ferramentas para a execução do trabalho.

Por fim, agradeço à universidade dos meus sonhos, por me permitir estudar o que amo de forma gratuita e com qualidade. Sem dúvidas, a experiência da graduação na USP será, no futuro, uma das memórias mais bonitas que poderei ter!

%!TeX root=../tese.tex
%("dica" para o editor de texto: este arquivo é parte de um documento maior)
% para saber mais: https://tex.stackexchange.com/q/78101

% As palavras-chave são obrigatórias, em português e em inglês, e devem ser
% definidas antes do resumo/abstract. Acrescente quantas forem necessárias.
\palavrachave{ENEM}
\palavrachave{Inteligência Artificial}
\palavrachave{Processamento de Linguagem Natural}
\palavrachave{Avaliação Automática de Redações}
\palavrachave{Transformadores}
\palavrachave{BERTimbau}

\keyword{ENEM}
\keyword{Artificial Intelligence}
\keyword{Natural Language Processing}
\keyword{Automated Essay Scoring}
\keyword{Transformers}
\keyword{BERTimbau}

% O resumo é obrigatório, em português e inglês. Estes comandos também
% geram automaticamente a referência para o próprio documento, conforme
% as normas sugeridas da USP.
\resumo{A avaliação automatizada de redações, que integra as ferramentas de inteligência artificial à educação, tem obtido um notável crescimento nos últimos anos, impulsionada pelo uso de técnicas avançadas de processamento de linguagem natural. Esta abordagem, caracterizada pela atribuição de notas a produções textuais de estudantes, ganha destaque no âmbito do Exame Nacional do Ensino Médio (ENEM), principal meio de admissão ao ensino superior no Brasil. Nesse contexto, a tecnologia demonstra grande potencial para agilizar a cadeia logística de avaliações, oferecendo meios eficazes de qualificar as competências de escrita dos alunos em escala nacional. Este trabalho explora o desenvolvimento de sistemas automáticos de correção de redações em língua portuguesa, alinhados ao formato do ENEM. A partir de coletâneas abertas de textos avaliados por especialistas, elaboraram-se estruturas que se utilizam do BERTimbau, um modelo de linguagem fundamentado em \textit{transformers} e adaptado ao português, para codificar redações com riqueza semântica, empregando-as no treinamento de redes neurais capazes de atribuir notas correspondentes. Em particular, cinco sistemas independentes foram construídos com o intuito de avaliar os textos a partir de uma matriz de competências, como exigido pelo ENEM. Além disso, ao longo do projeto, aprimorou-se a eficácia dos modelos por meio da utilização de técnicas especializadas de aprendizado. Isso incluiu a investigação automática de configurações ideais de hiperparâmetros para as redes, buscando aprimorar a precisão dos resultados obtidos. Apesar da constatada ineficácia no processo de otimização de hiperparâmetros, a opção por arquiteturas fixas revelou-se promissora, aproximando os resultados da avaliação automática da qualidade das correções humanas.}

\abstract{The automated essay scoring, integrating artificial intelligence tools into education, has experienced remarkable growth in recent years, driven by the use of advanced natural language processing techniques. This approach, characterized by grading students' textual productions, stands out within the scope of the \textit{Exame Nacional do Ensino Médio} (ENEM), the primary means of admission to higher education in Brazil. In this context, technology demonstrates great potential to speed up the assessment logistics chain, offering effective means of qualifying students' writing skills on a national scale. This work explores the development of automatic essay correction systems in the Portuguese language, aligned with the ENEM format. The structures were built from open collections of texts evaluated by experts, utilizing BERTimbau, a large language model based on transformers and adapted to Portuguese, to generate essays embeddings with semantic richness. These structures are employed in the training of neural networks capable of assigning corresponding grades. Specifically, five independent systems were constructed to evaluate texts based on a competency matrix, as required by the ENEM. Throughout the project, the effectiveness of the models was enhanced through the use of specialized learning techniques. This included the automated hypertuning procedure for the networks, aiming to improve the accuracy of the results obtained. Despite the observed inefficiency in the hyperparameter optimization, the choice of fixed architectures proved promising, bringing the results of automatic assessment closer to the quality of human evaluations.}



%%%%%%%%%%%%%%%%%%%%%%%%%%% LISTAS DE FIGURAS ETC. %%%%%%%%%%%%%%%%%%%%%%%%%%%%%

% Como as listas que se seguem podem não incluir uma quebra de página
% obrigatória, inserimos uma quebra manualmente aqui.
\makeatletter
\if@openright\cleardoublepage\else\clearpage\fi
\makeatother

% Todas as listas são opcionais; Usando "\chapter*" elas não são incluídas
% no sumário. As listas geradas automaticamente também não são incluídas por
% conta das opções "notlot" e "notlof" que usamos para a package tocbibind.

% Normalmente, "\chapter*" faz o novo capítulo iniciar em uma nova página, e as
% listas geradas automaticamente também por padrão ficam em páginas separadas.
% Como cada uma destas listas é muito curta, não faz muito sentido fazer isso
% aqui, então usamos este comando para desabilitar essas quebras de página.
% Se você preferir, comente as linhas com esse comando e des-comente as linhas
% sem ele para criar as listas em páginas separadas. Observe que você também
% pode inserir quebras de página manualmente (com \clearpage, veja o exemplo
% mais abaixo).
\newcommand\disablenewpage[1]{{\let\clearpage\par\let\cleardoublepage\par #1}}

% Nestas listas, é melhor usar "raggedbottom" (veja basics.tex). Colocamos
% a opção correspondente e as listas dentro de um grupo para ativar
% raggedbottom apenas temporariamente.
\bgroup
\raggedbottom

%%%%% Listas criadas manualmente

%\chapter*{Lista de abreviaturas}
\disablenewpage{\chapter*{Lista de abreviaturas}}

\begin{tabular}{rl}
    ENEM & Exame Nacional do Ensino Médio\\
    INEP & Instituto Nacional de Estudos e Pesquisas Educacionais Anísio Teixeira\\
    SiSU & Sistema de Seleção Unificada\\
    TOEFL & \emph{Test of English as a Foreign Language}\\
    NLP & Processamento de Linguagem Natural (\emph{Natural Language Processing})\\
    AES & Avaliação Automática de Redações (\emph{Automated Essay Scoring})\\
    PEG & \textit{Project Essay Grader}\\
    SVM & Máquina de Vetores de Suporte (\emph{Support Vector Machine})\\
    RNN & Rede Neural Recorrente (\emph{Recurrent Neural Network})\\
    BRNN & \emph{Bidirectional Recurrent Neural Network}\\
    BPTT & \emph{Backpropagation Through Time}\\
    LSTM & \emph{Long Short-Term Memory}\\
    BiLSTM & \emph{Bidirectional Long Short-Term Memory}\\
    GRU & \emph{Gated Recurrent Unit}\\
    BERT & \emph{Bidirectional Encoder Representations from Transformers}\\
    MLM & Modelagem de Linguagem Mascarada \emph{Masked Language Model}\\
    NSP & Previsão de Próxima Frase \emph{Next Sentence Prediction}\\
    SGD & Gradiente Descendente Estocástico (\emph{Stochastic Gradient Descent})\\
    MSE & Erro Quadrático Médio (\emph{Mean Squared Error})\\
    PCE & Proporção de Correspondência Exata\\
    QWK & \emph{Quadratic Weighted Kappa}\\
    DCC & Departamento de Ciência da Computação\\
    IME & Instituto de Matemática e Estatística\\
    USP & Universidade de São Paulo
\end{tabular}

%\chapter*{Lista de símbolos}
%\disablenewpage{\chapter*{Lista de símbolos}}

%\begin{tabular}{rl}
%  $\omega$ & Frequência angular\\
%    $\psi$ & Função de análise \emph{wavelet}\\
%    $\Psi$ & Transformada de Fourier de $\psi$\\
%\end{tabular}

% Quebra de página manual
\clearpage

%%%%% Listas criadas automaticamente

% Você pode escolher se quer ou não permitir a quebra de página
%\listoffigures
\disablenewpage{\listoffigures}

% Você pode escolher se quer ou não permitir a quebra de página
%\listoftables
\disablenewpage{\listoftables}

% Esta lista é criada "automaticamente" pela package float quando
% definimos o novo tipo de float "program" (em utils.tex)
% Você pode escolher se quer ou não permitir a quebra de página
%\listof{program}{\programlistname}
\disablenewpage{\listof{program}{\programlistname}}

% Sumário (obrigatório)
\tableofcontents

\egroup % Final de "raggedbottom"

% Referências indiretas ("x", veja "y") para o índice remissivo (opcionais,
% pois o índice é opcional). É comum colocar esses itens no final do documento,
% junto com o comando \printindex, mas em alguns casos isso torna necessário
% executar texindy (ou makeindex) mais de uma vez, então colocar aqui é melhor.
\index{Inglês|see{Língua estrangeira}}
\index{Figuras|see{Floats}}
\index{Tabelas|see{Floats}}
\index{Código-fonte|see{Floats}}
\index{Subcaptions|see{Subfiguras}}
\index{Sublegendas|see{Subfiguras}}
\index{Equações|see{Modo matemático}}
\index{Fórmulas|see{Modo matemático}}
\index{Rodapé, notas|see{Notas de rodapé}}
\index{Captions|see{Legendas}}
\index{Versão original|see{Tese/Dissertação, versões}}
\index{Versão corrigida|see{Tese/Dissertação, versões}}
\index{Palavras estrangeiras|see{Língua estrangeira}}
\index{Floats!Algoritmo|see{Floats, ordem}}


%%%%%%%%%%%%%%%%%%%%%%%%%%%%%%%% CAPÍTULOS %%%%%%%%%%%%%%%%%%%%%%%%%%%%%%%%%%%%%

% Aqui vai o conteúdo principal do trabalho, ou seja, os capítulos que compõem
% a dissertação/tese. O comando mainmatter reinicia a contagem de páginas,
% modifica a numeração para números arábicos e ativa a contagem de capítulos.
\mainmatter

\pagestyle{mainmatter}

% Espaçamento simples
\singlespacing

% TODO: Adicionar referências ao longo de toda a introdução

\chapter{Introdução}

Tradicionalmente, técnicas de classificação de texto têm desempenhado um papel fundamental na avaliação e atribuição de notas a redações com base em vários parâmetros, como conteúdo, estrutura, coerência e proficiência linguística. Algoritmos clássicos de aprendizado de máquina têm sido utilizados para identificar e avaliar características-chave dentro das redações, com o intuito de mimetizar o processo humano de correção manual dos textos. Essas técnicas muitas vezes dependem de padrões avaliados sob medida e de conhecimento específico do domínio, o que pode ser trabalhoso e menos adaptável à natureza dinâmica da linguagem.

Além disso, os modelos tradicionais de avaliação textual automática enfrentam desafios na compreensão da semântica e do contexto das redações. Por exemplo, podem ter dificuldade em reconhecer nuances, metáforas ou referências culturais, que são aspectos críticos de uma correção eficaz. Tal limitação abriu caminho para modelos mais sofisticados, capazes de capturar a essência da linguagem humana de maneira mais abrangente.

Em particular, o surgimento recente de modelos de linguagem como o \textit{Bidirectional Encoder Representations from Transformers} (BERT) e as arquiteturas de \textit{transformers} auxiliou na mitigação desse problema. O BERT, introduzido pelo Google em 2018, revolucionou a área de Processamento de Linguagem Natural (NLP) ao introduzir os \textit{embeddings}\footnote{\textit{Embeddings}, no âmbito do NLP, são representações vetoriais numéricas de palavras, cuja codificação tem o intuito de garantir que o computador entenda seu contexto e semântica.} contextualizados de palavras. Os \textit{transformers}, arquitetura subjacente ao BERT, se destacaram, por sua vez, em várias tarefas de NLP ao considerar o contexto e as relações entre palavras, tornando-os excepcionalmente adequados para a avaliação de redações.

Esses modelos operam com base nos princípios de pré-treinamento e de ajuste fino. Durante a fase do pré-treinamento, aprendem as complexidades da linguagem ao prever palavras ausentes em frases e compreender o contexto em que são usadas. Posteriormente, o ajuste fino permite que os modelos se adaptem a tarefas específicas, como a correção automática, a partir do treinamento com dados já rotulados. A capacidade de considerar a redação como uma unidade contextual, em oposição a palavras ou frases isoladas, capacita os \textit{transformers} a capturar as complexidades da linguagem, incluindo semântica, coerência e até mesmo significados subjacentes ao texto.

Tais qualidades desempenham um papel fundamental na correção automática de testes de conhecimento e produção textual, como no caso do Exame Nacional do Ensino Médio (ENEM), um dos principais instrumentos de avaliação educacional do Brasil. Na prova de redação desse exame, são estabelecidos critérios objetivos de pontuação que abordam competências como coesão textual, argumentação e domínio temático, características que podem ser assimiladas na etapa de pré-treinamento, por exemplo.

Utilizando-se do BERT e dos modelos de linguagem baseados na arquitetura de \textit{transformers}, os sistemas de avaliação automática de redações testemunham um avanço significativo na capacidade de imitar os avaliadores humanos. Esses modelos podem avaliar a qualidade completa do texto, reconhecer nuances da língua portuguesa e fornecer uma nota coerente de acordo com critérios objetivos de correção. Ao longo desta monografia, exploraremos os princípios do ajuste fino do BERT na correção automática de redações com enfoque no modelo do ENEM, os desafios abarcados por essa tarefa e, por fim, o potencial dessa abordagem em relação aos métodos tradicionais e humanos de avaliação.

\section{Motivação}

O Exame Nacional do Ensino Médio (ENEM) é uma avaliação educacional de grande importância no Brasil, realizada anualmente pelo Instituto Nacional de Estudos e Pesquisas Educacionais Anísio Teixeira (INEP). Desde sua criação em 1998, o exame tem evoluído para se tornar uma referência no processo de acesso ao ensino superior no país. A pontuação obtida pelos estudantes no ENEM é utilizada para diversos fins, sendo um dos mais notáveis o ingresso em universidades públicas.

A redação é uma das principais provas do exame, cujo intuito é avaliar a defesa de um ponto de vista com respeito a um tema definido \textit{a priori}. Ela é capaz de mensurar a capacidade dos estudantes de comunicar ideias de forma clara, coerente e persuasiva, respeitando o gênero dissertativo-argumentativo e a norma padrão da língua portuguesa. Além disso, as habilidades críticas, como a análise de informações a construção de argumentos sólidos, também são levadas em conta no processo de correção da prova.

A avaliação das redações é baseada em cinco competências, cujas notas podem variar de 0 a 200 em intervalos de 40 pontos. A pontuação final é calculada somando-se todos esses valores, de modo que o resultado final esteja entre 0 e 1000. O processo de correção de cada redação é feito por dois avaliadores independentes e, caso as notas divirjam em mais de 100 pontos no total ou em mais de 80 pontos em cada competência, o texto é corrigido por um terceiro profissional. Por fim, se a pontuação ainda for divergente com ambas as avaliações, a redação é avaliada por uma banca de três professores.

Entretanto, a correção manual de redações produzidas por estudantes de todo o país representa um desafio logístico significativo. A disparidade entre a quantidade de profissionais e o número de estudantes que participam do exame a cada ano leva a uma sobrecarga que, além de aumentar o tempo de espera dos resultados, exige um esforço excessivo por parte dos avaliadores. % (ref)

Nessa perspectiva, a automação da correção de redações utilizando modelos de linguagem pode configurar uma solução promissora para o problema. O BERT e outras arquiteturas baseadas em \textit{transformers} têm o potencial de oferecer uma avaliação objetiva das redações dos estudantes, dado que eles podem ser refinados com uma gama de exemplos de redações já avaliadas por profissionais especializados e, assim, aprender a reconhecer critérios de qualidade para cada competência com base em dados reais.

Além disso, a avaliação automática permite uma resposta mais rápida e eficiente, uma vez que as notas podem ser geradas em um período mais curto. Isso facilita o processo logístico de divulgação dos resultados, reduzindo significativamente o tempo necessário para que os estudantes tenham acesso às suas pontuações.

\section{Contextualização}

% Falar sobre o dataset

\section{Objetivos}

% Citar quais são os objetivos do trabalho:
%
% - Desenvolver um modelo de correção automática de redações baseado no BERT (mais especificamente o BERTimbau)
% - Comparar o modelo desenvolvido com métodos tradicionais de extração de features
% - Comparar a acurácia e métricas de avaliação do modelo desenvolvido com as notas atribuídas pelos avaliadores humanos
% - Analisar os resultados obtidos e discutir as vantagens e desvantagens da abordagem proposta
% - Documentar o processo de escolha de arquitetura, treinamento com hypertuning, dificuldades encontradas e resultados

\chapter{Fundamentação Teórica}

\section{A Redação no ENEM}

A prova de redação no ENEM desempenha um papel fundamental na qualificação das habilidades dos estudantes, medindo a capacidade de expressarem ideias de maneira clara, coerente e persuasiva. Para avaliar as redações, são estabelecidos critérios objetivos que consideram a estrutura do texto e métricas específicas do exame, baseadas em uma matriz de referência.

\subsection{Estrutura do Texto}

A estrutura da redação no ENEM segue o formato dissertativo-argumentativo, exigindo que os participantes defendam um ponto de vista sobre um tema previamente estabelecido, conforme abordado em \cite{cartilha-redacao}. A composição textual deve respeitar a norma padrão da língua portuguesa e o gênero determinado, mantendo uma introdução, desenvolvimento e conclusão. A clareza na exposição de ideias, a adequação ao tema e a coesão textual são elementos essenciais para uma boa pontuação.

Para os autores \citet{platao-e-fiorin}, a dissertação deve explicitar uma visão concreta da realidade, recorrendo a referências para ilustrar afirmações ou para sustentar argumentos. No contexto do ENEM, esse é o papel exercido pela coesão e articulação da redação, já que o texto deve manter uma progressão lógica e um encadeamento de ideias com o fim de defender o ponto de vista do aluno. Além disso, os autores defendem que a dissertação sempre deve seguir um eixo temático à medida que aborda conceitos do mundo de modo abstrato, sem relações temporais ou espaciais. Nesse sentido, na prova, é imperativo que o texto apresente uma coerência temática, alinhando-se ao tópico proposto.

Ainda ao longo do texto, os estudantes devem articular informações externas para embasar seus discursos, o que é definido, em \cite{platao-e-fiorin}, como argumentos de autoridade e argumentos baseados em provas concretas. Assim, aliado à seleção de elementos coesivos adequados, o propósito da redação deve ser o de convencer o interlocutor em torno do tema abordado.

Por fim, é exigido que os alunos apresentem uma proposta de intervenção na conclusão do texto, que deve conter as ações necessárias para mitigar o problema levantado, os agentes responsáveis, os meios de solução, os possíveis desdobramentos da aplicação da proposta e, por fim, um detalhamento mais refinado da mediação (\cite[p.~20]{cartilha-redacao}).

\subsection{Métricas de Avaliação}

A avaliação das redações no ENEM adota uma abordagem multifacetada, incorporando diversas métricas para assegurar uma análise abrangente, incluindo desde a coesão textual até a capacidade de argumentação e a adequação ao padrão culto da língua. A seleção e organização eficazes de informações também são critérios fundamentais. A combinação desses parâmetros contribui significativamente para a construção de uma avaliação equitativa e holística das habilidades de escrita dos participantes.

Cada redação no ENEM é avaliada com base na matriz de referência do exame, composta por cinco competências, delineadas em suas respectivas habilidades, conforme detalhado em \cite{cartilha-redacao}. A primeira competência concentra-se na avaliação do emprego correto da língua portuguesa, abrangendo aspectos como ortografia, pontuação, concordância, regência verbal, etc. A segunda competência analisa a capacidade de interpretação do participante em relação ao tema proposto. Na terceira competência, são avaliados critérios como seleção e organização de informações, articulação de ideias e construção de argumentos. A quarta competência incide sobre o encadeamento dos argumentos e a progressão temática do texto. Por fim, a quinta competência avalia a habilidade do participante em elaborar uma proposta de intervenção para o problema abordado, considerando a adequação ao tema, a viabilidade prática de aplicação e o respeito aos direitos humanos. As cinco competências podem ser sintetizadas, de acordo com a Cartilha de Redação do ENEM (\cite{cartilha-redacao}), em:

\begin{enumerate}
    \item[\textbf{I.}] Demonstrar domínio da modalidade escrita formal da língua portuguesa.
    \item[\textbf{II.}] Compreender a proposta de redação e aplicar conceitos das várias áreas de conhecimento para desenvolver o tema, dentro dos limites estruturais do texto dissertativo-argumentativo em prosa.
    \item[\textbf{III.}] Selecionar, relacionar, organizar e interpretar informações, fatos, opiniões e argumentos em defesa de um ponto de vista.
    \item[\textbf{IV.}] Demonstrar conhecimento dos mecanismos linguísticos necessários para a construção da argumentação.
    \item[\textbf{V.}] Elaborar proposta de intervenção para o problema abordado, respeitando os direitos humanos.
\end{enumerate}

\subsection{Atribuição de Notas}

A atribuição de notas na redação do ENEM segue um processo rigoroso. Cada uma das cinco competências é avaliada em uma escala de 0 a 200 pontos, variando, nesse intervalo, em valores múltiplos de 40. As redações são corrigidas, inicialmente, por dois especialistas independentes e todo o processo é meticulosamente estruturado para prevenir possíveis divergências nas pontuações finais.

Duas avaliações são consideradas divergentes caso a nota atribuida a qualquer uma das competências difira em mais de 80 pontos ou se a diferença total entre as notas seja superior a 100 pontos. Nesses casos, uma terceira correção é realizada, e a nota final é a média aritmética das duas avaliações mais próximas. Caso todas as avaliações ainda sejam discrepantes entre si, a redação é submetida a uma banca independente, responsável por atribuir a nota final ao texto.

As notas levam em consideração as métricas da matriz de referência do exame, conforme estabelecido em \cite[p.~9-22]{cartilha-redacao}. A pontuação final varia de 0 a 1000 pontos, sendo calculada a partir da soma dos pontos atribuídos a cada competência. Este sistema de avaliação visa garantir uma análise abrangente e justa das redações dos participantes.

A tabela \ref{tab:competencia-1} abaixo, extraída da Cartilha de Redação do ENEM, ilustra a associação entre as pontuações atribuídas à competência I e os níveis de desempenho esperados dos alunos. As demais competências possuem uma associação similar de acordo com seus respectivos campos de avaliação.

\begin{table}[H]
    \centering
    \caption{Níveis de desempenho esperados para a competência I e notas associadas. Tabela extraída de \cite[p.~10]{cartilha-redacao}}
    \label{tab:competencia-1}
    \begin{tabularx}{\textwidth}{|c|X|}
        \hline
        \textbf{Pontuação} & \textbf{Níveis de desempenho} \\
        \hline
        200 pontos & Demonstra excelente domínio da modalidade escrita formal da língua portuguesa e de escolha de registro. Desvios gramaticais ou de convenções da escrita serão aceitos somente como excepcionalidade e quando não caracterizarem reincidência. \\
        \hline
        160 pontos & Demonstra bom domínio da modalidade escrita formal da língua portuguesa e de escolha de registro, com poucos desvios gramaticais e de convenções da escrita. \\
        \hline
        120 pontos & Demonstra domínio mediano da modalidade escrita formal da língua portuguesa e de escolha de registro, com alguns desvios gramaticais e de convenções da escrita. \\
        \hline
        80 pontos & Demonstra domínio insuficiente da modalidade escrita formal da língua portuguesa, com muitos desvios gramaticais, de escolha de registro e de convenções da escrita. \\
        \hline
        40 pontos & Demonstra domínio precário da modalidade escrita formal da língua portuguesa, de forma sistemática, com diversificados e frequentes desvios gramaticais, de escolha de registro e de convenções da escrita. \\
        \hline
        0 pontos & Demonstra desconhecimento da modalidade escrita formal da língua portuguesa. \\
        \hline
    \end{tabularx}
\end{table}

%\section{Avaliação Automática de Redações}

%\subsection{Definição e Importância}

%\subsection{Breve Histórico}

%\section{Representação de Textos para o Computador}

%\subsection{N-gramas}

%\subsection{Word Embeddings}

%\subsubsection{Word2Vec}

%\subsubsection{GloVe}

%\subsubsection{FastText}

%\section{Modelos Iniciais de Avaliação Automática de Redações}

%\subsection{Modelos Baseados em Extração de Características}

%\subsection{Modelos Baseados em Análise de Estruturas}

%\subsection{Modelos Baseados em Regressão}

%\section{Modelos de Linguagem}

%\subsection{Modelos Baseados em N-gramas}

%\subsection{Redes Neurais Recorrentes}

%\subsection{Long Short-Term Memory}

%\subsection{Gated Recurrent Unit}

%\section{Modelos de Linguagem Baseados em Transformadores}

%\subsection{Arquitetura}

%\subsection{Mecanismo de Atenção}

%\section{BERT: Bidirectional Encoder Representations from Transformers}

%\subsection{Arquitetura}

%\subsection{Pré-treinamento}

%\subsection{Fine-tuning}

%\subsection{BERTimbau: BERT para o Português Brasileiro}

%\subsection{O BERT na Avaliação Automática de Redações}

\chapter{Metodologia}
\label{chap:work}

Com a fundamentação teórica fornecida na seção \ref{chap:fund}, exploramos as potencialidades do aprendizado profundo e do processamento de linguagem natural para desenvolver um modelo especializado na avaliação automática de redações, seguindo o formato do ENEM.

A base de dados Essay-BR, disponível em suas versões simples (\cite{marinho-et-al-21}) e estendida (\cite{marinho-et-al-22}), desempenhou um papel crucial no treinamento dos modelos avaliadores. Cada modelo foi treinado para atribuir notas a competências específicas do exame, oferecendo uma abordagem granular e detalhada para a avaliação automatizada de redações.

A arquitetura de cada sistema de avaliação foi concebida com base no processo de \textit{fine-tuning} do BERT. A estrutura envolve uma camada de entrada para a codificação dos textos e uma camada de redes neurais especializadas na atribuição de notas. O modelo escolhido para o refinamento foi o BERTimbau, pré-treinado em uma ampla gama de textos em língua portuguesa, com o objetivo principal de capturar nuances linguísticas específicas --- essenciais para a correta avaliação de redações no contexto do ENEM.

Após a fase de treinamento, submetemos cada modelo a uma avaliação rigorosa, empregando métricas de desempenho padrão, como proporção de correspondência exata e perda, além de métricas específicas do ENEM, como a avaliação consistente entre competências. Este processo visa assegurar não só a precisão geral dos sistemas avaliadores, mas também a coerência e alinhamento com os critérios estabelecidos pelo exame.

\section{Bases de dados}

Para treinar modelos de avaliação automática, partimos de uma abordagem de aprendizado supervisionado, utilizando conjuntos de redações já avaliadas por especialistas humanos. A base de dados que orientou a elaboração desse trabalho foi a Essay-BR, uma coletânea de textos no modelo do ENEM com avaliações que seguem a a matriz de referência do exame (\cite{marinho-et-al-21}; \cite{marinho-et-al-22}). % add-custom-info

As redações foram extraídas de dois portais educacionais abertos do Brasil, \href{https://vestibular.brasilescola.uol.com.br/banco-de-redacoes}{Vestibular UOL}\footnote{\url{https://vestibular.brasilescola.uol.com.br}} e \href{https://educacao.uol.com.br/bancoderedacoes}{Educação UOL}\footnote{\url{https://educacao.uol.com.br/bancoderedacoes}}, que disponibilizam a correção dos textos por especialistas da área publicamente. Em ambos os \textit{websites}, existem diferentes propostas de temas semelhantes às do ENEM, de modo que alunos podem submeter suas escritas para a avaliação de acordo com sua preferência.

Para popular a base, os autores da Essay-BR utilizaram um \textit{web crawler}, sistema encarregado de raspar de dados de sites, que coletou o conteúdo das páginas de avaliação individual em diferentes eixos temáticos. Os resultados obtidos, então, foram filtrados, com o intuito de extrair o texto e outras informações relevantes das redações, como as avaliações por competências e a pontuação geral.

Neste trabalho, visando lidar com a base de textos, utilizamos a biblioteca Pandas\footnote{\url{https://pandas.pydata.org}}, que fornece estruturas e ferramentas de análise de dados de alto desempenho, e a biblioteca Matplotlib\footnote{\url{https://matplotlib.org}}, que permite a visualização das informações em diferentes formatos de gráficos. A Essay-BR possui duas variantes de tamanho, as quais detalharemos nas seções \ref{subsec:essay-br-basic} e \ref{subsec:essay-br-extended} a seguir.

\subsection{Essay-BR Básica}
\label{subsec:essay-br-basic}

A primeira versão da Essay-BR, a que vamos referenciar como básica, contém uma quantidade de 4570 textos coletados de dezembro de 2015 a abril de 2020, contemplando uma variedade de 86 temas distintos. Desse total, 798 redações foram coletadas do Educação UOL, enquanto que 3772 redações foram obtidas do Vestibular UOL, produzindo uma coletânea rica em diferentes conteúdos e estilos de escrita.

Cada instância na base de dados é caracterizada por cinco informações relacionadas às redações: o identificador do tema (\texttt{prompt}), o título (\texttt{title}), o texto propriamente dito (\texttt{essay}), as notas atribuídas às competências (\texttt{competence}), e a pontuação final (\texttt{score}). O título, por ser opcional, pode ser vazio. Além disso, a redação é representada como uma lista de parágrafos isolados.

A avaliação das redações é expressa por meio de notas atribuídas a diferentes competências (\texttt{competence}), representadas por uma lista de 5 números inteiros que variam de 0 a 200, com incrementos de 40. Tais notas, essenciais para uma correção multifacetada, são as responsáveis por formar a pontuação final (\texttt{score}). Embora o valor de \texttt{score} seja, em essência, uma informação redundante, já que pode ser derivado da soma das notas de cada competência, sua inclusão na base de dados revela-se valiosa para a análise exploratória dos dados.

A Figura \ref{fig:essay-br-basic-analysis} ilustra, de forma visual, a distribuição das pontuações finais e das notas individuais das competências na Essay-BR básica. Essa representação gráfica revela padrões iniciais sobre a variabilidade e a dispersão dos resultados obtidos pelos alunos, proporcionando uma visão panorâmica das características gerais de avaliação.

\begin{figure}[H]
    \caption{Gráficos com a distribuição da pontuação final (à esquerda) e a distribuição das notas das competências (à direita) da Essay-BR básica.}
    \label{fig:essay-br-basic-analysis}
    \centering
    \resizebox{\textwidth}{!}{%% Creator: Matplotlib, PGF backend
%%
%% To include the figure in your LaTeX document, write
%%   \input{<filename>.pgf}
%%
%% Make sure the required packages are loaded in your preamble
%%   \usepackage{pgf}
%%
%% Also ensure that all the required font packages are loaded; for instance,
%% the lmodern package is sometimes necessary when using math font.
%%   \usepackage{lmodern}
%%
%% Figures using additional raster images can only be included by \input if
%% they are in the same directory as the main LaTeX file. For loading figures
%% from other directories you can use the `import` package
%%   \usepackage{import}
%%
%% and then include the figures with
%%   \import{<path to file>}{<filename>.pgf}
%%
%% Matplotlib used the following preamble
%%   \def\mathdefault#1{#1}
%%   \everymath=\expandafter{\the\everymath\displaystyle}
%%
%%   \usepackage{fontspec}
%%   \setmainfont{DejaVuSerif.ttf}[Path=\detokenize{/Users/josemayer/Documents/Pacotes/mambaforge/lib/python3.10/site-packages/matplotlib/mpl-data/fonts/ttf/}]
%%   \setsansfont{DejaVuSans.ttf}[Path=\detokenize{/Users/josemayer/Documents/Pacotes/mambaforge/lib/python3.10/site-packages/matplotlib/mpl-data/fonts/ttf/}]
%%   \setmonofont{DejaVuSansMono.ttf}[Path=\detokenize{/Users/josemayer/Documents/Pacotes/mambaforge/lib/python3.10/site-packages/matplotlib/mpl-data/fonts/ttf/}]
%%   \makeatletter\@ifpackageloaded{underscore}{}{\usepackage[strings]{underscore}}\makeatother
%%
\begingroup%
\makeatletter%
\begin{pgfpicture}%
\pgfpathrectangle{\pgfpointorigin}{\pgfqpoint{12.000000in}{6.000000in}}%
\pgfusepath{use as bounding box, clip}%
\begin{pgfscope}%
\pgfsetbuttcap%
\pgfsetmiterjoin%
\definecolor{currentfill}{rgb}{1.000000,1.000000,1.000000}%
\pgfsetfillcolor{currentfill}%
\pgfsetlinewidth{0.000000pt}%
\definecolor{currentstroke}{rgb}{1.000000,1.000000,1.000000}%
\pgfsetstrokecolor{currentstroke}%
\pgfsetdash{}{0pt}%
\pgfpathmoveto{\pgfqpoint{0.000000in}{0.000000in}}%
\pgfpathlineto{\pgfqpoint{12.000000in}{0.000000in}}%
\pgfpathlineto{\pgfqpoint{12.000000in}{6.000000in}}%
\pgfpathlineto{\pgfqpoint{0.000000in}{6.000000in}}%
\pgfpathlineto{\pgfqpoint{0.000000in}{0.000000in}}%
\pgfpathclose%
\pgfusepath{fill}%
\end{pgfscope}%
\begin{pgfscope}%
\pgfsetbuttcap%
\pgfsetmiterjoin%
\definecolor{currentfill}{rgb}{1.000000,1.000000,1.000000}%
\pgfsetfillcolor{currentfill}%
\pgfsetlinewidth{0.000000pt}%
\definecolor{currentstroke}{rgb}{0.000000,0.000000,0.000000}%
\pgfsetstrokecolor{currentstroke}%
\pgfsetstrokeopacity{0.000000}%
\pgfsetdash{}{0pt}%
\pgfpathmoveto{\pgfqpoint{0.709028in}{0.387222in}}%
\pgfpathlineto{\pgfqpoint{5.978715in}{0.387222in}}%
\pgfpathlineto{\pgfqpoint{5.978715in}{5.631667in}}%
\pgfpathlineto{\pgfqpoint{0.709028in}{5.631667in}}%
\pgfpathlineto{\pgfqpoint{0.709028in}{0.387222in}}%
\pgfpathclose%
\pgfusepath{fill}%
\end{pgfscope}%
\begin{pgfscope}%
\pgfpathrectangle{\pgfqpoint{0.709028in}{0.387222in}}{\pgfqpoint{5.269687in}{5.244444in}}%
\pgfusepath{clip}%
\pgfsetbuttcap%
\pgfsetmiterjoin%
\definecolor{currentfill}{rgb}{0.121569,0.466667,0.705882}%
\pgfsetfillcolor{currentfill}%
\pgfsetfillopacity{0.500000}%
\pgfsetlinewidth{1.003750pt}%
\definecolor{currentstroke}{rgb}{0.000000,0.000000,0.000000}%
\pgfsetstrokecolor{currentstroke}%
\pgfsetdash{}{0pt}%
\pgfpathmoveto{\pgfqpoint{0.948559in}{0.387222in}}%
\pgfpathlineto{\pgfqpoint{1.132814in}{0.387222in}}%
\pgfpathlineto{\pgfqpoint{1.132814in}{1.076728in}}%
\pgfpathlineto{\pgfqpoint{0.948559in}{1.076728in}}%
\pgfpathlineto{\pgfqpoint{0.948559in}{0.387222in}}%
\pgfpathclose%
\pgfusepath{stroke,fill}%
\end{pgfscope}%
\begin{pgfscope}%
\pgfpathrectangle{\pgfqpoint{0.709028in}{0.387222in}}{\pgfqpoint{5.269687in}{5.244444in}}%
\pgfusepath{clip}%
\pgfsetbuttcap%
\pgfsetmiterjoin%
\definecolor{currentfill}{rgb}{0.121569,0.466667,0.705882}%
\pgfsetfillcolor{currentfill}%
\pgfsetfillopacity{0.500000}%
\pgfsetlinewidth{1.003750pt}%
\definecolor{currentstroke}{rgb}{0.000000,0.000000,0.000000}%
\pgfsetstrokecolor{currentstroke}%
\pgfsetdash{}{0pt}%
\pgfpathmoveto{\pgfqpoint{1.132814in}{0.387222in}}%
\pgfpathlineto{\pgfqpoint{1.317069in}{0.387222in}}%
\pgfpathlineto{\pgfqpoint{1.317069in}{0.395631in}}%
\pgfpathlineto{\pgfqpoint{1.132814in}{0.395631in}}%
\pgfpathlineto{\pgfqpoint{1.132814in}{0.387222in}}%
\pgfpathclose%
\pgfusepath{stroke,fill}%
\end{pgfscope}%
\begin{pgfscope}%
\pgfpathrectangle{\pgfqpoint{0.709028in}{0.387222in}}{\pgfqpoint{5.269687in}{5.244444in}}%
\pgfusepath{clip}%
\pgfsetbuttcap%
\pgfsetmiterjoin%
\definecolor{currentfill}{rgb}{0.121569,0.466667,0.705882}%
\pgfsetfillcolor{currentfill}%
\pgfsetfillopacity{0.500000}%
\pgfsetlinewidth{1.003750pt}%
\definecolor{currentstroke}{rgb}{0.000000,0.000000,0.000000}%
\pgfsetstrokecolor{currentstroke}%
\pgfsetdash{}{0pt}%
\pgfpathmoveto{\pgfqpoint{1.317069in}{0.387222in}}%
\pgfpathlineto{\pgfqpoint{1.501323in}{0.387222in}}%
\pgfpathlineto{\pgfqpoint{1.501323in}{0.437674in}}%
\pgfpathlineto{\pgfqpoint{1.317069in}{0.437674in}}%
\pgfpathlineto{\pgfqpoint{1.317069in}{0.387222in}}%
\pgfpathclose%
\pgfusepath{stroke,fill}%
\end{pgfscope}%
\begin{pgfscope}%
\pgfpathrectangle{\pgfqpoint{0.709028in}{0.387222in}}{\pgfqpoint{5.269687in}{5.244444in}}%
\pgfusepath{clip}%
\pgfsetbuttcap%
\pgfsetmiterjoin%
\definecolor{currentfill}{rgb}{0.121569,0.466667,0.705882}%
\pgfsetfillcolor{currentfill}%
\pgfsetfillopacity{0.500000}%
\pgfsetlinewidth{1.003750pt}%
\definecolor{currentstroke}{rgb}{0.000000,0.000000,0.000000}%
\pgfsetstrokecolor{currentstroke}%
\pgfsetdash{}{0pt}%
\pgfpathmoveto{\pgfqpoint{1.501323in}{0.387222in}}%
\pgfpathlineto{\pgfqpoint{1.685578in}{0.387222in}}%
\pgfpathlineto{\pgfqpoint{1.685578in}{0.437674in}}%
\pgfpathlineto{\pgfqpoint{1.501323in}{0.437674in}}%
\pgfpathlineto{\pgfqpoint{1.501323in}{0.387222in}}%
\pgfpathclose%
\pgfusepath{stroke,fill}%
\end{pgfscope}%
\begin{pgfscope}%
\pgfpathrectangle{\pgfqpoint{0.709028in}{0.387222in}}{\pgfqpoint{5.269687in}{5.244444in}}%
\pgfusepath{clip}%
\pgfsetbuttcap%
\pgfsetmiterjoin%
\definecolor{currentfill}{rgb}{0.121569,0.466667,0.705882}%
\pgfsetfillcolor{currentfill}%
\pgfsetfillopacity{0.500000}%
\pgfsetlinewidth{1.003750pt}%
\definecolor{currentstroke}{rgb}{0.000000,0.000000,0.000000}%
\pgfsetstrokecolor{currentstroke}%
\pgfsetdash{}{0pt}%
\pgfpathmoveto{\pgfqpoint{1.685578in}{0.387222in}}%
\pgfpathlineto{\pgfqpoint{1.869833in}{0.387222in}}%
\pgfpathlineto{\pgfqpoint{1.869833in}{0.546986in}}%
\pgfpathlineto{\pgfqpoint{1.685578in}{0.546986in}}%
\pgfpathlineto{\pgfqpoint{1.685578in}{0.387222in}}%
\pgfpathclose%
\pgfusepath{stroke,fill}%
\end{pgfscope}%
\begin{pgfscope}%
\pgfpathrectangle{\pgfqpoint{0.709028in}{0.387222in}}{\pgfqpoint{5.269687in}{5.244444in}}%
\pgfusepath{clip}%
\pgfsetbuttcap%
\pgfsetmiterjoin%
\definecolor{currentfill}{rgb}{0.121569,0.466667,0.705882}%
\pgfsetfillcolor{currentfill}%
\pgfsetfillopacity{0.500000}%
\pgfsetlinewidth{1.003750pt}%
\definecolor{currentstroke}{rgb}{0.000000,0.000000,0.000000}%
\pgfsetstrokecolor{currentstroke}%
\pgfsetdash{}{0pt}%
\pgfpathmoveto{\pgfqpoint{1.869833in}{0.387222in}}%
\pgfpathlineto{\pgfqpoint{2.054088in}{0.387222in}}%
\pgfpathlineto{\pgfqpoint{2.054088in}{0.563803in}}%
\pgfpathlineto{\pgfqpoint{1.869833in}{0.563803in}}%
\pgfpathlineto{\pgfqpoint{1.869833in}{0.387222in}}%
\pgfpathclose%
\pgfusepath{stroke,fill}%
\end{pgfscope}%
\begin{pgfscope}%
\pgfpathrectangle{\pgfqpoint{0.709028in}{0.387222in}}{\pgfqpoint{5.269687in}{5.244444in}}%
\pgfusepath{clip}%
\pgfsetbuttcap%
\pgfsetmiterjoin%
\definecolor{currentfill}{rgb}{0.121569,0.466667,0.705882}%
\pgfsetfillcolor{currentfill}%
\pgfsetfillopacity{0.500000}%
\pgfsetlinewidth{1.003750pt}%
\definecolor{currentstroke}{rgb}{0.000000,0.000000,0.000000}%
\pgfsetstrokecolor{currentstroke}%
\pgfsetdash{}{0pt}%
\pgfpathmoveto{\pgfqpoint{2.054088in}{0.387222in}}%
\pgfpathlineto{\pgfqpoint{2.238343in}{0.387222in}}%
\pgfpathlineto{\pgfqpoint{2.238343in}{0.664706in}}%
\pgfpathlineto{\pgfqpoint{2.054088in}{0.664706in}}%
\pgfpathlineto{\pgfqpoint{2.054088in}{0.387222in}}%
\pgfpathclose%
\pgfusepath{stroke,fill}%
\end{pgfscope}%
\begin{pgfscope}%
\pgfpathrectangle{\pgfqpoint{0.709028in}{0.387222in}}{\pgfqpoint{5.269687in}{5.244444in}}%
\pgfusepath{clip}%
\pgfsetbuttcap%
\pgfsetmiterjoin%
\definecolor{currentfill}{rgb}{0.121569,0.466667,0.705882}%
\pgfsetfillcolor{currentfill}%
\pgfsetfillopacity{0.500000}%
\pgfsetlinewidth{1.003750pt}%
\definecolor{currentstroke}{rgb}{0.000000,0.000000,0.000000}%
\pgfsetstrokecolor{currentstroke}%
\pgfsetdash{}{0pt}%
\pgfpathmoveto{\pgfqpoint{2.238343in}{0.387222in}}%
\pgfpathlineto{\pgfqpoint{2.422597in}{0.387222in}}%
\pgfpathlineto{\pgfqpoint{2.422597in}{0.622663in}}%
\pgfpathlineto{\pgfqpoint{2.238343in}{0.622663in}}%
\pgfpathlineto{\pgfqpoint{2.238343in}{0.387222in}}%
\pgfpathclose%
\pgfusepath{stroke,fill}%
\end{pgfscope}%
\begin{pgfscope}%
\pgfpathrectangle{\pgfqpoint{0.709028in}{0.387222in}}{\pgfqpoint{5.269687in}{5.244444in}}%
\pgfusepath{clip}%
\pgfsetbuttcap%
\pgfsetmiterjoin%
\definecolor{currentfill}{rgb}{0.121569,0.466667,0.705882}%
\pgfsetfillcolor{currentfill}%
\pgfsetfillopacity{0.500000}%
\pgfsetlinewidth{1.003750pt}%
\definecolor{currentstroke}{rgb}{0.000000,0.000000,0.000000}%
\pgfsetstrokecolor{currentstroke}%
\pgfsetdash{}{0pt}%
\pgfpathmoveto{\pgfqpoint{2.422597in}{0.387222in}}%
\pgfpathlineto{\pgfqpoint{2.606852in}{0.387222in}}%
\pgfpathlineto{\pgfqpoint{2.606852in}{0.807652in}}%
\pgfpathlineto{\pgfqpoint{2.422597in}{0.807652in}}%
\pgfpathlineto{\pgfqpoint{2.422597in}{0.387222in}}%
\pgfpathclose%
\pgfusepath{stroke,fill}%
\end{pgfscope}%
\begin{pgfscope}%
\pgfpathrectangle{\pgfqpoint{0.709028in}{0.387222in}}{\pgfqpoint{5.269687in}{5.244444in}}%
\pgfusepath{clip}%
\pgfsetbuttcap%
\pgfsetmiterjoin%
\definecolor{currentfill}{rgb}{0.121569,0.466667,0.705882}%
\pgfsetfillcolor{currentfill}%
\pgfsetfillopacity{0.500000}%
\pgfsetlinewidth{1.003750pt}%
\definecolor{currentstroke}{rgb}{0.000000,0.000000,0.000000}%
\pgfsetstrokecolor{currentstroke}%
\pgfsetdash{}{0pt}%
\pgfpathmoveto{\pgfqpoint{2.606852in}{0.387222in}}%
\pgfpathlineto{\pgfqpoint{2.791107in}{0.387222in}}%
\pgfpathlineto{\pgfqpoint{2.791107in}{0.916964in}}%
\pgfpathlineto{\pgfqpoint{2.606852in}{0.916964in}}%
\pgfpathlineto{\pgfqpoint{2.606852in}{0.387222in}}%
\pgfpathclose%
\pgfusepath{stroke,fill}%
\end{pgfscope}%
\begin{pgfscope}%
\pgfpathrectangle{\pgfqpoint{0.709028in}{0.387222in}}{\pgfqpoint{5.269687in}{5.244444in}}%
\pgfusepath{clip}%
\pgfsetbuttcap%
\pgfsetmiterjoin%
\definecolor{currentfill}{rgb}{0.121569,0.466667,0.705882}%
\pgfsetfillcolor{currentfill}%
\pgfsetfillopacity{0.500000}%
\pgfsetlinewidth{1.003750pt}%
\definecolor{currentstroke}{rgb}{0.000000,0.000000,0.000000}%
\pgfsetstrokecolor{currentstroke}%
\pgfsetdash{}{0pt}%
\pgfpathmoveto{\pgfqpoint{2.791107in}{0.387222in}}%
\pgfpathlineto{\pgfqpoint{2.975362in}{0.387222in}}%
\pgfpathlineto{\pgfqpoint{2.975362in}{2.110985in}}%
\pgfpathlineto{\pgfqpoint{2.791107in}{2.110985in}}%
\pgfpathlineto{\pgfqpoint{2.791107in}{0.387222in}}%
\pgfpathclose%
\pgfusepath{stroke,fill}%
\end{pgfscope}%
\begin{pgfscope}%
\pgfpathrectangle{\pgfqpoint{0.709028in}{0.387222in}}{\pgfqpoint{5.269687in}{5.244444in}}%
\pgfusepath{clip}%
\pgfsetbuttcap%
\pgfsetmiterjoin%
\definecolor{currentfill}{rgb}{0.121569,0.466667,0.705882}%
\pgfsetfillcolor{currentfill}%
\pgfsetfillopacity{0.500000}%
\pgfsetlinewidth{1.003750pt}%
\definecolor{currentstroke}{rgb}{0.000000,0.000000,0.000000}%
\pgfsetstrokecolor{currentstroke}%
\pgfsetdash{}{0pt}%
\pgfpathmoveto{\pgfqpoint{2.975362in}{0.387222in}}%
\pgfpathlineto{\pgfqpoint{3.159617in}{0.387222in}}%
\pgfpathlineto{\pgfqpoint{3.159617in}{2.186663in}}%
\pgfpathlineto{\pgfqpoint{2.975362in}{2.186663in}}%
\pgfpathlineto{\pgfqpoint{2.975362in}{0.387222in}}%
\pgfpathclose%
\pgfusepath{stroke,fill}%
\end{pgfscope}%
\begin{pgfscope}%
\pgfpathrectangle{\pgfqpoint{0.709028in}{0.387222in}}{\pgfqpoint{5.269687in}{5.244444in}}%
\pgfusepath{clip}%
\pgfsetbuttcap%
\pgfsetmiterjoin%
\definecolor{currentfill}{rgb}{0.121569,0.466667,0.705882}%
\pgfsetfillcolor{currentfill}%
\pgfsetfillopacity{0.500000}%
\pgfsetlinewidth{1.003750pt}%
\definecolor{currentstroke}{rgb}{0.000000,0.000000,0.000000}%
\pgfsetstrokecolor{currentstroke}%
\pgfsetdash{}{0pt}%
\pgfpathmoveto{\pgfqpoint{3.159617in}{0.387222in}}%
\pgfpathlineto{\pgfqpoint{3.343872in}{0.387222in}}%
\pgfpathlineto{\pgfqpoint{3.343872in}{2.724813in}}%
\pgfpathlineto{\pgfqpoint{3.159617in}{2.724813in}}%
\pgfpathlineto{\pgfqpoint{3.159617in}{0.387222in}}%
\pgfpathclose%
\pgfusepath{stroke,fill}%
\end{pgfscope}%
\begin{pgfscope}%
\pgfpathrectangle{\pgfqpoint{0.709028in}{0.387222in}}{\pgfqpoint{5.269687in}{5.244444in}}%
\pgfusepath{clip}%
\pgfsetbuttcap%
\pgfsetmiterjoin%
\definecolor{currentfill}{rgb}{0.121569,0.466667,0.705882}%
\pgfsetfillcolor{currentfill}%
\pgfsetfillopacity{0.500000}%
\pgfsetlinewidth{1.003750pt}%
\definecolor{currentstroke}{rgb}{0.000000,0.000000,0.000000}%
\pgfsetstrokecolor{currentstroke}%
\pgfsetdash{}{0pt}%
\pgfpathmoveto{\pgfqpoint{3.343872in}{0.387222in}}%
\pgfpathlineto{\pgfqpoint{3.528126in}{0.387222in}}%
\pgfpathlineto{\pgfqpoint{3.528126in}{3.649759in}}%
\pgfpathlineto{\pgfqpoint{3.343872in}{3.649759in}}%
\pgfpathlineto{\pgfqpoint{3.343872in}{0.387222in}}%
\pgfpathclose%
\pgfusepath{stroke,fill}%
\end{pgfscope}%
\begin{pgfscope}%
\pgfpathrectangle{\pgfqpoint{0.709028in}{0.387222in}}{\pgfqpoint{5.269687in}{5.244444in}}%
\pgfusepath{clip}%
\pgfsetbuttcap%
\pgfsetmiterjoin%
\definecolor{currentfill}{rgb}{0.121569,0.466667,0.705882}%
\pgfsetfillcolor{currentfill}%
\pgfsetfillopacity{0.500000}%
\pgfsetlinewidth{1.003750pt}%
\definecolor{currentstroke}{rgb}{0.000000,0.000000,0.000000}%
\pgfsetstrokecolor{currentstroke}%
\pgfsetdash{}{0pt}%
\pgfpathmoveto{\pgfqpoint{3.528126in}{0.387222in}}%
\pgfpathlineto{\pgfqpoint{3.712381in}{0.387222in}}%
\pgfpathlineto{\pgfqpoint{3.712381in}{3.321824in}}%
\pgfpathlineto{\pgfqpoint{3.528126in}{3.321824in}}%
\pgfpathlineto{\pgfqpoint{3.528126in}{0.387222in}}%
\pgfpathclose%
\pgfusepath{stroke,fill}%
\end{pgfscope}%
\begin{pgfscope}%
\pgfpathrectangle{\pgfqpoint{0.709028in}{0.387222in}}{\pgfqpoint{5.269687in}{5.244444in}}%
\pgfusepath{clip}%
\pgfsetbuttcap%
\pgfsetmiterjoin%
\definecolor{currentfill}{rgb}{0.121569,0.466667,0.705882}%
\pgfsetfillcolor{currentfill}%
\pgfsetfillopacity{0.500000}%
\pgfsetlinewidth{1.003750pt}%
\definecolor{currentstroke}{rgb}{0.000000,0.000000,0.000000}%
\pgfsetstrokecolor{currentstroke}%
\pgfsetdash{}{0pt}%
\pgfpathmoveto{\pgfqpoint{3.712381in}{0.387222in}}%
\pgfpathlineto{\pgfqpoint{3.896636in}{0.387222in}}%
\pgfpathlineto{\pgfqpoint{3.896636in}{5.381931in}}%
\pgfpathlineto{\pgfqpoint{3.712381in}{5.381931in}}%
\pgfpathlineto{\pgfqpoint{3.712381in}{0.387222in}}%
\pgfpathclose%
\pgfusepath{stroke,fill}%
\end{pgfscope}%
\begin{pgfscope}%
\pgfpathrectangle{\pgfqpoint{0.709028in}{0.387222in}}{\pgfqpoint{5.269687in}{5.244444in}}%
\pgfusepath{clip}%
\pgfsetbuttcap%
\pgfsetmiterjoin%
\definecolor{currentfill}{rgb}{0.121569,0.466667,0.705882}%
\pgfsetfillcolor{currentfill}%
\pgfsetfillopacity{0.500000}%
\pgfsetlinewidth{1.003750pt}%
\definecolor{currentstroke}{rgb}{0.000000,0.000000,0.000000}%
\pgfsetstrokecolor{currentstroke}%
\pgfsetdash{}{0pt}%
\pgfpathmoveto{\pgfqpoint{3.896636in}{0.387222in}}%
\pgfpathlineto{\pgfqpoint{4.080891in}{0.387222in}}%
\pgfpathlineto{\pgfqpoint{4.080891in}{2.817308in}}%
\pgfpathlineto{\pgfqpoint{3.896636in}{2.817308in}}%
\pgfpathlineto{\pgfqpoint{3.896636in}{0.387222in}}%
\pgfpathclose%
\pgfusepath{stroke,fill}%
\end{pgfscope}%
\begin{pgfscope}%
\pgfpathrectangle{\pgfqpoint{0.709028in}{0.387222in}}{\pgfqpoint{5.269687in}{5.244444in}}%
\pgfusepath{clip}%
\pgfsetbuttcap%
\pgfsetmiterjoin%
\definecolor{currentfill}{rgb}{0.121569,0.466667,0.705882}%
\pgfsetfillcolor{currentfill}%
\pgfsetfillopacity{0.500000}%
\pgfsetlinewidth{1.003750pt}%
\definecolor{currentstroke}{rgb}{0.000000,0.000000,0.000000}%
\pgfsetstrokecolor{currentstroke}%
\pgfsetdash{}{0pt}%
\pgfpathmoveto{\pgfqpoint{4.080891in}{0.387222in}}%
\pgfpathlineto{\pgfqpoint{4.265146in}{0.387222in}}%
\pgfpathlineto{\pgfqpoint{4.265146in}{4.490620in}}%
\pgfpathlineto{\pgfqpoint{4.080891in}{4.490620in}}%
\pgfpathlineto{\pgfqpoint{4.080891in}{0.387222in}}%
\pgfpathclose%
\pgfusepath{stroke,fill}%
\end{pgfscope}%
\begin{pgfscope}%
\pgfpathrectangle{\pgfqpoint{0.709028in}{0.387222in}}{\pgfqpoint{5.269687in}{5.244444in}}%
\pgfusepath{clip}%
\pgfsetbuttcap%
\pgfsetmiterjoin%
\definecolor{currentfill}{rgb}{0.121569,0.466667,0.705882}%
\pgfsetfillcolor{currentfill}%
\pgfsetfillopacity{0.500000}%
\pgfsetlinewidth{1.003750pt}%
\definecolor{currentstroke}{rgb}{0.000000,0.000000,0.000000}%
\pgfsetstrokecolor{currentstroke}%
\pgfsetdash{}{0pt}%
\pgfpathmoveto{\pgfqpoint{4.265146in}{0.387222in}}%
\pgfpathlineto{\pgfqpoint{4.449400in}{0.387222in}}%
\pgfpathlineto{\pgfqpoint{4.449400in}{3.717028in}}%
\pgfpathlineto{\pgfqpoint{4.265146in}{3.717028in}}%
\pgfpathlineto{\pgfqpoint{4.265146in}{0.387222in}}%
\pgfpathclose%
\pgfusepath{stroke,fill}%
\end{pgfscope}%
\begin{pgfscope}%
\pgfpathrectangle{\pgfqpoint{0.709028in}{0.387222in}}{\pgfqpoint{5.269687in}{5.244444in}}%
\pgfusepath{clip}%
\pgfsetbuttcap%
\pgfsetmiterjoin%
\definecolor{currentfill}{rgb}{0.121569,0.466667,0.705882}%
\pgfsetfillcolor{currentfill}%
\pgfsetfillopacity{0.500000}%
\pgfsetlinewidth{1.003750pt}%
\definecolor{currentstroke}{rgb}{0.000000,0.000000,0.000000}%
\pgfsetstrokecolor{currentstroke}%
\pgfsetdash{}{0pt}%
\pgfpathmoveto{\pgfqpoint{4.449400in}{0.387222in}}%
\pgfpathlineto{\pgfqpoint{4.633655in}{0.387222in}}%
\pgfpathlineto{\pgfqpoint{4.633655in}{2.699587in}}%
\pgfpathlineto{\pgfqpoint{4.449400in}{2.699587in}}%
\pgfpathlineto{\pgfqpoint{4.449400in}{0.387222in}}%
\pgfpathclose%
\pgfusepath{stroke,fill}%
\end{pgfscope}%
\begin{pgfscope}%
\pgfpathrectangle{\pgfqpoint{0.709028in}{0.387222in}}{\pgfqpoint{5.269687in}{5.244444in}}%
\pgfusepath{clip}%
\pgfsetbuttcap%
\pgfsetmiterjoin%
\definecolor{currentfill}{rgb}{0.121569,0.466667,0.705882}%
\pgfsetfillcolor{currentfill}%
\pgfsetfillopacity{0.500000}%
\pgfsetlinewidth{1.003750pt}%
\definecolor{currentstroke}{rgb}{0.000000,0.000000,0.000000}%
\pgfsetstrokecolor{currentstroke}%
\pgfsetdash{}{0pt}%
\pgfpathmoveto{\pgfqpoint{4.633655in}{0.387222in}}%
\pgfpathlineto{\pgfqpoint{4.817910in}{0.387222in}}%
\pgfpathlineto{\pgfqpoint{4.817910in}{2.775265in}}%
\pgfpathlineto{\pgfqpoint{4.633655in}{2.775265in}}%
\pgfpathlineto{\pgfqpoint{4.633655in}{0.387222in}}%
\pgfpathclose%
\pgfusepath{stroke,fill}%
\end{pgfscope}%
\begin{pgfscope}%
\pgfpathrectangle{\pgfqpoint{0.709028in}{0.387222in}}{\pgfqpoint{5.269687in}{5.244444in}}%
\pgfusepath{clip}%
\pgfsetbuttcap%
\pgfsetmiterjoin%
\definecolor{currentfill}{rgb}{0.121569,0.466667,0.705882}%
\pgfsetfillcolor{currentfill}%
\pgfsetfillopacity{0.500000}%
\pgfsetlinewidth{1.003750pt}%
\definecolor{currentstroke}{rgb}{0.000000,0.000000,0.000000}%
\pgfsetstrokecolor{currentstroke}%
\pgfsetdash{}{0pt}%
\pgfpathmoveto{\pgfqpoint{4.817910in}{0.387222in}}%
\pgfpathlineto{\pgfqpoint{5.002165in}{0.387222in}}%
\pgfpathlineto{\pgfqpoint{5.002165in}{2.094168in}}%
\pgfpathlineto{\pgfqpoint{4.817910in}{2.094168in}}%
\pgfpathlineto{\pgfqpoint{4.817910in}{0.387222in}}%
\pgfpathclose%
\pgfusepath{stroke,fill}%
\end{pgfscope}%
\begin{pgfscope}%
\pgfpathrectangle{\pgfqpoint{0.709028in}{0.387222in}}{\pgfqpoint{5.269687in}{5.244444in}}%
\pgfusepath{clip}%
\pgfsetbuttcap%
\pgfsetmiterjoin%
\definecolor{currentfill}{rgb}{0.121569,0.466667,0.705882}%
\pgfsetfillcolor{currentfill}%
\pgfsetfillopacity{0.500000}%
\pgfsetlinewidth{1.003750pt}%
\definecolor{currentstroke}{rgb}{0.000000,0.000000,0.000000}%
\pgfsetstrokecolor{currentstroke}%
\pgfsetdash{}{0pt}%
\pgfpathmoveto{\pgfqpoint{5.002165in}{0.387222in}}%
\pgfpathlineto{\pgfqpoint{5.186420in}{0.387222in}}%
\pgfpathlineto{\pgfqpoint{5.186420in}{1.656921in}}%
\pgfpathlineto{\pgfqpoint{5.002165in}{1.656921in}}%
\pgfpathlineto{\pgfqpoint{5.002165in}{0.387222in}}%
\pgfpathclose%
\pgfusepath{stroke,fill}%
\end{pgfscope}%
\begin{pgfscope}%
\pgfpathrectangle{\pgfqpoint{0.709028in}{0.387222in}}{\pgfqpoint{5.269687in}{5.244444in}}%
\pgfusepath{clip}%
\pgfsetbuttcap%
\pgfsetmiterjoin%
\definecolor{currentfill}{rgb}{0.121569,0.466667,0.705882}%
\pgfsetfillcolor{currentfill}%
\pgfsetfillopacity{0.500000}%
\pgfsetlinewidth{1.003750pt}%
\definecolor{currentstroke}{rgb}{0.000000,0.000000,0.000000}%
\pgfsetstrokecolor{currentstroke}%
\pgfsetdash{}{0pt}%
\pgfpathmoveto{\pgfqpoint{5.186420in}{0.387222in}}%
\pgfpathlineto{\pgfqpoint{5.370674in}{0.387222in}}%
\pgfpathlineto{\pgfqpoint{5.370674in}{1.143996in}}%
\pgfpathlineto{\pgfqpoint{5.186420in}{1.143996in}}%
\pgfpathlineto{\pgfqpoint{5.186420in}{0.387222in}}%
\pgfpathclose%
\pgfusepath{stroke,fill}%
\end{pgfscope}%
\begin{pgfscope}%
\pgfpathrectangle{\pgfqpoint{0.709028in}{0.387222in}}{\pgfqpoint{5.269687in}{5.244444in}}%
\pgfusepath{clip}%
\pgfsetbuttcap%
\pgfsetmiterjoin%
\definecolor{currentfill}{rgb}{0.121569,0.466667,0.705882}%
\pgfsetfillcolor{currentfill}%
\pgfsetfillopacity{0.500000}%
\pgfsetlinewidth{1.003750pt}%
\definecolor{currentstroke}{rgb}{0.000000,0.000000,0.000000}%
\pgfsetstrokecolor{currentstroke}%
\pgfsetdash{}{0pt}%
\pgfpathmoveto{\pgfqpoint{5.370674in}{0.387222in}}%
\pgfpathlineto{\pgfqpoint{5.554929in}{0.387222in}}%
\pgfpathlineto{\pgfqpoint{5.554929in}{0.664706in}}%
\pgfpathlineto{\pgfqpoint{5.370674in}{0.664706in}}%
\pgfpathlineto{\pgfqpoint{5.370674in}{0.387222in}}%
\pgfpathclose%
\pgfusepath{stroke,fill}%
\end{pgfscope}%
\begin{pgfscope}%
\pgfpathrectangle{\pgfqpoint{0.709028in}{0.387222in}}{\pgfqpoint{5.269687in}{5.244444in}}%
\pgfusepath{clip}%
\pgfsetbuttcap%
\pgfsetmiterjoin%
\definecolor{currentfill}{rgb}{0.121569,0.466667,0.705882}%
\pgfsetfillcolor{currentfill}%
\pgfsetfillopacity{0.500000}%
\pgfsetlinewidth{1.003750pt}%
\definecolor{currentstroke}{rgb}{0.000000,0.000000,0.000000}%
\pgfsetstrokecolor{currentstroke}%
\pgfsetdash{}{0pt}%
\pgfpathmoveto{\pgfqpoint{5.554929in}{0.387222in}}%
\pgfpathlineto{\pgfqpoint{5.739184in}{0.387222in}}%
\pgfpathlineto{\pgfqpoint{5.739184in}{0.589029in}}%
\pgfpathlineto{\pgfqpoint{5.554929in}{0.589029in}}%
\pgfpathlineto{\pgfqpoint{5.554929in}{0.387222in}}%
\pgfpathclose%
\pgfusepath{stroke,fill}%
\end{pgfscope}%
\begin{pgfscope}%
\pgfsetbuttcap%
\pgfsetroundjoin%
\definecolor{currentfill}{rgb}{0.000000,0.000000,0.000000}%
\pgfsetfillcolor{currentfill}%
\pgfsetlinewidth{0.803000pt}%
\definecolor{currentstroke}{rgb}{0.000000,0.000000,0.000000}%
\pgfsetstrokecolor{currentstroke}%
\pgfsetdash{}{0pt}%
\pgfsys@defobject{currentmarker}{\pgfqpoint{0.000000in}{-0.048611in}}{\pgfqpoint{0.000000in}{0.000000in}}{%
\pgfpathmoveto{\pgfqpoint{0.000000in}{0.000000in}}%
\pgfpathlineto{\pgfqpoint{0.000000in}{-0.048611in}}%
\pgfusepath{stroke,fill}%
}%
\begin{pgfscope}%
\pgfsys@transformshift{0.948559in}{0.387222in}%
\pgfsys@useobject{currentmarker}{}%
\end{pgfscope}%
\end{pgfscope}%
\begin{pgfscope}%
\definecolor{textcolor}{rgb}{0.000000,0.000000,0.000000}%
\pgfsetstrokecolor{textcolor}%
\pgfsetfillcolor{textcolor}%
\pgftext[x=0.948559in,y=0.290000in,,top]{\color{textcolor}{\sffamily\fontsize{10.000000}{12.000000}\selectfont\catcode`\^=\active\def^{\ifmmode\sp\else\^{}\fi}\catcode`\%=\active\def%{\%}0}}%
\end{pgfscope}%
\begin{pgfscope}%
\pgfsetbuttcap%
\pgfsetroundjoin%
\definecolor{currentfill}{rgb}{0.000000,0.000000,0.000000}%
\pgfsetfillcolor{currentfill}%
\pgfsetlinewidth{0.803000pt}%
\definecolor{currentstroke}{rgb}{0.000000,0.000000,0.000000}%
\pgfsetstrokecolor{currentstroke}%
\pgfsetdash{}{0pt}%
\pgfsys@defobject{currentmarker}{\pgfqpoint{0.000000in}{-0.048611in}}{\pgfqpoint{0.000000in}{0.000000in}}{%
\pgfpathmoveto{\pgfqpoint{0.000000in}{0.000000in}}%
\pgfpathlineto{\pgfqpoint{0.000000in}{-0.048611in}}%
\pgfusepath{stroke,fill}%
}%
\begin{pgfscope}%
\pgfsys@transformshift{1.869833in}{0.387222in}%
\pgfsys@useobject{currentmarker}{}%
\end{pgfscope}%
\end{pgfscope}%
\begin{pgfscope}%
\definecolor{textcolor}{rgb}{0.000000,0.000000,0.000000}%
\pgfsetstrokecolor{textcolor}%
\pgfsetfillcolor{textcolor}%
\pgftext[x=1.869833in,y=0.290000in,,top]{\color{textcolor}{\sffamily\fontsize{10.000000}{12.000000}\selectfont\catcode`\^=\active\def^{\ifmmode\sp\else\^{}\fi}\catcode`\%=\active\def%{\%}200}}%
\end{pgfscope}%
\begin{pgfscope}%
\pgfsetbuttcap%
\pgfsetroundjoin%
\definecolor{currentfill}{rgb}{0.000000,0.000000,0.000000}%
\pgfsetfillcolor{currentfill}%
\pgfsetlinewidth{0.803000pt}%
\definecolor{currentstroke}{rgb}{0.000000,0.000000,0.000000}%
\pgfsetstrokecolor{currentstroke}%
\pgfsetdash{}{0pt}%
\pgfsys@defobject{currentmarker}{\pgfqpoint{0.000000in}{-0.048611in}}{\pgfqpoint{0.000000in}{0.000000in}}{%
\pgfpathmoveto{\pgfqpoint{0.000000in}{0.000000in}}%
\pgfpathlineto{\pgfqpoint{0.000000in}{-0.048611in}}%
\pgfusepath{stroke,fill}%
}%
\begin{pgfscope}%
\pgfsys@transformshift{2.791107in}{0.387222in}%
\pgfsys@useobject{currentmarker}{}%
\end{pgfscope}%
\end{pgfscope}%
\begin{pgfscope}%
\definecolor{textcolor}{rgb}{0.000000,0.000000,0.000000}%
\pgfsetstrokecolor{textcolor}%
\pgfsetfillcolor{textcolor}%
\pgftext[x=2.791107in,y=0.290000in,,top]{\color{textcolor}{\sffamily\fontsize{10.000000}{12.000000}\selectfont\catcode`\^=\active\def^{\ifmmode\sp\else\^{}\fi}\catcode`\%=\active\def%{\%}400}}%
\end{pgfscope}%
\begin{pgfscope}%
\pgfsetbuttcap%
\pgfsetroundjoin%
\definecolor{currentfill}{rgb}{0.000000,0.000000,0.000000}%
\pgfsetfillcolor{currentfill}%
\pgfsetlinewidth{0.803000pt}%
\definecolor{currentstroke}{rgb}{0.000000,0.000000,0.000000}%
\pgfsetstrokecolor{currentstroke}%
\pgfsetdash{}{0pt}%
\pgfsys@defobject{currentmarker}{\pgfqpoint{0.000000in}{-0.048611in}}{\pgfqpoint{0.000000in}{0.000000in}}{%
\pgfpathmoveto{\pgfqpoint{0.000000in}{0.000000in}}%
\pgfpathlineto{\pgfqpoint{0.000000in}{-0.048611in}}%
\pgfusepath{stroke,fill}%
}%
\begin{pgfscope}%
\pgfsys@transformshift{3.712381in}{0.387222in}%
\pgfsys@useobject{currentmarker}{}%
\end{pgfscope}%
\end{pgfscope}%
\begin{pgfscope}%
\definecolor{textcolor}{rgb}{0.000000,0.000000,0.000000}%
\pgfsetstrokecolor{textcolor}%
\pgfsetfillcolor{textcolor}%
\pgftext[x=3.712381in,y=0.290000in,,top]{\color{textcolor}{\sffamily\fontsize{10.000000}{12.000000}\selectfont\catcode`\^=\active\def^{\ifmmode\sp\else\^{}\fi}\catcode`\%=\active\def%{\%}600}}%
\end{pgfscope}%
\begin{pgfscope}%
\pgfsetbuttcap%
\pgfsetroundjoin%
\definecolor{currentfill}{rgb}{0.000000,0.000000,0.000000}%
\pgfsetfillcolor{currentfill}%
\pgfsetlinewidth{0.803000pt}%
\definecolor{currentstroke}{rgb}{0.000000,0.000000,0.000000}%
\pgfsetstrokecolor{currentstroke}%
\pgfsetdash{}{0pt}%
\pgfsys@defobject{currentmarker}{\pgfqpoint{0.000000in}{-0.048611in}}{\pgfqpoint{0.000000in}{0.000000in}}{%
\pgfpathmoveto{\pgfqpoint{0.000000in}{0.000000in}}%
\pgfpathlineto{\pgfqpoint{0.000000in}{-0.048611in}}%
\pgfusepath{stroke,fill}%
}%
\begin{pgfscope}%
\pgfsys@transformshift{4.633655in}{0.387222in}%
\pgfsys@useobject{currentmarker}{}%
\end{pgfscope}%
\end{pgfscope}%
\begin{pgfscope}%
\definecolor{textcolor}{rgb}{0.000000,0.000000,0.000000}%
\pgfsetstrokecolor{textcolor}%
\pgfsetfillcolor{textcolor}%
\pgftext[x=4.633655in,y=0.290000in,,top]{\color{textcolor}{\sffamily\fontsize{10.000000}{12.000000}\selectfont\catcode`\^=\active\def^{\ifmmode\sp\else\^{}\fi}\catcode`\%=\active\def%{\%}800}}%
\end{pgfscope}%
\begin{pgfscope}%
\pgfsetbuttcap%
\pgfsetroundjoin%
\definecolor{currentfill}{rgb}{0.000000,0.000000,0.000000}%
\pgfsetfillcolor{currentfill}%
\pgfsetlinewidth{0.803000pt}%
\definecolor{currentstroke}{rgb}{0.000000,0.000000,0.000000}%
\pgfsetstrokecolor{currentstroke}%
\pgfsetdash{}{0pt}%
\pgfsys@defobject{currentmarker}{\pgfqpoint{0.000000in}{-0.048611in}}{\pgfqpoint{0.000000in}{0.000000in}}{%
\pgfpathmoveto{\pgfqpoint{0.000000in}{0.000000in}}%
\pgfpathlineto{\pgfqpoint{0.000000in}{-0.048611in}}%
\pgfusepath{stroke,fill}%
}%
\begin{pgfscope}%
\pgfsys@transformshift{5.554929in}{0.387222in}%
\pgfsys@useobject{currentmarker}{}%
\end{pgfscope}%
\end{pgfscope}%
\begin{pgfscope}%
\definecolor{textcolor}{rgb}{0.000000,0.000000,0.000000}%
\pgfsetstrokecolor{textcolor}%
\pgfsetfillcolor{textcolor}%
\pgftext[x=5.554929in,y=0.290000in,,top]{\color{textcolor}{\sffamily\fontsize{10.000000}{12.000000}\selectfont\catcode`\^=\active\def^{\ifmmode\sp\else\^{}\fi}\catcode`\%=\active\def%{\%}1000}}%
\end{pgfscope}%
\begin{pgfscope}%
\pgfsetbuttcap%
\pgfsetroundjoin%
\definecolor{currentfill}{rgb}{0.000000,0.000000,0.000000}%
\pgfsetfillcolor{currentfill}%
\pgfsetlinewidth{0.803000pt}%
\definecolor{currentstroke}{rgb}{0.000000,0.000000,0.000000}%
\pgfsetstrokecolor{currentstroke}%
\pgfsetdash{}{0pt}%
\pgfsys@defobject{currentmarker}{\pgfqpoint{-0.048611in}{0.000000in}}{\pgfqpoint{-0.000000in}{0.000000in}}{%
\pgfpathmoveto{\pgfqpoint{-0.000000in}{0.000000in}}%
\pgfpathlineto{\pgfqpoint{-0.048611in}{0.000000in}}%
\pgfusepath{stroke,fill}%
}%
\begin{pgfscope}%
\pgfsys@transformshift{0.709028in}{0.387222in}%
\pgfsys@useobject{currentmarker}{}%
\end{pgfscope}%
\end{pgfscope}%
\begin{pgfscope}%
\definecolor{textcolor}{rgb}{0.000000,0.000000,0.000000}%
\pgfsetstrokecolor{textcolor}%
\pgfsetfillcolor{textcolor}%
\pgftext[x=0.523440in, y=0.334461in, left, base]{\color{textcolor}{\sffamily\fontsize{10.000000}{12.000000}\selectfont\catcode`\^=\active\def^{\ifmmode\sp\else\^{}\fi}\catcode`\%=\active\def%{\%}0}}%
\end{pgfscope}%
\begin{pgfscope}%
\pgfsetbuttcap%
\pgfsetroundjoin%
\definecolor{currentfill}{rgb}{0.000000,0.000000,0.000000}%
\pgfsetfillcolor{currentfill}%
\pgfsetlinewidth{0.803000pt}%
\definecolor{currentstroke}{rgb}{0.000000,0.000000,0.000000}%
\pgfsetstrokecolor{currentstroke}%
\pgfsetdash{}{0pt}%
\pgfsys@defobject{currentmarker}{\pgfqpoint{-0.048611in}{0.000000in}}{\pgfqpoint{-0.000000in}{0.000000in}}{%
\pgfpathmoveto{\pgfqpoint{-0.000000in}{0.000000in}}%
\pgfpathlineto{\pgfqpoint{-0.048611in}{0.000000in}}%
\pgfusepath{stroke,fill}%
}%
\begin{pgfscope}%
\pgfsys@transformshift{0.709028in}{1.228082in}%
\pgfsys@useobject{currentmarker}{}%
\end{pgfscope}%
\end{pgfscope}%
\begin{pgfscope}%
\definecolor{textcolor}{rgb}{0.000000,0.000000,0.000000}%
\pgfsetstrokecolor{textcolor}%
\pgfsetfillcolor{textcolor}%
\pgftext[x=0.346710in, y=1.175321in, left, base]{\color{textcolor}{\sffamily\fontsize{10.000000}{12.000000}\selectfont\catcode`\^=\active\def^{\ifmmode\sp\else\^{}\fi}\catcode`\%=\active\def%{\%}100}}%
\end{pgfscope}%
\begin{pgfscope}%
\pgfsetbuttcap%
\pgfsetroundjoin%
\definecolor{currentfill}{rgb}{0.000000,0.000000,0.000000}%
\pgfsetfillcolor{currentfill}%
\pgfsetlinewidth{0.803000pt}%
\definecolor{currentstroke}{rgb}{0.000000,0.000000,0.000000}%
\pgfsetstrokecolor{currentstroke}%
\pgfsetdash{}{0pt}%
\pgfsys@defobject{currentmarker}{\pgfqpoint{-0.048611in}{0.000000in}}{\pgfqpoint{-0.000000in}{0.000000in}}{%
\pgfpathmoveto{\pgfqpoint{-0.000000in}{0.000000in}}%
\pgfpathlineto{\pgfqpoint{-0.048611in}{0.000000in}}%
\pgfusepath{stroke,fill}%
}%
\begin{pgfscope}%
\pgfsys@transformshift{0.709028in}{2.068942in}%
\pgfsys@useobject{currentmarker}{}%
\end{pgfscope}%
\end{pgfscope}%
\begin{pgfscope}%
\definecolor{textcolor}{rgb}{0.000000,0.000000,0.000000}%
\pgfsetstrokecolor{textcolor}%
\pgfsetfillcolor{textcolor}%
\pgftext[x=0.346710in, y=2.016181in, left, base]{\color{textcolor}{\sffamily\fontsize{10.000000}{12.000000}\selectfont\catcode`\^=\active\def^{\ifmmode\sp\else\^{}\fi}\catcode`\%=\active\def%{\%}200}}%
\end{pgfscope}%
\begin{pgfscope}%
\pgfsetbuttcap%
\pgfsetroundjoin%
\definecolor{currentfill}{rgb}{0.000000,0.000000,0.000000}%
\pgfsetfillcolor{currentfill}%
\pgfsetlinewidth{0.803000pt}%
\definecolor{currentstroke}{rgb}{0.000000,0.000000,0.000000}%
\pgfsetstrokecolor{currentstroke}%
\pgfsetdash{}{0pt}%
\pgfsys@defobject{currentmarker}{\pgfqpoint{-0.048611in}{0.000000in}}{\pgfqpoint{-0.000000in}{0.000000in}}{%
\pgfpathmoveto{\pgfqpoint{-0.000000in}{0.000000in}}%
\pgfpathlineto{\pgfqpoint{-0.048611in}{0.000000in}}%
\pgfusepath{stroke,fill}%
}%
\begin{pgfscope}%
\pgfsys@transformshift{0.709028in}{2.909803in}%
\pgfsys@useobject{currentmarker}{}%
\end{pgfscope}%
\end{pgfscope}%
\begin{pgfscope}%
\definecolor{textcolor}{rgb}{0.000000,0.000000,0.000000}%
\pgfsetstrokecolor{textcolor}%
\pgfsetfillcolor{textcolor}%
\pgftext[x=0.346710in, y=2.857041in, left, base]{\color{textcolor}{\sffamily\fontsize{10.000000}{12.000000}\selectfont\catcode`\^=\active\def^{\ifmmode\sp\else\^{}\fi}\catcode`\%=\active\def%{\%}300}}%
\end{pgfscope}%
\begin{pgfscope}%
\pgfsetbuttcap%
\pgfsetroundjoin%
\definecolor{currentfill}{rgb}{0.000000,0.000000,0.000000}%
\pgfsetfillcolor{currentfill}%
\pgfsetlinewidth{0.803000pt}%
\definecolor{currentstroke}{rgb}{0.000000,0.000000,0.000000}%
\pgfsetstrokecolor{currentstroke}%
\pgfsetdash{}{0pt}%
\pgfsys@defobject{currentmarker}{\pgfqpoint{-0.048611in}{0.000000in}}{\pgfqpoint{-0.000000in}{0.000000in}}{%
\pgfpathmoveto{\pgfqpoint{-0.000000in}{0.000000in}}%
\pgfpathlineto{\pgfqpoint{-0.048611in}{0.000000in}}%
\pgfusepath{stroke,fill}%
}%
\begin{pgfscope}%
\pgfsys@transformshift{0.709028in}{3.750663in}%
\pgfsys@useobject{currentmarker}{}%
\end{pgfscope}%
\end{pgfscope}%
\begin{pgfscope}%
\definecolor{textcolor}{rgb}{0.000000,0.000000,0.000000}%
\pgfsetstrokecolor{textcolor}%
\pgfsetfillcolor{textcolor}%
\pgftext[x=0.346710in, y=3.697901in, left, base]{\color{textcolor}{\sffamily\fontsize{10.000000}{12.000000}\selectfont\catcode`\^=\active\def^{\ifmmode\sp\else\^{}\fi}\catcode`\%=\active\def%{\%}400}}%
\end{pgfscope}%
\begin{pgfscope}%
\pgfsetbuttcap%
\pgfsetroundjoin%
\definecolor{currentfill}{rgb}{0.000000,0.000000,0.000000}%
\pgfsetfillcolor{currentfill}%
\pgfsetlinewidth{0.803000pt}%
\definecolor{currentstroke}{rgb}{0.000000,0.000000,0.000000}%
\pgfsetstrokecolor{currentstroke}%
\pgfsetdash{}{0pt}%
\pgfsys@defobject{currentmarker}{\pgfqpoint{-0.048611in}{0.000000in}}{\pgfqpoint{-0.000000in}{0.000000in}}{%
\pgfpathmoveto{\pgfqpoint{-0.000000in}{0.000000in}}%
\pgfpathlineto{\pgfqpoint{-0.048611in}{0.000000in}}%
\pgfusepath{stroke,fill}%
}%
\begin{pgfscope}%
\pgfsys@transformshift{0.709028in}{4.591523in}%
\pgfsys@useobject{currentmarker}{}%
\end{pgfscope}%
\end{pgfscope}%
\begin{pgfscope}%
\definecolor{textcolor}{rgb}{0.000000,0.000000,0.000000}%
\pgfsetstrokecolor{textcolor}%
\pgfsetfillcolor{textcolor}%
\pgftext[x=0.346710in, y=4.538761in, left, base]{\color{textcolor}{\sffamily\fontsize{10.000000}{12.000000}\selectfont\catcode`\^=\active\def^{\ifmmode\sp\else\^{}\fi}\catcode`\%=\active\def%{\%}500}}%
\end{pgfscope}%
\begin{pgfscope}%
\pgfsetbuttcap%
\pgfsetroundjoin%
\definecolor{currentfill}{rgb}{0.000000,0.000000,0.000000}%
\pgfsetfillcolor{currentfill}%
\pgfsetlinewidth{0.803000pt}%
\definecolor{currentstroke}{rgb}{0.000000,0.000000,0.000000}%
\pgfsetstrokecolor{currentstroke}%
\pgfsetdash{}{0pt}%
\pgfsys@defobject{currentmarker}{\pgfqpoint{-0.048611in}{0.000000in}}{\pgfqpoint{-0.000000in}{0.000000in}}{%
\pgfpathmoveto{\pgfqpoint{-0.000000in}{0.000000in}}%
\pgfpathlineto{\pgfqpoint{-0.048611in}{0.000000in}}%
\pgfusepath{stroke,fill}%
}%
\begin{pgfscope}%
\pgfsys@transformshift{0.709028in}{5.432383in}%
\pgfsys@useobject{currentmarker}{}%
\end{pgfscope}%
\end{pgfscope}%
\begin{pgfscope}%
\definecolor{textcolor}{rgb}{0.000000,0.000000,0.000000}%
\pgfsetstrokecolor{textcolor}%
\pgfsetfillcolor{textcolor}%
\pgftext[x=0.346710in, y=5.379621in, left, base]{\color{textcolor}{\sffamily\fontsize{10.000000}{12.000000}\selectfont\catcode`\^=\active\def^{\ifmmode\sp\else\^{}\fi}\catcode`\%=\active\def%{\%}600}}%
\end{pgfscope}%
\begin{pgfscope}%
\definecolor{textcolor}{rgb}{0.000000,0.000000,0.000000}%
\pgfsetstrokecolor{textcolor}%
\pgfsetfillcolor{textcolor}%
\pgftext[x=0.291154in,y=3.009444in,,bottom,rotate=90.000000]{\color{textcolor}{\sffamily\fontsize{10.000000}{12.000000}\selectfont\catcode`\^=\active\def^{\ifmmode\sp\else\^{}\fi}\catcode`\%=\active\def%{\%}Quantidade de redações}}%
\end{pgfscope}%
\begin{pgfscope}%
\pgfpathrectangle{\pgfqpoint{0.709028in}{0.387222in}}{\pgfqpoint{5.269687in}{5.244444in}}%
\pgfusepath{clip}%
\pgfsetrectcap%
\pgfsetroundjoin%
\pgfsetlinewidth{1.505625pt}%
\definecolor{currentstroke}{rgb}{0.121569,0.466667,0.705882}%
\pgfsetstrokecolor{currentstroke}%
\pgfsetdash{}{0pt}%
\pgfpathmoveto{\pgfqpoint{0.948559in}{0.726045in}}%
\pgfpathlineto{\pgfqpoint{0.971707in}{0.723067in}}%
\pgfpathlineto{\pgfqpoint{0.994854in}{0.712725in}}%
\pgfpathlineto{\pgfqpoint{1.018002in}{0.695826in}}%
\pgfpathlineto{\pgfqpoint{1.041149in}{0.673598in}}%
\pgfpathlineto{\pgfqpoint{1.064297in}{0.647556in}}%
\pgfpathlineto{\pgfqpoint{1.133740in}{0.562687in}}%
\pgfpathlineto{\pgfqpoint{1.156887in}{0.536855in}}%
\pgfpathlineto{\pgfqpoint{1.180035in}{0.513916in}}%
\pgfpathlineto{\pgfqpoint{1.203183in}{0.494368in}}%
\pgfpathlineto{\pgfqpoint{1.226330in}{0.478392in}}%
\pgfpathlineto{\pgfqpoint{1.249478in}{0.465913in}}%
\pgfpathlineto{\pgfqpoint{1.272625in}{0.456664in}}%
\pgfpathlineto{\pgfqpoint{1.295773in}{0.450272in}}%
\pgfpathlineto{\pgfqpoint{1.318920in}{0.446320in}}%
\pgfpathlineto{\pgfqpoint{1.342068in}{0.444402in}}%
\pgfpathlineto{\pgfqpoint{1.365216in}{0.444164in}}%
\pgfpathlineto{\pgfqpoint{1.388363in}{0.445320in}}%
\pgfpathlineto{\pgfqpoint{1.434658in}{0.451020in}}%
\pgfpathlineto{\pgfqpoint{1.480954in}{0.460421in}}%
\pgfpathlineto{\pgfqpoint{1.527249in}{0.472849in}}%
\pgfpathlineto{\pgfqpoint{1.573544in}{0.487516in}}%
\pgfpathlineto{\pgfqpoint{1.735577in}{0.541699in}}%
\pgfpathlineto{\pgfqpoint{1.920758in}{0.596931in}}%
\pgfpathlineto{\pgfqpoint{1.990201in}{0.616751in}}%
\pgfpathlineto{\pgfqpoint{2.036496in}{0.628305in}}%
\pgfpathlineto{\pgfqpoint{2.175381in}{0.659768in}}%
\pgfpathlineto{\pgfqpoint{2.198529in}{0.666804in}}%
\pgfpathlineto{\pgfqpoint{2.221676in}{0.675051in}}%
\pgfpathlineto{\pgfqpoint{2.244824in}{0.684711in}}%
\pgfpathlineto{\pgfqpoint{2.267972in}{0.695935in}}%
\pgfpathlineto{\pgfqpoint{2.291119in}{0.708829in}}%
\pgfpathlineto{\pgfqpoint{2.314267in}{0.723474in}}%
\pgfpathlineto{\pgfqpoint{2.337414in}{0.739960in}}%
\pgfpathlineto{\pgfqpoint{2.360562in}{0.758417in}}%
\pgfpathlineto{\pgfqpoint{2.383710in}{0.779058in}}%
\pgfpathlineto{\pgfqpoint{2.406857in}{0.802210in}}%
\pgfpathlineto{\pgfqpoint{2.430005in}{0.828332in}}%
\pgfpathlineto{\pgfqpoint{2.453152in}{0.858014in}}%
\pgfpathlineto{\pgfqpoint{2.476300in}{0.891949in}}%
\pgfpathlineto{\pgfqpoint{2.499447in}{0.930879in}}%
\pgfpathlineto{\pgfqpoint{2.522595in}{0.975519in}}%
\pgfpathlineto{\pgfqpoint{2.545743in}{1.026458in}}%
\pgfpathlineto{\pgfqpoint{2.568890in}{1.084063in}}%
\pgfpathlineto{\pgfqpoint{2.592038in}{1.148379in}}%
\pgfpathlineto{\pgfqpoint{2.615185in}{1.219066in}}%
\pgfpathlineto{\pgfqpoint{2.638333in}{1.295365in}}%
\pgfpathlineto{\pgfqpoint{2.684628in}{1.459818in}}%
\pgfpathlineto{\pgfqpoint{2.754071in}{1.711497in}}%
\pgfpathlineto{\pgfqpoint{2.777219in}{1.790220in}}%
\pgfpathlineto{\pgfqpoint{2.800366in}{1.864399in}}%
\pgfpathlineto{\pgfqpoint{2.823514in}{1.933524in}}%
\pgfpathlineto{\pgfqpoint{2.846661in}{1.997582in}}%
\pgfpathlineto{\pgfqpoint{2.869809in}{2.057032in}}%
\pgfpathlineto{\pgfqpoint{2.916104in}{2.165750in}}%
\pgfpathlineto{\pgfqpoint{2.985547in}{2.321657in}}%
\pgfpathlineto{\pgfqpoint{3.008694in}{2.376494in}}%
\pgfpathlineto{\pgfqpoint{3.031842in}{2.434280in}}%
\pgfpathlineto{\pgfqpoint{3.054990in}{2.495530in}}%
\pgfpathlineto{\pgfqpoint{3.078137in}{2.560462in}}%
\pgfpathlineto{\pgfqpoint{3.124432in}{2.700722in}}%
\pgfpathlineto{\pgfqpoint{3.170728in}{2.850940in}}%
\pgfpathlineto{\pgfqpoint{3.240170in}{3.077351in}}%
\pgfpathlineto{\pgfqpoint{3.263318in}{3.148475in}}%
\pgfpathlineto{\pgfqpoint{3.286466in}{3.215859in}}%
\pgfpathlineto{\pgfqpoint{3.309613in}{3.279074in}}%
\pgfpathlineto{\pgfqpoint{3.332761in}{3.338200in}}%
\pgfpathlineto{\pgfqpoint{3.379056in}{3.447331in}}%
\pgfpathlineto{\pgfqpoint{3.425351in}{3.554306in}}%
\pgfpathlineto{\pgfqpoint{3.448499in}{3.611570in}}%
\pgfpathlineto{\pgfqpoint{3.471646in}{3.673462in}}%
\pgfpathlineto{\pgfqpoint{3.494794in}{3.740791in}}%
\pgfpathlineto{\pgfqpoint{3.517941in}{3.813409in}}%
\pgfpathlineto{\pgfqpoint{3.587384in}{4.044375in}}%
\pgfpathlineto{\pgfqpoint{3.610532in}{4.114348in}}%
\pgfpathlineto{\pgfqpoint{3.633679in}{4.173565in}}%
\pgfpathlineto{\pgfqpoint{3.656827in}{4.217752in}}%
\pgfpathlineto{\pgfqpoint{3.679975in}{4.243475in}}%
\pgfpathlineto{\pgfqpoint{3.703122in}{4.248664in}}%
\pgfpathlineto{\pgfqpoint{3.726270in}{4.232994in}}%
\pgfpathlineto{\pgfqpoint{3.749417in}{4.198047in}}%
\pgfpathlineto{\pgfqpoint{3.772565in}{4.147232in}}%
\pgfpathlineto{\pgfqpoint{3.795712in}{4.085455in}}%
\pgfpathlineto{\pgfqpoint{3.842008in}{3.952771in}}%
\pgfpathlineto{\pgfqpoint{3.865155in}{3.893754in}}%
\pgfpathlineto{\pgfqpoint{3.888303in}{3.846204in}}%
\pgfpathlineto{\pgfqpoint{3.911450in}{3.813235in}}%
\pgfpathlineto{\pgfqpoint{3.934598in}{3.796114in}}%
\pgfpathlineto{\pgfqpoint{3.957746in}{3.794220in}}%
\pgfpathlineto{\pgfqpoint{3.980893in}{3.805226in}}%
\pgfpathlineto{\pgfqpoint{4.004041in}{3.825475in}}%
\pgfpathlineto{\pgfqpoint{4.050336in}{3.875452in}}%
\pgfpathlineto{\pgfqpoint{4.073484in}{3.895855in}}%
\pgfpathlineto{\pgfqpoint{4.096631in}{3.907792in}}%
\pgfpathlineto{\pgfqpoint{4.119779in}{3.908315in}}%
\pgfpathlineto{\pgfqpoint{4.142926in}{3.895562in}}%
\pgfpathlineto{\pgfqpoint{4.166074in}{3.868768in}}%
\pgfpathlineto{\pgfqpoint{4.189221in}{3.828160in}}%
\pgfpathlineto{\pgfqpoint{4.212369in}{3.774786in}}%
\pgfpathlineto{\pgfqpoint{4.235517in}{3.710321in}}%
\pgfpathlineto{\pgfqpoint{4.258664in}{3.636866in}}%
\pgfpathlineto{\pgfqpoint{4.281812in}{3.556770in}}%
\pgfpathlineto{\pgfqpoint{4.397550in}{3.138394in}}%
\pgfpathlineto{\pgfqpoint{4.420697in}{3.064515in}}%
\pgfpathlineto{\pgfqpoint{4.443845in}{2.996599in}}%
\pgfpathlineto{\pgfqpoint{4.466993in}{2.934830in}}%
\pgfpathlineto{\pgfqpoint{4.490140in}{2.878890in}}%
\pgfpathlineto{\pgfqpoint{4.513288in}{2.828011in}}%
\pgfpathlineto{\pgfqpoint{4.536435in}{2.781071in}}%
\pgfpathlineto{\pgfqpoint{4.652173in}{2.557953in}}%
\pgfpathlineto{\pgfqpoint{4.675321in}{2.507825in}}%
\pgfpathlineto{\pgfqpoint{4.698468in}{2.454579in}}%
\pgfpathlineto{\pgfqpoint{4.744764in}{2.339415in}}%
\pgfpathlineto{\pgfqpoint{4.791059in}{2.215824in}}%
\pgfpathlineto{\pgfqpoint{4.953092in}{1.773166in}}%
\pgfpathlineto{\pgfqpoint{5.091977in}{1.405333in}}%
\pgfpathlineto{\pgfqpoint{5.161420in}{1.225124in}}%
\pgfpathlineto{\pgfqpoint{5.207715in}{1.109678in}}%
\pgfpathlineto{\pgfqpoint{5.254011in}{1.001134in}}%
\pgfpathlineto{\pgfqpoint{5.277158in}{0.950388in}}%
\pgfpathlineto{\pgfqpoint{5.300306in}{0.902459in}}%
\pgfpathlineto{\pgfqpoint{5.323453in}{0.857627in}}%
\pgfpathlineto{\pgfqpoint{5.346601in}{0.816086in}}%
\pgfpathlineto{\pgfqpoint{5.369749in}{0.777917in}}%
\pgfpathlineto{\pgfqpoint{5.392896in}{0.743079in}}%
\pgfpathlineto{\pgfqpoint{5.416044in}{0.711410in}}%
\pgfpathlineto{\pgfqpoint{5.439191in}{0.682640in}}%
\pgfpathlineto{\pgfqpoint{5.462339in}{0.656416in}}%
\pgfpathlineto{\pgfqpoint{5.485486in}{0.632340in}}%
\pgfpathlineto{\pgfqpoint{5.531782in}{0.589045in}}%
\pgfpathlineto{\pgfqpoint{5.554929in}{0.569145in}}%
\pgfpathlineto{\pgfqpoint{5.554929in}{0.569145in}}%
\pgfusepath{stroke}%
\end{pgfscope}%
\begin{pgfscope}%
\pgfsetrectcap%
\pgfsetmiterjoin%
\pgfsetlinewidth{0.803000pt}%
\definecolor{currentstroke}{rgb}{0.000000,0.000000,0.000000}%
\pgfsetstrokecolor{currentstroke}%
\pgfsetdash{}{0pt}%
\pgfpathmoveto{\pgfqpoint{0.709028in}{0.387222in}}%
\pgfpathlineto{\pgfqpoint{0.709028in}{5.631667in}}%
\pgfusepath{stroke}%
\end{pgfscope}%
\begin{pgfscope}%
\pgfsetrectcap%
\pgfsetmiterjoin%
\pgfsetlinewidth{0.803000pt}%
\definecolor{currentstroke}{rgb}{0.000000,0.000000,0.000000}%
\pgfsetstrokecolor{currentstroke}%
\pgfsetdash{}{0pt}%
\pgfpathmoveto{\pgfqpoint{5.978715in}{0.387222in}}%
\pgfpathlineto{\pgfqpoint{5.978715in}{5.631667in}}%
\pgfusepath{stroke}%
\end{pgfscope}%
\begin{pgfscope}%
\pgfsetrectcap%
\pgfsetmiterjoin%
\pgfsetlinewidth{0.803000pt}%
\definecolor{currentstroke}{rgb}{0.000000,0.000000,0.000000}%
\pgfsetstrokecolor{currentstroke}%
\pgfsetdash{}{0pt}%
\pgfpathmoveto{\pgfqpoint{0.709028in}{0.387222in}}%
\pgfpathlineto{\pgfqpoint{5.978715in}{0.387222in}}%
\pgfusepath{stroke}%
\end{pgfscope}%
\begin{pgfscope}%
\pgfsetrectcap%
\pgfsetmiterjoin%
\pgfsetlinewidth{0.803000pt}%
\definecolor{currentstroke}{rgb}{0.000000,0.000000,0.000000}%
\pgfsetstrokecolor{currentstroke}%
\pgfsetdash{}{0pt}%
\pgfpathmoveto{\pgfqpoint{0.709028in}{5.631667in}}%
\pgfpathlineto{\pgfqpoint{5.978715in}{5.631667in}}%
\pgfusepath{stroke}%
\end{pgfscope}%
\begin{pgfscope}%
\definecolor{textcolor}{rgb}{0.000000,0.000000,0.000000}%
\pgfsetstrokecolor{textcolor}%
\pgfsetfillcolor{textcolor}%
\pgftext[x=3.343872in,y=5.715000in,,base]{\color{textcolor}{\sffamily\fontsize{12.000000}{14.400000}\selectfont\catcode`\^=\active\def^{\ifmmode\sp\else\^{}\fi}\catcode`\%=\active\def%{\%}Distribuição de score}}%
\end{pgfscope}%
\begin{pgfscope}%
\pgfsetbuttcap%
\pgfsetmiterjoin%
\definecolor{currentfill}{rgb}{1.000000,1.000000,1.000000}%
\pgfsetfillcolor{currentfill}%
\pgfsetlinewidth{0.000000pt}%
\definecolor{currentstroke}{rgb}{0.000000,0.000000,0.000000}%
\pgfsetstrokecolor{currentstroke}%
\pgfsetstrokeopacity{0.000000}%
\pgfsetdash{}{0pt}%
\pgfpathmoveto{\pgfqpoint{6.580312in}{0.387222in}}%
\pgfpathlineto{\pgfqpoint{11.850000in}{0.387222in}}%
\pgfpathlineto{\pgfqpoint{11.850000in}{5.631667in}}%
\pgfpathlineto{\pgfqpoint{6.580312in}{5.631667in}}%
\pgfpathlineto{\pgfqpoint{6.580312in}{0.387222in}}%
\pgfpathclose%
\pgfusepath{fill}%
\end{pgfscope}%
\begin{pgfscope}%
\pgfpathrectangle{\pgfqpoint{6.580312in}{0.387222in}}{\pgfqpoint{5.269687in}{5.244444in}}%
\pgfusepath{clip}%
\pgfsetbuttcap%
\pgfsetmiterjoin%
\definecolor{currentfill}{rgb}{0.194608,0.453431,0.632843}%
\pgfsetfillcolor{currentfill}%
\pgfsetlinewidth{0.000000pt}%
\definecolor{currentstroke}{rgb}{0.000000,0.000000,0.000000}%
\pgfsetstrokecolor{currentstroke}%
\pgfsetstrokeopacity{0.000000}%
\pgfsetdash{}{0pt}%
\pgfpathmoveto{\pgfqpoint{6.668141in}{0.387222in}}%
\pgfpathlineto{\pgfqpoint{6.808666in}{0.387222in}}%
\pgfpathlineto{\pgfqpoint{6.808666in}{0.571438in}}%
\pgfpathlineto{\pgfqpoint{6.668141in}{0.571438in}}%
\pgfpathlineto{\pgfqpoint{6.668141in}{0.387222in}}%
\pgfpathclose%
\pgfusepath{fill}%
\end{pgfscope}%
\begin{pgfscope}%
\pgfpathrectangle{\pgfqpoint{6.580312in}{0.387222in}}{\pgfqpoint{5.269687in}{5.244444in}}%
\pgfusepath{clip}%
\pgfsetbuttcap%
\pgfsetmiterjoin%
\definecolor{currentfill}{rgb}{0.194608,0.453431,0.632843}%
\pgfsetfillcolor{currentfill}%
\pgfsetlinewidth{0.000000pt}%
\definecolor{currentstroke}{rgb}{0.000000,0.000000,0.000000}%
\pgfsetstrokecolor{currentstroke}%
\pgfsetstrokeopacity{0.000000}%
\pgfsetdash{}{0pt}%
\pgfpathmoveto{\pgfqpoint{7.546422in}{0.387222in}}%
\pgfpathlineto{\pgfqpoint{7.686947in}{0.387222in}}%
\pgfpathlineto{\pgfqpoint{7.686947in}{0.432801in}}%
\pgfpathlineto{\pgfqpoint{7.546422in}{0.432801in}}%
\pgfpathlineto{\pgfqpoint{7.546422in}{0.387222in}}%
\pgfpathclose%
\pgfusepath{fill}%
\end{pgfscope}%
\begin{pgfscope}%
\pgfpathrectangle{\pgfqpoint{6.580312in}{0.387222in}}{\pgfqpoint{5.269687in}{5.244444in}}%
\pgfusepath{clip}%
\pgfsetbuttcap%
\pgfsetmiterjoin%
\definecolor{currentfill}{rgb}{0.194608,0.453431,0.632843}%
\pgfsetfillcolor{currentfill}%
\pgfsetlinewidth{0.000000pt}%
\definecolor{currentstroke}{rgb}{0.000000,0.000000,0.000000}%
\pgfsetstrokecolor{currentstroke}%
\pgfsetstrokeopacity{0.000000}%
\pgfsetdash{}{0pt}%
\pgfpathmoveto{\pgfqpoint{8.424703in}{0.387222in}}%
\pgfpathlineto{\pgfqpoint{8.565228in}{0.387222in}}%
\pgfpathlineto{\pgfqpoint{8.565228in}{1.069009in}}%
\pgfpathlineto{\pgfqpoint{8.424703in}{1.069009in}}%
\pgfpathlineto{\pgfqpoint{8.424703in}{0.387222in}}%
\pgfpathclose%
\pgfusepath{fill}%
\end{pgfscope}%
\begin{pgfscope}%
\pgfpathrectangle{\pgfqpoint{6.580312in}{0.387222in}}{\pgfqpoint{5.269687in}{5.244444in}}%
\pgfusepath{clip}%
\pgfsetbuttcap%
\pgfsetmiterjoin%
\definecolor{currentfill}{rgb}{0.194608,0.453431,0.632843}%
\pgfsetfillcolor{currentfill}%
\pgfsetlinewidth{0.000000pt}%
\definecolor{currentstroke}{rgb}{0.000000,0.000000,0.000000}%
\pgfsetstrokecolor{currentstroke}%
\pgfsetstrokeopacity{0.000000}%
\pgfsetdash{}{0pt}%
\pgfpathmoveto{\pgfqpoint{9.302984in}{0.387222in}}%
\pgfpathlineto{\pgfqpoint{9.443509in}{0.387222in}}%
\pgfpathlineto{\pgfqpoint{9.443509in}{5.381931in}}%
\pgfpathlineto{\pgfqpoint{9.302984in}{5.381931in}}%
\pgfpathlineto{\pgfqpoint{9.302984in}{0.387222in}}%
\pgfpathclose%
\pgfusepath{fill}%
\end{pgfscope}%
\begin{pgfscope}%
\pgfpathrectangle{\pgfqpoint{6.580312in}{0.387222in}}{\pgfqpoint{5.269687in}{5.244444in}}%
\pgfusepath{clip}%
\pgfsetbuttcap%
\pgfsetmiterjoin%
\definecolor{currentfill}{rgb}{0.194608,0.453431,0.632843}%
\pgfsetfillcolor{currentfill}%
\pgfsetlinewidth{0.000000pt}%
\definecolor{currentstroke}{rgb}{0.000000,0.000000,0.000000}%
\pgfsetstrokecolor{currentstroke}%
\pgfsetstrokeopacity{0.000000}%
\pgfsetdash{}{0pt}%
\pgfpathmoveto{\pgfqpoint{10.181266in}{0.387222in}}%
\pgfpathlineto{\pgfqpoint{10.321791in}{0.387222in}}%
\pgfpathlineto{\pgfqpoint{10.321791in}{2.928257in}}%
\pgfpathlineto{\pgfqpoint{10.181266in}{2.928257in}}%
\pgfpathlineto{\pgfqpoint{10.181266in}{0.387222in}}%
\pgfpathclose%
\pgfusepath{fill}%
\end{pgfscope}%
\begin{pgfscope}%
\pgfpathrectangle{\pgfqpoint{6.580312in}{0.387222in}}{\pgfqpoint{5.269687in}{5.244444in}}%
\pgfusepath{clip}%
\pgfsetbuttcap%
\pgfsetmiterjoin%
\definecolor{currentfill}{rgb}{0.194608,0.453431,0.632843}%
\pgfsetfillcolor{currentfill}%
\pgfsetlinewidth{0.000000pt}%
\definecolor{currentstroke}{rgb}{0.000000,0.000000,0.000000}%
\pgfsetstrokecolor{currentstroke}%
\pgfsetstrokeopacity{0.000000}%
\pgfsetdash{}{0pt}%
\pgfpathmoveto{\pgfqpoint{11.059547in}{0.387222in}}%
\pgfpathlineto{\pgfqpoint{11.200072in}{0.387222in}}%
\pgfpathlineto{\pgfqpoint{11.200072in}{0.618916in}}%
\pgfpathlineto{\pgfqpoint{11.059547in}{0.618916in}}%
\pgfpathlineto{\pgfqpoint{11.059547in}{0.387222in}}%
\pgfpathclose%
\pgfusepath{fill}%
\end{pgfscope}%
\begin{pgfscope}%
\pgfpathrectangle{\pgfqpoint{6.580312in}{0.387222in}}{\pgfqpoint{5.269687in}{5.244444in}}%
\pgfusepath{clip}%
\pgfsetbuttcap%
\pgfsetmiterjoin%
\definecolor{currentfill}{rgb}{0.881863,0.505392,0.173039}%
\pgfsetfillcolor{currentfill}%
\pgfsetlinewidth{0.000000pt}%
\definecolor{currentstroke}{rgb}{0.000000,0.000000,0.000000}%
\pgfsetstrokecolor{currentstroke}%
\pgfsetstrokeopacity{0.000000}%
\pgfsetdash{}{0pt}%
\pgfpathmoveto{\pgfqpoint{6.808666in}{0.387222in}}%
\pgfpathlineto{\pgfqpoint{6.949191in}{0.387222in}}%
\pgfpathlineto{\pgfqpoint{6.949191in}{0.594227in}}%
\pgfpathlineto{\pgfqpoint{6.808666in}{0.594227in}}%
\pgfpathlineto{\pgfqpoint{6.808666in}{0.387222in}}%
\pgfpathclose%
\pgfusepath{fill}%
\end{pgfscope}%
\begin{pgfscope}%
\pgfpathrectangle{\pgfqpoint{6.580312in}{0.387222in}}{\pgfqpoint{5.269687in}{5.244444in}}%
\pgfusepath{clip}%
\pgfsetbuttcap%
\pgfsetmiterjoin%
\definecolor{currentfill}{rgb}{0.881863,0.505392,0.173039}%
\pgfsetfillcolor{currentfill}%
\pgfsetlinewidth{0.000000pt}%
\definecolor{currentstroke}{rgb}{0.000000,0.000000,0.000000}%
\pgfsetstrokecolor{currentstroke}%
\pgfsetstrokeopacity{0.000000}%
\pgfsetdash{}{0pt}%
\pgfpathmoveto{\pgfqpoint{7.686947in}{0.387222in}}%
\pgfpathlineto{\pgfqpoint{7.827472in}{0.387222in}}%
\pgfpathlineto{\pgfqpoint{7.827472in}{0.537253in}}%
\pgfpathlineto{\pgfqpoint{7.686947in}{0.537253in}}%
\pgfpathlineto{\pgfqpoint{7.686947in}{0.387222in}}%
\pgfpathclose%
\pgfusepath{fill}%
\end{pgfscope}%
\begin{pgfscope}%
\pgfpathrectangle{\pgfqpoint{6.580312in}{0.387222in}}{\pgfqpoint{5.269687in}{5.244444in}}%
\pgfusepath{clip}%
\pgfsetbuttcap%
\pgfsetmiterjoin%
\definecolor{currentfill}{rgb}{0.881863,0.505392,0.173039}%
\pgfsetfillcolor{currentfill}%
\pgfsetlinewidth{0.000000pt}%
\definecolor{currentstroke}{rgb}{0.000000,0.000000,0.000000}%
\pgfsetstrokecolor{currentstroke}%
\pgfsetstrokeopacity{0.000000}%
\pgfsetdash{}{0pt}%
\pgfpathmoveto{\pgfqpoint{8.565228in}{0.387222in}}%
\pgfpathlineto{\pgfqpoint{8.705753in}{0.387222in}}%
\pgfpathlineto{\pgfqpoint{8.705753in}{1.695722in}}%
\pgfpathlineto{\pgfqpoint{8.565228in}{1.695722in}}%
\pgfpathlineto{\pgfqpoint{8.565228in}{0.387222in}}%
\pgfpathclose%
\pgfusepath{fill}%
\end{pgfscope}%
\begin{pgfscope}%
\pgfpathrectangle{\pgfqpoint{6.580312in}{0.387222in}}{\pgfqpoint{5.269687in}{5.244444in}}%
\pgfusepath{clip}%
\pgfsetbuttcap%
\pgfsetmiterjoin%
\definecolor{currentfill}{rgb}{0.881863,0.505392,0.173039}%
\pgfsetfillcolor{currentfill}%
\pgfsetlinewidth{0.000000pt}%
\definecolor{currentstroke}{rgb}{0.000000,0.000000,0.000000}%
\pgfsetstrokecolor{currentstroke}%
\pgfsetstrokeopacity{0.000000}%
\pgfsetdash{}{0pt}%
\pgfpathmoveto{\pgfqpoint{9.443509in}{0.387222in}}%
\pgfpathlineto{\pgfqpoint{9.584034in}{0.387222in}}%
\pgfpathlineto{\pgfqpoint{9.584034in}{3.636632in}}%
\pgfpathlineto{\pgfqpoint{9.443509in}{3.636632in}}%
\pgfpathlineto{\pgfqpoint{9.443509in}{0.387222in}}%
\pgfpathclose%
\pgfusepath{fill}%
\end{pgfscope}%
\begin{pgfscope}%
\pgfpathrectangle{\pgfqpoint{6.580312in}{0.387222in}}{\pgfqpoint{5.269687in}{5.244444in}}%
\pgfusepath{clip}%
\pgfsetbuttcap%
\pgfsetmiterjoin%
\definecolor{currentfill}{rgb}{0.881863,0.505392,0.173039}%
\pgfsetfillcolor{currentfill}%
\pgfsetlinewidth{0.000000pt}%
\definecolor{currentstroke}{rgb}{0.000000,0.000000,0.000000}%
\pgfsetstrokecolor{currentstroke}%
\pgfsetstrokeopacity{0.000000}%
\pgfsetdash{}{0pt}%
\pgfpathmoveto{\pgfqpoint{10.321791in}{0.387222in}}%
\pgfpathlineto{\pgfqpoint{10.462316in}{0.387222in}}%
\pgfpathlineto{\pgfqpoint{10.462316in}{3.625237in}}%
\pgfpathlineto{\pgfqpoint{10.321791in}{3.625237in}}%
\pgfpathlineto{\pgfqpoint{10.321791in}{0.387222in}}%
\pgfpathclose%
\pgfusepath{fill}%
\end{pgfscope}%
\begin{pgfscope}%
\pgfpathrectangle{\pgfqpoint{6.580312in}{0.387222in}}{\pgfqpoint{5.269687in}{5.244444in}}%
\pgfusepath{clip}%
\pgfsetbuttcap%
\pgfsetmiterjoin%
\definecolor{currentfill}{rgb}{0.881863,0.505392,0.173039}%
\pgfsetfillcolor{currentfill}%
\pgfsetlinewidth{0.000000pt}%
\definecolor{currentstroke}{rgb}{0.000000,0.000000,0.000000}%
\pgfsetstrokecolor{currentstroke}%
\pgfsetstrokeopacity{0.000000}%
\pgfsetdash{}{0pt}%
\pgfpathmoveto{\pgfqpoint{11.200072in}{0.387222in}}%
\pgfpathlineto{\pgfqpoint{11.340597in}{0.387222in}}%
\pgfpathlineto{\pgfqpoint{11.340597in}{0.913281in}}%
\pgfpathlineto{\pgfqpoint{11.200072in}{0.913281in}}%
\pgfpathlineto{\pgfqpoint{11.200072in}{0.387222in}}%
\pgfpathclose%
\pgfusepath{fill}%
\end{pgfscope}%
\begin{pgfscope}%
\pgfpathrectangle{\pgfqpoint{6.580312in}{0.387222in}}{\pgfqpoint{5.269687in}{5.244444in}}%
\pgfusepath{clip}%
\pgfsetbuttcap%
\pgfsetmiterjoin%
\definecolor{currentfill}{rgb}{0.229412,0.570588,0.229412}%
\pgfsetfillcolor{currentfill}%
\pgfsetlinewidth{0.000000pt}%
\definecolor{currentstroke}{rgb}{0.000000,0.000000,0.000000}%
\pgfsetstrokecolor{currentstroke}%
\pgfsetstrokeopacity{0.000000}%
\pgfsetdash{}{0pt}%
\pgfpathmoveto{\pgfqpoint{6.949191in}{0.387222in}}%
\pgfpathlineto{\pgfqpoint{7.089716in}{0.387222in}}%
\pgfpathlineto{\pgfqpoint{7.089716in}{0.618916in}}%
\pgfpathlineto{\pgfqpoint{6.949191in}{0.618916in}}%
\pgfpathlineto{\pgfqpoint{6.949191in}{0.387222in}}%
\pgfpathclose%
\pgfusepath{fill}%
\end{pgfscope}%
\begin{pgfscope}%
\pgfpathrectangle{\pgfqpoint{6.580312in}{0.387222in}}{\pgfqpoint{5.269687in}{5.244444in}}%
\pgfusepath{clip}%
\pgfsetbuttcap%
\pgfsetmiterjoin%
\definecolor{currentfill}{rgb}{0.229412,0.570588,0.229412}%
\pgfsetfillcolor{currentfill}%
\pgfsetlinewidth{0.000000pt}%
\definecolor{currentstroke}{rgb}{0.000000,0.000000,0.000000}%
\pgfsetstrokecolor{currentstroke}%
\pgfsetstrokeopacity{0.000000}%
\pgfsetdash{}{0pt}%
\pgfpathmoveto{\pgfqpoint{7.827472in}{0.387222in}}%
\pgfpathlineto{\pgfqpoint{7.967997in}{0.387222in}}%
\pgfpathlineto{\pgfqpoint{7.967997in}{0.664495in}}%
\pgfpathlineto{\pgfqpoint{7.827472in}{0.664495in}}%
\pgfpathlineto{\pgfqpoint{7.827472in}{0.387222in}}%
\pgfpathclose%
\pgfusepath{fill}%
\end{pgfscope}%
\begin{pgfscope}%
\pgfpathrectangle{\pgfqpoint{6.580312in}{0.387222in}}{\pgfqpoint{5.269687in}{5.244444in}}%
\pgfusepath{clip}%
\pgfsetbuttcap%
\pgfsetmiterjoin%
\definecolor{currentfill}{rgb}{0.229412,0.570588,0.229412}%
\pgfsetfillcolor{currentfill}%
\pgfsetlinewidth{0.000000pt}%
\definecolor{currentstroke}{rgb}{0.000000,0.000000,0.000000}%
\pgfsetstrokecolor{currentstroke}%
\pgfsetstrokeopacity{0.000000}%
\pgfsetdash{}{0pt}%
\pgfpathmoveto{\pgfqpoint{8.705753in}{0.387222in}}%
\pgfpathlineto{\pgfqpoint{8.846278in}{0.387222in}}%
\pgfpathlineto{\pgfqpoint{8.846278in}{2.677572in}}%
\pgfpathlineto{\pgfqpoint{8.705753in}{2.677572in}}%
\pgfpathlineto{\pgfqpoint{8.705753in}{0.387222in}}%
\pgfpathclose%
\pgfusepath{fill}%
\end{pgfscope}%
\begin{pgfscope}%
\pgfpathrectangle{\pgfqpoint{6.580312in}{0.387222in}}{\pgfqpoint{5.269687in}{5.244444in}}%
\pgfusepath{clip}%
\pgfsetbuttcap%
\pgfsetmiterjoin%
\definecolor{currentfill}{rgb}{0.229412,0.570588,0.229412}%
\pgfsetfillcolor{currentfill}%
\pgfsetlinewidth{0.000000pt}%
\definecolor{currentstroke}{rgb}{0.000000,0.000000,0.000000}%
\pgfsetstrokecolor{currentstroke}%
\pgfsetstrokeopacity{0.000000}%
\pgfsetdash{}{0pt}%
\pgfpathmoveto{\pgfqpoint{9.584034in}{0.387222in}}%
\pgfpathlineto{\pgfqpoint{9.724559in}{0.387222in}}%
\pgfpathlineto{\pgfqpoint{9.724559in}{4.432367in}}%
\pgfpathlineto{\pgfqpoint{9.584034in}{4.432367in}}%
\pgfpathlineto{\pgfqpoint{9.584034in}{0.387222in}}%
\pgfpathclose%
\pgfusepath{fill}%
\end{pgfscope}%
\begin{pgfscope}%
\pgfpathrectangle{\pgfqpoint{6.580312in}{0.387222in}}{\pgfqpoint{5.269687in}{5.244444in}}%
\pgfusepath{clip}%
\pgfsetbuttcap%
\pgfsetmiterjoin%
\definecolor{currentfill}{rgb}{0.229412,0.570588,0.229412}%
\pgfsetfillcolor{currentfill}%
\pgfsetlinewidth{0.000000pt}%
\definecolor{currentstroke}{rgb}{0.000000,0.000000,0.000000}%
\pgfsetstrokecolor{currentstroke}%
\pgfsetstrokeopacity{0.000000}%
\pgfsetdash{}{0pt}%
\pgfpathmoveto{\pgfqpoint{10.462316in}{0.387222in}}%
\pgfpathlineto{\pgfqpoint{10.602841in}{0.387222in}}%
\pgfpathlineto{\pgfqpoint{10.602841in}{1.957802in}}%
\pgfpathlineto{\pgfqpoint{10.462316in}{1.957802in}}%
\pgfpathlineto{\pgfqpoint{10.462316in}{0.387222in}}%
\pgfpathclose%
\pgfusepath{fill}%
\end{pgfscope}%
\begin{pgfscope}%
\pgfpathrectangle{\pgfqpoint{6.580312in}{0.387222in}}{\pgfqpoint{5.269687in}{5.244444in}}%
\pgfusepath{clip}%
\pgfsetbuttcap%
\pgfsetmiterjoin%
\definecolor{currentfill}{rgb}{0.229412,0.570588,0.229412}%
\pgfsetfillcolor{currentfill}%
\pgfsetlinewidth{0.000000pt}%
\definecolor{currentstroke}{rgb}{0.000000,0.000000,0.000000}%
\pgfsetstrokecolor{currentstroke}%
\pgfsetstrokeopacity{0.000000}%
\pgfsetdash{}{0pt}%
\pgfpathmoveto{\pgfqpoint{11.340597in}{0.387222in}}%
\pgfpathlineto{\pgfqpoint{11.481122in}{0.387222in}}%
\pgfpathlineto{\pgfqpoint{11.481122in}{0.651201in}}%
\pgfpathlineto{\pgfqpoint{11.340597in}{0.651201in}}%
\pgfpathlineto{\pgfqpoint{11.340597in}{0.387222in}}%
\pgfpathclose%
\pgfusepath{fill}%
\end{pgfscope}%
\begin{pgfscope}%
\pgfpathrectangle{\pgfqpoint{6.580312in}{0.387222in}}{\pgfqpoint{5.269687in}{5.244444in}}%
\pgfusepath{clip}%
\pgfsetbuttcap%
\pgfsetmiterjoin%
\definecolor{currentfill}{rgb}{0.753431,0.238725,0.241667}%
\pgfsetfillcolor{currentfill}%
\pgfsetlinewidth{0.000000pt}%
\definecolor{currentstroke}{rgb}{0.000000,0.000000,0.000000}%
\pgfsetstrokecolor{currentstroke}%
\pgfsetstrokeopacity{0.000000}%
\pgfsetdash{}{0pt}%
\pgfpathmoveto{\pgfqpoint{7.089716in}{0.387222in}}%
\pgfpathlineto{\pgfqpoint{7.230241in}{0.387222in}}%
\pgfpathlineto{\pgfqpoint{7.230241in}{0.641705in}}%
\pgfpathlineto{\pgfqpoint{7.089716in}{0.641705in}}%
\pgfpathlineto{\pgfqpoint{7.089716in}{0.387222in}}%
\pgfpathclose%
\pgfusepath{fill}%
\end{pgfscope}%
\begin{pgfscope}%
\pgfpathrectangle{\pgfqpoint{6.580312in}{0.387222in}}{\pgfqpoint{5.269687in}{5.244444in}}%
\pgfusepath{clip}%
\pgfsetbuttcap%
\pgfsetmiterjoin%
\definecolor{currentfill}{rgb}{0.753431,0.238725,0.241667}%
\pgfsetfillcolor{currentfill}%
\pgfsetlinewidth{0.000000pt}%
\definecolor{currentstroke}{rgb}{0.000000,0.000000,0.000000}%
\pgfsetstrokecolor{currentstroke}%
\pgfsetstrokeopacity{0.000000}%
\pgfsetdash{}{0pt}%
\pgfpathmoveto{\pgfqpoint{7.967997in}{0.387222in}}%
\pgfpathlineto{\pgfqpoint{8.108522in}{0.387222in}}%
\pgfpathlineto{\pgfqpoint{8.108522in}{0.503069in}}%
\pgfpathlineto{\pgfqpoint{7.967997in}{0.503069in}}%
\pgfpathlineto{\pgfqpoint{7.967997in}{0.387222in}}%
\pgfpathclose%
\pgfusepath{fill}%
\end{pgfscope}%
\begin{pgfscope}%
\pgfpathrectangle{\pgfqpoint{6.580312in}{0.387222in}}{\pgfqpoint{5.269687in}{5.244444in}}%
\pgfusepath{clip}%
\pgfsetbuttcap%
\pgfsetmiterjoin%
\definecolor{currentfill}{rgb}{0.753431,0.238725,0.241667}%
\pgfsetfillcolor{currentfill}%
\pgfsetlinewidth{0.000000pt}%
\definecolor{currentstroke}{rgb}{0.000000,0.000000,0.000000}%
\pgfsetstrokecolor{currentstroke}%
\pgfsetstrokeopacity{0.000000}%
\pgfsetdash{}{0pt}%
\pgfpathmoveto{\pgfqpoint{8.846278in}{0.387222in}}%
\pgfpathlineto{\pgfqpoint{8.986803in}{0.387222in}}%
\pgfpathlineto{\pgfqpoint{8.986803in}{1.507708in}}%
\pgfpathlineto{\pgfqpoint{8.846278in}{1.507708in}}%
\pgfpathlineto{\pgfqpoint{8.846278in}{0.387222in}}%
\pgfpathclose%
\pgfusepath{fill}%
\end{pgfscope}%
\begin{pgfscope}%
\pgfpathrectangle{\pgfqpoint{6.580312in}{0.387222in}}{\pgfqpoint{5.269687in}{5.244444in}}%
\pgfusepath{clip}%
\pgfsetbuttcap%
\pgfsetmiterjoin%
\definecolor{currentfill}{rgb}{0.753431,0.238725,0.241667}%
\pgfsetfillcolor{currentfill}%
\pgfsetlinewidth{0.000000pt}%
\definecolor{currentstroke}{rgb}{0.000000,0.000000,0.000000}%
\pgfsetstrokecolor{currentstroke}%
\pgfsetstrokeopacity{0.000000}%
\pgfsetdash{}{0pt}%
\pgfpathmoveto{\pgfqpoint{9.724559in}{0.387222in}}%
\pgfpathlineto{\pgfqpoint{9.865084in}{0.387222in}}%
\pgfpathlineto{\pgfqpoint{9.865084in}{4.185480in}}%
\pgfpathlineto{\pgfqpoint{9.724559in}{4.185480in}}%
\pgfpathlineto{\pgfqpoint{9.724559in}{0.387222in}}%
\pgfpathclose%
\pgfusepath{fill}%
\end{pgfscope}%
\begin{pgfscope}%
\pgfpathrectangle{\pgfqpoint{6.580312in}{0.387222in}}{\pgfqpoint{5.269687in}{5.244444in}}%
\pgfusepath{clip}%
\pgfsetbuttcap%
\pgfsetmiterjoin%
\definecolor{currentfill}{rgb}{0.753431,0.238725,0.241667}%
\pgfsetfillcolor{currentfill}%
\pgfsetlinewidth{0.000000pt}%
\definecolor{currentstroke}{rgb}{0.000000,0.000000,0.000000}%
\pgfsetstrokecolor{currentstroke}%
\pgfsetstrokeopacity{0.000000}%
\pgfsetdash{}{0pt}%
\pgfpathmoveto{\pgfqpoint{10.602841in}{0.387222in}}%
\pgfpathlineto{\pgfqpoint{10.743366in}{0.387222in}}%
\pgfpathlineto{\pgfqpoint{10.743366in}{2.744041in}}%
\pgfpathlineto{\pgfqpoint{10.602841in}{2.744041in}}%
\pgfpathlineto{\pgfqpoint{10.602841in}{0.387222in}}%
\pgfpathclose%
\pgfusepath{fill}%
\end{pgfscope}%
\begin{pgfscope}%
\pgfpathrectangle{\pgfqpoint{6.580312in}{0.387222in}}{\pgfqpoint{5.269687in}{5.244444in}}%
\pgfusepath{clip}%
\pgfsetbuttcap%
\pgfsetmiterjoin%
\definecolor{currentfill}{rgb}{0.753431,0.238725,0.241667}%
\pgfsetfillcolor{currentfill}%
\pgfsetlinewidth{0.000000pt}%
\definecolor{currentstroke}{rgb}{0.000000,0.000000,0.000000}%
\pgfsetstrokecolor{currentstroke}%
\pgfsetstrokeopacity{0.000000}%
\pgfsetdash{}{0pt}%
\pgfpathmoveto{\pgfqpoint{11.481122in}{0.387222in}}%
\pgfpathlineto{\pgfqpoint{11.621647in}{0.387222in}}%
\pgfpathlineto{\pgfqpoint{11.621647in}{1.420348in}}%
\pgfpathlineto{\pgfqpoint{11.481122in}{1.420348in}}%
\pgfpathlineto{\pgfqpoint{11.481122in}{0.387222in}}%
\pgfpathclose%
\pgfusepath{fill}%
\end{pgfscope}%
\begin{pgfscope}%
\pgfpathrectangle{\pgfqpoint{6.580312in}{0.387222in}}{\pgfqpoint{5.269687in}{5.244444in}}%
\pgfusepath{clip}%
\pgfsetbuttcap%
\pgfsetmiterjoin%
\definecolor{currentfill}{rgb}{0.578431,0.446078,0.699020}%
\pgfsetfillcolor{currentfill}%
\pgfsetlinewidth{0.000000pt}%
\definecolor{currentstroke}{rgb}{0.000000,0.000000,0.000000}%
\pgfsetstrokecolor{currentstroke}%
\pgfsetstrokeopacity{0.000000}%
\pgfsetdash{}{0pt}%
\pgfpathmoveto{\pgfqpoint{7.230241in}{0.387222in}}%
\pgfpathlineto{\pgfqpoint{7.370766in}{0.387222in}}%
\pgfpathlineto{\pgfqpoint{7.370766in}{0.960759in}}%
\pgfpathlineto{\pgfqpoint{7.230241in}{0.960759in}}%
\pgfpathlineto{\pgfqpoint{7.230241in}{0.387222in}}%
\pgfpathclose%
\pgfusepath{fill}%
\end{pgfscope}%
\begin{pgfscope}%
\pgfpathrectangle{\pgfqpoint{6.580312in}{0.387222in}}{\pgfqpoint{5.269687in}{5.244444in}}%
\pgfusepath{clip}%
\pgfsetbuttcap%
\pgfsetmiterjoin%
\definecolor{currentfill}{rgb}{0.578431,0.446078,0.699020}%
\pgfsetfillcolor{currentfill}%
\pgfsetlinewidth{0.000000pt}%
\definecolor{currentstroke}{rgb}{0.000000,0.000000,0.000000}%
\pgfsetstrokecolor{currentstroke}%
\pgfsetstrokeopacity{0.000000}%
\pgfsetdash{}{0pt}%
\pgfpathmoveto{\pgfqpoint{8.108522in}{0.387222in}}%
\pgfpathlineto{\pgfqpoint{8.249047in}{0.387222in}}%
\pgfpathlineto{\pgfqpoint{8.249047in}{0.911382in}}%
\pgfpathlineto{\pgfqpoint{8.108522in}{0.911382in}}%
\pgfpathlineto{\pgfqpoint{8.108522in}{0.387222in}}%
\pgfpathclose%
\pgfusepath{fill}%
\end{pgfscope}%
\begin{pgfscope}%
\pgfpathrectangle{\pgfqpoint{6.580312in}{0.387222in}}{\pgfqpoint{5.269687in}{5.244444in}}%
\pgfusepath{clip}%
\pgfsetbuttcap%
\pgfsetmiterjoin%
\definecolor{currentfill}{rgb}{0.578431,0.446078,0.699020}%
\pgfsetfillcolor{currentfill}%
\pgfsetlinewidth{0.000000pt}%
\definecolor{currentstroke}{rgb}{0.000000,0.000000,0.000000}%
\pgfsetstrokecolor{currentstroke}%
\pgfsetstrokeopacity{0.000000}%
\pgfsetdash{}{0pt}%
\pgfpathmoveto{\pgfqpoint{8.986803in}{0.387222in}}%
\pgfpathlineto{\pgfqpoint{9.127328in}{0.387222in}}%
\pgfpathlineto{\pgfqpoint{9.127328in}{2.330031in}}%
\pgfpathlineto{\pgfqpoint{8.986803in}{2.330031in}}%
\pgfpathlineto{\pgfqpoint{8.986803in}{0.387222in}}%
\pgfpathclose%
\pgfusepath{fill}%
\end{pgfscope}%
\begin{pgfscope}%
\pgfpathrectangle{\pgfqpoint{6.580312in}{0.387222in}}{\pgfqpoint{5.269687in}{5.244444in}}%
\pgfusepath{clip}%
\pgfsetbuttcap%
\pgfsetmiterjoin%
\definecolor{currentfill}{rgb}{0.578431,0.446078,0.699020}%
\pgfsetfillcolor{currentfill}%
\pgfsetlinewidth{0.000000pt}%
\definecolor{currentstroke}{rgb}{0.000000,0.000000,0.000000}%
\pgfsetstrokecolor{currentstroke}%
\pgfsetstrokeopacity{0.000000}%
\pgfsetdash{}{0pt}%
\pgfpathmoveto{\pgfqpoint{9.865084in}{0.387222in}}%
\pgfpathlineto{\pgfqpoint{10.005609in}{0.387222in}}%
\pgfpathlineto{\pgfqpoint{10.005609in}{3.676513in}}%
\pgfpathlineto{\pgfqpoint{9.865084in}{3.676513in}}%
\pgfpathlineto{\pgfqpoint{9.865084in}{0.387222in}}%
\pgfpathclose%
\pgfusepath{fill}%
\end{pgfscope}%
\begin{pgfscope}%
\pgfpathrectangle{\pgfqpoint{6.580312in}{0.387222in}}{\pgfqpoint{5.269687in}{5.244444in}}%
\pgfusepath{clip}%
\pgfsetbuttcap%
\pgfsetmiterjoin%
\definecolor{currentfill}{rgb}{0.578431,0.446078,0.699020}%
\pgfsetfillcolor{currentfill}%
\pgfsetlinewidth{0.000000pt}%
\definecolor{currentstroke}{rgb}{0.000000,0.000000,0.000000}%
\pgfsetstrokecolor{currentstroke}%
\pgfsetstrokeopacity{0.000000}%
\pgfsetdash{}{0pt}%
\pgfpathmoveto{\pgfqpoint{10.743366in}{0.387222in}}%
\pgfpathlineto{\pgfqpoint{10.883891in}{0.387222in}}%
\pgfpathlineto{\pgfqpoint{10.883891in}{1.935012in}}%
\pgfpathlineto{\pgfqpoint{10.743366in}{1.935012in}}%
\pgfpathlineto{\pgfqpoint{10.743366in}{0.387222in}}%
\pgfpathclose%
\pgfusepath{fill}%
\end{pgfscope}%
\begin{pgfscope}%
\pgfpathrectangle{\pgfqpoint{6.580312in}{0.387222in}}{\pgfqpoint{5.269687in}{5.244444in}}%
\pgfusepath{clip}%
\pgfsetbuttcap%
\pgfsetmiterjoin%
\definecolor{currentfill}{rgb}{0.578431,0.446078,0.699020}%
\pgfsetfillcolor{currentfill}%
\pgfsetlinewidth{0.000000pt}%
\definecolor{currentstroke}{rgb}{0.000000,0.000000,0.000000}%
\pgfsetstrokecolor{currentstroke}%
\pgfsetstrokeopacity{0.000000}%
\pgfsetdash{}{0pt}%
\pgfpathmoveto{\pgfqpoint{11.621647in}{0.387222in}}%
\pgfpathlineto{\pgfqpoint{11.762172in}{0.387222in}}%
\pgfpathlineto{\pgfqpoint{11.762172in}{1.188655in}}%
\pgfpathlineto{\pgfqpoint{11.621647in}{1.188655in}}%
\pgfpathlineto{\pgfqpoint{11.621647in}{0.387222in}}%
\pgfpathclose%
\pgfusepath{fill}%
\end{pgfscope}%
\begin{pgfscope}%
\pgfpathrectangle{\pgfqpoint{6.580312in}{0.387222in}}{\pgfqpoint{5.269687in}{5.244444in}}%
\pgfusepath{clip}%
\pgfsetbuttcap%
\pgfsetmiterjoin%
\definecolor{currentfill}{rgb}{0.194608,0.453431,0.632843}%
\pgfsetfillcolor{currentfill}%
\pgfsetlinewidth{0.000000pt}%
\definecolor{currentstroke}{rgb}{0.000000,0.000000,0.000000}%
\pgfsetstrokecolor{currentstroke}%
\pgfsetstrokeopacity{0.000000}%
\pgfsetdash{}{0pt}%
\pgfpathmoveto{\pgfqpoint{7.019453in}{0.387222in}}%
\pgfpathlineto{\pgfqpoint{7.019453in}{0.387222in}}%
\pgfpathlineto{\pgfqpoint{7.019453in}{0.387222in}}%
\pgfpathlineto{\pgfqpoint{7.019453in}{0.387222in}}%
\pgfpathlineto{\pgfqpoint{7.019453in}{0.387222in}}%
\pgfpathclose%
\pgfusepath{fill}%
\end{pgfscope}%
\begin{pgfscope}%
\pgfpathrectangle{\pgfqpoint{6.580312in}{0.387222in}}{\pgfqpoint{5.269687in}{5.244444in}}%
\pgfusepath{clip}%
\pgfsetbuttcap%
\pgfsetmiterjoin%
\definecolor{currentfill}{rgb}{0.881863,0.505392,0.173039}%
\pgfsetfillcolor{currentfill}%
\pgfsetlinewidth{0.000000pt}%
\definecolor{currentstroke}{rgb}{0.000000,0.000000,0.000000}%
\pgfsetstrokecolor{currentstroke}%
\pgfsetstrokeopacity{0.000000}%
\pgfsetdash{}{0pt}%
\pgfpathmoveto{\pgfqpoint{7.019453in}{0.387222in}}%
\pgfpathlineto{\pgfqpoint{7.019453in}{0.387222in}}%
\pgfpathlineto{\pgfqpoint{7.019453in}{0.387222in}}%
\pgfpathlineto{\pgfqpoint{7.019453in}{0.387222in}}%
\pgfpathlineto{\pgfqpoint{7.019453in}{0.387222in}}%
\pgfpathclose%
\pgfusepath{fill}%
\end{pgfscope}%
\begin{pgfscope}%
\pgfpathrectangle{\pgfqpoint{6.580312in}{0.387222in}}{\pgfqpoint{5.269687in}{5.244444in}}%
\pgfusepath{clip}%
\pgfsetbuttcap%
\pgfsetmiterjoin%
\definecolor{currentfill}{rgb}{0.229412,0.570588,0.229412}%
\pgfsetfillcolor{currentfill}%
\pgfsetlinewidth{0.000000pt}%
\definecolor{currentstroke}{rgb}{0.000000,0.000000,0.000000}%
\pgfsetstrokecolor{currentstroke}%
\pgfsetstrokeopacity{0.000000}%
\pgfsetdash{}{0pt}%
\pgfpathmoveto{\pgfqpoint{7.019453in}{0.387222in}}%
\pgfpathlineto{\pgfqpoint{7.019453in}{0.387222in}}%
\pgfpathlineto{\pgfqpoint{7.019453in}{0.387222in}}%
\pgfpathlineto{\pgfqpoint{7.019453in}{0.387222in}}%
\pgfpathlineto{\pgfqpoint{7.019453in}{0.387222in}}%
\pgfpathclose%
\pgfusepath{fill}%
\end{pgfscope}%
\begin{pgfscope}%
\pgfpathrectangle{\pgfqpoint{6.580312in}{0.387222in}}{\pgfqpoint{5.269687in}{5.244444in}}%
\pgfusepath{clip}%
\pgfsetbuttcap%
\pgfsetmiterjoin%
\definecolor{currentfill}{rgb}{0.753431,0.238725,0.241667}%
\pgfsetfillcolor{currentfill}%
\pgfsetlinewidth{0.000000pt}%
\definecolor{currentstroke}{rgb}{0.000000,0.000000,0.000000}%
\pgfsetstrokecolor{currentstroke}%
\pgfsetstrokeopacity{0.000000}%
\pgfsetdash{}{0pt}%
\pgfpathmoveto{\pgfqpoint{7.019453in}{0.387222in}}%
\pgfpathlineto{\pgfqpoint{7.019453in}{0.387222in}}%
\pgfpathlineto{\pgfqpoint{7.019453in}{0.387222in}}%
\pgfpathlineto{\pgfqpoint{7.019453in}{0.387222in}}%
\pgfpathlineto{\pgfqpoint{7.019453in}{0.387222in}}%
\pgfpathclose%
\pgfusepath{fill}%
\end{pgfscope}%
\begin{pgfscope}%
\pgfpathrectangle{\pgfqpoint{6.580312in}{0.387222in}}{\pgfqpoint{5.269687in}{5.244444in}}%
\pgfusepath{clip}%
\pgfsetbuttcap%
\pgfsetmiterjoin%
\definecolor{currentfill}{rgb}{0.578431,0.446078,0.699020}%
\pgfsetfillcolor{currentfill}%
\pgfsetlinewidth{0.000000pt}%
\definecolor{currentstroke}{rgb}{0.000000,0.000000,0.000000}%
\pgfsetstrokecolor{currentstroke}%
\pgfsetstrokeopacity{0.000000}%
\pgfsetdash{}{0pt}%
\pgfpathmoveto{\pgfqpoint{7.019453in}{0.387222in}}%
\pgfpathlineto{\pgfqpoint{7.019453in}{0.387222in}}%
\pgfpathlineto{\pgfqpoint{7.019453in}{0.387222in}}%
\pgfpathlineto{\pgfqpoint{7.019453in}{0.387222in}}%
\pgfpathlineto{\pgfqpoint{7.019453in}{0.387222in}}%
\pgfpathclose%
\pgfusepath{fill}%
\end{pgfscope}%
\begin{pgfscope}%
\pgfsetbuttcap%
\pgfsetroundjoin%
\definecolor{currentfill}{rgb}{0.000000,0.000000,0.000000}%
\pgfsetfillcolor{currentfill}%
\pgfsetlinewidth{0.803000pt}%
\definecolor{currentstroke}{rgb}{0.000000,0.000000,0.000000}%
\pgfsetstrokecolor{currentstroke}%
\pgfsetdash{}{0pt}%
\pgfsys@defobject{currentmarker}{\pgfqpoint{0.000000in}{-0.048611in}}{\pgfqpoint{0.000000in}{0.000000in}}{%
\pgfpathmoveto{\pgfqpoint{0.000000in}{0.000000in}}%
\pgfpathlineto{\pgfqpoint{0.000000in}{-0.048611in}}%
\pgfusepath{stroke,fill}%
}%
\begin{pgfscope}%
\pgfsys@transformshift{7.019453in}{0.387222in}%
\pgfsys@useobject{currentmarker}{}%
\end{pgfscope}%
\end{pgfscope}%
\begin{pgfscope}%
\definecolor{textcolor}{rgb}{0.000000,0.000000,0.000000}%
\pgfsetstrokecolor{textcolor}%
\pgfsetfillcolor{textcolor}%
\pgftext[x=7.019453in,y=0.290000in,,top]{\color{textcolor}{\sffamily\fontsize{10.000000}{12.000000}\selectfont\catcode`\^=\active\def^{\ifmmode\sp\else\^{}\fi}\catcode`\%=\active\def%{\%}0}}%
\end{pgfscope}%
\begin{pgfscope}%
\pgfsetbuttcap%
\pgfsetroundjoin%
\definecolor{currentfill}{rgb}{0.000000,0.000000,0.000000}%
\pgfsetfillcolor{currentfill}%
\pgfsetlinewidth{0.803000pt}%
\definecolor{currentstroke}{rgb}{0.000000,0.000000,0.000000}%
\pgfsetstrokecolor{currentstroke}%
\pgfsetdash{}{0pt}%
\pgfsys@defobject{currentmarker}{\pgfqpoint{0.000000in}{-0.048611in}}{\pgfqpoint{0.000000in}{0.000000in}}{%
\pgfpathmoveto{\pgfqpoint{0.000000in}{0.000000in}}%
\pgfpathlineto{\pgfqpoint{0.000000in}{-0.048611in}}%
\pgfusepath{stroke,fill}%
}%
\begin{pgfscope}%
\pgfsys@transformshift{7.897734in}{0.387222in}%
\pgfsys@useobject{currentmarker}{}%
\end{pgfscope}%
\end{pgfscope}%
\begin{pgfscope}%
\definecolor{textcolor}{rgb}{0.000000,0.000000,0.000000}%
\pgfsetstrokecolor{textcolor}%
\pgfsetfillcolor{textcolor}%
\pgftext[x=7.897734in,y=0.290000in,,top]{\color{textcolor}{\sffamily\fontsize{10.000000}{12.000000}\selectfont\catcode`\^=\active\def^{\ifmmode\sp\else\^{}\fi}\catcode`\%=\active\def%{\%}40}}%
\end{pgfscope}%
\begin{pgfscope}%
\pgfsetbuttcap%
\pgfsetroundjoin%
\definecolor{currentfill}{rgb}{0.000000,0.000000,0.000000}%
\pgfsetfillcolor{currentfill}%
\pgfsetlinewidth{0.803000pt}%
\definecolor{currentstroke}{rgb}{0.000000,0.000000,0.000000}%
\pgfsetstrokecolor{currentstroke}%
\pgfsetdash{}{0pt}%
\pgfsys@defobject{currentmarker}{\pgfqpoint{0.000000in}{-0.048611in}}{\pgfqpoint{0.000000in}{0.000000in}}{%
\pgfpathmoveto{\pgfqpoint{0.000000in}{0.000000in}}%
\pgfpathlineto{\pgfqpoint{0.000000in}{-0.048611in}}%
\pgfusepath{stroke,fill}%
}%
\begin{pgfscope}%
\pgfsys@transformshift{8.776016in}{0.387222in}%
\pgfsys@useobject{currentmarker}{}%
\end{pgfscope}%
\end{pgfscope}%
\begin{pgfscope}%
\definecolor{textcolor}{rgb}{0.000000,0.000000,0.000000}%
\pgfsetstrokecolor{textcolor}%
\pgfsetfillcolor{textcolor}%
\pgftext[x=8.776016in,y=0.290000in,,top]{\color{textcolor}{\sffamily\fontsize{10.000000}{12.000000}\selectfont\catcode`\^=\active\def^{\ifmmode\sp\else\^{}\fi}\catcode`\%=\active\def%{\%}80}}%
\end{pgfscope}%
\begin{pgfscope}%
\pgfsetbuttcap%
\pgfsetroundjoin%
\definecolor{currentfill}{rgb}{0.000000,0.000000,0.000000}%
\pgfsetfillcolor{currentfill}%
\pgfsetlinewidth{0.803000pt}%
\definecolor{currentstroke}{rgb}{0.000000,0.000000,0.000000}%
\pgfsetstrokecolor{currentstroke}%
\pgfsetdash{}{0pt}%
\pgfsys@defobject{currentmarker}{\pgfqpoint{0.000000in}{-0.048611in}}{\pgfqpoint{0.000000in}{0.000000in}}{%
\pgfpathmoveto{\pgfqpoint{0.000000in}{0.000000in}}%
\pgfpathlineto{\pgfqpoint{0.000000in}{-0.048611in}}%
\pgfusepath{stroke,fill}%
}%
\begin{pgfscope}%
\pgfsys@transformshift{9.654297in}{0.387222in}%
\pgfsys@useobject{currentmarker}{}%
\end{pgfscope}%
\end{pgfscope}%
\begin{pgfscope}%
\definecolor{textcolor}{rgb}{0.000000,0.000000,0.000000}%
\pgfsetstrokecolor{textcolor}%
\pgfsetfillcolor{textcolor}%
\pgftext[x=9.654297in,y=0.290000in,,top]{\color{textcolor}{\sffamily\fontsize{10.000000}{12.000000}\selectfont\catcode`\^=\active\def^{\ifmmode\sp\else\^{}\fi}\catcode`\%=\active\def%{\%}120}}%
\end{pgfscope}%
\begin{pgfscope}%
\pgfsetbuttcap%
\pgfsetroundjoin%
\definecolor{currentfill}{rgb}{0.000000,0.000000,0.000000}%
\pgfsetfillcolor{currentfill}%
\pgfsetlinewidth{0.803000pt}%
\definecolor{currentstroke}{rgb}{0.000000,0.000000,0.000000}%
\pgfsetstrokecolor{currentstroke}%
\pgfsetdash{}{0pt}%
\pgfsys@defobject{currentmarker}{\pgfqpoint{0.000000in}{-0.048611in}}{\pgfqpoint{0.000000in}{0.000000in}}{%
\pgfpathmoveto{\pgfqpoint{0.000000in}{0.000000in}}%
\pgfpathlineto{\pgfqpoint{0.000000in}{-0.048611in}}%
\pgfusepath{stroke,fill}%
}%
\begin{pgfscope}%
\pgfsys@transformshift{10.532578in}{0.387222in}%
\pgfsys@useobject{currentmarker}{}%
\end{pgfscope}%
\end{pgfscope}%
\begin{pgfscope}%
\definecolor{textcolor}{rgb}{0.000000,0.000000,0.000000}%
\pgfsetstrokecolor{textcolor}%
\pgfsetfillcolor{textcolor}%
\pgftext[x=10.532578in,y=0.290000in,,top]{\color{textcolor}{\sffamily\fontsize{10.000000}{12.000000}\selectfont\catcode`\^=\active\def^{\ifmmode\sp\else\^{}\fi}\catcode`\%=\active\def%{\%}160}}%
\end{pgfscope}%
\begin{pgfscope}%
\pgfsetbuttcap%
\pgfsetroundjoin%
\definecolor{currentfill}{rgb}{0.000000,0.000000,0.000000}%
\pgfsetfillcolor{currentfill}%
\pgfsetlinewidth{0.803000pt}%
\definecolor{currentstroke}{rgb}{0.000000,0.000000,0.000000}%
\pgfsetstrokecolor{currentstroke}%
\pgfsetdash{}{0pt}%
\pgfsys@defobject{currentmarker}{\pgfqpoint{0.000000in}{-0.048611in}}{\pgfqpoint{0.000000in}{0.000000in}}{%
\pgfpathmoveto{\pgfqpoint{0.000000in}{0.000000in}}%
\pgfpathlineto{\pgfqpoint{0.000000in}{-0.048611in}}%
\pgfusepath{stroke,fill}%
}%
\begin{pgfscope}%
\pgfsys@transformshift{11.410859in}{0.387222in}%
\pgfsys@useobject{currentmarker}{}%
\end{pgfscope}%
\end{pgfscope}%
\begin{pgfscope}%
\definecolor{textcolor}{rgb}{0.000000,0.000000,0.000000}%
\pgfsetstrokecolor{textcolor}%
\pgfsetfillcolor{textcolor}%
\pgftext[x=11.410859in,y=0.290000in,,top]{\color{textcolor}{\sffamily\fontsize{10.000000}{12.000000}\selectfont\catcode`\^=\active\def^{\ifmmode\sp\else\^{}\fi}\catcode`\%=\active\def%{\%}200}}%
\end{pgfscope}%
\begin{pgfscope}%
\pgfsetbuttcap%
\pgfsetroundjoin%
\definecolor{currentfill}{rgb}{0.000000,0.000000,0.000000}%
\pgfsetfillcolor{currentfill}%
\pgfsetlinewidth{0.803000pt}%
\definecolor{currentstroke}{rgb}{0.000000,0.000000,0.000000}%
\pgfsetstrokecolor{currentstroke}%
\pgfsetdash{}{0pt}%
\pgfsys@defobject{currentmarker}{\pgfqpoint{-0.048611in}{0.000000in}}{\pgfqpoint{-0.000000in}{0.000000in}}{%
\pgfpathmoveto{\pgfqpoint{-0.000000in}{0.000000in}}%
\pgfpathlineto{\pgfqpoint{-0.048611in}{0.000000in}}%
\pgfusepath{stroke,fill}%
}%
\begin{pgfscope}%
\pgfsys@transformshift{6.580312in}{0.387222in}%
\pgfsys@useobject{currentmarker}{}%
\end{pgfscope}%
\end{pgfscope}%
\begin{pgfscope}%
\definecolor{textcolor}{rgb}{0.000000,0.000000,0.000000}%
\pgfsetstrokecolor{textcolor}%
\pgfsetfillcolor{textcolor}%
\pgftext[x=6.394725in, y=0.334461in, left, base]{\color{textcolor}{\sffamily\fontsize{10.000000}{12.000000}\selectfont\catcode`\^=\active\def^{\ifmmode\sp\else\^{}\fi}\catcode`\%=\active\def%{\%}0}}%
\end{pgfscope}%
\begin{pgfscope}%
\pgfsetbuttcap%
\pgfsetroundjoin%
\definecolor{currentfill}{rgb}{0.000000,0.000000,0.000000}%
\pgfsetfillcolor{currentfill}%
\pgfsetlinewidth{0.803000pt}%
\definecolor{currentstroke}{rgb}{0.000000,0.000000,0.000000}%
\pgfsetstrokecolor{currentstroke}%
\pgfsetdash{}{0pt}%
\pgfsys@defobject{currentmarker}{\pgfqpoint{-0.048611in}{0.000000in}}{\pgfqpoint{-0.000000in}{0.000000in}}{%
\pgfpathmoveto{\pgfqpoint{-0.000000in}{0.000000in}}%
\pgfpathlineto{\pgfqpoint{-0.048611in}{0.000000in}}%
\pgfusepath{stroke,fill}%
}%
\begin{pgfscope}%
\pgfsys@transformshift{6.580312in}{1.336787in}%
\pgfsys@useobject{currentmarker}{}%
\end{pgfscope}%
\end{pgfscope}%
\begin{pgfscope}%
\definecolor{textcolor}{rgb}{0.000000,0.000000,0.000000}%
\pgfsetstrokecolor{textcolor}%
\pgfsetfillcolor{textcolor}%
\pgftext[x=6.217994in, y=1.284025in, left, base]{\color{textcolor}{\sffamily\fontsize{10.000000}{12.000000}\selectfont\catcode`\^=\active\def^{\ifmmode\sp\else\^{}\fi}\catcode`\%=\active\def%{\%}500}}%
\end{pgfscope}%
\begin{pgfscope}%
\pgfsetbuttcap%
\pgfsetroundjoin%
\definecolor{currentfill}{rgb}{0.000000,0.000000,0.000000}%
\pgfsetfillcolor{currentfill}%
\pgfsetlinewidth{0.803000pt}%
\definecolor{currentstroke}{rgb}{0.000000,0.000000,0.000000}%
\pgfsetstrokecolor{currentstroke}%
\pgfsetdash{}{0pt}%
\pgfsys@defobject{currentmarker}{\pgfqpoint{-0.048611in}{0.000000in}}{\pgfqpoint{-0.000000in}{0.000000in}}{%
\pgfpathmoveto{\pgfqpoint{-0.000000in}{0.000000in}}%
\pgfpathlineto{\pgfqpoint{-0.048611in}{0.000000in}}%
\pgfusepath{stroke,fill}%
}%
\begin{pgfscope}%
\pgfsys@transformshift{6.580312in}{2.286351in}%
\pgfsys@useobject{currentmarker}{}%
\end{pgfscope}%
\end{pgfscope}%
\begin{pgfscope}%
\definecolor{textcolor}{rgb}{0.000000,0.000000,0.000000}%
\pgfsetstrokecolor{textcolor}%
\pgfsetfillcolor{textcolor}%
\pgftext[x=6.129629in, y=2.233590in, left, base]{\color{textcolor}{\sffamily\fontsize{10.000000}{12.000000}\selectfont\catcode`\^=\active\def^{\ifmmode\sp\else\^{}\fi}\catcode`\%=\active\def%{\%}1000}}%
\end{pgfscope}%
\begin{pgfscope}%
\pgfsetbuttcap%
\pgfsetroundjoin%
\definecolor{currentfill}{rgb}{0.000000,0.000000,0.000000}%
\pgfsetfillcolor{currentfill}%
\pgfsetlinewidth{0.803000pt}%
\definecolor{currentstroke}{rgb}{0.000000,0.000000,0.000000}%
\pgfsetstrokecolor{currentstroke}%
\pgfsetdash{}{0pt}%
\pgfsys@defobject{currentmarker}{\pgfqpoint{-0.048611in}{0.000000in}}{\pgfqpoint{-0.000000in}{0.000000in}}{%
\pgfpathmoveto{\pgfqpoint{-0.000000in}{0.000000in}}%
\pgfpathlineto{\pgfqpoint{-0.048611in}{0.000000in}}%
\pgfusepath{stroke,fill}%
}%
\begin{pgfscope}%
\pgfsys@transformshift{6.580312in}{3.235916in}%
\pgfsys@useobject{currentmarker}{}%
\end{pgfscope}%
\end{pgfscope}%
\begin{pgfscope}%
\definecolor{textcolor}{rgb}{0.000000,0.000000,0.000000}%
\pgfsetstrokecolor{textcolor}%
\pgfsetfillcolor{textcolor}%
\pgftext[x=6.129629in, y=3.183154in, left, base]{\color{textcolor}{\sffamily\fontsize{10.000000}{12.000000}\selectfont\catcode`\^=\active\def^{\ifmmode\sp\else\^{}\fi}\catcode`\%=\active\def%{\%}1500}}%
\end{pgfscope}%
\begin{pgfscope}%
\pgfsetbuttcap%
\pgfsetroundjoin%
\definecolor{currentfill}{rgb}{0.000000,0.000000,0.000000}%
\pgfsetfillcolor{currentfill}%
\pgfsetlinewidth{0.803000pt}%
\definecolor{currentstroke}{rgb}{0.000000,0.000000,0.000000}%
\pgfsetstrokecolor{currentstroke}%
\pgfsetdash{}{0pt}%
\pgfsys@defobject{currentmarker}{\pgfqpoint{-0.048611in}{0.000000in}}{\pgfqpoint{-0.000000in}{0.000000in}}{%
\pgfpathmoveto{\pgfqpoint{-0.000000in}{0.000000in}}%
\pgfpathlineto{\pgfqpoint{-0.048611in}{0.000000in}}%
\pgfusepath{stroke,fill}%
}%
\begin{pgfscope}%
\pgfsys@transformshift{6.580312in}{4.185480in}%
\pgfsys@useobject{currentmarker}{}%
\end{pgfscope}%
\end{pgfscope}%
\begin{pgfscope}%
\definecolor{textcolor}{rgb}{0.000000,0.000000,0.000000}%
\pgfsetstrokecolor{textcolor}%
\pgfsetfillcolor{textcolor}%
\pgftext[x=6.129629in, y=4.132718in, left, base]{\color{textcolor}{\sffamily\fontsize{10.000000}{12.000000}\selectfont\catcode`\^=\active\def^{\ifmmode\sp\else\^{}\fi}\catcode`\%=\active\def%{\%}2000}}%
\end{pgfscope}%
\begin{pgfscope}%
\pgfsetbuttcap%
\pgfsetroundjoin%
\definecolor{currentfill}{rgb}{0.000000,0.000000,0.000000}%
\pgfsetfillcolor{currentfill}%
\pgfsetlinewidth{0.803000pt}%
\definecolor{currentstroke}{rgb}{0.000000,0.000000,0.000000}%
\pgfsetstrokecolor{currentstroke}%
\pgfsetdash{}{0pt}%
\pgfsys@defobject{currentmarker}{\pgfqpoint{-0.048611in}{0.000000in}}{\pgfqpoint{-0.000000in}{0.000000in}}{%
\pgfpathmoveto{\pgfqpoint{-0.000000in}{0.000000in}}%
\pgfpathlineto{\pgfqpoint{-0.048611in}{0.000000in}}%
\pgfusepath{stroke,fill}%
}%
\begin{pgfscope}%
\pgfsys@transformshift{6.580312in}{5.135044in}%
\pgfsys@useobject{currentmarker}{}%
\end{pgfscope}%
\end{pgfscope}%
\begin{pgfscope}%
\definecolor{textcolor}{rgb}{0.000000,0.000000,0.000000}%
\pgfsetstrokecolor{textcolor}%
\pgfsetfillcolor{textcolor}%
\pgftext[x=6.129629in, y=5.082283in, left, base]{\color{textcolor}{\sffamily\fontsize{10.000000}{12.000000}\selectfont\catcode`\^=\active\def^{\ifmmode\sp\else\^{}\fi}\catcode`\%=\active\def%{\%}2500}}%
\end{pgfscope}%
\begin{pgfscope}%
\pgfpathrectangle{\pgfqpoint{6.580312in}{0.387222in}}{\pgfqpoint{5.269687in}{5.244444in}}%
\pgfusepath{clip}%
\pgfsetrectcap%
\pgfsetroundjoin%
\pgfsetlinewidth{2.258437pt}%
\definecolor{currentstroke}{rgb}{0.260000,0.260000,0.260000}%
\pgfsetstrokecolor{currentstroke}%
\pgfsetdash{}{0pt}%
\pgfusepath{stroke}%
\end{pgfscope}%
\begin{pgfscope}%
\pgfpathrectangle{\pgfqpoint{6.580312in}{0.387222in}}{\pgfqpoint{5.269687in}{5.244444in}}%
\pgfusepath{clip}%
\pgfsetrectcap%
\pgfsetroundjoin%
\pgfsetlinewidth{2.258437pt}%
\definecolor{currentstroke}{rgb}{0.260000,0.260000,0.260000}%
\pgfsetstrokecolor{currentstroke}%
\pgfsetdash{}{0pt}%
\pgfusepath{stroke}%
\end{pgfscope}%
\begin{pgfscope}%
\pgfpathrectangle{\pgfqpoint{6.580312in}{0.387222in}}{\pgfqpoint{5.269687in}{5.244444in}}%
\pgfusepath{clip}%
\pgfsetrectcap%
\pgfsetroundjoin%
\pgfsetlinewidth{2.258437pt}%
\definecolor{currentstroke}{rgb}{0.260000,0.260000,0.260000}%
\pgfsetstrokecolor{currentstroke}%
\pgfsetdash{}{0pt}%
\pgfusepath{stroke}%
\end{pgfscope}%
\begin{pgfscope}%
\pgfpathrectangle{\pgfqpoint{6.580312in}{0.387222in}}{\pgfqpoint{5.269687in}{5.244444in}}%
\pgfusepath{clip}%
\pgfsetrectcap%
\pgfsetroundjoin%
\pgfsetlinewidth{2.258437pt}%
\definecolor{currentstroke}{rgb}{0.260000,0.260000,0.260000}%
\pgfsetstrokecolor{currentstroke}%
\pgfsetdash{}{0pt}%
\pgfusepath{stroke}%
\end{pgfscope}%
\begin{pgfscope}%
\pgfpathrectangle{\pgfqpoint{6.580312in}{0.387222in}}{\pgfqpoint{5.269687in}{5.244444in}}%
\pgfusepath{clip}%
\pgfsetrectcap%
\pgfsetroundjoin%
\pgfsetlinewidth{2.258437pt}%
\definecolor{currentstroke}{rgb}{0.260000,0.260000,0.260000}%
\pgfsetstrokecolor{currentstroke}%
\pgfsetdash{}{0pt}%
\pgfusepath{stroke}%
\end{pgfscope}%
\begin{pgfscope}%
\pgfpathrectangle{\pgfqpoint{6.580312in}{0.387222in}}{\pgfqpoint{5.269687in}{5.244444in}}%
\pgfusepath{clip}%
\pgfsetrectcap%
\pgfsetroundjoin%
\pgfsetlinewidth{2.258437pt}%
\definecolor{currentstroke}{rgb}{0.260000,0.260000,0.260000}%
\pgfsetstrokecolor{currentstroke}%
\pgfsetdash{}{0pt}%
\pgfusepath{stroke}%
\end{pgfscope}%
\begin{pgfscope}%
\pgfpathrectangle{\pgfqpoint{6.580312in}{0.387222in}}{\pgfqpoint{5.269687in}{5.244444in}}%
\pgfusepath{clip}%
\pgfsetrectcap%
\pgfsetroundjoin%
\pgfsetlinewidth{2.258437pt}%
\definecolor{currentstroke}{rgb}{0.260000,0.260000,0.260000}%
\pgfsetstrokecolor{currentstroke}%
\pgfsetdash{}{0pt}%
\pgfusepath{stroke}%
\end{pgfscope}%
\begin{pgfscope}%
\pgfpathrectangle{\pgfqpoint{6.580312in}{0.387222in}}{\pgfqpoint{5.269687in}{5.244444in}}%
\pgfusepath{clip}%
\pgfsetrectcap%
\pgfsetroundjoin%
\pgfsetlinewidth{2.258437pt}%
\definecolor{currentstroke}{rgb}{0.260000,0.260000,0.260000}%
\pgfsetstrokecolor{currentstroke}%
\pgfsetdash{}{0pt}%
\pgfusepath{stroke}%
\end{pgfscope}%
\begin{pgfscope}%
\pgfpathrectangle{\pgfqpoint{6.580312in}{0.387222in}}{\pgfqpoint{5.269687in}{5.244444in}}%
\pgfusepath{clip}%
\pgfsetrectcap%
\pgfsetroundjoin%
\pgfsetlinewidth{2.258437pt}%
\definecolor{currentstroke}{rgb}{0.260000,0.260000,0.260000}%
\pgfsetstrokecolor{currentstroke}%
\pgfsetdash{}{0pt}%
\pgfusepath{stroke}%
\end{pgfscope}%
\begin{pgfscope}%
\pgfpathrectangle{\pgfqpoint{6.580312in}{0.387222in}}{\pgfqpoint{5.269687in}{5.244444in}}%
\pgfusepath{clip}%
\pgfsetrectcap%
\pgfsetroundjoin%
\pgfsetlinewidth{2.258437pt}%
\definecolor{currentstroke}{rgb}{0.260000,0.260000,0.260000}%
\pgfsetstrokecolor{currentstroke}%
\pgfsetdash{}{0pt}%
\pgfusepath{stroke}%
\end{pgfscope}%
\begin{pgfscope}%
\pgfpathrectangle{\pgfqpoint{6.580312in}{0.387222in}}{\pgfqpoint{5.269687in}{5.244444in}}%
\pgfusepath{clip}%
\pgfsetrectcap%
\pgfsetroundjoin%
\pgfsetlinewidth{2.258437pt}%
\definecolor{currentstroke}{rgb}{0.260000,0.260000,0.260000}%
\pgfsetstrokecolor{currentstroke}%
\pgfsetdash{}{0pt}%
\pgfusepath{stroke}%
\end{pgfscope}%
\begin{pgfscope}%
\pgfpathrectangle{\pgfqpoint{6.580312in}{0.387222in}}{\pgfqpoint{5.269687in}{5.244444in}}%
\pgfusepath{clip}%
\pgfsetrectcap%
\pgfsetroundjoin%
\pgfsetlinewidth{2.258437pt}%
\definecolor{currentstroke}{rgb}{0.260000,0.260000,0.260000}%
\pgfsetstrokecolor{currentstroke}%
\pgfsetdash{}{0pt}%
\pgfusepath{stroke}%
\end{pgfscope}%
\begin{pgfscope}%
\pgfpathrectangle{\pgfqpoint{6.580312in}{0.387222in}}{\pgfqpoint{5.269687in}{5.244444in}}%
\pgfusepath{clip}%
\pgfsetrectcap%
\pgfsetroundjoin%
\pgfsetlinewidth{2.258437pt}%
\definecolor{currentstroke}{rgb}{0.260000,0.260000,0.260000}%
\pgfsetstrokecolor{currentstroke}%
\pgfsetdash{}{0pt}%
\pgfusepath{stroke}%
\end{pgfscope}%
\begin{pgfscope}%
\pgfpathrectangle{\pgfqpoint{6.580312in}{0.387222in}}{\pgfqpoint{5.269687in}{5.244444in}}%
\pgfusepath{clip}%
\pgfsetrectcap%
\pgfsetroundjoin%
\pgfsetlinewidth{2.258437pt}%
\definecolor{currentstroke}{rgb}{0.260000,0.260000,0.260000}%
\pgfsetstrokecolor{currentstroke}%
\pgfsetdash{}{0pt}%
\pgfusepath{stroke}%
\end{pgfscope}%
\begin{pgfscope}%
\pgfpathrectangle{\pgfqpoint{6.580312in}{0.387222in}}{\pgfqpoint{5.269687in}{5.244444in}}%
\pgfusepath{clip}%
\pgfsetrectcap%
\pgfsetroundjoin%
\pgfsetlinewidth{2.258437pt}%
\definecolor{currentstroke}{rgb}{0.260000,0.260000,0.260000}%
\pgfsetstrokecolor{currentstroke}%
\pgfsetdash{}{0pt}%
\pgfusepath{stroke}%
\end{pgfscope}%
\begin{pgfscope}%
\pgfpathrectangle{\pgfqpoint{6.580312in}{0.387222in}}{\pgfqpoint{5.269687in}{5.244444in}}%
\pgfusepath{clip}%
\pgfsetrectcap%
\pgfsetroundjoin%
\pgfsetlinewidth{2.258437pt}%
\definecolor{currentstroke}{rgb}{0.260000,0.260000,0.260000}%
\pgfsetstrokecolor{currentstroke}%
\pgfsetdash{}{0pt}%
\pgfusepath{stroke}%
\end{pgfscope}%
\begin{pgfscope}%
\pgfpathrectangle{\pgfqpoint{6.580312in}{0.387222in}}{\pgfqpoint{5.269687in}{5.244444in}}%
\pgfusepath{clip}%
\pgfsetrectcap%
\pgfsetroundjoin%
\pgfsetlinewidth{2.258437pt}%
\definecolor{currentstroke}{rgb}{0.260000,0.260000,0.260000}%
\pgfsetstrokecolor{currentstroke}%
\pgfsetdash{}{0pt}%
\pgfusepath{stroke}%
\end{pgfscope}%
\begin{pgfscope}%
\pgfpathrectangle{\pgfqpoint{6.580312in}{0.387222in}}{\pgfqpoint{5.269687in}{5.244444in}}%
\pgfusepath{clip}%
\pgfsetrectcap%
\pgfsetroundjoin%
\pgfsetlinewidth{2.258437pt}%
\definecolor{currentstroke}{rgb}{0.260000,0.260000,0.260000}%
\pgfsetstrokecolor{currentstroke}%
\pgfsetdash{}{0pt}%
\pgfusepath{stroke}%
\end{pgfscope}%
\begin{pgfscope}%
\pgfpathrectangle{\pgfqpoint{6.580312in}{0.387222in}}{\pgfqpoint{5.269687in}{5.244444in}}%
\pgfusepath{clip}%
\pgfsetrectcap%
\pgfsetroundjoin%
\pgfsetlinewidth{2.258437pt}%
\definecolor{currentstroke}{rgb}{0.260000,0.260000,0.260000}%
\pgfsetstrokecolor{currentstroke}%
\pgfsetdash{}{0pt}%
\pgfusepath{stroke}%
\end{pgfscope}%
\begin{pgfscope}%
\pgfpathrectangle{\pgfqpoint{6.580312in}{0.387222in}}{\pgfqpoint{5.269687in}{5.244444in}}%
\pgfusepath{clip}%
\pgfsetrectcap%
\pgfsetroundjoin%
\pgfsetlinewidth{2.258437pt}%
\definecolor{currentstroke}{rgb}{0.260000,0.260000,0.260000}%
\pgfsetstrokecolor{currentstroke}%
\pgfsetdash{}{0pt}%
\pgfusepath{stroke}%
\end{pgfscope}%
\begin{pgfscope}%
\pgfpathrectangle{\pgfqpoint{6.580312in}{0.387222in}}{\pgfqpoint{5.269687in}{5.244444in}}%
\pgfusepath{clip}%
\pgfsetrectcap%
\pgfsetroundjoin%
\pgfsetlinewidth{2.258437pt}%
\definecolor{currentstroke}{rgb}{0.260000,0.260000,0.260000}%
\pgfsetstrokecolor{currentstroke}%
\pgfsetdash{}{0pt}%
\pgfusepath{stroke}%
\end{pgfscope}%
\begin{pgfscope}%
\pgfpathrectangle{\pgfqpoint{6.580312in}{0.387222in}}{\pgfqpoint{5.269687in}{5.244444in}}%
\pgfusepath{clip}%
\pgfsetrectcap%
\pgfsetroundjoin%
\pgfsetlinewidth{2.258437pt}%
\definecolor{currentstroke}{rgb}{0.260000,0.260000,0.260000}%
\pgfsetstrokecolor{currentstroke}%
\pgfsetdash{}{0pt}%
\pgfusepath{stroke}%
\end{pgfscope}%
\begin{pgfscope}%
\pgfpathrectangle{\pgfqpoint{6.580312in}{0.387222in}}{\pgfqpoint{5.269687in}{5.244444in}}%
\pgfusepath{clip}%
\pgfsetrectcap%
\pgfsetroundjoin%
\pgfsetlinewidth{2.258437pt}%
\definecolor{currentstroke}{rgb}{0.260000,0.260000,0.260000}%
\pgfsetstrokecolor{currentstroke}%
\pgfsetdash{}{0pt}%
\pgfusepath{stroke}%
\end{pgfscope}%
\begin{pgfscope}%
\pgfpathrectangle{\pgfqpoint{6.580312in}{0.387222in}}{\pgfqpoint{5.269687in}{5.244444in}}%
\pgfusepath{clip}%
\pgfsetrectcap%
\pgfsetroundjoin%
\pgfsetlinewidth{2.258437pt}%
\definecolor{currentstroke}{rgb}{0.260000,0.260000,0.260000}%
\pgfsetstrokecolor{currentstroke}%
\pgfsetdash{}{0pt}%
\pgfusepath{stroke}%
\end{pgfscope}%
\begin{pgfscope}%
\pgfpathrectangle{\pgfqpoint{6.580312in}{0.387222in}}{\pgfqpoint{5.269687in}{5.244444in}}%
\pgfusepath{clip}%
\pgfsetrectcap%
\pgfsetroundjoin%
\pgfsetlinewidth{2.258437pt}%
\definecolor{currentstroke}{rgb}{0.260000,0.260000,0.260000}%
\pgfsetstrokecolor{currentstroke}%
\pgfsetdash{}{0pt}%
\pgfusepath{stroke}%
\end{pgfscope}%
\begin{pgfscope}%
\pgfpathrectangle{\pgfqpoint{6.580312in}{0.387222in}}{\pgfqpoint{5.269687in}{5.244444in}}%
\pgfusepath{clip}%
\pgfsetrectcap%
\pgfsetroundjoin%
\pgfsetlinewidth{2.258437pt}%
\definecolor{currentstroke}{rgb}{0.260000,0.260000,0.260000}%
\pgfsetstrokecolor{currentstroke}%
\pgfsetdash{}{0pt}%
\pgfusepath{stroke}%
\end{pgfscope}%
\begin{pgfscope}%
\pgfpathrectangle{\pgfqpoint{6.580312in}{0.387222in}}{\pgfqpoint{5.269687in}{5.244444in}}%
\pgfusepath{clip}%
\pgfsetrectcap%
\pgfsetroundjoin%
\pgfsetlinewidth{2.258437pt}%
\definecolor{currentstroke}{rgb}{0.260000,0.260000,0.260000}%
\pgfsetstrokecolor{currentstroke}%
\pgfsetdash{}{0pt}%
\pgfusepath{stroke}%
\end{pgfscope}%
\begin{pgfscope}%
\pgfpathrectangle{\pgfqpoint{6.580312in}{0.387222in}}{\pgfqpoint{5.269687in}{5.244444in}}%
\pgfusepath{clip}%
\pgfsetrectcap%
\pgfsetroundjoin%
\pgfsetlinewidth{2.258437pt}%
\definecolor{currentstroke}{rgb}{0.260000,0.260000,0.260000}%
\pgfsetstrokecolor{currentstroke}%
\pgfsetdash{}{0pt}%
\pgfusepath{stroke}%
\end{pgfscope}%
\begin{pgfscope}%
\pgfpathrectangle{\pgfqpoint{6.580312in}{0.387222in}}{\pgfqpoint{5.269687in}{5.244444in}}%
\pgfusepath{clip}%
\pgfsetrectcap%
\pgfsetroundjoin%
\pgfsetlinewidth{2.258437pt}%
\definecolor{currentstroke}{rgb}{0.260000,0.260000,0.260000}%
\pgfsetstrokecolor{currentstroke}%
\pgfsetdash{}{0pt}%
\pgfusepath{stroke}%
\end{pgfscope}%
\begin{pgfscope}%
\pgfpathrectangle{\pgfqpoint{6.580312in}{0.387222in}}{\pgfqpoint{5.269687in}{5.244444in}}%
\pgfusepath{clip}%
\pgfsetrectcap%
\pgfsetroundjoin%
\pgfsetlinewidth{2.258437pt}%
\definecolor{currentstroke}{rgb}{0.260000,0.260000,0.260000}%
\pgfsetstrokecolor{currentstroke}%
\pgfsetdash{}{0pt}%
\pgfusepath{stroke}%
\end{pgfscope}%
\begin{pgfscope}%
\pgfsetrectcap%
\pgfsetmiterjoin%
\pgfsetlinewidth{0.803000pt}%
\definecolor{currentstroke}{rgb}{0.000000,0.000000,0.000000}%
\pgfsetstrokecolor{currentstroke}%
\pgfsetdash{}{0pt}%
\pgfpathmoveto{\pgfqpoint{6.580312in}{0.387222in}}%
\pgfpathlineto{\pgfqpoint{6.580312in}{5.631667in}}%
\pgfusepath{stroke}%
\end{pgfscope}%
\begin{pgfscope}%
\pgfsetrectcap%
\pgfsetmiterjoin%
\pgfsetlinewidth{0.803000pt}%
\definecolor{currentstroke}{rgb}{0.000000,0.000000,0.000000}%
\pgfsetstrokecolor{currentstroke}%
\pgfsetdash{}{0pt}%
\pgfpathmoveto{\pgfqpoint{11.850000in}{0.387222in}}%
\pgfpathlineto{\pgfqpoint{11.850000in}{5.631667in}}%
\pgfusepath{stroke}%
\end{pgfscope}%
\begin{pgfscope}%
\pgfsetrectcap%
\pgfsetmiterjoin%
\pgfsetlinewidth{0.803000pt}%
\definecolor{currentstroke}{rgb}{0.000000,0.000000,0.000000}%
\pgfsetstrokecolor{currentstroke}%
\pgfsetdash{}{0pt}%
\pgfpathmoveto{\pgfqpoint{6.580312in}{0.387222in}}%
\pgfpathlineto{\pgfqpoint{11.850000in}{0.387222in}}%
\pgfusepath{stroke}%
\end{pgfscope}%
\begin{pgfscope}%
\pgfsetrectcap%
\pgfsetmiterjoin%
\pgfsetlinewidth{0.803000pt}%
\definecolor{currentstroke}{rgb}{0.000000,0.000000,0.000000}%
\pgfsetstrokecolor{currentstroke}%
\pgfsetdash{}{0pt}%
\pgfpathmoveto{\pgfqpoint{6.580312in}{5.631667in}}%
\pgfpathlineto{\pgfqpoint{11.850000in}{5.631667in}}%
\pgfusepath{stroke}%
\end{pgfscope}%
\begin{pgfscope}%
\definecolor{textcolor}{rgb}{0.000000,0.000000,0.000000}%
\pgfsetstrokecolor{textcolor}%
\pgfsetfillcolor{textcolor}%
\pgftext[x=9.215156in,y=5.715000in,,base]{\color{textcolor}{\sffamily\fontsize{12.000000}{14.400000}\selectfont\catcode`\^=\active\def^{\ifmmode\sp\else\^{}\fi}\catcode`\%=\active\def%{\%}Distribuição de competence}}%
\end{pgfscope}%
\begin{pgfscope}%
\pgfsetbuttcap%
\pgfsetmiterjoin%
\definecolor{currentfill}{rgb}{1.000000,1.000000,1.000000}%
\pgfsetfillcolor{currentfill}%
\pgfsetfillopacity{0.800000}%
\pgfsetlinewidth{1.003750pt}%
\definecolor{currentstroke}{rgb}{0.800000,0.800000,0.800000}%
\pgfsetstrokecolor{currentstroke}%
\pgfsetstrokeopacity{0.800000}%
\pgfsetdash{}{0pt}%
\pgfpathmoveto{\pgfqpoint{10.209361in}{4.473465in}}%
\pgfpathlineto{\pgfqpoint{11.752778in}{4.473465in}}%
\pgfpathquadraticcurveto{\pgfqpoint{11.780556in}{4.473465in}}{\pgfqpoint{11.780556in}{4.501242in}}%
\pgfpathlineto{\pgfqpoint{11.780556in}{5.534444in}}%
\pgfpathquadraticcurveto{\pgfqpoint{11.780556in}{5.562222in}}{\pgfqpoint{11.752778in}{5.562222in}}%
\pgfpathlineto{\pgfqpoint{10.209361in}{5.562222in}}%
\pgfpathquadraticcurveto{\pgfqpoint{10.181584in}{5.562222in}}{\pgfqpoint{10.181584in}{5.534444in}}%
\pgfpathlineto{\pgfqpoint{10.181584in}{4.501242in}}%
\pgfpathquadraticcurveto{\pgfqpoint{10.181584in}{4.473465in}}{\pgfqpoint{10.209361in}{4.473465in}}%
\pgfpathlineto{\pgfqpoint{10.209361in}{4.473465in}}%
\pgfpathclose%
\pgfusepath{stroke,fill}%
\end{pgfscope}%
\begin{pgfscope}%
\pgfsetbuttcap%
\pgfsetmiterjoin%
\definecolor{currentfill}{rgb}{0.194608,0.453431,0.632843}%
\pgfsetfillcolor{currentfill}%
\pgfsetlinewidth{0.000000pt}%
\definecolor{currentstroke}{rgb}{0.000000,0.000000,0.000000}%
\pgfsetstrokecolor{currentstroke}%
\pgfsetstrokeopacity{0.000000}%
\pgfsetdash{}{0pt}%
\pgfpathmoveto{\pgfqpoint{10.237139in}{5.395583in}}%
\pgfpathlineto{\pgfqpoint{10.514917in}{5.395583in}}%
\pgfpathlineto{\pgfqpoint{10.514917in}{5.492805in}}%
\pgfpathlineto{\pgfqpoint{10.237139in}{5.492805in}}%
\pgfpathlineto{\pgfqpoint{10.237139in}{5.395583in}}%
\pgfpathclose%
\pgfusepath{fill}%
\end{pgfscope}%
\begin{pgfscope}%
\definecolor{textcolor}{rgb}{0.000000,0.000000,0.000000}%
\pgfsetstrokecolor{textcolor}%
\pgfsetfillcolor{textcolor}%
\pgftext[x=10.626028in,y=5.395583in,left,base]{\color{textcolor}{\sffamily\fontsize{10.000000}{12.000000}\selectfont\catcode`\^=\active\def^{\ifmmode\sp\else\^{}\fi}\catcode`\%=\active\def%{\%}Competência I}}%
\end{pgfscope}%
\begin{pgfscope}%
\pgfsetbuttcap%
\pgfsetmiterjoin%
\definecolor{currentfill}{rgb}{0.881863,0.505392,0.173039}%
\pgfsetfillcolor{currentfill}%
\pgfsetlinewidth{0.000000pt}%
\definecolor{currentstroke}{rgb}{0.000000,0.000000,0.000000}%
\pgfsetstrokecolor{currentstroke}%
\pgfsetstrokeopacity{0.000000}%
\pgfsetdash{}{0pt}%
\pgfpathmoveto{\pgfqpoint{10.237139in}{5.186164in}}%
\pgfpathlineto{\pgfqpoint{10.514917in}{5.186164in}}%
\pgfpathlineto{\pgfqpoint{10.514917in}{5.283387in}}%
\pgfpathlineto{\pgfqpoint{10.237139in}{5.283387in}}%
\pgfpathlineto{\pgfqpoint{10.237139in}{5.186164in}}%
\pgfpathclose%
\pgfusepath{fill}%
\end{pgfscope}%
\begin{pgfscope}%
\definecolor{textcolor}{rgb}{0.000000,0.000000,0.000000}%
\pgfsetstrokecolor{textcolor}%
\pgfsetfillcolor{textcolor}%
\pgftext[x=10.626028in,y=5.186164in,left,base]{\color{textcolor}{\sffamily\fontsize{10.000000}{12.000000}\selectfont\catcode`\^=\active\def^{\ifmmode\sp\else\^{}\fi}\catcode`\%=\active\def%{\%}Competência II}}%
\end{pgfscope}%
\begin{pgfscope}%
\pgfsetbuttcap%
\pgfsetmiterjoin%
\definecolor{currentfill}{rgb}{0.229412,0.570588,0.229412}%
\pgfsetfillcolor{currentfill}%
\pgfsetlinewidth{0.000000pt}%
\definecolor{currentstroke}{rgb}{0.000000,0.000000,0.000000}%
\pgfsetstrokecolor{currentstroke}%
\pgfsetstrokeopacity{0.000000}%
\pgfsetdash{}{0pt}%
\pgfpathmoveto{\pgfqpoint{10.237139in}{4.976746in}}%
\pgfpathlineto{\pgfqpoint{10.514917in}{4.976746in}}%
\pgfpathlineto{\pgfqpoint{10.514917in}{5.073968in}}%
\pgfpathlineto{\pgfqpoint{10.237139in}{5.073968in}}%
\pgfpathlineto{\pgfqpoint{10.237139in}{4.976746in}}%
\pgfpathclose%
\pgfusepath{fill}%
\end{pgfscope}%
\begin{pgfscope}%
\definecolor{textcolor}{rgb}{0.000000,0.000000,0.000000}%
\pgfsetstrokecolor{textcolor}%
\pgfsetfillcolor{textcolor}%
\pgftext[x=10.626028in,y=4.976746in,left,base]{\color{textcolor}{\sffamily\fontsize{10.000000}{12.000000}\selectfont\catcode`\^=\active\def^{\ifmmode\sp\else\^{}\fi}\catcode`\%=\active\def%{\%}Competência III}}%
\end{pgfscope}%
\begin{pgfscope}%
\pgfsetbuttcap%
\pgfsetmiterjoin%
\definecolor{currentfill}{rgb}{0.753431,0.238725,0.241667}%
\pgfsetfillcolor{currentfill}%
\pgfsetlinewidth{0.000000pt}%
\definecolor{currentstroke}{rgb}{0.000000,0.000000,0.000000}%
\pgfsetstrokecolor{currentstroke}%
\pgfsetstrokeopacity{0.000000}%
\pgfsetdash{}{0pt}%
\pgfpathmoveto{\pgfqpoint{10.237139in}{4.767328in}}%
\pgfpathlineto{\pgfqpoint{10.514917in}{4.767328in}}%
\pgfpathlineto{\pgfqpoint{10.514917in}{4.864550in}}%
\pgfpathlineto{\pgfqpoint{10.237139in}{4.864550in}}%
\pgfpathlineto{\pgfqpoint{10.237139in}{4.767328in}}%
\pgfpathclose%
\pgfusepath{fill}%
\end{pgfscope}%
\begin{pgfscope}%
\definecolor{textcolor}{rgb}{0.000000,0.000000,0.000000}%
\pgfsetstrokecolor{textcolor}%
\pgfsetfillcolor{textcolor}%
\pgftext[x=10.626028in,y=4.767328in,left,base]{\color{textcolor}{\sffamily\fontsize{10.000000}{12.000000}\selectfont\catcode`\^=\active\def^{\ifmmode\sp\else\^{}\fi}\catcode`\%=\active\def%{\%}Competência IV}}%
\end{pgfscope}%
\begin{pgfscope}%
\pgfsetbuttcap%
\pgfsetmiterjoin%
\definecolor{currentfill}{rgb}{0.578431,0.446078,0.699020}%
\pgfsetfillcolor{currentfill}%
\pgfsetlinewidth{0.000000pt}%
\definecolor{currentstroke}{rgb}{0.000000,0.000000,0.000000}%
\pgfsetstrokecolor{currentstroke}%
\pgfsetstrokeopacity{0.000000}%
\pgfsetdash{}{0pt}%
\pgfpathmoveto{\pgfqpoint{10.237139in}{4.557910in}}%
\pgfpathlineto{\pgfqpoint{10.514917in}{4.557910in}}%
\pgfpathlineto{\pgfqpoint{10.514917in}{4.655132in}}%
\pgfpathlineto{\pgfqpoint{10.237139in}{4.655132in}}%
\pgfpathlineto{\pgfqpoint{10.237139in}{4.557910in}}%
\pgfpathclose%
\pgfusepath{fill}%
\end{pgfscope}%
\begin{pgfscope}%
\definecolor{textcolor}{rgb}{0.000000,0.000000,0.000000}%
\pgfsetstrokecolor{textcolor}%
\pgfsetfillcolor{textcolor}%
\pgftext[x=10.626028in,y=4.557910in,left,base]{\color{textcolor}{\sffamily\fontsize{10.000000}{12.000000}\selectfont\catcode`\^=\active\def^{\ifmmode\sp\else\^{}\fi}\catcode`\%=\active\def%{\%}Competência V}}%
\end{pgfscope}%
\end{pgfpicture}%
\makeatother%
\endgroup%
}
\end{figure}

Pela figura \ref{fig:essay-br-basic-analysis}, nota-se que o conjunto de dados apresenta uma distribuição com características próximas a de uma normal para as pontuações finais e notas de competências, contendo médias ligeiramente superiores às medianas. Destaca-se, ainda, a variabilidade nas notas de competências, sendo algumas mais concentradas --- como a competência I --- e outras mais dispersas --- como a competência V.

Para a Essay-BR básica, os autores \citet{marinho-et-al-21} disponibilizaram uma biblioteca em Python que visa facilitar a manipulação dos dados das redações, permitindo desde a criação de \textit{dataframes}\footnote{Os \textit{dataframes} são as estruturas de dados de tabela do Pandas, que permitem a visualização, manipulação e extração de informações de uma base.} até a divisão das instâncias em conjuntos de treino, teste e validação. O código-fonte e o conteúdo correspondente estão disponíveis no GitHub\footnote{\url{https://github.com/rafaelanchieta/essay}}.

\subsection{Essay-BR Estendida}
\label{subsec:essay-br-extended}

A versão estendida da Essay-BR contém uma quantidade de 6579 textos em 151 temas distintos, coletados de dezembro de 2015 a agosto de 2021. Além das instâncias já mencionadas em \ref{subsec:essay-br-basic}, ela contém mais 1160 redações reunidas do portal Vestibular UOL, em um período de tempo mais atualizado, e 849 redações extraídas do trabalho de \citet{amorim-et-al-2017}.

A base de dados contém instâncias com 9 informações diferentes a respeito das redações: o identificador do tema (\texttt{prompt}), o título (\texttt{title}), o texto (\texttt{essay}), as notas das competências (\texttt{c1}, \texttt{c2}, \texttt{c3}, \texttt{c4} e \texttt{c5}) e a pontuação final (\texttt{score}). Ao contrário da organização anterior, as notas das competências são separadas em 5 colunas distintas na versão estendida.

Para a análise exploratória dos dados, utilizamos a mesma abordagem da seção \ref{subsec:essay-br-basic}. A figura \ref{fig:essay-br-extended-analysis} mostra a distribuição de pontuações finais e das notas das competências da Essay-BR estendida.

\begin{figure}[H]
    \caption{Gráficos com a distribuição da pontuação final (à esquerda) e a distribuição das notas das competências (à direita) da Essay-BR estendida.}
    \label{fig:essay-br-extended-analysis}
    \centering
    \resizebox{\textwidth}{!}{%% Creator: Matplotlib, PGF backend
%%
%% To include the figure in your LaTeX document, write
%%   \input{<filename>.pgf}
%%
%% Make sure the required packages are loaded in your preamble
%%   \usepackage{pgf}
%%
%% Also ensure that all the required font packages are loaded; for instance,
%% the lmodern package is sometimes necessary when using math font.
%%   \usepackage{lmodern}
%%
%% Figures using additional raster images can only be included by \input if
%% they are in the same directory as the main LaTeX file. For loading figures
%% from other directories you can use the `import` package
%%   \usepackage{import}
%%
%% and then include the figures with
%%   \import{<path to file>}{<filename>.pgf}
%%
%% Matplotlib used the following preamble
%%   \def\mathdefault#1{#1}
%%   \everymath=\expandafter{\the\everymath\displaystyle}
%%   
%%   \usepackage{fontspec}
%%   \setmainfont{DejaVuSerif.ttf}[Path=\detokenize{/Users/josemayer/Documents/Pacotes/mambaforge/lib/python3.10/site-packages/matplotlib/mpl-data/fonts/ttf/}]
%%   \setsansfont{DejaVuSans.ttf}[Path=\detokenize{/Users/josemayer/Documents/Pacotes/mambaforge/lib/python3.10/site-packages/matplotlib/mpl-data/fonts/ttf/}]
%%   \setmonofont{DejaVuSansMono.ttf}[Path=\detokenize{/Users/josemayer/Documents/Pacotes/mambaforge/lib/python3.10/site-packages/matplotlib/mpl-data/fonts/ttf/}]
%%   \makeatletter\@ifpackageloaded{underscore}{}{\usepackage[strings]{underscore}}\makeatother
%%
\begingroup%
\makeatletter%
\begin{pgfpicture}%
\pgfpathrectangle{\pgfpointorigin}{\pgfqpoint{12.000000in}{6.000000in}}%
\pgfusepath{use as bounding box, clip}%
\begin{pgfscope}%
\pgfsetbuttcap%
\pgfsetmiterjoin%
\definecolor{currentfill}{rgb}{1.000000,1.000000,1.000000}%
\pgfsetfillcolor{currentfill}%
\pgfsetlinewidth{0.000000pt}%
\definecolor{currentstroke}{rgb}{1.000000,1.000000,1.000000}%
\pgfsetstrokecolor{currentstroke}%
\pgfsetdash{}{0pt}%
\pgfpathmoveto{\pgfqpoint{0.000000in}{0.000000in}}%
\pgfpathlineto{\pgfqpoint{12.000000in}{0.000000in}}%
\pgfpathlineto{\pgfqpoint{12.000000in}{6.000000in}}%
\pgfpathlineto{\pgfqpoint{0.000000in}{6.000000in}}%
\pgfpathlineto{\pgfqpoint{0.000000in}{0.000000in}}%
\pgfpathclose%
\pgfusepath{fill}%
\end{pgfscope}%
\begin{pgfscope}%
\pgfsetbuttcap%
\pgfsetmiterjoin%
\definecolor{currentfill}{rgb}{1.000000,1.000000,1.000000}%
\pgfsetfillcolor{currentfill}%
\pgfsetlinewidth{0.000000pt}%
\definecolor{currentstroke}{rgb}{0.000000,0.000000,0.000000}%
\pgfsetstrokecolor{currentstroke}%
\pgfsetstrokeopacity{0.000000}%
\pgfsetdash{}{0pt}%
\pgfpathmoveto{\pgfqpoint{0.709028in}{0.387222in}}%
\pgfpathlineto{\pgfqpoint{5.978403in}{0.387222in}}%
\pgfpathlineto{\pgfqpoint{5.978403in}{5.631667in}}%
\pgfpathlineto{\pgfqpoint{0.709028in}{5.631667in}}%
\pgfpathlineto{\pgfqpoint{0.709028in}{0.387222in}}%
\pgfpathclose%
\pgfusepath{fill}%
\end{pgfscope}%
\begin{pgfscope}%
\pgfpathrectangle{\pgfqpoint{0.709028in}{0.387222in}}{\pgfqpoint{5.269375in}{5.244444in}}%
\pgfusepath{clip}%
\pgfsetbuttcap%
\pgfsetmiterjoin%
\definecolor{currentfill}{rgb}{0.121569,0.466667,0.705882}%
\pgfsetfillcolor{currentfill}%
\pgfsetfillopacity{0.500000}%
\pgfsetlinewidth{1.003750pt}%
\definecolor{currentstroke}{rgb}{0.000000,0.000000,0.000000}%
\pgfsetstrokecolor{currentstroke}%
\pgfsetdash{}{0pt}%
\pgfpathmoveto{\pgfqpoint{0.948545in}{0.387222in}}%
\pgfpathlineto{\pgfqpoint{1.132789in}{0.387222in}}%
\pgfpathlineto{\pgfqpoint{1.132789in}{1.031701in}}%
\pgfpathlineto{\pgfqpoint{0.948545in}{1.031701in}}%
\pgfpathlineto{\pgfqpoint{0.948545in}{0.387222in}}%
\pgfpathclose%
\pgfusepath{stroke,fill}%
\end{pgfscope}%
\begin{pgfscope}%
\pgfpathrectangle{\pgfqpoint{0.709028in}{0.387222in}}{\pgfqpoint{5.269375in}{5.244444in}}%
\pgfusepath{clip}%
\pgfsetbuttcap%
\pgfsetmiterjoin%
\definecolor{currentfill}{rgb}{0.121569,0.466667,0.705882}%
\pgfsetfillcolor{currentfill}%
\pgfsetfillopacity{0.500000}%
\pgfsetlinewidth{1.003750pt}%
\definecolor{currentstroke}{rgb}{0.000000,0.000000,0.000000}%
\pgfsetstrokecolor{currentstroke}%
\pgfsetdash{}{0pt}%
\pgfpathmoveto{\pgfqpoint{1.132789in}{0.387222in}}%
\pgfpathlineto{\pgfqpoint{1.317033in}{0.387222in}}%
\pgfpathlineto{\pgfqpoint{1.317033in}{0.387222in}}%
\pgfpathlineto{\pgfqpoint{1.132789in}{0.387222in}}%
\pgfpathlineto{\pgfqpoint{1.132789in}{0.387222in}}%
\pgfpathclose%
\pgfusepath{stroke,fill}%
\end{pgfscope}%
\begin{pgfscope}%
\pgfpathrectangle{\pgfqpoint{0.709028in}{0.387222in}}{\pgfqpoint{5.269375in}{5.244444in}}%
\pgfusepath{clip}%
\pgfsetbuttcap%
\pgfsetmiterjoin%
\definecolor{currentfill}{rgb}{0.121569,0.466667,0.705882}%
\pgfsetfillcolor{currentfill}%
\pgfsetfillopacity{0.500000}%
\pgfsetlinewidth{1.003750pt}%
\definecolor{currentstroke}{rgb}{0.000000,0.000000,0.000000}%
\pgfsetstrokecolor{currentstroke}%
\pgfsetdash{}{0pt}%
\pgfpathmoveto{\pgfqpoint{1.317033in}{0.387222in}}%
\pgfpathlineto{\pgfqpoint{1.501276in}{0.387222in}}%
\pgfpathlineto{\pgfqpoint{1.501276in}{0.457274in}}%
\pgfpathlineto{\pgfqpoint{1.317033in}{0.457274in}}%
\pgfpathlineto{\pgfqpoint{1.317033in}{0.387222in}}%
\pgfpathclose%
\pgfusepath{stroke,fill}%
\end{pgfscope}%
\begin{pgfscope}%
\pgfpathrectangle{\pgfqpoint{0.709028in}{0.387222in}}{\pgfqpoint{5.269375in}{5.244444in}}%
\pgfusepath{clip}%
\pgfsetbuttcap%
\pgfsetmiterjoin%
\definecolor{currentfill}{rgb}{0.121569,0.466667,0.705882}%
\pgfsetfillcolor{currentfill}%
\pgfsetfillopacity{0.500000}%
\pgfsetlinewidth{1.003750pt}%
\definecolor{currentstroke}{rgb}{0.000000,0.000000,0.000000}%
\pgfsetstrokecolor{currentstroke}%
\pgfsetdash{}{0pt}%
\pgfpathmoveto{\pgfqpoint{1.501276in}{0.387222in}}%
\pgfpathlineto{\pgfqpoint{1.685520in}{0.387222in}}%
\pgfpathlineto{\pgfqpoint{1.685520in}{0.415243in}}%
\pgfpathlineto{\pgfqpoint{1.501276in}{0.415243in}}%
\pgfpathlineto{\pgfqpoint{1.501276in}{0.387222in}}%
\pgfpathclose%
\pgfusepath{stroke,fill}%
\end{pgfscope}%
\begin{pgfscope}%
\pgfpathrectangle{\pgfqpoint{0.709028in}{0.387222in}}{\pgfqpoint{5.269375in}{5.244444in}}%
\pgfusepath{clip}%
\pgfsetbuttcap%
\pgfsetmiterjoin%
\definecolor{currentfill}{rgb}{0.121569,0.466667,0.705882}%
\pgfsetfillcolor{currentfill}%
\pgfsetfillopacity{0.500000}%
\pgfsetlinewidth{1.003750pt}%
\definecolor{currentstroke}{rgb}{0.000000,0.000000,0.000000}%
\pgfsetstrokecolor{currentstroke}%
\pgfsetdash{}{0pt}%
\pgfpathmoveto{\pgfqpoint{1.685520in}{0.387222in}}%
\pgfpathlineto{\pgfqpoint{1.869764in}{0.387222in}}%
\pgfpathlineto{\pgfqpoint{1.869764in}{0.646415in}}%
\pgfpathlineto{\pgfqpoint{1.685520in}{0.646415in}}%
\pgfpathlineto{\pgfqpoint{1.685520in}{0.387222in}}%
\pgfpathclose%
\pgfusepath{stroke,fill}%
\end{pgfscope}%
\begin{pgfscope}%
\pgfpathrectangle{\pgfqpoint{0.709028in}{0.387222in}}{\pgfqpoint{5.269375in}{5.244444in}}%
\pgfusepath{clip}%
\pgfsetbuttcap%
\pgfsetmiterjoin%
\definecolor{currentfill}{rgb}{0.121569,0.466667,0.705882}%
\pgfsetfillcolor{currentfill}%
\pgfsetfillopacity{0.500000}%
\pgfsetlinewidth{1.003750pt}%
\definecolor{currentstroke}{rgb}{0.000000,0.000000,0.000000}%
\pgfsetstrokecolor{currentstroke}%
\pgfsetdash{}{0pt}%
\pgfpathmoveto{\pgfqpoint{1.869764in}{0.387222in}}%
\pgfpathlineto{\pgfqpoint{2.054008in}{0.387222in}}%
\pgfpathlineto{\pgfqpoint{2.054008in}{0.562352in}}%
\pgfpathlineto{\pgfqpoint{1.869764in}{0.562352in}}%
\pgfpathlineto{\pgfqpoint{1.869764in}{0.387222in}}%
\pgfpathclose%
\pgfusepath{stroke,fill}%
\end{pgfscope}%
\begin{pgfscope}%
\pgfpathrectangle{\pgfqpoint{0.709028in}{0.387222in}}{\pgfqpoint{5.269375in}{5.244444in}}%
\pgfusepath{clip}%
\pgfsetbuttcap%
\pgfsetmiterjoin%
\definecolor{currentfill}{rgb}{0.121569,0.466667,0.705882}%
\pgfsetfillcolor{currentfill}%
\pgfsetfillopacity{0.500000}%
\pgfsetlinewidth{1.003750pt}%
\definecolor{currentstroke}{rgb}{0.000000,0.000000,0.000000}%
\pgfsetstrokecolor{currentstroke}%
\pgfsetdash{}{0pt}%
\pgfpathmoveto{\pgfqpoint{2.054008in}{0.387222in}}%
\pgfpathlineto{\pgfqpoint{2.238252in}{0.387222in}}%
\pgfpathlineto{\pgfqpoint{2.238252in}{0.765503in}}%
\pgfpathlineto{\pgfqpoint{2.054008in}{0.765503in}}%
\pgfpathlineto{\pgfqpoint{2.054008in}{0.387222in}}%
\pgfpathclose%
\pgfusepath{stroke,fill}%
\end{pgfscope}%
\begin{pgfscope}%
\pgfpathrectangle{\pgfqpoint{0.709028in}{0.387222in}}{\pgfqpoint{5.269375in}{5.244444in}}%
\pgfusepath{clip}%
\pgfsetbuttcap%
\pgfsetmiterjoin%
\definecolor{currentfill}{rgb}{0.121569,0.466667,0.705882}%
\pgfsetfillcolor{currentfill}%
\pgfsetfillopacity{0.500000}%
\pgfsetlinewidth{1.003750pt}%
\definecolor{currentstroke}{rgb}{0.000000,0.000000,0.000000}%
\pgfsetstrokecolor{currentstroke}%
\pgfsetdash{}{0pt}%
\pgfpathmoveto{\pgfqpoint{2.238252in}{0.387222in}}%
\pgfpathlineto{\pgfqpoint{2.422496in}{0.387222in}}%
\pgfpathlineto{\pgfqpoint{2.422496in}{0.653420in}}%
\pgfpathlineto{\pgfqpoint{2.238252in}{0.653420in}}%
\pgfpathlineto{\pgfqpoint{2.238252in}{0.387222in}}%
\pgfpathclose%
\pgfusepath{stroke,fill}%
\end{pgfscope}%
\begin{pgfscope}%
\pgfpathrectangle{\pgfqpoint{0.709028in}{0.387222in}}{\pgfqpoint{5.269375in}{5.244444in}}%
\pgfusepath{clip}%
\pgfsetbuttcap%
\pgfsetmiterjoin%
\definecolor{currentfill}{rgb}{0.121569,0.466667,0.705882}%
\pgfsetfillcolor{currentfill}%
\pgfsetfillopacity{0.500000}%
\pgfsetlinewidth{1.003750pt}%
\definecolor{currentstroke}{rgb}{0.000000,0.000000,0.000000}%
\pgfsetstrokecolor{currentstroke}%
\pgfsetdash{}{0pt}%
\pgfpathmoveto{\pgfqpoint{2.422496in}{0.387222in}}%
\pgfpathlineto{\pgfqpoint{2.606740in}{0.387222in}}%
\pgfpathlineto{\pgfqpoint{2.606740in}{0.947638in}}%
\pgfpathlineto{\pgfqpoint{2.422496in}{0.947638in}}%
\pgfpathlineto{\pgfqpoint{2.422496in}{0.387222in}}%
\pgfpathclose%
\pgfusepath{stroke,fill}%
\end{pgfscope}%
\begin{pgfscope}%
\pgfpathrectangle{\pgfqpoint{0.709028in}{0.387222in}}{\pgfqpoint{5.269375in}{5.244444in}}%
\pgfusepath{clip}%
\pgfsetbuttcap%
\pgfsetmiterjoin%
\definecolor{currentfill}{rgb}{0.121569,0.466667,0.705882}%
\pgfsetfillcolor{currentfill}%
\pgfsetfillopacity{0.500000}%
\pgfsetlinewidth{1.003750pt}%
\definecolor{currentstroke}{rgb}{0.000000,0.000000,0.000000}%
\pgfsetstrokecolor{currentstroke}%
\pgfsetdash{}{0pt}%
\pgfpathmoveto{\pgfqpoint{2.606740in}{0.387222in}}%
\pgfpathlineto{\pgfqpoint{2.790984in}{0.387222in}}%
\pgfpathlineto{\pgfqpoint{2.790984in}{1.129774in}}%
\pgfpathlineto{\pgfqpoint{2.606740in}{1.129774in}}%
\pgfpathlineto{\pgfqpoint{2.606740in}{0.387222in}}%
\pgfpathclose%
\pgfusepath{stroke,fill}%
\end{pgfscope}%
\begin{pgfscope}%
\pgfpathrectangle{\pgfqpoint{0.709028in}{0.387222in}}{\pgfqpoint{5.269375in}{5.244444in}}%
\pgfusepath{clip}%
\pgfsetbuttcap%
\pgfsetmiterjoin%
\definecolor{currentfill}{rgb}{0.121569,0.466667,0.705882}%
\pgfsetfillcolor{currentfill}%
\pgfsetfillopacity{0.500000}%
\pgfsetlinewidth{1.003750pt}%
\definecolor{currentstroke}{rgb}{0.000000,0.000000,0.000000}%
\pgfsetstrokecolor{currentstroke}%
\pgfsetdash{}{0pt}%
\pgfpathmoveto{\pgfqpoint{2.790984in}{0.387222in}}%
\pgfpathlineto{\pgfqpoint{2.975228in}{0.387222in}}%
\pgfpathlineto{\pgfqpoint{2.975228in}{2.145528in}}%
\pgfpathlineto{\pgfqpoint{2.790984in}{2.145528in}}%
\pgfpathlineto{\pgfqpoint{2.790984in}{0.387222in}}%
\pgfpathclose%
\pgfusepath{stroke,fill}%
\end{pgfscope}%
\begin{pgfscope}%
\pgfpathrectangle{\pgfqpoint{0.709028in}{0.387222in}}{\pgfqpoint{5.269375in}{5.244444in}}%
\pgfusepath{clip}%
\pgfsetbuttcap%
\pgfsetmiterjoin%
\definecolor{currentfill}{rgb}{0.121569,0.466667,0.705882}%
\pgfsetfillcolor{currentfill}%
\pgfsetfillopacity{0.500000}%
\pgfsetlinewidth{1.003750pt}%
\definecolor{currentstroke}{rgb}{0.000000,0.000000,0.000000}%
\pgfsetstrokecolor{currentstroke}%
\pgfsetdash{}{0pt}%
\pgfpathmoveto{\pgfqpoint{2.975228in}{0.387222in}}%
\pgfpathlineto{\pgfqpoint{3.159471in}{0.387222in}}%
\pgfpathlineto{\pgfqpoint{3.159471in}{2.404720in}}%
\pgfpathlineto{\pgfqpoint{2.975228in}{2.404720in}}%
\pgfpathlineto{\pgfqpoint{2.975228in}{0.387222in}}%
\pgfpathclose%
\pgfusepath{stroke,fill}%
\end{pgfscope}%
\begin{pgfscope}%
\pgfpathrectangle{\pgfqpoint{0.709028in}{0.387222in}}{\pgfqpoint{5.269375in}{5.244444in}}%
\pgfusepath{clip}%
\pgfsetbuttcap%
\pgfsetmiterjoin%
\definecolor{currentfill}{rgb}{0.121569,0.466667,0.705882}%
\pgfsetfillcolor{currentfill}%
\pgfsetfillopacity{0.500000}%
\pgfsetlinewidth{1.003750pt}%
\definecolor{currentstroke}{rgb}{0.000000,0.000000,0.000000}%
\pgfsetstrokecolor{currentstroke}%
\pgfsetdash{}{0pt}%
\pgfpathmoveto{\pgfqpoint{3.159471in}{0.387222in}}%
\pgfpathlineto{\pgfqpoint{3.343715in}{0.387222in}}%
\pgfpathlineto{\pgfqpoint{3.343715in}{3.133261in}}%
\pgfpathlineto{\pgfqpoint{3.159471in}{3.133261in}}%
\pgfpathlineto{\pgfqpoint{3.159471in}{0.387222in}}%
\pgfpathclose%
\pgfusepath{stroke,fill}%
\end{pgfscope}%
\begin{pgfscope}%
\pgfpathrectangle{\pgfqpoint{0.709028in}{0.387222in}}{\pgfqpoint{5.269375in}{5.244444in}}%
\pgfusepath{clip}%
\pgfsetbuttcap%
\pgfsetmiterjoin%
\definecolor{currentfill}{rgb}{0.121569,0.466667,0.705882}%
\pgfsetfillcolor{currentfill}%
\pgfsetfillopacity{0.500000}%
\pgfsetlinewidth{1.003750pt}%
\definecolor{currentstroke}{rgb}{0.000000,0.000000,0.000000}%
\pgfsetstrokecolor{currentstroke}%
\pgfsetdash{}{0pt}%
\pgfpathmoveto{\pgfqpoint{3.343715in}{0.387222in}}%
\pgfpathlineto{\pgfqpoint{3.527959in}{0.387222in}}%
\pgfpathlineto{\pgfqpoint{3.527959in}{3.777740in}}%
\pgfpathlineto{\pgfqpoint{3.343715in}{3.777740in}}%
\pgfpathlineto{\pgfqpoint{3.343715in}{0.387222in}}%
\pgfpathclose%
\pgfusepath{stroke,fill}%
\end{pgfscope}%
\begin{pgfscope}%
\pgfpathrectangle{\pgfqpoint{0.709028in}{0.387222in}}{\pgfqpoint{5.269375in}{5.244444in}}%
\pgfusepath{clip}%
\pgfsetbuttcap%
\pgfsetmiterjoin%
\definecolor{currentfill}{rgb}{0.121569,0.466667,0.705882}%
\pgfsetfillcolor{currentfill}%
\pgfsetfillopacity{0.500000}%
\pgfsetlinewidth{1.003750pt}%
\definecolor{currentstroke}{rgb}{0.000000,0.000000,0.000000}%
\pgfsetstrokecolor{currentstroke}%
\pgfsetdash{}{0pt}%
\pgfpathmoveto{\pgfqpoint{3.527959in}{0.387222in}}%
\pgfpathlineto{\pgfqpoint{3.712203in}{0.387222in}}%
\pgfpathlineto{\pgfqpoint{3.712203in}{3.392454in}}%
\pgfpathlineto{\pgfqpoint{3.527959in}{3.392454in}}%
\pgfpathlineto{\pgfqpoint{3.527959in}{0.387222in}}%
\pgfpathclose%
\pgfusepath{stroke,fill}%
\end{pgfscope}%
\begin{pgfscope}%
\pgfpathrectangle{\pgfqpoint{0.709028in}{0.387222in}}{\pgfqpoint{5.269375in}{5.244444in}}%
\pgfusepath{clip}%
\pgfsetbuttcap%
\pgfsetmiterjoin%
\definecolor{currentfill}{rgb}{0.121569,0.466667,0.705882}%
\pgfsetfillcolor{currentfill}%
\pgfsetfillopacity{0.500000}%
\pgfsetlinewidth{1.003750pt}%
\definecolor{currentstroke}{rgb}{0.000000,0.000000,0.000000}%
\pgfsetstrokecolor{currentstroke}%
\pgfsetdash{}{0pt}%
\pgfpathmoveto{\pgfqpoint{3.712203in}{0.387222in}}%
\pgfpathlineto{\pgfqpoint{3.896447in}{0.387222in}}%
\pgfpathlineto{\pgfqpoint{3.896447in}{5.381931in}}%
\pgfpathlineto{\pgfqpoint{3.712203in}{5.381931in}}%
\pgfpathlineto{\pgfqpoint{3.712203in}{0.387222in}}%
\pgfpathclose%
\pgfusepath{stroke,fill}%
\end{pgfscope}%
\begin{pgfscope}%
\pgfpathrectangle{\pgfqpoint{0.709028in}{0.387222in}}{\pgfqpoint{5.269375in}{5.244444in}}%
\pgfusepath{clip}%
\pgfsetbuttcap%
\pgfsetmiterjoin%
\definecolor{currentfill}{rgb}{0.121569,0.466667,0.705882}%
\pgfsetfillcolor{currentfill}%
\pgfsetfillopacity{0.500000}%
\pgfsetlinewidth{1.003750pt}%
\definecolor{currentstroke}{rgb}{0.000000,0.000000,0.000000}%
\pgfsetstrokecolor{currentstroke}%
\pgfsetdash{}{0pt}%
\pgfpathmoveto{\pgfqpoint{3.896447in}{0.387222in}}%
\pgfpathlineto{\pgfqpoint{4.080691in}{0.387222in}}%
\pgfpathlineto{\pgfqpoint{4.080691in}{3.273365in}}%
\pgfpathlineto{\pgfqpoint{3.896447in}{3.273365in}}%
\pgfpathlineto{\pgfqpoint{3.896447in}{0.387222in}}%
\pgfpathclose%
\pgfusepath{stroke,fill}%
\end{pgfscope}%
\begin{pgfscope}%
\pgfpathrectangle{\pgfqpoint{0.709028in}{0.387222in}}{\pgfqpoint{5.269375in}{5.244444in}}%
\pgfusepath{clip}%
\pgfsetbuttcap%
\pgfsetmiterjoin%
\definecolor{currentfill}{rgb}{0.121569,0.466667,0.705882}%
\pgfsetfillcolor{currentfill}%
\pgfsetfillopacity{0.500000}%
\pgfsetlinewidth{1.003750pt}%
\definecolor{currentstroke}{rgb}{0.000000,0.000000,0.000000}%
\pgfsetstrokecolor{currentstroke}%
\pgfsetdash{}{0pt}%
\pgfpathmoveto{\pgfqpoint{4.080691in}{0.387222in}}%
\pgfpathlineto{\pgfqpoint{4.264935in}{0.387222in}}%
\pgfpathlineto{\pgfqpoint{4.264935in}{5.143754in}}%
\pgfpathlineto{\pgfqpoint{4.080691in}{5.143754in}}%
\pgfpathlineto{\pgfqpoint{4.080691in}{0.387222in}}%
\pgfpathclose%
\pgfusepath{stroke,fill}%
\end{pgfscope}%
\begin{pgfscope}%
\pgfpathrectangle{\pgfqpoint{0.709028in}{0.387222in}}{\pgfqpoint{5.269375in}{5.244444in}}%
\pgfusepath{clip}%
\pgfsetbuttcap%
\pgfsetmiterjoin%
\definecolor{currentfill}{rgb}{0.121569,0.466667,0.705882}%
\pgfsetfillcolor{currentfill}%
\pgfsetfillopacity{0.500000}%
\pgfsetlinewidth{1.003750pt}%
\definecolor{currentstroke}{rgb}{0.000000,0.000000,0.000000}%
\pgfsetstrokecolor{currentstroke}%
\pgfsetdash{}{0pt}%
\pgfpathmoveto{\pgfqpoint{4.264935in}{0.387222in}}%
\pgfpathlineto{\pgfqpoint{4.449179in}{0.387222in}}%
\pgfpathlineto{\pgfqpoint{4.449179in}{4.380187in}}%
\pgfpathlineto{\pgfqpoint{4.264935in}{4.380187in}}%
\pgfpathlineto{\pgfqpoint{4.264935in}{0.387222in}}%
\pgfpathclose%
\pgfusepath{stroke,fill}%
\end{pgfscope}%
\begin{pgfscope}%
\pgfpathrectangle{\pgfqpoint{0.709028in}{0.387222in}}{\pgfqpoint{5.269375in}{5.244444in}}%
\pgfusepath{clip}%
\pgfsetbuttcap%
\pgfsetmiterjoin%
\definecolor{currentfill}{rgb}{0.121569,0.466667,0.705882}%
\pgfsetfillcolor{currentfill}%
\pgfsetfillopacity{0.500000}%
\pgfsetlinewidth{1.003750pt}%
\definecolor{currentstroke}{rgb}{0.000000,0.000000,0.000000}%
\pgfsetstrokecolor{currentstroke}%
\pgfsetdash{}{0pt}%
\pgfpathmoveto{\pgfqpoint{4.449179in}{0.387222in}}%
\pgfpathlineto{\pgfqpoint{4.633422in}{0.387222in}}%
\pgfpathlineto{\pgfqpoint{4.633422in}{3.700683in}}%
\pgfpathlineto{\pgfqpoint{4.449179in}{3.700683in}}%
\pgfpathlineto{\pgfqpoint{4.449179in}{0.387222in}}%
\pgfpathclose%
\pgfusepath{stroke,fill}%
\end{pgfscope}%
\begin{pgfscope}%
\pgfpathrectangle{\pgfqpoint{0.709028in}{0.387222in}}{\pgfqpoint{5.269375in}{5.244444in}}%
\pgfusepath{clip}%
\pgfsetbuttcap%
\pgfsetmiterjoin%
\definecolor{currentfill}{rgb}{0.121569,0.466667,0.705882}%
\pgfsetfillcolor{currentfill}%
\pgfsetfillopacity{0.500000}%
\pgfsetlinewidth{1.003750pt}%
\definecolor{currentstroke}{rgb}{0.000000,0.000000,0.000000}%
\pgfsetstrokecolor{currentstroke}%
\pgfsetdash{}{0pt}%
\pgfpathmoveto{\pgfqpoint{4.633422in}{0.387222in}}%
\pgfpathlineto{\pgfqpoint{4.817666in}{0.387222in}}%
\pgfpathlineto{\pgfqpoint{4.817666in}{3.819771in}}%
\pgfpathlineto{\pgfqpoint{4.633422in}{3.819771in}}%
\pgfpathlineto{\pgfqpoint{4.633422in}{0.387222in}}%
\pgfpathclose%
\pgfusepath{stroke,fill}%
\end{pgfscope}%
\begin{pgfscope}%
\pgfpathrectangle{\pgfqpoint{0.709028in}{0.387222in}}{\pgfqpoint{5.269375in}{5.244444in}}%
\pgfusepath{clip}%
\pgfsetbuttcap%
\pgfsetmiterjoin%
\definecolor{currentfill}{rgb}{0.121569,0.466667,0.705882}%
\pgfsetfillcolor{currentfill}%
\pgfsetfillopacity{0.500000}%
\pgfsetlinewidth{1.003750pt}%
\definecolor{currentstroke}{rgb}{0.000000,0.000000,0.000000}%
\pgfsetstrokecolor{currentstroke}%
\pgfsetdash{}{0pt}%
\pgfpathmoveto{\pgfqpoint{4.817666in}{0.387222in}}%
\pgfpathlineto{\pgfqpoint{5.001910in}{0.387222in}}%
\pgfpathlineto{\pgfqpoint{5.001910in}{3.049199in}}%
\pgfpathlineto{\pgfqpoint{4.817666in}{3.049199in}}%
\pgfpathlineto{\pgfqpoint{4.817666in}{0.387222in}}%
\pgfpathclose%
\pgfusepath{stroke,fill}%
\end{pgfscope}%
\begin{pgfscope}%
\pgfpathrectangle{\pgfqpoint{0.709028in}{0.387222in}}{\pgfqpoint{5.269375in}{5.244444in}}%
\pgfusepath{clip}%
\pgfsetbuttcap%
\pgfsetmiterjoin%
\definecolor{currentfill}{rgb}{0.121569,0.466667,0.705882}%
\pgfsetfillcolor{currentfill}%
\pgfsetfillopacity{0.500000}%
\pgfsetlinewidth{1.003750pt}%
\definecolor{currentstroke}{rgb}{0.000000,0.000000,0.000000}%
\pgfsetstrokecolor{currentstroke}%
\pgfsetdash{}{0pt}%
\pgfpathmoveto{\pgfqpoint{5.001910in}{0.387222in}}%
\pgfpathlineto{\pgfqpoint{5.186154in}{0.387222in}}%
\pgfpathlineto{\pgfqpoint{5.186154in}{2.551830in}}%
\pgfpathlineto{\pgfqpoint{5.001910in}{2.551830in}}%
\pgfpathlineto{\pgfqpoint{5.001910in}{0.387222in}}%
\pgfpathclose%
\pgfusepath{stroke,fill}%
\end{pgfscope}%
\begin{pgfscope}%
\pgfpathrectangle{\pgfqpoint{0.709028in}{0.387222in}}{\pgfqpoint{5.269375in}{5.244444in}}%
\pgfusepath{clip}%
\pgfsetbuttcap%
\pgfsetmiterjoin%
\definecolor{currentfill}{rgb}{0.121569,0.466667,0.705882}%
\pgfsetfillcolor{currentfill}%
\pgfsetfillopacity{0.500000}%
\pgfsetlinewidth{1.003750pt}%
\definecolor{currentstroke}{rgb}{0.000000,0.000000,0.000000}%
\pgfsetstrokecolor{currentstroke}%
\pgfsetdash{}{0pt}%
\pgfpathmoveto{\pgfqpoint{5.186154in}{0.387222in}}%
\pgfpathlineto{\pgfqpoint{5.370398in}{0.387222in}}%
\pgfpathlineto{\pgfqpoint{5.370398in}{1.564096in}}%
\pgfpathlineto{\pgfqpoint{5.186154in}{1.564096in}}%
\pgfpathlineto{\pgfqpoint{5.186154in}{0.387222in}}%
\pgfpathclose%
\pgfusepath{stroke,fill}%
\end{pgfscope}%
\begin{pgfscope}%
\pgfpathrectangle{\pgfqpoint{0.709028in}{0.387222in}}{\pgfqpoint{5.269375in}{5.244444in}}%
\pgfusepath{clip}%
\pgfsetbuttcap%
\pgfsetmiterjoin%
\definecolor{currentfill}{rgb}{0.121569,0.466667,0.705882}%
\pgfsetfillcolor{currentfill}%
\pgfsetfillopacity{0.500000}%
\pgfsetlinewidth{1.003750pt}%
\definecolor{currentstroke}{rgb}{0.000000,0.000000,0.000000}%
\pgfsetstrokecolor{currentstroke}%
\pgfsetdash{}{0pt}%
\pgfpathmoveto{\pgfqpoint{5.370398in}{0.387222in}}%
\pgfpathlineto{\pgfqpoint{5.554642in}{0.387222in}}%
\pgfpathlineto{\pgfqpoint{5.554642in}{0.772508in}}%
\pgfpathlineto{\pgfqpoint{5.370398in}{0.772508in}}%
\pgfpathlineto{\pgfqpoint{5.370398in}{0.387222in}}%
\pgfpathclose%
\pgfusepath{stroke,fill}%
\end{pgfscope}%
\begin{pgfscope}%
\pgfpathrectangle{\pgfqpoint{0.709028in}{0.387222in}}{\pgfqpoint{5.269375in}{5.244444in}}%
\pgfusepath{clip}%
\pgfsetbuttcap%
\pgfsetmiterjoin%
\definecolor{currentfill}{rgb}{0.121569,0.466667,0.705882}%
\pgfsetfillcolor{currentfill}%
\pgfsetfillopacity{0.500000}%
\pgfsetlinewidth{1.003750pt}%
\definecolor{currentstroke}{rgb}{0.000000,0.000000,0.000000}%
\pgfsetstrokecolor{currentstroke}%
\pgfsetdash{}{0pt}%
\pgfpathmoveto{\pgfqpoint{5.554642in}{0.387222in}}%
\pgfpathlineto{\pgfqpoint{5.738886in}{0.387222in}}%
\pgfpathlineto{\pgfqpoint{5.738886in}{0.653420in}}%
\pgfpathlineto{\pgfqpoint{5.554642in}{0.653420in}}%
\pgfpathlineto{\pgfqpoint{5.554642in}{0.387222in}}%
\pgfpathclose%
\pgfusepath{stroke,fill}%
\end{pgfscope}%
\begin{pgfscope}%
\pgfsetbuttcap%
\pgfsetroundjoin%
\definecolor{currentfill}{rgb}{0.000000,0.000000,0.000000}%
\pgfsetfillcolor{currentfill}%
\pgfsetlinewidth{0.803000pt}%
\definecolor{currentstroke}{rgb}{0.000000,0.000000,0.000000}%
\pgfsetstrokecolor{currentstroke}%
\pgfsetdash{}{0pt}%
\pgfsys@defobject{currentmarker}{\pgfqpoint{0.000000in}{-0.048611in}}{\pgfqpoint{0.000000in}{0.000000in}}{%
\pgfpathmoveto{\pgfqpoint{0.000000in}{0.000000in}}%
\pgfpathlineto{\pgfqpoint{0.000000in}{-0.048611in}}%
\pgfusepath{stroke,fill}%
}%
\begin{pgfscope}%
\pgfsys@transformshift{0.948545in}{0.387222in}%
\pgfsys@useobject{currentmarker}{}%
\end{pgfscope}%
\end{pgfscope}%
\begin{pgfscope}%
\definecolor{textcolor}{rgb}{0.000000,0.000000,0.000000}%
\pgfsetstrokecolor{textcolor}%
\pgfsetfillcolor{textcolor}%
\pgftext[x=0.948545in,y=0.290000in,,top]{\color{textcolor}{\sffamily\fontsize{10.000000}{12.000000}\selectfont\catcode`\^=\active\def^{\ifmmode\sp\else\^{}\fi}\catcode`\%=\active\def%{\%}0}}%
\end{pgfscope}%
\begin{pgfscope}%
\pgfsetbuttcap%
\pgfsetroundjoin%
\definecolor{currentfill}{rgb}{0.000000,0.000000,0.000000}%
\pgfsetfillcolor{currentfill}%
\pgfsetlinewidth{0.803000pt}%
\definecolor{currentstroke}{rgb}{0.000000,0.000000,0.000000}%
\pgfsetstrokecolor{currentstroke}%
\pgfsetdash{}{0pt}%
\pgfsys@defobject{currentmarker}{\pgfqpoint{0.000000in}{-0.048611in}}{\pgfqpoint{0.000000in}{0.000000in}}{%
\pgfpathmoveto{\pgfqpoint{0.000000in}{0.000000in}}%
\pgfpathlineto{\pgfqpoint{0.000000in}{-0.048611in}}%
\pgfusepath{stroke,fill}%
}%
\begin{pgfscope}%
\pgfsys@transformshift{1.869764in}{0.387222in}%
\pgfsys@useobject{currentmarker}{}%
\end{pgfscope}%
\end{pgfscope}%
\begin{pgfscope}%
\definecolor{textcolor}{rgb}{0.000000,0.000000,0.000000}%
\pgfsetstrokecolor{textcolor}%
\pgfsetfillcolor{textcolor}%
\pgftext[x=1.869764in,y=0.290000in,,top]{\color{textcolor}{\sffamily\fontsize{10.000000}{12.000000}\selectfont\catcode`\^=\active\def^{\ifmmode\sp\else\^{}\fi}\catcode`\%=\active\def%{\%}200}}%
\end{pgfscope}%
\begin{pgfscope}%
\pgfsetbuttcap%
\pgfsetroundjoin%
\definecolor{currentfill}{rgb}{0.000000,0.000000,0.000000}%
\pgfsetfillcolor{currentfill}%
\pgfsetlinewidth{0.803000pt}%
\definecolor{currentstroke}{rgb}{0.000000,0.000000,0.000000}%
\pgfsetstrokecolor{currentstroke}%
\pgfsetdash{}{0pt}%
\pgfsys@defobject{currentmarker}{\pgfqpoint{0.000000in}{-0.048611in}}{\pgfqpoint{0.000000in}{0.000000in}}{%
\pgfpathmoveto{\pgfqpoint{0.000000in}{0.000000in}}%
\pgfpathlineto{\pgfqpoint{0.000000in}{-0.048611in}}%
\pgfusepath{stroke,fill}%
}%
\begin{pgfscope}%
\pgfsys@transformshift{2.790984in}{0.387222in}%
\pgfsys@useobject{currentmarker}{}%
\end{pgfscope}%
\end{pgfscope}%
\begin{pgfscope}%
\definecolor{textcolor}{rgb}{0.000000,0.000000,0.000000}%
\pgfsetstrokecolor{textcolor}%
\pgfsetfillcolor{textcolor}%
\pgftext[x=2.790984in,y=0.290000in,,top]{\color{textcolor}{\sffamily\fontsize{10.000000}{12.000000}\selectfont\catcode`\^=\active\def^{\ifmmode\sp\else\^{}\fi}\catcode`\%=\active\def%{\%}400}}%
\end{pgfscope}%
\begin{pgfscope}%
\pgfsetbuttcap%
\pgfsetroundjoin%
\definecolor{currentfill}{rgb}{0.000000,0.000000,0.000000}%
\pgfsetfillcolor{currentfill}%
\pgfsetlinewidth{0.803000pt}%
\definecolor{currentstroke}{rgb}{0.000000,0.000000,0.000000}%
\pgfsetstrokecolor{currentstroke}%
\pgfsetdash{}{0pt}%
\pgfsys@defobject{currentmarker}{\pgfqpoint{0.000000in}{-0.048611in}}{\pgfqpoint{0.000000in}{0.000000in}}{%
\pgfpathmoveto{\pgfqpoint{0.000000in}{0.000000in}}%
\pgfpathlineto{\pgfqpoint{0.000000in}{-0.048611in}}%
\pgfusepath{stroke,fill}%
}%
\begin{pgfscope}%
\pgfsys@transformshift{3.712203in}{0.387222in}%
\pgfsys@useobject{currentmarker}{}%
\end{pgfscope}%
\end{pgfscope}%
\begin{pgfscope}%
\definecolor{textcolor}{rgb}{0.000000,0.000000,0.000000}%
\pgfsetstrokecolor{textcolor}%
\pgfsetfillcolor{textcolor}%
\pgftext[x=3.712203in,y=0.290000in,,top]{\color{textcolor}{\sffamily\fontsize{10.000000}{12.000000}\selectfont\catcode`\^=\active\def^{\ifmmode\sp\else\^{}\fi}\catcode`\%=\active\def%{\%}600}}%
\end{pgfscope}%
\begin{pgfscope}%
\pgfsetbuttcap%
\pgfsetroundjoin%
\definecolor{currentfill}{rgb}{0.000000,0.000000,0.000000}%
\pgfsetfillcolor{currentfill}%
\pgfsetlinewidth{0.803000pt}%
\definecolor{currentstroke}{rgb}{0.000000,0.000000,0.000000}%
\pgfsetstrokecolor{currentstroke}%
\pgfsetdash{}{0pt}%
\pgfsys@defobject{currentmarker}{\pgfqpoint{0.000000in}{-0.048611in}}{\pgfqpoint{0.000000in}{0.000000in}}{%
\pgfpathmoveto{\pgfqpoint{0.000000in}{0.000000in}}%
\pgfpathlineto{\pgfqpoint{0.000000in}{-0.048611in}}%
\pgfusepath{stroke,fill}%
}%
\begin{pgfscope}%
\pgfsys@transformshift{4.633422in}{0.387222in}%
\pgfsys@useobject{currentmarker}{}%
\end{pgfscope}%
\end{pgfscope}%
\begin{pgfscope}%
\definecolor{textcolor}{rgb}{0.000000,0.000000,0.000000}%
\pgfsetstrokecolor{textcolor}%
\pgfsetfillcolor{textcolor}%
\pgftext[x=4.633422in,y=0.290000in,,top]{\color{textcolor}{\sffamily\fontsize{10.000000}{12.000000}\selectfont\catcode`\^=\active\def^{\ifmmode\sp\else\^{}\fi}\catcode`\%=\active\def%{\%}800}}%
\end{pgfscope}%
\begin{pgfscope}%
\pgfsetbuttcap%
\pgfsetroundjoin%
\definecolor{currentfill}{rgb}{0.000000,0.000000,0.000000}%
\pgfsetfillcolor{currentfill}%
\pgfsetlinewidth{0.803000pt}%
\definecolor{currentstroke}{rgb}{0.000000,0.000000,0.000000}%
\pgfsetstrokecolor{currentstroke}%
\pgfsetdash{}{0pt}%
\pgfsys@defobject{currentmarker}{\pgfqpoint{0.000000in}{-0.048611in}}{\pgfqpoint{0.000000in}{0.000000in}}{%
\pgfpathmoveto{\pgfqpoint{0.000000in}{0.000000in}}%
\pgfpathlineto{\pgfqpoint{0.000000in}{-0.048611in}}%
\pgfusepath{stroke,fill}%
}%
\begin{pgfscope}%
\pgfsys@transformshift{5.554642in}{0.387222in}%
\pgfsys@useobject{currentmarker}{}%
\end{pgfscope}%
\end{pgfscope}%
\begin{pgfscope}%
\definecolor{textcolor}{rgb}{0.000000,0.000000,0.000000}%
\pgfsetstrokecolor{textcolor}%
\pgfsetfillcolor{textcolor}%
\pgftext[x=5.554642in,y=0.290000in,,top]{\color{textcolor}{\sffamily\fontsize{10.000000}{12.000000}\selectfont\catcode`\^=\active\def^{\ifmmode\sp\else\^{}\fi}\catcode`\%=\active\def%{\%}1000}}%
\end{pgfscope}%
\begin{pgfscope}%
\pgfsetbuttcap%
\pgfsetroundjoin%
\definecolor{currentfill}{rgb}{0.000000,0.000000,0.000000}%
\pgfsetfillcolor{currentfill}%
\pgfsetlinewidth{0.803000pt}%
\definecolor{currentstroke}{rgb}{0.000000,0.000000,0.000000}%
\pgfsetstrokecolor{currentstroke}%
\pgfsetdash{}{0pt}%
\pgfsys@defobject{currentmarker}{\pgfqpoint{-0.048611in}{0.000000in}}{\pgfqpoint{-0.000000in}{0.000000in}}{%
\pgfpathmoveto{\pgfqpoint{-0.000000in}{0.000000in}}%
\pgfpathlineto{\pgfqpoint{-0.048611in}{0.000000in}}%
\pgfusepath{stroke,fill}%
}%
\begin{pgfscope}%
\pgfsys@transformshift{0.709028in}{0.387222in}%
\pgfsys@useobject{currentmarker}{}%
\end{pgfscope}%
\end{pgfscope}%
\begin{pgfscope}%
\definecolor{textcolor}{rgb}{0.000000,0.000000,0.000000}%
\pgfsetstrokecolor{textcolor}%
\pgfsetfillcolor{textcolor}%
\pgftext[x=0.523440in, y=0.334461in, left, base]{\color{textcolor}{\sffamily\fontsize{10.000000}{12.000000}\selectfont\catcode`\^=\active\def^{\ifmmode\sp\else\^{}\fi}\catcode`\%=\active\def%{\%}0}}%
\end{pgfscope}%
\begin{pgfscope}%
\pgfsetbuttcap%
\pgfsetroundjoin%
\definecolor{currentfill}{rgb}{0.000000,0.000000,0.000000}%
\pgfsetfillcolor{currentfill}%
\pgfsetlinewidth{0.803000pt}%
\definecolor{currentstroke}{rgb}{0.000000,0.000000,0.000000}%
\pgfsetstrokecolor{currentstroke}%
\pgfsetdash{}{0pt}%
\pgfsys@defobject{currentmarker}{\pgfqpoint{-0.048611in}{0.000000in}}{\pgfqpoint{-0.000000in}{0.000000in}}{%
\pgfpathmoveto{\pgfqpoint{-0.000000in}{0.000000in}}%
\pgfpathlineto{\pgfqpoint{-0.048611in}{0.000000in}}%
\pgfusepath{stroke,fill}%
}%
\begin{pgfscope}%
\pgfsys@transformshift{0.709028in}{1.087742in}%
\pgfsys@useobject{currentmarker}{}%
\end{pgfscope}%
\end{pgfscope}%
\begin{pgfscope}%
\definecolor{textcolor}{rgb}{0.000000,0.000000,0.000000}%
\pgfsetstrokecolor{textcolor}%
\pgfsetfillcolor{textcolor}%
\pgftext[x=0.346710in, y=1.034981in, left, base]{\color{textcolor}{\sffamily\fontsize{10.000000}{12.000000}\selectfont\catcode`\^=\active\def^{\ifmmode\sp\else\^{}\fi}\catcode`\%=\active\def%{\%}100}}%
\end{pgfscope}%
\begin{pgfscope}%
\pgfsetbuttcap%
\pgfsetroundjoin%
\definecolor{currentfill}{rgb}{0.000000,0.000000,0.000000}%
\pgfsetfillcolor{currentfill}%
\pgfsetlinewidth{0.803000pt}%
\definecolor{currentstroke}{rgb}{0.000000,0.000000,0.000000}%
\pgfsetstrokecolor{currentstroke}%
\pgfsetdash{}{0pt}%
\pgfsys@defobject{currentmarker}{\pgfqpoint{-0.048611in}{0.000000in}}{\pgfqpoint{-0.000000in}{0.000000in}}{%
\pgfpathmoveto{\pgfqpoint{-0.000000in}{0.000000in}}%
\pgfpathlineto{\pgfqpoint{-0.048611in}{0.000000in}}%
\pgfusepath{stroke,fill}%
}%
\begin{pgfscope}%
\pgfsys@transformshift{0.709028in}{1.788263in}%
\pgfsys@useobject{currentmarker}{}%
\end{pgfscope}%
\end{pgfscope}%
\begin{pgfscope}%
\definecolor{textcolor}{rgb}{0.000000,0.000000,0.000000}%
\pgfsetstrokecolor{textcolor}%
\pgfsetfillcolor{textcolor}%
\pgftext[x=0.346710in, y=1.735501in, left, base]{\color{textcolor}{\sffamily\fontsize{10.000000}{12.000000}\selectfont\catcode`\^=\active\def^{\ifmmode\sp\else\^{}\fi}\catcode`\%=\active\def%{\%}200}}%
\end{pgfscope}%
\begin{pgfscope}%
\pgfsetbuttcap%
\pgfsetroundjoin%
\definecolor{currentfill}{rgb}{0.000000,0.000000,0.000000}%
\pgfsetfillcolor{currentfill}%
\pgfsetlinewidth{0.803000pt}%
\definecolor{currentstroke}{rgb}{0.000000,0.000000,0.000000}%
\pgfsetstrokecolor{currentstroke}%
\pgfsetdash{}{0pt}%
\pgfsys@defobject{currentmarker}{\pgfqpoint{-0.048611in}{0.000000in}}{\pgfqpoint{-0.000000in}{0.000000in}}{%
\pgfpathmoveto{\pgfqpoint{-0.000000in}{0.000000in}}%
\pgfpathlineto{\pgfqpoint{-0.048611in}{0.000000in}}%
\pgfusepath{stroke,fill}%
}%
\begin{pgfscope}%
\pgfsys@transformshift{0.709028in}{2.488783in}%
\pgfsys@useobject{currentmarker}{}%
\end{pgfscope}%
\end{pgfscope}%
\begin{pgfscope}%
\definecolor{textcolor}{rgb}{0.000000,0.000000,0.000000}%
\pgfsetstrokecolor{textcolor}%
\pgfsetfillcolor{textcolor}%
\pgftext[x=0.346710in, y=2.436021in, left, base]{\color{textcolor}{\sffamily\fontsize{10.000000}{12.000000}\selectfont\catcode`\^=\active\def^{\ifmmode\sp\else\^{}\fi}\catcode`\%=\active\def%{\%}300}}%
\end{pgfscope}%
\begin{pgfscope}%
\pgfsetbuttcap%
\pgfsetroundjoin%
\definecolor{currentfill}{rgb}{0.000000,0.000000,0.000000}%
\pgfsetfillcolor{currentfill}%
\pgfsetlinewidth{0.803000pt}%
\definecolor{currentstroke}{rgb}{0.000000,0.000000,0.000000}%
\pgfsetstrokecolor{currentstroke}%
\pgfsetdash{}{0pt}%
\pgfsys@defobject{currentmarker}{\pgfqpoint{-0.048611in}{0.000000in}}{\pgfqpoint{-0.000000in}{0.000000in}}{%
\pgfpathmoveto{\pgfqpoint{-0.000000in}{0.000000in}}%
\pgfpathlineto{\pgfqpoint{-0.048611in}{0.000000in}}%
\pgfusepath{stroke,fill}%
}%
\begin{pgfscope}%
\pgfsys@transformshift{0.709028in}{3.189303in}%
\pgfsys@useobject{currentmarker}{}%
\end{pgfscope}%
\end{pgfscope}%
\begin{pgfscope}%
\definecolor{textcolor}{rgb}{0.000000,0.000000,0.000000}%
\pgfsetstrokecolor{textcolor}%
\pgfsetfillcolor{textcolor}%
\pgftext[x=0.346710in, y=3.136541in, left, base]{\color{textcolor}{\sffamily\fontsize{10.000000}{12.000000}\selectfont\catcode`\^=\active\def^{\ifmmode\sp\else\^{}\fi}\catcode`\%=\active\def%{\%}400}}%
\end{pgfscope}%
\begin{pgfscope}%
\pgfsetbuttcap%
\pgfsetroundjoin%
\definecolor{currentfill}{rgb}{0.000000,0.000000,0.000000}%
\pgfsetfillcolor{currentfill}%
\pgfsetlinewidth{0.803000pt}%
\definecolor{currentstroke}{rgb}{0.000000,0.000000,0.000000}%
\pgfsetstrokecolor{currentstroke}%
\pgfsetdash{}{0pt}%
\pgfsys@defobject{currentmarker}{\pgfqpoint{-0.048611in}{0.000000in}}{\pgfqpoint{-0.000000in}{0.000000in}}{%
\pgfpathmoveto{\pgfqpoint{-0.000000in}{0.000000in}}%
\pgfpathlineto{\pgfqpoint{-0.048611in}{0.000000in}}%
\pgfusepath{stroke,fill}%
}%
\begin{pgfscope}%
\pgfsys@transformshift{0.709028in}{3.889823in}%
\pgfsys@useobject{currentmarker}{}%
\end{pgfscope}%
\end{pgfscope}%
\begin{pgfscope}%
\definecolor{textcolor}{rgb}{0.000000,0.000000,0.000000}%
\pgfsetstrokecolor{textcolor}%
\pgfsetfillcolor{textcolor}%
\pgftext[x=0.346710in, y=3.837062in, left, base]{\color{textcolor}{\sffamily\fontsize{10.000000}{12.000000}\selectfont\catcode`\^=\active\def^{\ifmmode\sp\else\^{}\fi}\catcode`\%=\active\def%{\%}500}}%
\end{pgfscope}%
\begin{pgfscope}%
\pgfsetbuttcap%
\pgfsetroundjoin%
\definecolor{currentfill}{rgb}{0.000000,0.000000,0.000000}%
\pgfsetfillcolor{currentfill}%
\pgfsetlinewidth{0.803000pt}%
\definecolor{currentstroke}{rgb}{0.000000,0.000000,0.000000}%
\pgfsetstrokecolor{currentstroke}%
\pgfsetdash{}{0pt}%
\pgfsys@defobject{currentmarker}{\pgfqpoint{-0.048611in}{0.000000in}}{\pgfqpoint{-0.000000in}{0.000000in}}{%
\pgfpathmoveto{\pgfqpoint{-0.000000in}{0.000000in}}%
\pgfpathlineto{\pgfqpoint{-0.048611in}{0.000000in}}%
\pgfusepath{stroke,fill}%
}%
\begin{pgfscope}%
\pgfsys@transformshift{0.709028in}{4.590343in}%
\pgfsys@useobject{currentmarker}{}%
\end{pgfscope}%
\end{pgfscope}%
\begin{pgfscope}%
\definecolor{textcolor}{rgb}{0.000000,0.000000,0.000000}%
\pgfsetstrokecolor{textcolor}%
\pgfsetfillcolor{textcolor}%
\pgftext[x=0.346710in, y=4.537582in, left, base]{\color{textcolor}{\sffamily\fontsize{10.000000}{12.000000}\selectfont\catcode`\^=\active\def^{\ifmmode\sp\else\^{}\fi}\catcode`\%=\active\def%{\%}600}}%
\end{pgfscope}%
\begin{pgfscope}%
\pgfsetbuttcap%
\pgfsetroundjoin%
\definecolor{currentfill}{rgb}{0.000000,0.000000,0.000000}%
\pgfsetfillcolor{currentfill}%
\pgfsetlinewidth{0.803000pt}%
\definecolor{currentstroke}{rgb}{0.000000,0.000000,0.000000}%
\pgfsetstrokecolor{currentstroke}%
\pgfsetdash{}{0pt}%
\pgfsys@defobject{currentmarker}{\pgfqpoint{-0.048611in}{0.000000in}}{\pgfqpoint{-0.000000in}{0.000000in}}{%
\pgfpathmoveto{\pgfqpoint{-0.000000in}{0.000000in}}%
\pgfpathlineto{\pgfqpoint{-0.048611in}{0.000000in}}%
\pgfusepath{stroke,fill}%
}%
\begin{pgfscope}%
\pgfsys@transformshift{0.709028in}{5.290864in}%
\pgfsys@useobject{currentmarker}{}%
\end{pgfscope}%
\end{pgfscope}%
\begin{pgfscope}%
\definecolor{textcolor}{rgb}{0.000000,0.000000,0.000000}%
\pgfsetstrokecolor{textcolor}%
\pgfsetfillcolor{textcolor}%
\pgftext[x=0.346710in, y=5.238102in, left, base]{\color{textcolor}{\sffamily\fontsize{10.000000}{12.000000}\selectfont\catcode`\^=\active\def^{\ifmmode\sp\else\^{}\fi}\catcode`\%=\active\def%{\%}700}}%
\end{pgfscope}%
\begin{pgfscope}%
\definecolor{textcolor}{rgb}{0.000000,0.000000,0.000000}%
\pgfsetstrokecolor{textcolor}%
\pgfsetfillcolor{textcolor}%
\pgftext[x=0.291154in,y=3.009444in,,bottom,rotate=90.000000]{\color{textcolor}{\sffamily\fontsize{10.000000}{12.000000}\selectfont\catcode`\^=\active\def^{\ifmmode\sp\else\^{}\fi}\catcode`\%=\active\def%{\%}Quantidade de redações}}%
\end{pgfscope}%
\begin{pgfscope}%
\pgfpathrectangle{\pgfqpoint{0.709028in}{0.387222in}}{\pgfqpoint{5.269375in}{5.244444in}}%
\pgfusepath{clip}%
\pgfsetrectcap%
\pgfsetroundjoin%
\pgfsetlinewidth{1.505625pt}%
\definecolor{currentstroke}{rgb}{0.121569,0.466667,0.705882}%
\pgfsetstrokecolor{currentstroke}%
\pgfsetdash{}{0pt}%
\pgfpathmoveto{\pgfqpoint{0.948545in}{0.720383in}}%
\pgfpathlineto{\pgfqpoint{0.971691in}{0.716700in}}%
\pgfpathlineto{\pgfqpoint{0.994837in}{0.704937in}}%
\pgfpathlineto{\pgfqpoint{1.017983in}{0.686121in}}%
\pgfpathlineto{\pgfqpoint{1.041130in}{0.661787in}}%
\pgfpathlineto{\pgfqpoint{1.087422in}{0.604085in}}%
\pgfpathlineto{\pgfqpoint{1.133715in}{0.546703in}}%
\pgfpathlineto{\pgfqpoint{1.156861in}{0.521782in}}%
\pgfpathlineto{\pgfqpoint{1.180007in}{0.500500in}}%
\pgfpathlineto{\pgfqpoint{1.203153in}{0.483158in}}%
\pgfpathlineto{\pgfqpoint{1.226299in}{0.469687in}}%
\pgfpathlineto{\pgfqpoint{1.249446in}{0.459752in}}%
\pgfpathlineto{\pgfqpoint{1.272592in}{0.452861in}}%
\pgfpathlineto{\pgfqpoint{1.295738in}{0.448478in}}%
\pgfpathlineto{\pgfqpoint{1.318884in}{0.446114in}}%
\pgfpathlineto{\pgfqpoint{1.342030in}{0.445394in}}%
\pgfpathlineto{\pgfqpoint{1.365177in}{0.446080in}}%
\pgfpathlineto{\pgfqpoint{1.388323in}{0.448084in}}%
\pgfpathlineto{\pgfqpoint{1.411469in}{0.451430in}}%
\pgfpathlineto{\pgfqpoint{1.434615in}{0.456216in}}%
\pgfpathlineto{\pgfqpoint{1.457762in}{0.462558in}}%
\pgfpathlineto{\pgfqpoint{1.480908in}{0.470525in}}%
\pgfpathlineto{\pgfqpoint{1.504054in}{0.480100in}}%
\pgfpathlineto{\pgfqpoint{1.550346in}{0.503355in}}%
\pgfpathlineto{\pgfqpoint{1.642931in}{0.554978in}}%
\pgfpathlineto{\pgfqpoint{1.666078in}{0.566157in}}%
\pgfpathlineto{\pgfqpoint{1.689224in}{0.575962in}}%
\pgfpathlineto{\pgfqpoint{1.712370in}{0.584329in}}%
\pgfpathlineto{\pgfqpoint{1.758662in}{0.597418in}}%
\pgfpathlineto{\pgfqpoint{1.851247in}{0.620015in}}%
\pgfpathlineto{\pgfqpoint{1.897540in}{0.635410in}}%
\pgfpathlineto{\pgfqpoint{1.943832in}{0.654119in}}%
\pgfpathlineto{\pgfqpoint{2.013271in}{0.682617in}}%
\pgfpathlineto{\pgfqpoint{2.059563in}{0.697686in}}%
\pgfpathlineto{\pgfqpoint{2.105856in}{0.708912in}}%
\pgfpathlineto{\pgfqpoint{2.152148in}{0.719114in}}%
\pgfpathlineto{\pgfqpoint{2.175294in}{0.725323in}}%
\pgfpathlineto{\pgfqpoint{2.198440in}{0.733136in}}%
\pgfpathlineto{\pgfqpoint{2.221587in}{0.743118in}}%
\pgfpathlineto{\pgfqpoint{2.244733in}{0.755690in}}%
\pgfpathlineto{\pgfqpoint{2.267879in}{0.771100in}}%
\pgfpathlineto{\pgfqpoint{2.291025in}{0.789416in}}%
\pgfpathlineto{\pgfqpoint{2.314172in}{0.810552in}}%
\pgfpathlineto{\pgfqpoint{2.337318in}{0.834318in}}%
\pgfpathlineto{\pgfqpoint{2.360464in}{0.860492in}}%
\pgfpathlineto{\pgfqpoint{2.383610in}{0.888895in}}%
\pgfpathlineto{\pgfqpoint{2.406756in}{0.919468in}}%
\pgfpathlineto{\pgfqpoint{2.429903in}{0.952322in}}%
\pgfpathlineto{\pgfqpoint{2.453049in}{0.987765in}}%
\pgfpathlineto{\pgfqpoint{2.476195in}{1.026294in}}%
\pgfpathlineto{\pgfqpoint{2.499341in}{1.068552in}}%
\pgfpathlineto{\pgfqpoint{2.522488in}{1.115250in}}%
\pgfpathlineto{\pgfqpoint{2.545634in}{1.167065in}}%
\pgfpathlineto{\pgfqpoint{2.568780in}{1.224525in}}%
\pgfpathlineto{\pgfqpoint{2.591926in}{1.287906in}}%
\pgfpathlineto{\pgfqpoint{2.615072in}{1.357130in}}%
\pgfpathlineto{\pgfqpoint{2.638219in}{1.431724in}}%
\pgfpathlineto{\pgfqpoint{2.684511in}{1.593144in}}%
\pgfpathlineto{\pgfqpoint{2.777096in}{1.925463in}}%
\pgfpathlineto{\pgfqpoint{2.823388in}{2.076810in}}%
\pgfpathlineto{\pgfqpoint{2.869681in}{2.214744in}}%
\pgfpathlineto{\pgfqpoint{3.008558in}{2.610427in}}%
\pgfpathlineto{\pgfqpoint{3.054850in}{2.755871in}}%
\pgfpathlineto{\pgfqpoint{3.193728in}{3.208570in}}%
\pgfpathlineto{\pgfqpoint{3.216874in}{3.276530in}}%
\pgfpathlineto{\pgfqpoint{3.240020in}{3.339642in}}%
\pgfpathlineto{\pgfqpoint{3.263166in}{3.397138in}}%
\pgfpathlineto{\pgfqpoint{3.286313in}{3.448546in}}%
\pgfpathlineto{\pgfqpoint{3.309459in}{3.493831in}}%
\pgfpathlineto{\pgfqpoint{3.332605in}{3.533521in}}%
\pgfpathlineto{\pgfqpoint{3.355751in}{3.568797in}}%
\pgfpathlineto{\pgfqpoint{3.402044in}{3.634079in}}%
\pgfpathlineto{\pgfqpoint{3.425190in}{3.669276in}}%
\pgfpathlineto{\pgfqpoint{3.448336in}{3.709903in}}%
\pgfpathlineto{\pgfqpoint{3.471482in}{3.758397in}}%
\pgfpathlineto{\pgfqpoint{3.494629in}{3.816412in}}%
\pgfpathlineto{\pgfqpoint{3.517775in}{3.884447in}}%
\pgfpathlineto{\pgfqpoint{3.540921in}{3.961574in}}%
\pgfpathlineto{\pgfqpoint{3.633506in}{4.292595in}}%
\pgfpathlineto{\pgfqpoint{3.656652in}{4.355840in}}%
\pgfpathlineto{\pgfqpoint{3.679798in}{4.401298in}}%
\pgfpathlineto{\pgfqpoint{3.702945in}{4.425997in}}%
\pgfpathlineto{\pgfqpoint{3.726091in}{4.429080in}}%
\pgfpathlineto{\pgfqpoint{3.749237in}{4.412059in}}%
\pgfpathlineto{\pgfqpoint{3.772383in}{4.378742in}}%
\pgfpathlineto{\pgfqpoint{3.795529in}{4.334811in}}%
\pgfpathlineto{\pgfqpoint{3.818676in}{4.287128in}}%
\pgfpathlineto{\pgfqpoint{3.841822in}{4.242864in}}%
\pgfpathlineto{\pgfqpoint{3.864968in}{4.208591in}}%
\pgfpathlineto{\pgfqpoint{3.888114in}{4.189460in}}%
\pgfpathlineto{\pgfqpoint{3.911260in}{4.188598in}}%
\pgfpathlineto{\pgfqpoint{3.934407in}{4.206784in}}%
\pgfpathlineto{\pgfqpoint{3.957553in}{4.242440in}}%
\pgfpathlineto{\pgfqpoint{3.980699in}{4.291939in}}%
\pgfpathlineto{\pgfqpoint{4.050138in}{4.468838in}}%
\pgfpathlineto{\pgfqpoint{4.073284in}{4.517964in}}%
\pgfpathlineto{\pgfqpoint{4.096430in}{4.554273in}}%
\pgfpathlineto{\pgfqpoint{4.119576in}{4.575011in}}%
\pgfpathlineto{\pgfqpoint{4.142723in}{4.578958in}}%
\pgfpathlineto{\pgfqpoint{4.165869in}{4.566299in}}%
\pgfpathlineto{\pgfqpoint{4.189015in}{4.538353in}}%
\pgfpathlineto{\pgfqpoint{4.212161in}{4.497234in}}%
\pgfpathlineto{\pgfqpoint{4.235308in}{4.445524in}}%
\pgfpathlineto{\pgfqpoint{4.258454in}{4.385993in}}%
\pgfpathlineto{\pgfqpoint{4.304746in}{4.254305in}}%
\pgfpathlineto{\pgfqpoint{4.351039in}{4.121661in}}%
\pgfpathlineto{\pgfqpoint{4.374185in}{4.059718in}}%
\pgfpathlineto{\pgfqpoint{4.397331in}{4.002406in}}%
\pgfpathlineto{\pgfqpoint{4.420477in}{3.950424in}}%
\pgfpathlineto{\pgfqpoint{4.443623in}{3.903965in}}%
\pgfpathlineto{\pgfqpoint{4.466770in}{3.862718in}}%
\pgfpathlineto{\pgfqpoint{4.489916in}{3.825895in}}%
\pgfpathlineto{\pgfqpoint{4.536208in}{3.760385in}}%
\pgfpathlineto{\pgfqpoint{4.582501in}{3.694819in}}%
\pgfpathlineto{\pgfqpoint{4.605647in}{3.657856in}}%
\pgfpathlineto{\pgfqpoint{4.628793in}{3.616316in}}%
\pgfpathlineto{\pgfqpoint{4.651939in}{3.569370in}}%
\pgfpathlineto{\pgfqpoint{4.675086in}{3.516678in}}%
\pgfpathlineto{\pgfqpoint{4.698232in}{3.458375in}}%
\pgfpathlineto{\pgfqpoint{4.721378in}{3.394989in}}%
\pgfpathlineto{\pgfqpoint{4.767670in}{3.256208in}}%
\pgfpathlineto{\pgfqpoint{4.813963in}{3.107017in}}%
\pgfpathlineto{\pgfqpoint{4.883402in}{2.872530in}}%
\pgfpathlineto{\pgfqpoint{4.929694in}{2.708719in}}%
\pgfpathlineto{\pgfqpoint{4.975986in}{2.535162in}}%
\pgfpathlineto{\pgfqpoint{5.022279in}{2.348787in}}%
\pgfpathlineto{\pgfqpoint{5.068571in}{2.149520in}}%
\pgfpathlineto{\pgfqpoint{5.138010in}{1.835795in}}%
\pgfpathlineto{\pgfqpoint{5.207449in}{1.525816in}}%
\pgfpathlineto{\pgfqpoint{5.253741in}{1.334033in}}%
\pgfpathlineto{\pgfqpoint{5.276887in}{1.245228in}}%
\pgfpathlineto{\pgfqpoint{5.300033in}{1.162127in}}%
\pgfpathlineto{\pgfqpoint{5.323180in}{1.085291in}}%
\pgfpathlineto{\pgfqpoint{5.346326in}{1.015084in}}%
\pgfpathlineto{\pgfqpoint{5.369472in}{0.951641in}}%
\pgfpathlineto{\pgfqpoint{5.392618in}{0.894839in}}%
\pgfpathlineto{\pgfqpoint{5.415765in}{0.844307in}}%
\pgfpathlineto{\pgfqpoint{5.438911in}{0.799457in}}%
\pgfpathlineto{\pgfqpoint{5.462057in}{0.759541in}}%
\pgfpathlineto{\pgfqpoint{5.485203in}{0.723722in}}%
\pgfpathlineto{\pgfqpoint{5.508349in}{0.691155in}}%
\pgfpathlineto{\pgfqpoint{5.531496in}{0.661066in}}%
\pgfpathlineto{\pgfqpoint{5.554642in}{0.632814in}}%
\pgfpathlineto{\pgfqpoint{5.554642in}{0.632814in}}%
\pgfusepath{stroke}%
\end{pgfscope}%
\begin{pgfscope}%
\pgfsetrectcap%
\pgfsetmiterjoin%
\pgfsetlinewidth{0.803000pt}%
\definecolor{currentstroke}{rgb}{0.000000,0.000000,0.000000}%
\pgfsetstrokecolor{currentstroke}%
\pgfsetdash{}{0pt}%
\pgfpathmoveto{\pgfqpoint{0.709028in}{0.387222in}}%
\pgfpathlineto{\pgfqpoint{0.709028in}{5.631667in}}%
\pgfusepath{stroke}%
\end{pgfscope}%
\begin{pgfscope}%
\pgfsetrectcap%
\pgfsetmiterjoin%
\pgfsetlinewidth{0.803000pt}%
\definecolor{currentstroke}{rgb}{0.000000,0.000000,0.000000}%
\pgfsetstrokecolor{currentstroke}%
\pgfsetdash{}{0pt}%
\pgfpathmoveto{\pgfqpoint{5.978403in}{0.387222in}}%
\pgfpathlineto{\pgfqpoint{5.978403in}{5.631667in}}%
\pgfusepath{stroke}%
\end{pgfscope}%
\begin{pgfscope}%
\pgfsetrectcap%
\pgfsetmiterjoin%
\pgfsetlinewidth{0.803000pt}%
\definecolor{currentstroke}{rgb}{0.000000,0.000000,0.000000}%
\pgfsetstrokecolor{currentstroke}%
\pgfsetdash{}{0pt}%
\pgfpathmoveto{\pgfqpoint{0.709028in}{0.387222in}}%
\pgfpathlineto{\pgfqpoint{5.978403in}{0.387222in}}%
\pgfusepath{stroke}%
\end{pgfscope}%
\begin{pgfscope}%
\pgfsetrectcap%
\pgfsetmiterjoin%
\pgfsetlinewidth{0.803000pt}%
\definecolor{currentstroke}{rgb}{0.000000,0.000000,0.000000}%
\pgfsetstrokecolor{currentstroke}%
\pgfsetdash{}{0pt}%
\pgfpathmoveto{\pgfqpoint{0.709028in}{5.631667in}}%
\pgfpathlineto{\pgfqpoint{5.978403in}{5.631667in}}%
\pgfusepath{stroke}%
\end{pgfscope}%
\begin{pgfscope}%
\definecolor{textcolor}{rgb}{0.000000,0.000000,0.000000}%
\pgfsetstrokecolor{textcolor}%
\pgfsetfillcolor{textcolor}%
\pgftext[x=3.343715in,y=5.715000in,,base]{\color{textcolor}{\sffamily\fontsize{12.000000}{14.400000}\selectfont\catcode`\^=\active\def^{\ifmmode\sp\else\^{}\fi}\catcode`\%=\active\def%{\%}Distribuição de score}}%
\end{pgfscope}%
\begin{pgfscope}%
\pgfsetbuttcap%
\pgfsetmiterjoin%
\definecolor{currentfill}{rgb}{1.000000,1.000000,1.000000}%
\pgfsetfillcolor{currentfill}%
\pgfsetlinewidth{0.000000pt}%
\definecolor{currentstroke}{rgb}{0.000000,0.000000,0.000000}%
\pgfsetstrokecolor{currentstroke}%
\pgfsetstrokeopacity{0.000000}%
\pgfsetdash{}{0pt}%
\pgfpathmoveto{\pgfqpoint{6.580625in}{0.387222in}}%
\pgfpathlineto{\pgfqpoint{11.850000in}{0.387222in}}%
\pgfpathlineto{\pgfqpoint{11.850000in}{5.631667in}}%
\pgfpathlineto{\pgfqpoint{6.580625in}{5.631667in}}%
\pgfpathlineto{\pgfqpoint{6.580625in}{0.387222in}}%
\pgfpathclose%
\pgfusepath{fill}%
\end{pgfscope}%
\begin{pgfscope}%
\pgfpathrectangle{\pgfqpoint{6.580625in}{0.387222in}}{\pgfqpoint{5.269375in}{5.244444in}}%
\pgfusepath{clip}%
\pgfsetbuttcap%
\pgfsetmiterjoin%
\definecolor{currentfill}{rgb}{0.194608,0.453431,0.632843}%
\pgfsetfillcolor{currentfill}%
\pgfsetlinewidth{0.000000pt}%
\definecolor{currentstroke}{rgb}{0.000000,0.000000,0.000000}%
\pgfsetstrokecolor{currentstroke}%
\pgfsetstrokeopacity{0.000000}%
\pgfsetdash{}{0pt}%
\pgfpathmoveto{\pgfqpoint{6.668448in}{0.387222in}}%
\pgfpathlineto{\pgfqpoint{6.808965in}{0.387222in}}%
\pgfpathlineto{\pgfqpoint{6.808965in}{0.557153in}}%
\pgfpathlineto{\pgfqpoint{6.668448in}{0.557153in}}%
\pgfpathlineto{\pgfqpoint{6.668448in}{0.387222in}}%
\pgfpathclose%
\pgfusepath{fill}%
\end{pgfscope}%
\begin{pgfscope}%
\pgfpathrectangle{\pgfqpoint{6.580625in}{0.387222in}}{\pgfqpoint{5.269375in}{5.244444in}}%
\pgfusepath{clip}%
\pgfsetbuttcap%
\pgfsetmiterjoin%
\definecolor{currentfill}{rgb}{0.194608,0.453431,0.632843}%
\pgfsetfillcolor{currentfill}%
\pgfsetlinewidth{0.000000pt}%
\definecolor{currentstroke}{rgb}{0.000000,0.000000,0.000000}%
\pgfsetstrokecolor{currentstroke}%
\pgfsetstrokeopacity{0.000000}%
\pgfsetdash{}{0pt}%
\pgfpathmoveto{\pgfqpoint{7.546677in}{0.387222in}}%
\pgfpathlineto{\pgfqpoint{7.687194in}{0.387222in}}%
\pgfpathlineto{\pgfqpoint{7.687194in}{0.425338in}}%
\pgfpathlineto{\pgfqpoint{7.546677in}{0.425338in}}%
\pgfpathlineto{\pgfqpoint{7.546677in}{0.387222in}}%
\pgfpathclose%
\pgfusepath{fill}%
\end{pgfscope}%
\begin{pgfscope}%
\pgfpathrectangle{\pgfqpoint{6.580625in}{0.387222in}}{\pgfqpoint{5.269375in}{5.244444in}}%
\pgfusepath{clip}%
\pgfsetbuttcap%
\pgfsetmiterjoin%
\definecolor{currentfill}{rgb}{0.194608,0.453431,0.632843}%
\pgfsetfillcolor{currentfill}%
\pgfsetlinewidth{0.000000pt}%
\definecolor{currentstroke}{rgb}{0.000000,0.000000,0.000000}%
\pgfsetstrokecolor{currentstroke}%
\pgfsetstrokeopacity{0.000000}%
\pgfsetdash{}{0pt}%
\pgfpathmoveto{\pgfqpoint{8.424906in}{0.387222in}}%
\pgfpathlineto{\pgfqpoint{8.565423in}{0.387222in}}%
\pgfpathlineto{\pgfqpoint{8.565423in}{1.219409in}}%
\pgfpathlineto{\pgfqpoint{8.424906in}{1.219409in}}%
\pgfpathlineto{\pgfqpoint{8.424906in}{0.387222in}}%
\pgfpathclose%
\pgfusepath{fill}%
\end{pgfscope}%
\begin{pgfscope}%
\pgfpathrectangle{\pgfqpoint{6.580625in}{0.387222in}}{\pgfqpoint{5.269375in}{5.244444in}}%
\pgfusepath{clip}%
\pgfsetbuttcap%
\pgfsetmiterjoin%
\definecolor{currentfill}{rgb}{0.194608,0.453431,0.632843}%
\pgfsetfillcolor{currentfill}%
\pgfsetlinewidth{0.000000pt}%
\definecolor{currentstroke}{rgb}{0.000000,0.000000,0.000000}%
\pgfsetstrokecolor{currentstroke}%
\pgfsetstrokeopacity{0.000000}%
\pgfsetdash{}{0pt}%
\pgfpathmoveto{\pgfqpoint{9.303135in}{0.387222in}}%
\pgfpathlineto{\pgfqpoint{9.443652in}{0.387222in}}%
\pgfpathlineto{\pgfqpoint{9.443652in}{5.381931in}}%
\pgfpathlineto{\pgfqpoint{9.303135in}{5.381931in}}%
\pgfpathlineto{\pgfqpoint{9.303135in}{0.387222in}}%
\pgfpathclose%
\pgfusepath{fill}%
\end{pgfscope}%
\begin{pgfscope}%
\pgfpathrectangle{\pgfqpoint{6.580625in}{0.387222in}}{\pgfqpoint{5.269375in}{5.244444in}}%
\pgfusepath{clip}%
\pgfsetbuttcap%
\pgfsetmiterjoin%
\definecolor{currentfill}{rgb}{0.194608,0.453431,0.632843}%
\pgfsetfillcolor{currentfill}%
\pgfsetlinewidth{0.000000pt}%
\definecolor{currentstroke}{rgb}{0.000000,0.000000,0.000000}%
\pgfsetstrokecolor{currentstroke}%
\pgfsetstrokeopacity{0.000000}%
\pgfsetdash{}{0pt}%
\pgfpathmoveto{\pgfqpoint{10.181365in}{0.387222in}}%
\pgfpathlineto{\pgfqpoint{10.321881in}{0.387222in}}%
\pgfpathlineto{\pgfqpoint{10.321881in}{4.330581in}}%
\pgfpathlineto{\pgfqpoint{10.181365in}{4.330581in}}%
\pgfpathlineto{\pgfqpoint{10.181365in}{0.387222in}}%
\pgfpathclose%
\pgfusepath{fill}%
\end{pgfscope}%
\begin{pgfscope}%
\pgfpathrectangle{\pgfqpoint{6.580625in}{0.387222in}}{\pgfqpoint{5.269375in}{5.244444in}}%
\pgfusepath{clip}%
\pgfsetbuttcap%
\pgfsetmiterjoin%
\definecolor{currentfill}{rgb}{0.194608,0.453431,0.632843}%
\pgfsetfillcolor{currentfill}%
\pgfsetlinewidth{0.000000pt}%
\definecolor{currentstroke}{rgb}{0.000000,0.000000,0.000000}%
\pgfsetstrokecolor{currentstroke}%
\pgfsetstrokeopacity{0.000000}%
\pgfsetdash{}{0pt}%
\pgfpathmoveto{\pgfqpoint{11.059594in}{0.387222in}}%
\pgfpathlineto{\pgfqpoint{11.200110in}{0.387222in}}%
\pgfpathlineto{\pgfqpoint{11.200110in}{0.854136in}}%
\pgfpathlineto{\pgfqpoint{11.059594in}{0.854136in}}%
\pgfpathlineto{\pgfqpoint{11.059594in}{0.387222in}}%
\pgfpathclose%
\pgfusepath{fill}%
\end{pgfscope}%
\begin{pgfscope}%
\pgfpathrectangle{\pgfqpoint{6.580625in}{0.387222in}}{\pgfqpoint{5.269375in}{5.244444in}}%
\pgfusepath{clip}%
\pgfsetbuttcap%
\pgfsetmiterjoin%
\definecolor{currentfill}{rgb}{0.881863,0.505392,0.173039}%
\pgfsetfillcolor{currentfill}%
\pgfsetlinewidth{0.000000pt}%
\definecolor{currentstroke}{rgb}{0.000000,0.000000,0.000000}%
\pgfsetstrokecolor{currentstroke}%
\pgfsetstrokeopacity{0.000000}%
\pgfsetdash{}{0pt}%
\pgfpathmoveto{\pgfqpoint{6.808965in}{0.387222in}}%
\pgfpathlineto{\pgfqpoint{6.949481in}{0.387222in}}%
\pgfpathlineto{\pgfqpoint{6.949481in}{0.582564in}}%
\pgfpathlineto{\pgfqpoint{6.808965in}{0.582564in}}%
\pgfpathlineto{\pgfqpoint{6.808965in}{0.387222in}}%
\pgfpathclose%
\pgfusepath{fill}%
\end{pgfscope}%
\begin{pgfscope}%
\pgfpathrectangle{\pgfqpoint{6.580625in}{0.387222in}}{\pgfqpoint{5.269375in}{5.244444in}}%
\pgfusepath{clip}%
\pgfsetbuttcap%
\pgfsetmiterjoin%
\definecolor{currentfill}{rgb}{0.881863,0.505392,0.173039}%
\pgfsetfillcolor{currentfill}%
\pgfsetlinewidth{0.000000pt}%
\definecolor{currentstroke}{rgb}{0.000000,0.000000,0.000000}%
\pgfsetstrokecolor{currentstroke}%
\pgfsetstrokeopacity{0.000000}%
\pgfsetdash{}{0pt}%
\pgfpathmoveto{\pgfqpoint{7.687194in}{0.387222in}}%
\pgfpathlineto{\pgfqpoint{7.827710in}{0.387222in}}%
\pgfpathlineto{\pgfqpoint{7.827710in}{0.534919in}}%
\pgfpathlineto{\pgfqpoint{7.687194in}{0.534919in}}%
\pgfpathlineto{\pgfqpoint{7.687194in}{0.387222in}}%
\pgfpathclose%
\pgfusepath{fill}%
\end{pgfscope}%
\begin{pgfscope}%
\pgfpathrectangle{\pgfqpoint{6.580625in}{0.387222in}}{\pgfqpoint{5.269375in}{5.244444in}}%
\pgfusepath{clip}%
\pgfsetbuttcap%
\pgfsetmiterjoin%
\definecolor{currentfill}{rgb}{0.881863,0.505392,0.173039}%
\pgfsetfillcolor{currentfill}%
\pgfsetlinewidth{0.000000pt}%
\definecolor{currentstroke}{rgb}{0.000000,0.000000,0.000000}%
\pgfsetstrokecolor{currentstroke}%
\pgfsetstrokeopacity{0.000000}%
\pgfsetdash{}{0pt}%
\pgfpathmoveto{\pgfqpoint{8.565423in}{0.387222in}}%
\pgfpathlineto{\pgfqpoint{8.705940in}{0.387222in}}%
\pgfpathlineto{\pgfqpoint{8.705940in}{1.845137in}}%
\pgfpathlineto{\pgfqpoint{8.565423in}{1.845137in}}%
\pgfpathlineto{\pgfqpoint{8.565423in}{0.387222in}}%
\pgfpathclose%
\pgfusepath{fill}%
\end{pgfscope}%
\begin{pgfscope}%
\pgfpathrectangle{\pgfqpoint{6.580625in}{0.387222in}}{\pgfqpoint{5.269375in}{5.244444in}}%
\pgfusepath{clip}%
\pgfsetbuttcap%
\pgfsetmiterjoin%
\definecolor{currentfill}{rgb}{0.881863,0.505392,0.173039}%
\pgfsetfillcolor{currentfill}%
\pgfsetlinewidth{0.000000pt}%
\definecolor{currentstroke}{rgb}{0.000000,0.000000,0.000000}%
\pgfsetstrokecolor{currentstroke}%
\pgfsetstrokeopacity{0.000000}%
\pgfsetdash{}{0pt}%
\pgfpathmoveto{\pgfqpoint{9.443652in}{0.387222in}}%
\pgfpathlineto{\pgfqpoint{9.584169in}{0.387222in}}%
\pgfpathlineto{\pgfqpoint{9.584169in}{4.271819in}}%
\pgfpathlineto{\pgfqpoint{9.443652in}{4.271819in}}%
\pgfpathlineto{\pgfqpoint{9.443652in}{0.387222in}}%
\pgfpathclose%
\pgfusepath{fill}%
\end{pgfscope}%
\begin{pgfscope}%
\pgfpathrectangle{\pgfqpoint{6.580625in}{0.387222in}}{\pgfqpoint{5.269375in}{5.244444in}}%
\pgfusepath{clip}%
\pgfsetbuttcap%
\pgfsetmiterjoin%
\definecolor{currentfill}{rgb}{0.881863,0.505392,0.173039}%
\pgfsetfillcolor{currentfill}%
\pgfsetlinewidth{0.000000pt}%
\definecolor{currentstroke}{rgb}{0.000000,0.000000,0.000000}%
\pgfsetstrokecolor{currentstroke}%
\pgfsetstrokeopacity{0.000000}%
\pgfsetdash{}{0pt}%
\pgfpathmoveto{\pgfqpoint{10.321881in}{0.387222in}}%
\pgfpathlineto{\pgfqpoint{10.462398in}{0.387222in}}%
\pgfpathlineto{\pgfqpoint{10.462398in}{4.238468in}}%
\pgfpathlineto{\pgfqpoint{10.321881in}{4.238468in}}%
\pgfpathlineto{\pgfqpoint{10.321881in}{0.387222in}}%
\pgfpathclose%
\pgfusepath{fill}%
\end{pgfscope}%
\begin{pgfscope}%
\pgfpathrectangle{\pgfqpoint{6.580625in}{0.387222in}}{\pgfqpoint{5.269375in}{5.244444in}}%
\pgfusepath{clip}%
\pgfsetbuttcap%
\pgfsetmiterjoin%
\definecolor{currentfill}{rgb}{0.881863,0.505392,0.173039}%
\pgfsetfillcolor{currentfill}%
\pgfsetlinewidth{0.000000pt}%
\definecolor{currentstroke}{rgb}{0.000000,0.000000,0.000000}%
\pgfsetstrokecolor{currentstroke}%
\pgfsetstrokeopacity{0.000000}%
\pgfsetdash{}{0pt}%
\pgfpathmoveto{\pgfqpoint{11.200110in}{0.387222in}}%
\pgfpathlineto{\pgfqpoint{11.340627in}{0.387222in}}%
\pgfpathlineto{\pgfqpoint{11.340627in}{1.295640in}}%
\pgfpathlineto{\pgfqpoint{11.200110in}{1.295640in}}%
\pgfpathlineto{\pgfqpoint{11.200110in}{0.387222in}}%
\pgfpathclose%
\pgfusepath{fill}%
\end{pgfscope}%
\begin{pgfscope}%
\pgfpathrectangle{\pgfqpoint{6.580625in}{0.387222in}}{\pgfqpoint{5.269375in}{5.244444in}}%
\pgfusepath{clip}%
\pgfsetbuttcap%
\pgfsetmiterjoin%
\definecolor{currentfill}{rgb}{0.229412,0.570588,0.229412}%
\pgfsetfillcolor{currentfill}%
\pgfsetlinewidth{0.000000pt}%
\definecolor{currentstroke}{rgb}{0.000000,0.000000,0.000000}%
\pgfsetstrokecolor{currentstroke}%
\pgfsetstrokeopacity{0.000000}%
\pgfsetdash{}{0pt}%
\pgfpathmoveto{\pgfqpoint{6.949481in}{0.387222in}}%
\pgfpathlineto{\pgfqpoint{7.089998in}{0.387222in}}%
\pgfpathlineto{\pgfqpoint{7.089998in}{0.681029in}}%
\pgfpathlineto{\pgfqpoint{6.949481in}{0.681029in}}%
\pgfpathlineto{\pgfqpoint{6.949481in}{0.387222in}}%
\pgfpathclose%
\pgfusepath{fill}%
\end{pgfscope}%
\begin{pgfscope}%
\pgfpathrectangle{\pgfqpoint{6.580625in}{0.387222in}}{\pgfqpoint{5.269375in}{5.244444in}}%
\pgfusepath{clip}%
\pgfsetbuttcap%
\pgfsetmiterjoin%
\definecolor{currentfill}{rgb}{0.229412,0.570588,0.229412}%
\pgfsetfillcolor{currentfill}%
\pgfsetlinewidth{0.000000pt}%
\definecolor{currentstroke}{rgb}{0.000000,0.000000,0.000000}%
\pgfsetstrokecolor{currentstroke}%
\pgfsetstrokeopacity{0.000000}%
\pgfsetdash{}{0pt}%
\pgfpathmoveto{\pgfqpoint{7.827710in}{0.387222in}}%
\pgfpathlineto{\pgfqpoint{7.968227in}{0.387222in}}%
\pgfpathlineto{\pgfqpoint{7.968227in}{0.647678in}}%
\pgfpathlineto{\pgfqpoint{7.827710in}{0.647678in}}%
\pgfpathlineto{\pgfqpoint{7.827710in}{0.387222in}}%
\pgfpathclose%
\pgfusepath{fill}%
\end{pgfscope}%
\begin{pgfscope}%
\pgfpathrectangle{\pgfqpoint{6.580625in}{0.387222in}}{\pgfqpoint{5.269375in}{5.244444in}}%
\pgfusepath{clip}%
\pgfsetbuttcap%
\pgfsetmiterjoin%
\definecolor{currentfill}{rgb}{0.229412,0.570588,0.229412}%
\pgfsetfillcolor{currentfill}%
\pgfsetlinewidth{0.000000pt}%
\definecolor{currentstroke}{rgb}{0.000000,0.000000,0.000000}%
\pgfsetstrokecolor{currentstroke}%
\pgfsetstrokeopacity{0.000000}%
\pgfsetdash{}{0pt}%
\pgfpathmoveto{\pgfqpoint{8.705940in}{0.387222in}}%
\pgfpathlineto{\pgfqpoint{8.846456in}{0.387222in}}%
\pgfpathlineto{\pgfqpoint{8.846456in}{2.937780in}}%
\pgfpathlineto{\pgfqpoint{8.705940in}{2.937780in}}%
\pgfpathlineto{\pgfqpoint{8.705940in}{0.387222in}}%
\pgfpathclose%
\pgfusepath{fill}%
\end{pgfscope}%
\begin{pgfscope}%
\pgfpathrectangle{\pgfqpoint{6.580625in}{0.387222in}}{\pgfqpoint{5.269375in}{5.244444in}}%
\pgfusepath{clip}%
\pgfsetbuttcap%
\pgfsetmiterjoin%
\definecolor{currentfill}{rgb}{0.229412,0.570588,0.229412}%
\pgfsetfillcolor{currentfill}%
\pgfsetlinewidth{0.000000pt}%
\definecolor{currentstroke}{rgb}{0.000000,0.000000,0.000000}%
\pgfsetstrokecolor{currentstroke}%
\pgfsetstrokeopacity{0.000000}%
\pgfsetdash{}{0pt}%
\pgfpathmoveto{\pgfqpoint{9.584169in}{0.387222in}}%
\pgfpathlineto{\pgfqpoint{9.724685in}{0.387222in}}%
\pgfpathlineto{\pgfqpoint{9.724685in}{5.237410in}}%
\pgfpathlineto{\pgfqpoint{9.584169in}{5.237410in}}%
\pgfpathlineto{\pgfqpoint{9.584169in}{0.387222in}}%
\pgfpathclose%
\pgfusepath{fill}%
\end{pgfscope}%
\begin{pgfscope}%
\pgfpathrectangle{\pgfqpoint{6.580625in}{0.387222in}}{\pgfqpoint{5.269375in}{5.244444in}}%
\pgfusepath{clip}%
\pgfsetbuttcap%
\pgfsetmiterjoin%
\definecolor{currentfill}{rgb}{0.229412,0.570588,0.229412}%
\pgfsetfillcolor{currentfill}%
\pgfsetlinewidth{0.000000pt}%
\definecolor{currentstroke}{rgb}{0.000000,0.000000,0.000000}%
\pgfsetstrokecolor{currentstroke}%
\pgfsetstrokeopacity{0.000000}%
\pgfsetdash{}{0pt}%
\pgfpathmoveto{\pgfqpoint{10.462398in}{0.387222in}}%
\pgfpathlineto{\pgfqpoint{10.602915in}{0.387222in}}%
\pgfpathlineto{\pgfqpoint{10.602915in}{2.574095in}}%
\pgfpathlineto{\pgfqpoint{10.462398in}{2.574095in}}%
\pgfpathlineto{\pgfqpoint{10.462398in}{0.387222in}}%
\pgfpathclose%
\pgfusepath{fill}%
\end{pgfscope}%
\begin{pgfscope}%
\pgfpathrectangle{\pgfqpoint{6.580625in}{0.387222in}}{\pgfqpoint{5.269375in}{5.244444in}}%
\pgfusepath{clip}%
\pgfsetbuttcap%
\pgfsetmiterjoin%
\definecolor{currentfill}{rgb}{0.229412,0.570588,0.229412}%
\pgfsetfillcolor{currentfill}%
\pgfsetlinewidth{0.000000pt}%
\definecolor{currentstroke}{rgb}{0.000000,0.000000,0.000000}%
\pgfsetstrokecolor{currentstroke}%
\pgfsetstrokeopacity{0.000000}%
\pgfsetdash{}{0pt}%
\pgfpathmoveto{\pgfqpoint{11.340627in}{0.387222in}}%
\pgfpathlineto{\pgfqpoint{11.481144in}{0.387222in}}%
\pgfpathlineto{\pgfqpoint{11.481144in}{0.690557in}}%
\pgfpathlineto{\pgfqpoint{11.340627in}{0.690557in}}%
\pgfpathlineto{\pgfqpoint{11.340627in}{0.387222in}}%
\pgfpathclose%
\pgfusepath{fill}%
\end{pgfscope}%
\begin{pgfscope}%
\pgfpathrectangle{\pgfqpoint{6.580625in}{0.387222in}}{\pgfqpoint{5.269375in}{5.244444in}}%
\pgfusepath{clip}%
\pgfsetbuttcap%
\pgfsetmiterjoin%
\definecolor{currentfill}{rgb}{0.753431,0.238725,0.241667}%
\pgfsetfillcolor{currentfill}%
\pgfsetlinewidth{0.000000pt}%
\definecolor{currentstroke}{rgb}{0.000000,0.000000,0.000000}%
\pgfsetstrokecolor{currentstroke}%
\pgfsetstrokeopacity{0.000000}%
\pgfsetdash{}{0pt}%
\pgfpathmoveto{\pgfqpoint{7.089998in}{0.387222in}}%
\pgfpathlineto{\pgfqpoint{7.230515in}{0.387222in}}%
\pgfpathlineto{\pgfqpoint{7.230515in}{0.715968in}}%
\pgfpathlineto{\pgfqpoint{7.089998in}{0.715968in}}%
\pgfpathlineto{\pgfqpoint{7.089998in}{0.387222in}}%
\pgfpathclose%
\pgfusepath{fill}%
\end{pgfscope}%
\begin{pgfscope}%
\pgfpathrectangle{\pgfqpoint{6.580625in}{0.387222in}}{\pgfqpoint{5.269375in}{5.244444in}}%
\pgfusepath{clip}%
\pgfsetbuttcap%
\pgfsetmiterjoin%
\definecolor{currentfill}{rgb}{0.753431,0.238725,0.241667}%
\pgfsetfillcolor{currentfill}%
\pgfsetlinewidth{0.000000pt}%
\definecolor{currentstroke}{rgb}{0.000000,0.000000,0.000000}%
\pgfsetstrokecolor{currentstroke}%
\pgfsetstrokeopacity{0.000000}%
\pgfsetdash{}{0pt}%
\pgfpathmoveto{\pgfqpoint{7.968227in}{0.387222in}}%
\pgfpathlineto{\pgfqpoint{8.108744in}{0.387222in}}%
\pgfpathlineto{\pgfqpoint{8.108744in}{0.490452in}}%
\pgfpathlineto{\pgfqpoint{7.968227in}{0.490452in}}%
\pgfpathlineto{\pgfqpoint{7.968227in}{0.387222in}}%
\pgfpathclose%
\pgfusepath{fill}%
\end{pgfscope}%
\begin{pgfscope}%
\pgfpathrectangle{\pgfqpoint{6.580625in}{0.387222in}}{\pgfqpoint{5.269375in}{5.244444in}}%
\pgfusepath{clip}%
\pgfsetbuttcap%
\pgfsetmiterjoin%
\definecolor{currentfill}{rgb}{0.753431,0.238725,0.241667}%
\pgfsetfillcolor{currentfill}%
\pgfsetlinewidth{0.000000pt}%
\definecolor{currentstroke}{rgb}{0.000000,0.000000,0.000000}%
\pgfsetstrokecolor{currentstroke}%
\pgfsetstrokeopacity{0.000000}%
\pgfsetdash{}{0pt}%
\pgfpathmoveto{\pgfqpoint{8.846456in}{0.387222in}}%
\pgfpathlineto{\pgfqpoint{8.986973in}{0.387222in}}%
\pgfpathlineto{\pgfqpoint{8.986973in}{1.792729in}}%
\pgfpathlineto{\pgfqpoint{8.846456in}{1.792729in}}%
\pgfpathlineto{\pgfqpoint{8.846456in}{0.387222in}}%
\pgfpathclose%
\pgfusepath{fill}%
\end{pgfscope}%
\begin{pgfscope}%
\pgfpathrectangle{\pgfqpoint{6.580625in}{0.387222in}}{\pgfqpoint{5.269375in}{5.244444in}}%
\pgfusepath{clip}%
\pgfsetbuttcap%
\pgfsetmiterjoin%
\definecolor{currentfill}{rgb}{0.753431,0.238725,0.241667}%
\pgfsetfillcolor{currentfill}%
\pgfsetlinewidth{0.000000pt}%
\definecolor{currentstroke}{rgb}{0.000000,0.000000,0.000000}%
\pgfsetstrokecolor{currentstroke}%
\pgfsetstrokeopacity{0.000000}%
\pgfsetdash{}{0pt}%
\pgfpathmoveto{\pgfqpoint{9.724685in}{0.387222in}}%
\pgfpathlineto{\pgfqpoint{9.865202in}{0.387222in}}%
\pgfpathlineto{\pgfqpoint{9.865202in}{4.292465in}}%
\pgfpathlineto{\pgfqpoint{9.724685in}{4.292465in}}%
\pgfpathlineto{\pgfqpoint{9.724685in}{0.387222in}}%
\pgfpathclose%
\pgfusepath{fill}%
\end{pgfscope}%
\begin{pgfscope}%
\pgfpathrectangle{\pgfqpoint{6.580625in}{0.387222in}}{\pgfqpoint{5.269375in}{5.244444in}}%
\pgfusepath{clip}%
\pgfsetbuttcap%
\pgfsetmiterjoin%
\definecolor{currentfill}{rgb}{0.753431,0.238725,0.241667}%
\pgfsetfillcolor{currentfill}%
\pgfsetlinewidth{0.000000pt}%
\definecolor{currentstroke}{rgb}{0.000000,0.000000,0.000000}%
\pgfsetstrokecolor{currentstroke}%
\pgfsetstrokeopacity{0.000000}%
\pgfsetdash{}{0pt}%
\pgfpathmoveto{\pgfqpoint{10.602915in}{0.387222in}}%
\pgfpathlineto{\pgfqpoint{10.743431in}{0.387222in}}%
\pgfpathlineto{\pgfqpoint{10.743431in}{3.280818in}}%
\pgfpathlineto{\pgfqpoint{10.602915in}{3.280818in}}%
\pgfpathlineto{\pgfqpoint{10.602915in}{0.387222in}}%
\pgfpathclose%
\pgfusepath{fill}%
\end{pgfscope}%
\begin{pgfscope}%
\pgfpathrectangle{\pgfqpoint{6.580625in}{0.387222in}}{\pgfqpoint{5.269375in}{5.244444in}}%
\pgfusepath{clip}%
\pgfsetbuttcap%
\pgfsetmiterjoin%
\definecolor{currentfill}{rgb}{0.753431,0.238725,0.241667}%
\pgfsetfillcolor{currentfill}%
\pgfsetlinewidth{0.000000pt}%
\definecolor{currentstroke}{rgb}{0.000000,0.000000,0.000000}%
\pgfsetstrokecolor{currentstroke}%
\pgfsetstrokeopacity{0.000000}%
\pgfsetdash{}{0pt}%
\pgfpathmoveto{\pgfqpoint{11.481144in}{0.387222in}}%
\pgfpathlineto{\pgfqpoint{11.621660in}{0.387222in}}%
\pgfpathlineto{\pgfqpoint{11.621660in}{2.196117in}}%
\pgfpathlineto{\pgfqpoint{11.481144in}{2.196117in}}%
\pgfpathlineto{\pgfqpoint{11.481144in}{0.387222in}}%
\pgfpathclose%
\pgfusepath{fill}%
\end{pgfscope}%
\begin{pgfscope}%
\pgfpathrectangle{\pgfqpoint{6.580625in}{0.387222in}}{\pgfqpoint{5.269375in}{5.244444in}}%
\pgfusepath{clip}%
\pgfsetbuttcap%
\pgfsetmiterjoin%
\definecolor{currentfill}{rgb}{0.578431,0.446078,0.699020}%
\pgfsetfillcolor{currentfill}%
\pgfsetlinewidth{0.000000pt}%
\definecolor{currentstroke}{rgb}{0.000000,0.000000,0.000000}%
\pgfsetstrokecolor{currentstroke}%
\pgfsetstrokeopacity{0.000000}%
\pgfsetdash{}{0pt}%
\pgfpathmoveto{\pgfqpoint{7.230515in}{0.387222in}}%
\pgfpathlineto{\pgfqpoint{7.371031in}{0.387222in}}%
\pgfpathlineto{\pgfqpoint{7.371031in}{1.198763in}}%
\pgfpathlineto{\pgfqpoint{7.230515in}{1.198763in}}%
\pgfpathlineto{\pgfqpoint{7.230515in}{0.387222in}}%
\pgfpathclose%
\pgfusepath{fill}%
\end{pgfscope}%
\begin{pgfscope}%
\pgfpathrectangle{\pgfqpoint{6.580625in}{0.387222in}}{\pgfqpoint{5.269375in}{5.244444in}}%
\pgfusepath{clip}%
\pgfsetbuttcap%
\pgfsetmiterjoin%
\definecolor{currentfill}{rgb}{0.578431,0.446078,0.699020}%
\pgfsetfillcolor{currentfill}%
\pgfsetlinewidth{0.000000pt}%
\definecolor{currentstroke}{rgb}{0.000000,0.000000,0.000000}%
\pgfsetstrokecolor{currentstroke}%
\pgfsetstrokeopacity{0.000000}%
\pgfsetdash{}{0pt}%
\pgfpathmoveto{\pgfqpoint{8.108744in}{0.387222in}}%
\pgfpathlineto{\pgfqpoint{8.249260in}{0.387222in}}%
\pgfpathlineto{\pgfqpoint{8.249260in}{0.858901in}}%
\pgfpathlineto{\pgfqpoint{8.108744in}{0.858901in}}%
\pgfpathlineto{\pgfqpoint{8.108744in}{0.387222in}}%
\pgfpathclose%
\pgfusepath{fill}%
\end{pgfscope}%
\begin{pgfscope}%
\pgfpathrectangle{\pgfqpoint{6.580625in}{0.387222in}}{\pgfqpoint{5.269375in}{5.244444in}}%
\pgfusepath{clip}%
\pgfsetbuttcap%
\pgfsetmiterjoin%
\definecolor{currentfill}{rgb}{0.578431,0.446078,0.699020}%
\pgfsetfillcolor{currentfill}%
\pgfsetlinewidth{0.000000pt}%
\definecolor{currentstroke}{rgb}{0.000000,0.000000,0.000000}%
\pgfsetstrokecolor{currentstroke}%
\pgfsetstrokeopacity{0.000000}%
\pgfsetdash{}{0pt}%
\pgfpathmoveto{\pgfqpoint{8.986973in}{0.387222in}}%
\pgfpathlineto{\pgfqpoint{9.127490in}{0.387222in}}%
\pgfpathlineto{\pgfqpoint{9.127490in}{2.507393in}}%
\pgfpathlineto{\pgfqpoint{8.986973in}{2.507393in}}%
\pgfpathlineto{\pgfqpoint{8.986973in}{0.387222in}}%
\pgfpathclose%
\pgfusepath{fill}%
\end{pgfscope}%
\begin{pgfscope}%
\pgfpathrectangle{\pgfqpoint{6.580625in}{0.387222in}}{\pgfqpoint{5.269375in}{5.244444in}}%
\pgfusepath{clip}%
\pgfsetbuttcap%
\pgfsetmiterjoin%
\definecolor{currentfill}{rgb}{0.578431,0.446078,0.699020}%
\pgfsetfillcolor{currentfill}%
\pgfsetlinewidth{0.000000pt}%
\definecolor{currentstroke}{rgb}{0.000000,0.000000,0.000000}%
\pgfsetstrokecolor{currentstroke}%
\pgfsetstrokeopacity{0.000000}%
\pgfsetdash{}{0pt}%
\pgfpathmoveto{\pgfqpoint{9.865202in}{0.387222in}}%
\pgfpathlineto{\pgfqpoint{10.005719in}{0.387222in}}%
\pgfpathlineto{\pgfqpoint{10.005719in}{4.020893in}}%
\pgfpathlineto{\pgfqpoint{9.865202in}{4.020893in}}%
\pgfpathlineto{\pgfqpoint{9.865202in}{0.387222in}}%
\pgfpathclose%
\pgfusepath{fill}%
\end{pgfscope}%
\begin{pgfscope}%
\pgfpathrectangle{\pgfqpoint{6.580625in}{0.387222in}}{\pgfqpoint{5.269375in}{5.244444in}}%
\pgfusepath{clip}%
\pgfsetbuttcap%
\pgfsetmiterjoin%
\definecolor{currentfill}{rgb}{0.578431,0.446078,0.699020}%
\pgfsetfillcolor{currentfill}%
\pgfsetlinewidth{0.000000pt}%
\definecolor{currentstroke}{rgb}{0.000000,0.000000,0.000000}%
\pgfsetstrokecolor{currentstroke}%
\pgfsetstrokeopacity{0.000000}%
\pgfsetdash{}{0pt}%
\pgfpathmoveto{\pgfqpoint{10.743431in}{0.387222in}}%
\pgfpathlineto{\pgfqpoint{10.883948in}{0.387222in}}%
\pgfpathlineto{\pgfqpoint{10.883948in}{2.825021in}}%
\pgfpathlineto{\pgfqpoint{10.743431in}{2.825021in}}%
\pgfpathlineto{\pgfqpoint{10.743431in}{0.387222in}}%
\pgfpathclose%
\pgfusepath{fill}%
\end{pgfscope}%
\begin{pgfscope}%
\pgfpathrectangle{\pgfqpoint{6.580625in}{0.387222in}}{\pgfqpoint{5.269375in}{5.244444in}}%
\pgfusepath{clip}%
\pgfsetbuttcap%
\pgfsetmiterjoin%
\definecolor{currentfill}{rgb}{0.578431,0.446078,0.699020}%
\pgfsetfillcolor{currentfill}%
\pgfsetlinewidth{0.000000pt}%
\definecolor{currentstroke}{rgb}{0.000000,0.000000,0.000000}%
\pgfsetstrokecolor{currentstroke}%
\pgfsetstrokeopacity{0.000000}%
\pgfsetdash{}{0pt}%
\pgfpathmoveto{\pgfqpoint{11.621660in}{0.387222in}}%
\pgfpathlineto{\pgfqpoint{11.762177in}{0.387222in}}%
\pgfpathlineto{\pgfqpoint{11.762177in}{1.357577in}}%
\pgfpathlineto{\pgfqpoint{11.621660in}{1.357577in}}%
\pgfpathlineto{\pgfqpoint{11.621660in}{0.387222in}}%
\pgfpathclose%
\pgfusepath{fill}%
\end{pgfscope}%
\begin{pgfscope}%
\pgfpathrectangle{\pgfqpoint{6.580625in}{0.387222in}}{\pgfqpoint{5.269375in}{5.244444in}}%
\pgfusepath{clip}%
\pgfsetbuttcap%
\pgfsetmiterjoin%
\definecolor{currentfill}{rgb}{0.194608,0.453431,0.632843}%
\pgfsetfillcolor{currentfill}%
\pgfsetlinewidth{0.000000pt}%
\definecolor{currentstroke}{rgb}{0.000000,0.000000,0.000000}%
\pgfsetstrokecolor{currentstroke}%
\pgfsetstrokeopacity{0.000000}%
\pgfsetdash{}{0pt}%
\pgfpathmoveto{\pgfqpoint{7.019740in}{0.387222in}}%
\pgfpathlineto{\pgfqpoint{7.019740in}{0.387222in}}%
\pgfpathlineto{\pgfqpoint{7.019740in}{0.387222in}}%
\pgfpathlineto{\pgfqpoint{7.019740in}{0.387222in}}%
\pgfpathlineto{\pgfqpoint{7.019740in}{0.387222in}}%
\pgfpathclose%
\pgfusepath{fill}%
\end{pgfscope}%
\begin{pgfscope}%
\pgfpathrectangle{\pgfqpoint{6.580625in}{0.387222in}}{\pgfqpoint{5.269375in}{5.244444in}}%
\pgfusepath{clip}%
\pgfsetbuttcap%
\pgfsetmiterjoin%
\definecolor{currentfill}{rgb}{0.881863,0.505392,0.173039}%
\pgfsetfillcolor{currentfill}%
\pgfsetlinewidth{0.000000pt}%
\definecolor{currentstroke}{rgb}{0.000000,0.000000,0.000000}%
\pgfsetstrokecolor{currentstroke}%
\pgfsetstrokeopacity{0.000000}%
\pgfsetdash{}{0pt}%
\pgfpathmoveto{\pgfqpoint{7.019740in}{0.387222in}}%
\pgfpathlineto{\pgfqpoint{7.019740in}{0.387222in}}%
\pgfpathlineto{\pgfqpoint{7.019740in}{0.387222in}}%
\pgfpathlineto{\pgfqpoint{7.019740in}{0.387222in}}%
\pgfpathlineto{\pgfqpoint{7.019740in}{0.387222in}}%
\pgfpathclose%
\pgfusepath{fill}%
\end{pgfscope}%
\begin{pgfscope}%
\pgfpathrectangle{\pgfqpoint{6.580625in}{0.387222in}}{\pgfqpoint{5.269375in}{5.244444in}}%
\pgfusepath{clip}%
\pgfsetbuttcap%
\pgfsetmiterjoin%
\definecolor{currentfill}{rgb}{0.229412,0.570588,0.229412}%
\pgfsetfillcolor{currentfill}%
\pgfsetlinewidth{0.000000pt}%
\definecolor{currentstroke}{rgb}{0.000000,0.000000,0.000000}%
\pgfsetstrokecolor{currentstroke}%
\pgfsetstrokeopacity{0.000000}%
\pgfsetdash{}{0pt}%
\pgfpathmoveto{\pgfqpoint{7.019740in}{0.387222in}}%
\pgfpathlineto{\pgfqpoint{7.019740in}{0.387222in}}%
\pgfpathlineto{\pgfqpoint{7.019740in}{0.387222in}}%
\pgfpathlineto{\pgfqpoint{7.019740in}{0.387222in}}%
\pgfpathlineto{\pgfqpoint{7.019740in}{0.387222in}}%
\pgfpathclose%
\pgfusepath{fill}%
\end{pgfscope}%
\begin{pgfscope}%
\pgfpathrectangle{\pgfqpoint{6.580625in}{0.387222in}}{\pgfqpoint{5.269375in}{5.244444in}}%
\pgfusepath{clip}%
\pgfsetbuttcap%
\pgfsetmiterjoin%
\definecolor{currentfill}{rgb}{0.753431,0.238725,0.241667}%
\pgfsetfillcolor{currentfill}%
\pgfsetlinewidth{0.000000pt}%
\definecolor{currentstroke}{rgb}{0.000000,0.000000,0.000000}%
\pgfsetstrokecolor{currentstroke}%
\pgfsetstrokeopacity{0.000000}%
\pgfsetdash{}{0pt}%
\pgfpathmoveto{\pgfqpoint{7.019740in}{0.387222in}}%
\pgfpathlineto{\pgfqpoint{7.019740in}{0.387222in}}%
\pgfpathlineto{\pgfqpoint{7.019740in}{0.387222in}}%
\pgfpathlineto{\pgfqpoint{7.019740in}{0.387222in}}%
\pgfpathlineto{\pgfqpoint{7.019740in}{0.387222in}}%
\pgfpathclose%
\pgfusepath{fill}%
\end{pgfscope}%
\begin{pgfscope}%
\pgfpathrectangle{\pgfqpoint{6.580625in}{0.387222in}}{\pgfqpoint{5.269375in}{5.244444in}}%
\pgfusepath{clip}%
\pgfsetbuttcap%
\pgfsetmiterjoin%
\definecolor{currentfill}{rgb}{0.578431,0.446078,0.699020}%
\pgfsetfillcolor{currentfill}%
\pgfsetlinewidth{0.000000pt}%
\definecolor{currentstroke}{rgb}{0.000000,0.000000,0.000000}%
\pgfsetstrokecolor{currentstroke}%
\pgfsetstrokeopacity{0.000000}%
\pgfsetdash{}{0pt}%
\pgfpathmoveto{\pgfqpoint{7.019740in}{0.387222in}}%
\pgfpathlineto{\pgfqpoint{7.019740in}{0.387222in}}%
\pgfpathlineto{\pgfqpoint{7.019740in}{0.387222in}}%
\pgfpathlineto{\pgfqpoint{7.019740in}{0.387222in}}%
\pgfpathlineto{\pgfqpoint{7.019740in}{0.387222in}}%
\pgfpathclose%
\pgfusepath{fill}%
\end{pgfscope}%
\begin{pgfscope}%
\pgfsetbuttcap%
\pgfsetroundjoin%
\definecolor{currentfill}{rgb}{0.000000,0.000000,0.000000}%
\pgfsetfillcolor{currentfill}%
\pgfsetlinewidth{0.803000pt}%
\definecolor{currentstroke}{rgb}{0.000000,0.000000,0.000000}%
\pgfsetstrokecolor{currentstroke}%
\pgfsetdash{}{0pt}%
\pgfsys@defobject{currentmarker}{\pgfqpoint{0.000000in}{-0.048611in}}{\pgfqpoint{0.000000in}{0.000000in}}{%
\pgfpathmoveto{\pgfqpoint{0.000000in}{0.000000in}}%
\pgfpathlineto{\pgfqpoint{0.000000in}{-0.048611in}}%
\pgfusepath{stroke,fill}%
}%
\begin{pgfscope}%
\pgfsys@transformshift{7.019740in}{0.387222in}%
\pgfsys@useobject{currentmarker}{}%
\end{pgfscope}%
\end{pgfscope}%
\begin{pgfscope}%
\definecolor{textcolor}{rgb}{0.000000,0.000000,0.000000}%
\pgfsetstrokecolor{textcolor}%
\pgfsetfillcolor{textcolor}%
\pgftext[x=7.019740in,y=0.290000in,,top]{\color{textcolor}{\sffamily\fontsize{10.000000}{12.000000}\selectfont\catcode`\^=\active\def^{\ifmmode\sp\else\^{}\fi}\catcode`\%=\active\def%{\%}0}}%
\end{pgfscope}%
\begin{pgfscope}%
\pgfsetbuttcap%
\pgfsetroundjoin%
\definecolor{currentfill}{rgb}{0.000000,0.000000,0.000000}%
\pgfsetfillcolor{currentfill}%
\pgfsetlinewidth{0.803000pt}%
\definecolor{currentstroke}{rgb}{0.000000,0.000000,0.000000}%
\pgfsetstrokecolor{currentstroke}%
\pgfsetdash{}{0pt}%
\pgfsys@defobject{currentmarker}{\pgfqpoint{0.000000in}{-0.048611in}}{\pgfqpoint{0.000000in}{0.000000in}}{%
\pgfpathmoveto{\pgfqpoint{0.000000in}{0.000000in}}%
\pgfpathlineto{\pgfqpoint{0.000000in}{-0.048611in}}%
\pgfusepath{stroke,fill}%
}%
\begin{pgfscope}%
\pgfsys@transformshift{7.897969in}{0.387222in}%
\pgfsys@useobject{currentmarker}{}%
\end{pgfscope}%
\end{pgfscope}%
\begin{pgfscope}%
\definecolor{textcolor}{rgb}{0.000000,0.000000,0.000000}%
\pgfsetstrokecolor{textcolor}%
\pgfsetfillcolor{textcolor}%
\pgftext[x=7.897969in,y=0.290000in,,top]{\color{textcolor}{\sffamily\fontsize{10.000000}{12.000000}\selectfont\catcode`\^=\active\def^{\ifmmode\sp\else\^{}\fi}\catcode`\%=\active\def%{\%}40}}%
\end{pgfscope}%
\begin{pgfscope}%
\pgfsetbuttcap%
\pgfsetroundjoin%
\definecolor{currentfill}{rgb}{0.000000,0.000000,0.000000}%
\pgfsetfillcolor{currentfill}%
\pgfsetlinewidth{0.803000pt}%
\definecolor{currentstroke}{rgb}{0.000000,0.000000,0.000000}%
\pgfsetstrokecolor{currentstroke}%
\pgfsetdash{}{0pt}%
\pgfsys@defobject{currentmarker}{\pgfqpoint{0.000000in}{-0.048611in}}{\pgfqpoint{0.000000in}{0.000000in}}{%
\pgfpathmoveto{\pgfqpoint{0.000000in}{0.000000in}}%
\pgfpathlineto{\pgfqpoint{0.000000in}{-0.048611in}}%
\pgfusepath{stroke,fill}%
}%
\begin{pgfscope}%
\pgfsys@transformshift{8.776198in}{0.387222in}%
\pgfsys@useobject{currentmarker}{}%
\end{pgfscope}%
\end{pgfscope}%
\begin{pgfscope}%
\definecolor{textcolor}{rgb}{0.000000,0.000000,0.000000}%
\pgfsetstrokecolor{textcolor}%
\pgfsetfillcolor{textcolor}%
\pgftext[x=8.776198in,y=0.290000in,,top]{\color{textcolor}{\sffamily\fontsize{10.000000}{12.000000}\selectfont\catcode`\^=\active\def^{\ifmmode\sp\else\^{}\fi}\catcode`\%=\active\def%{\%}80}}%
\end{pgfscope}%
\begin{pgfscope}%
\pgfsetbuttcap%
\pgfsetroundjoin%
\definecolor{currentfill}{rgb}{0.000000,0.000000,0.000000}%
\pgfsetfillcolor{currentfill}%
\pgfsetlinewidth{0.803000pt}%
\definecolor{currentstroke}{rgb}{0.000000,0.000000,0.000000}%
\pgfsetstrokecolor{currentstroke}%
\pgfsetdash{}{0pt}%
\pgfsys@defobject{currentmarker}{\pgfqpoint{0.000000in}{-0.048611in}}{\pgfqpoint{0.000000in}{0.000000in}}{%
\pgfpathmoveto{\pgfqpoint{0.000000in}{0.000000in}}%
\pgfpathlineto{\pgfqpoint{0.000000in}{-0.048611in}}%
\pgfusepath{stroke,fill}%
}%
\begin{pgfscope}%
\pgfsys@transformshift{9.654427in}{0.387222in}%
\pgfsys@useobject{currentmarker}{}%
\end{pgfscope}%
\end{pgfscope}%
\begin{pgfscope}%
\definecolor{textcolor}{rgb}{0.000000,0.000000,0.000000}%
\pgfsetstrokecolor{textcolor}%
\pgfsetfillcolor{textcolor}%
\pgftext[x=9.654427in,y=0.290000in,,top]{\color{textcolor}{\sffamily\fontsize{10.000000}{12.000000}\selectfont\catcode`\^=\active\def^{\ifmmode\sp\else\^{}\fi}\catcode`\%=\active\def%{\%}120}}%
\end{pgfscope}%
\begin{pgfscope}%
\pgfsetbuttcap%
\pgfsetroundjoin%
\definecolor{currentfill}{rgb}{0.000000,0.000000,0.000000}%
\pgfsetfillcolor{currentfill}%
\pgfsetlinewidth{0.803000pt}%
\definecolor{currentstroke}{rgb}{0.000000,0.000000,0.000000}%
\pgfsetstrokecolor{currentstroke}%
\pgfsetdash{}{0pt}%
\pgfsys@defobject{currentmarker}{\pgfqpoint{0.000000in}{-0.048611in}}{\pgfqpoint{0.000000in}{0.000000in}}{%
\pgfpathmoveto{\pgfqpoint{0.000000in}{0.000000in}}%
\pgfpathlineto{\pgfqpoint{0.000000in}{-0.048611in}}%
\pgfusepath{stroke,fill}%
}%
\begin{pgfscope}%
\pgfsys@transformshift{10.532656in}{0.387222in}%
\pgfsys@useobject{currentmarker}{}%
\end{pgfscope}%
\end{pgfscope}%
\begin{pgfscope}%
\definecolor{textcolor}{rgb}{0.000000,0.000000,0.000000}%
\pgfsetstrokecolor{textcolor}%
\pgfsetfillcolor{textcolor}%
\pgftext[x=10.532656in,y=0.290000in,,top]{\color{textcolor}{\sffamily\fontsize{10.000000}{12.000000}\selectfont\catcode`\^=\active\def^{\ifmmode\sp\else\^{}\fi}\catcode`\%=\active\def%{\%}160}}%
\end{pgfscope}%
\begin{pgfscope}%
\pgfsetbuttcap%
\pgfsetroundjoin%
\definecolor{currentfill}{rgb}{0.000000,0.000000,0.000000}%
\pgfsetfillcolor{currentfill}%
\pgfsetlinewidth{0.803000pt}%
\definecolor{currentstroke}{rgb}{0.000000,0.000000,0.000000}%
\pgfsetstrokecolor{currentstroke}%
\pgfsetdash{}{0pt}%
\pgfsys@defobject{currentmarker}{\pgfqpoint{0.000000in}{-0.048611in}}{\pgfqpoint{0.000000in}{0.000000in}}{%
\pgfpathmoveto{\pgfqpoint{0.000000in}{0.000000in}}%
\pgfpathlineto{\pgfqpoint{0.000000in}{-0.048611in}}%
\pgfusepath{stroke,fill}%
}%
\begin{pgfscope}%
\pgfsys@transformshift{11.410885in}{0.387222in}%
\pgfsys@useobject{currentmarker}{}%
\end{pgfscope}%
\end{pgfscope}%
\begin{pgfscope}%
\definecolor{textcolor}{rgb}{0.000000,0.000000,0.000000}%
\pgfsetstrokecolor{textcolor}%
\pgfsetfillcolor{textcolor}%
\pgftext[x=11.410885in,y=0.290000in,,top]{\color{textcolor}{\sffamily\fontsize{10.000000}{12.000000}\selectfont\catcode`\^=\active\def^{\ifmmode\sp\else\^{}\fi}\catcode`\%=\active\def%{\%}200}}%
\end{pgfscope}%
\begin{pgfscope}%
\pgfsetbuttcap%
\pgfsetroundjoin%
\definecolor{currentfill}{rgb}{0.000000,0.000000,0.000000}%
\pgfsetfillcolor{currentfill}%
\pgfsetlinewidth{0.803000pt}%
\definecolor{currentstroke}{rgb}{0.000000,0.000000,0.000000}%
\pgfsetstrokecolor{currentstroke}%
\pgfsetdash{}{0pt}%
\pgfsys@defobject{currentmarker}{\pgfqpoint{-0.048611in}{0.000000in}}{\pgfqpoint{-0.000000in}{0.000000in}}{%
\pgfpathmoveto{\pgfqpoint{-0.000000in}{0.000000in}}%
\pgfpathlineto{\pgfqpoint{-0.048611in}{0.000000in}}%
\pgfusepath{stroke,fill}%
}%
\begin{pgfscope}%
\pgfsys@transformshift{6.580625in}{0.387222in}%
\pgfsys@useobject{currentmarker}{}%
\end{pgfscope}%
\end{pgfscope}%
\begin{pgfscope}%
\definecolor{textcolor}{rgb}{0.000000,0.000000,0.000000}%
\pgfsetstrokecolor{textcolor}%
\pgfsetfillcolor{textcolor}%
\pgftext[x=6.395038in, y=0.334461in, left, base]{\color{textcolor}{\sffamily\fontsize{10.000000}{12.000000}\selectfont\catcode`\^=\active\def^{\ifmmode\sp\else\^{}\fi}\catcode`\%=\active\def%{\%}0}}%
\end{pgfscope}%
\begin{pgfscope}%
\pgfsetbuttcap%
\pgfsetroundjoin%
\definecolor{currentfill}{rgb}{0.000000,0.000000,0.000000}%
\pgfsetfillcolor{currentfill}%
\pgfsetlinewidth{0.803000pt}%
\definecolor{currentstroke}{rgb}{0.000000,0.000000,0.000000}%
\pgfsetstrokecolor{currentstroke}%
\pgfsetdash{}{0pt}%
\pgfsys@defobject{currentmarker}{\pgfqpoint{-0.048611in}{0.000000in}}{\pgfqpoint{-0.000000in}{0.000000in}}{%
\pgfpathmoveto{\pgfqpoint{-0.000000in}{0.000000in}}%
\pgfpathlineto{\pgfqpoint{-0.048611in}{0.000000in}}%
\pgfusepath{stroke,fill}%
}%
\begin{pgfscope}%
\pgfsys@transformshift{6.580625in}{1.181294in}%
\pgfsys@useobject{currentmarker}{}%
\end{pgfscope}%
\end{pgfscope}%
\begin{pgfscope}%
\definecolor{textcolor}{rgb}{0.000000,0.000000,0.000000}%
\pgfsetstrokecolor{textcolor}%
\pgfsetfillcolor{textcolor}%
\pgftext[x=6.218307in, y=1.128532in, left, base]{\color{textcolor}{\sffamily\fontsize{10.000000}{12.000000}\selectfont\catcode`\^=\active\def^{\ifmmode\sp\else\^{}\fi}\catcode`\%=\active\def%{\%}500}}%
\end{pgfscope}%
\begin{pgfscope}%
\pgfsetbuttcap%
\pgfsetroundjoin%
\definecolor{currentfill}{rgb}{0.000000,0.000000,0.000000}%
\pgfsetfillcolor{currentfill}%
\pgfsetlinewidth{0.803000pt}%
\definecolor{currentstroke}{rgb}{0.000000,0.000000,0.000000}%
\pgfsetstrokecolor{currentstroke}%
\pgfsetdash{}{0pt}%
\pgfsys@defobject{currentmarker}{\pgfqpoint{-0.048611in}{0.000000in}}{\pgfqpoint{-0.000000in}{0.000000in}}{%
\pgfpathmoveto{\pgfqpoint{-0.000000in}{0.000000in}}%
\pgfpathlineto{\pgfqpoint{-0.048611in}{0.000000in}}%
\pgfusepath{stroke,fill}%
}%
\begin{pgfscope}%
\pgfsys@transformshift{6.580625in}{1.975365in}%
\pgfsys@useobject{currentmarker}{}%
\end{pgfscope}%
\end{pgfscope}%
\begin{pgfscope}%
\definecolor{textcolor}{rgb}{0.000000,0.000000,0.000000}%
\pgfsetstrokecolor{textcolor}%
\pgfsetfillcolor{textcolor}%
\pgftext[x=6.129941in, y=1.922603in, left, base]{\color{textcolor}{\sffamily\fontsize{10.000000}{12.000000}\selectfont\catcode`\^=\active\def^{\ifmmode\sp\else\^{}\fi}\catcode`\%=\active\def%{\%}1000}}%
\end{pgfscope}%
\begin{pgfscope}%
\pgfsetbuttcap%
\pgfsetroundjoin%
\definecolor{currentfill}{rgb}{0.000000,0.000000,0.000000}%
\pgfsetfillcolor{currentfill}%
\pgfsetlinewidth{0.803000pt}%
\definecolor{currentstroke}{rgb}{0.000000,0.000000,0.000000}%
\pgfsetstrokecolor{currentstroke}%
\pgfsetdash{}{0pt}%
\pgfsys@defobject{currentmarker}{\pgfqpoint{-0.048611in}{0.000000in}}{\pgfqpoint{-0.000000in}{0.000000in}}{%
\pgfpathmoveto{\pgfqpoint{-0.000000in}{0.000000in}}%
\pgfpathlineto{\pgfqpoint{-0.048611in}{0.000000in}}%
\pgfusepath{stroke,fill}%
}%
\begin{pgfscope}%
\pgfsys@transformshift{6.580625in}{2.769436in}%
\pgfsys@useobject{currentmarker}{}%
\end{pgfscope}%
\end{pgfscope}%
\begin{pgfscope}%
\definecolor{textcolor}{rgb}{0.000000,0.000000,0.000000}%
\pgfsetstrokecolor{textcolor}%
\pgfsetfillcolor{textcolor}%
\pgftext[x=6.129941in, y=2.716675in, left, base]{\color{textcolor}{\sffamily\fontsize{10.000000}{12.000000}\selectfont\catcode`\^=\active\def^{\ifmmode\sp\else\^{}\fi}\catcode`\%=\active\def%{\%}1500}}%
\end{pgfscope}%
\begin{pgfscope}%
\pgfsetbuttcap%
\pgfsetroundjoin%
\definecolor{currentfill}{rgb}{0.000000,0.000000,0.000000}%
\pgfsetfillcolor{currentfill}%
\pgfsetlinewidth{0.803000pt}%
\definecolor{currentstroke}{rgb}{0.000000,0.000000,0.000000}%
\pgfsetstrokecolor{currentstroke}%
\pgfsetdash{}{0pt}%
\pgfsys@defobject{currentmarker}{\pgfqpoint{-0.048611in}{0.000000in}}{\pgfqpoint{-0.000000in}{0.000000in}}{%
\pgfpathmoveto{\pgfqpoint{-0.000000in}{0.000000in}}%
\pgfpathlineto{\pgfqpoint{-0.048611in}{0.000000in}}%
\pgfusepath{stroke,fill}%
}%
\begin{pgfscope}%
\pgfsys@transformshift{6.580625in}{3.563508in}%
\pgfsys@useobject{currentmarker}{}%
\end{pgfscope}%
\end{pgfscope}%
\begin{pgfscope}%
\definecolor{textcolor}{rgb}{0.000000,0.000000,0.000000}%
\pgfsetstrokecolor{textcolor}%
\pgfsetfillcolor{textcolor}%
\pgftext[x=6.129941in, y=3.510746in, left, base]{\color{textcolor}{\sffamily\fontsize{10.000000}{12.000000}\selectfont\catcode`\^=\active\def^{\ifmmode\sp\else\^{}\fi}\catcode`\%=\active\def%{\%}2000}}%
\end{pgfscope}%
\begin{pgfscope}%
\pgfsetbuttcap%
\pgfsetroundjoin%
\definecolor{currentfill}{rgb}{0.000000,0.000000,0.000000}%
\pgfsetfillcolor{currentfill}%
\pgfsetlinewidth{0.803000pt}%
\definecolor{currentstroke}{rgb}{0.000000,0.000000,0.000000}%
\pgfsetstrokecolor{currentstroke}%
\pgfsetdash{}{0pt}%
\pgfsys@defobject{currentmarker}{\pgfqpoint{-0.048611in}{0.000000in}}{\pgfqpoint{-0.000000in}{0.000000in}}{%
\pgfpathmoveto{\pgfqpoint{-0.000000in}{0.000000in}}%
\pgfpathlineto{\pgfqpoint{-0.048611in}{0.000000in}}%
\pgfusepath{stroke,fill}%
}%
\begin{pgfscope}%
\pgfsys@transformshift{6.580625in}{4.357579in}%
\pgfsys@useobject{currentmarker}{}%
\end{pgfscope}%
\end{pgfscope}%
\begin{pgfscope}%
\definecolor{textcolor}{rgb}{0.000000,0.000000,0.000000}%
\pgfsetstrokecolor{textcolor}%
\pgfsetfillcolor{textcolor}%
\pgftext[x=6.129941in, y=4.304818in, left, base]{\color{textcolor}{\sffamily\fontsize{10.000000}{12.000000}\selectfont\catcode`\^=\active\def^{\ifmmode\sp\else\^{}\fi}\catcode`\%=\active\def%{\%}2500}}%
\end{pgfscope}%
\begin{pgfscope}%
\pgfsetbuttcap%
\pgfsetroundjoin%
\definecolor{currentfill}{rgb}{0.000000,0.000000,0.000000}%
\pgfsetfillcolor{currentfill}%
\pgfsetlinewidth{0.803000pt}%
\definecolor{currentstroke}{rgb}{0.000000,0.000000,0.000000}%
\pgfsetstrokecolor{currentstroke}%
\pgfsetdash{}{0pt}%
\pgfsys@defobject{currentmarker}{\pgfqpoint{-0.048611in}{0.000000in}}{\pgfqpoint{-0.000000in}{0.000000in}}{%
\pgfpathmoveto{\pgfqpoint{-0.000000in}{0.000000in}}%
\pgfpathlineto{\pgfqpoint{-0.048611in}{0.000000in}}%
\pgfusepath{stroke,fill}%
}%
\begin{pgfscope}%
\pgfsys@transformshift{6.580625in}{5.151651in}%
\pgfsys@useobject{currentmarker}{}%
\end{pgfscope}%
\end{pgfscope}%
\begin{pgfscope}%
\definecolor{textcolor}{rgb}{0.000000,0.000000,0.000000}%
\pgfsetstrokecolor{textcolor}%
\pgfsetfillcolor{textcolor}%
\pgftext[x=6.129941in, y=5.098889in, left, base]{\color{textcolor}{\sffamily\fontsize{10.000000}{12.000000}\selectfont\catcode`\^=\active\def^{\ifmmode\sp\else\^{}\fi}\catcode`\%=\active\def%{\%}3000}}%
\end{pgfscope}%
\begin{pgfscope}%
\pgfpathrectangle{\pgfqpoint{6.580625in}{0.387222in}}{\pgfqpoint{5.269375in}{5.244444in}}%
\pgfusepath{clip}%
\pgfsetrectcap%
\pgfsetroundjoin%
\pgfsetlinewidth{2.258437pt}%
\definecolor{currentstroke}{rgb}{0.260000,0.260000,0.260000}%
\pgfsetstrokecolor{currentstroke}%
\pgfsetdash{}{0pt}%
\pgfusepath{stroke}%
\end{pgfscope}%
\begin{pgfscope}%
\pgfpathrectangle{\pgfqpoint{6.580625in}{0.387222in}}{\pgfqpoint{5.269375in}{5.244444in}}%
\pgfusepath{clip}%
\pgfsetrectcap%
\pgfsetroundjoin%
\pgfsetlinewidth{2.258437pt}%
\definecolor{currentstroke}{rgb}{0.260000,0.260000,0.260000}%
\pgfsetstrokecolor{currentstroke}%
\pgfsetdash{}{0pt}%
\pgfusepath{stroke}%
\end{pgfscope}%
\begin{pgfscope}%
\pgfpathrectangle{\pgfqpoint{6.580625in}{0.387222in}}{\pgfqpoint{5.269375in}{5.244444in}}%
\pgfusepath{clip}%
\pgfsetrectcap%
\pgfsetroundjoin%
\pgfsetlinewidth{2.258437pt}%
\definecolor{currentstroke}{rgb}{0.260000,0.260000,0.260000}%
\pgfsetstrokecolor{currentstroke}%
\pgfsetdash{}{0pt}%
\pgfusepath{stroke}%
\end{pgfscope}%
\begin{pgfscope}%
\pgfpathrectangle{\pgfqpoint{6.580625in}{0.387222in}}{\pgfqpoint{5.269375in}{5.244444in}}%
\pgfusepath{clip}%
\pgfsetrectcap%
\pgfsetroundjoin%
\pgfsetlinewidth{2.258437pt}%
\definecolor{currentstroke}{rgb}{0.260000,0.260000,0.260000}%
\pgfsetstrokecolor{currentstroke}%
\pgfsetdash{}{0pt}%
\pgfusepath{stroke}%
\end{pgfscope}%
\begin{pgfscope}%
\pgfpathrectangle{\pgfqpoint{6.580625in}{0.387222in}}{\pgfqpoint{5.269375in}{5.244444in}}%
\pgfusepath{clip}%
\pgfsetrectcap%
\pgfsetroundjoin%
\pgfsetlinewidth{2.258437pt}%
\definecolor{currentstroke}{rgb}{0.260000,0.260000,0.260000}%
\pgfsetstrokecolor{currentstroke}%
\pgfsetdash{}{0pt}%
\pgfusepath{stroke}%
\end{pgfscope}%
\begin{pgfscope}%
\pgfpathrectangle{\pgfqpoint{6.580625in}{0.387222in}}{\pgfqpoint{5.269375in}{5.244444in}}%
\pgfusepath{clip}%
\pgfsetrectcap%
\pgfsetroundjoin%
\pgfsetlinewidth{2.258437pt}%
\definecolor{currentstroke}{rgb}{0.260000,0.260000,0.260000}%
\pgfsetstrokecolor{currentstroke}%
\pgfsetdash{}{0pt}%
\pgfusepath{stroke}%
\end{pgfscope}%
\begin{pgfscope}%
\pgfpathrectangle{\pgfqpoint{6.580625in}{0.387222in}}{\pgfqpoint{5.269375in}{5.244444in}}%
\pgfusepath{clip}%
\pgfsetrectcap%
\pgfsetroundjoin%
\pgfsetlinewidth{2.258437pt}%
\definecolor{currentstroke}{rgb}{0.260000,0.260000,0.260000}%
\pgfsetstrokecolor{currentstroke}%
\pgfsetdash{}{0pt}%
\pgfusepath{stroke}%
\end{pgfscope}%
\begin{pgfscope}%
\pgfpathrectangle{\pgfqpoint{6.580625in}{0.387222in}}{\pgfqpoint{5.269375in}{5.244444in}}%
\pgfusepath{clip}%
\pgfsetrectcap%
\pgfsetroundjoin%
\pgfsetlinewidth{2.258437pt}%
\definecolor{currentstroke}{rgb}{0.260000,0.260000,0.260000}%
\pgfsetstrokecolor{currentstroke}%
\pgfsetdash{}{0pt}%
\pgfusepath{stroke}%
\end{pgfscope}%
\begin{pgfscope}%
\pgfpathrectangle{\pgfqpoint{6.580625in}{0.387222in}}{\pgfqpoint{5.269375in}{5.244444in}}%
\pgfusepath{clip}%
\pgfsetrectcap%
\pgfsetroundjoin%
\pgfsetlinewidth{2.258437pt}%
\definecolor{currentstroke}{rgb}{0.260000,0.260000,0.260000}%
\pgfsetstrokecolor{currentstroke}%
\pgfsetdash{}{0pt}%
\pgfusepath{stroke}%
\end{pgfscope}%
\begin{pgfscope}%
\pgfpathrectangle{\pgfqpoint{6.580625in}{0.387222in}}{\pgfqpoint{5.269375in}{5.244444in}}%
\pgfusepath{clip}%
\pgfsetrectcap%
\pgfsetroundjoin%
\pgfsetlinewidth{2.258437pt}%
\definecolor{currentstroke}{rgb}{0.260000,0.260000,0.260000}%
\pgfsetstrokecolor{currentstroke}%
\pgfsetdash{}{0pt}%
\pgfusepath{stroke}%
\end{pgfscope}%
\begin{pgfscope}%
\pgfpathrectangle{\pgfqpoint{6.580625in}{0.387222in}}{\pgfqpoint{5.269375in}{5.244444in}}%
\pgfusepath{clip}%
\pgfsetrectcap%
\pgfsetroundjoin%
\pgfsetlinewidth{2.258437pt}%
\definecolor{currentstroke}{rgb}{0.260000,0.260000,0.260000}%
\pgfsetstrokecolor{currentstroke}%
\pgfsetdash{}{0pt}%
\pgfusepath{stroke}%
\end{pgfscope}%
\begin{pgfscope}%
\pgfpathrectangle{\pgfqpoint{6.580625in}{0.387222in}}{\pgfqpoint{5.269375in}{5.244444in}}%
\pgfusepath{clip}%
\pgfsetrectcap%
\pgfsetroundjoin%
\pgfsetlinewidth{2.258437pt}%
\definecolor{currentstroke}{rgb}{0.260000,0.260000,0.260000}%
\pgfsetstrokecolor{currentstroke}%
\pgfsetdash{}{0pt}%
\pgfusepath{stroke}%
\end{pgfscope}%
\begin{pgfscope}%
\pgfpathrectangle{\pgfqpoint{6.580625in}{0.387222in}}{\pgfqpoint{5.269375in}{5.244444in}}%
\pgfusepath{clip}%
\pgfsetrectcap%
\pgfsetroundjoin%
\pgfsetlinewidth{2.258437pt}%
\definecolor{currentstroke}{rgb}{0.260000,0.260000,0.260000}%
\pgfsetstrokecolor{currentstroke}%
\pgfsetdash{}{0pt}%
\pgfusepath{stroke}%
\end{pgfscope}%
\begin{pgfscope}%
\pgfpathrectangle{\pgfqpoint{6.580625in}{0.387222in}}{\pgfqpoint{5.269375in}{5.244444in}}%
\pgfusepath{clip}%
\pgfsetrectcap%
\pgfsetroundjoin%
\pgfsetlinewidth{2.258437pt}%
\definecolor{currentstroke}{rgb}{0.260000,0.260000,0.260000}%
\pgfsetstrokecolor{currentstroke}%
\pgfsetdash{}{0pt}%
\pgfusepath{stroke}%
\end{pgfscope}%
\begin{pgfscope}%
\pgfpathrectangle{\pgfqpoint{6.580625in}{0.387222in}}{\pgfqpoint{5.269375in}{5.244444in}}%
\pgfusepath{clip}%
\pgfsetrectcap%
\pgfsetroundjoin%
\pgfsetlinewidth{2.258437pt}%
\definecolor{currentstroke}{rgb}{0.260000,0.260000,0.260000}%
\pgfsetstrokecolor{currentstroke}%
\pgfsetdash{}{0pt}%
\pgfusepath{stroke}%
\end{pgfscope}%
\begin{pgfscope}%
\pgfpathrectangle{\pgfqpoint{6.580625in}{0.387222in}}{\pgfqpoint{5.269375in}{5.244444in}}%
\pgfusepath{clip}%
\pgfsetrectcap%
\pgfsetroundjoin%
\pgfsetlinewidth{2.258437pt}%
\definecolor{currentstroke}{rgb}{0.260000,0.260000,0.260000}%
\pgfsetstrokecolor{currentstroke}%
\pgfsetdash{}{0pt}%
\pgfusepath{stroke}%
\end{pgfscope}%
\begin{pgfscope}%
\pgfpathrectangle{\pgfqpoint{6.580625in}{0.387222in}}{\pgfqpoint{5.269375in}{5.244444in}}%
\pgfusepath{clip}%
\pgfsetrectcap%
\pgfsetroundjoin%
\pgfsetlinewidth{2.258437pt}%
\definecolor{currentstroke}{rgb}{0.260000,0.260000,0.260000}%
\pgfsetstrokecolor{currentstroke}%
\pgfsetdash{}{0pt}%
\pgfusepath{stroke}%
\end{pgfscope}%
\begin{pgfscope}%
\pgfpathrectangle{\pgfqpoint{6.580625in}{0.387222in}}{\pgfqpoint{5.269375in}{5.244444in}}%
\pgfusepath{clip}%
\pgfsetrectcap%
\pgfsetroundjoin%
\pgfsetlinewidth{2.258437pt}%
\definecolor{currentstroke}{rgb}{0.260000,0.260000,0.260000}%
\pgfsetstrokecolor{currentstroke}%
\pgfsetdash{}{0pt}%
\pgfusepath{stroke}%
\end{pgfscope}%
\begin{pgfscope}%
\pgfpathrectangle{\pgfqpoint{6.580625in}{0.387222in}}{\pgfqpoint{5.269375in}{5.244444in}}%
\pgfusepath{clip}%
\pgfsetrectcap%
\pgfsetroundjoin%
\pgfsetlinewidth{2.258437pt}%
\definecolor{currentstroke}{rgb}{0.260000,0.260000,0.260000}%
\pgfsetstrokecolor{currentstroke}%
\pgfsetdash{}{0pt}%
\pgfusepath{stroke}%
\end{pgfscope}%
\begin{pgfscope}%
\pgfpathrectangle{\pgfqpoint{6.580625in}{0.387222in}}{\pgfqpoint{5.269375in}{5.244444in}}%
\pgfusepath{clip}%
\pgfsetrectcap%
\pgfsetroundjoin%
\pgfsetlinewidth{2.258437pt}%
\definecolor{currentstroke}{rgb}{0.260000,0.260000,0.260000}%
\pgfsetstrokecolor{currentstroke}%
\pgfsetdash{}{0pt}%
\pgfusepath{stroke}%
\end{pgfscope}%
\begin{pgfscope}%
\pgfpathrectangle{\pgfqpoint{6.580625in}{0.387222in}}{\pgfqpoint{5.269375in}{5.244444in}}%
\pgfusepath{clip}%
\pgfsetrectcap%
\pgfsetroundjoin%
\pgfsetlinewidth{2.258437pt}%
\definecolor{currentstroke}{rgb}{0.260000,0.260000,0.260000}%
\pgfsetstrokecolor{currentstroke}%
\pgfsetdash{}{0pt}%
\pgfusepath{stroke}%
\end{pgfscope}%
\begin{pgfscope}%
\pgfpathrectangle{\pgfqpoint{6.580625in}{0.387222in}}{\pgfqpoint{5.269375in}{5.244444in}}%
\pgfusepath{clip}%
\pgfsetrectcap%
\pgfsetroundjoin%
\pgfsetlinewidth{2.258437pt}%
\definecolor{currentstroke}{rgb}{0.260000,0.260000,0.260000}%
\pgfsetstrokecolor{currentstroke}%
\pgfsetdash{}{0pt}%
\pgfusepath{stroke}%
\end{pgfscope}%
\begin{pgfscope}%
\pgfpathrectangle{\pgfqpoint{6.580625in}{0.387222in}}{\pgfqpoint{5.269375in}{5.244444in}}%
\pgfusepath{clip}%
\pgfsetrectcap%
\pgfsetroundjoin%
\pgfsetlinewidth{2.258437pt}%
\definecolor{currentstroke}{rgb}{0.260000,0.260000,0.260000}%
\pgfsetstrokecolor{currentstroke}%
\pgfsetdash{}{0pt}%
\pgfusepath{stroke}%
\end{pgfscope}%
\begin{pgfscope}%
\pgfpathrectangle{\pgfqpoint{6.580625in}{0.387222in}}{\pgfqpoint{5.269375in}{5.244444in}}%
\pgfusepath{clip}%
\pgfsetrectcap%
\pgfsetroundjoin%
\pgfsetlinewidth{2.258437pt}%
\definecolor{currentstroke}{rgb}{0.260000,0.260000,0.260000}%
\pgfsetstrokecolor{currentstroke}%
\pgfsetdash{}{0pt}%
\pgfusepath{stroke}%
\end{pgfscope}%
\begin{pgfscope}%
\pgfpathrectangle{\pgfqpoint{6.580625in}{0.387222in}}{\pgfqpoint{5.269375in}{5.244444in}}%
\pgfusepath{clip}%
\pgfsetrectcap%
\pgfsetroundjoin%
\pgfsetlinewidth{2.258437pt}%
\definecolor{currentstroke}{rgb}{0.260000,0.260000,0.260000}%
\pgfsetstrokecolor{currentstroke}%
\pgfsetdash{}{0pt}%
\pgfusepath{stroke}%
\end{pgfscope}%
\begin{pgfscope}%
\pgfpathrectangle{\pgfqpoint{6.580625in}{0.387222in}}{\pgfqpoint{5.269375in}{5.244444in}}%
\pgfusepath{clip}%
\pgfsetrectcap%
\pgfsetroundjoin%
\pgfsetlinewidth{2.258437pt}%
\definecolor{currentstroke}{rgb}{0.260000,0.260000,0.260000}%
\pgfsetstrokecolor{currentstroke}%
\pgfsetdash{}{0pt}%
\pgfusepath{stroke}%
\end{pgfscope}%
\begin{pgfscope}%
\pgfpathrectangle{\pgfqpoint{6.580625in}{0.387222in}}{\pgfqpoint{5.269375in}{5.244444in}}%
\pgfusepath{clip}%
\pgfsetrectcap%
\pgfsetroundjoin%
\pgfsetlinewidth{2.258437pt}%
\definecolor{currentstroke}{rgb}{0.260000,0.260000,0.260000}%
\pgfsetstrokecolor{currentstroke}%
\pgfsetdash{}{0pt}%
\pgfusepath{stroke}%
\end{pgfscope}%
\begin{pgfscope}%
\pgfpathrectangle{\pgfqpoint{6.580625in}{0.387222in}}{\pgfqpoint{5.269375in}{5.244444in}}%
\pgfusepath{clip}%
\pgfsetrectcap%
\pgfsetroundjoin%
\pgfsetlinewidth{2.258437pt}%
\definecolor{currentstroke}{rgb}{0.260000,0.260000,0.260000}%
\pgfsetstrokecolor{currentstroke}%
\pgfsetdash{}{0pt}%
\pgfusepath{stroke}%
\end{pgfscope}%
\begin{pgfscope}%
\pgfpathrectangle{\pgfqpoint{6.580625in}{0.387222in}}{\pgfqpoint{5.269375in}{5.244444in}}%
\pgfusepath{clip}%
\pgfsetrectcap%
\pgfsetroundjoin%
\pgfsetlinewidth{2.258437pt}%
\definecolor{currentstroke}{rgb}{0.260000,0.260000,0.260000}%
\pgfsetstrokecolor{currentstroke}%
\pgfsetdash{}{0pt}%
\pgfusepath{stroke}%
\end{pgfscope}%
\begin{pgfscope}%
\pgfpathrectangle{\pgfqpoint{6.580625in}{0.387222in}}{\pgfqpoint{5.269375in}{5.244444in}}%
\pgfusepath{clip}%
\pgfsetrectcap%
\pgfsetroundjoin%
\pgfsetlinewidth{2.258437pt}%
\definecolor{currentstroke}{rgb}{0.260000,0.260000,0.260000}%
\pgfsetstrokecolor{currentstroke}%
\pgfsetdash{}{0pt}%
\pgfusepath{stroke}%
\end{pgfscope}%
\begin{pgfscope}%
\pgfsetrectcap%
\pgfsetmiterjoin%
\pgfsetlinewidth{0.803000pt}%
\definecolor{currentstroke}{rgb}{0.000000,0.000000,0.000000}%
\pgfsetstrokecolor{currentstroke}%
\pgfsetdash{}{0pt}%
\pgfpathmoveto{\pgfqpoint{6.580625in}{0.387222in}}%
\pgfpathlineto{\pgfqpoint{6.580625in}{5.631667in}}%
\pgfusepath{stroke}%
\end{pgfscope}%
\begin{pgfscope}%
\pgfsetrectcap%
\pgfsetmiterjoin%
\pgfsetlinewidth{0.803000pt}%
\definecolor{currentstroke}{rgb}{0.000000,0.000000,0.000000}%
\pgfsetstrokecolor{currentstroke}%
\pgfsetdash{}{0pt}%
\pgfpathmoveto{\pgfqpoint{11.850000in}{0.387222in}}%
\pgfpathlineto{\pgfqpoint{11.850000in}{5.631667in}}%
\pgfusepath{stroke}%
\end{pgfscope}%
\begin{pgfscope}%
\pgfsetrectcap%
\pgfsetmiterjoin%
\pgfsetlinewidth{0.803000pt}%
\definecolor{currentstroke}{rgb}{0.000000,0.000000,0.000000}%
\pgfsetstrokecolor{currentstroke}%
\pgfsetdash{}{0pt}%
\pgfpathmoveto{\pgfqpoint{6.580625in}{0.387222in}}%
\pgfpathlineto{\pgfqpoint{11.850000in}{0.387222in}}%
\pgfusepath{stroke}%
\end{pgfscope}%
\begin{pgfscope}%
\pgfsetrectcap%
\pgfsetmiterjoin%
\pgfsetlinewidth{0.803000pt}%
\definecolor{currentstroke}{rgb}{0.000000,0.000000,0.000000}%
\pgfsetstrokecolor{currentstroke}%
\pgfsetdash{}{0pt}%
\pgfpathmoveto{\pgfqpoint{6.580625in}{5.631667in}}%
\pgfpathlineto{\pgfqpoint{11.850000in}{5.631667in}}%
\pgfusepath{stroke}%
\end{pgfscope}%
\begin{pgfscope}%
\definecolor{textcolor}{rgb}{0.000000,0.000000,0.000000}%
\pgfsetstrokecolor{textcolor}%
\pgfsetfillcolor{textcolor}%
\pgftext[x=9.215312in,y=5.715000in,,base]{\color{textcolor}{\sffamily\fontsize{12.000000}{14.400000}\selectfont\catcode`\^=\active\def^{\ifmmode\sp\else\^{}\fi}\catcode`\%=\active\def%{\%}Distribuição de competence}}%
\end{pgfscope}%
\begin{pgfscope}%
\pgfsetbuttcap%
\pgfsetmiterjoin%
\definecolor{currentfill}{rgb}{1.000000,1.000000,1.000000}%
\pgfsetfillcolor{currentfill}%
\pgfsetfillopacity{0.800000}%
\pgfsetlinewidth{1.003750pt}%
\definecolor{currentstroke}{rgb}{0.800000,0.800000,0.800000}%
\pgfsetstrokecolor{currentstroke}%
\pgfsetstrokeopacity{0.800000}%
\pgfsetdash{}{0pt}%
\pgfpathmoveto{\pgfqpoint{10.209361in}{4.473465in}}%
\pgfpathlineto{\pgfqpoint{11.752778in}{4.473465in}}%
\pgfpathquadraticcurveto{\pgfqpoint{11.780556in}{4.473465in}}{\pgfqpoint{11.780556in}{4.501242in}}%
\pgfpathlineto{\pgfqpoint{11.780556in}{5.534444in}}%
\pgfpathquadraticcurveto{\pgfqpoint{11.780556in}{5.562222in}}{\pgfqpoint{11.752778in}{5.562222in}}%
\pgfpathlineto{\pgfqpoint{10.209361in}{5.562222in}}%
\pgfpathquadraticcurveto{\pgfqpoint{10.181584in}{5.562222in}}{\pgfqpoint{10.181584in}{5.534444in}}%
\pgfpathlineto{\pgfqpoint{10.181584in}{4.501242in}}%
\pgfpathquadraticcurveto{\pgfqpoint{10.181584in}{4.473465in}}{\pgfqpoint{10.209361in}{4.473465in}}%
\pgfpathlineto{\pgfqpoint{10.209361in}{4.473465in}}%
\pgfpathclose%
\pgfusepath{stroke,fill}%
\end{pgfscope}%
\begin{pgfscope}%
\pgfsetbuttcap%
\pgfsetmiterjoin%
\definecolor{currentfill}{rgb}{0.194608,0.453431,0.632843}%
\pgfsetfillcolor{currentfill}%
\pgfsetlinewidth{0.000000pt}%
\definecolor{currentstroke}{rgb}{0.000000,0.000000,0.000000}%
\pgfsetstrokecolor{currentstroke}%
\pgfsetstrokeopacity{0.000000}%
\pgfsetdash{}{0pt}%
\pgfpathmoveto{\pgfqpoint{10.237139in}{5.395583in}}%
\pgfpathlineto{\pgfqpoint{10.514917in}{5.395583in}}%
\pgfpathlineto{\pgfqpoint{10.514917in}{5.492805in}}%
\pgfpathlineto{\pgfqpoint{10.237139in}{5.492805in}}%
\pgfpathlineto{\pgfqpoint{10.237139in}{5.395583in}}%
\pgfpathclose%
\pgfusepath{fill}%
\end{pgfscope}%
\begin{pgfscope}%
\definecolor{textcolor}{rgb}{0.000000,0.000000,0.000000}%
\pgfsetstrokecolor{textcolor}%
\pgfsetfillcolor{textcolor}%
\pgftext[x=10.626028in,y=5.395583in,left,base]{\color{textcolor}{\sffamily\fontsize{10.000000}{12.000000}\selectfont\catcode`\^=\active\def^{\ifmmode\sp\else\^{}\fi}\catcode`\%=\active\def%{\%}Competência I}}%
\end{pgfscope}%
\begin{pgfscope}%
\pgfsetbuttcap%
\pgfsetmiterjoin%
\definecolor{currentfill}{rgb}{0.881863,0.505392,0.173039}%
\pgfsetfillcolor{currentfill}%
\pgfsetlinewidth{0.000000pt}%
\definecolor{currentstroke}{rgb}{0.000000,0.000000,0.000000}%
\pgfsetstrokecolor{currentstroke}%
\pgfsetstrokeopacity{0.000000}%
\pgfsetdash{}{0pt}%
\pgfpathmoveto{\pgfqpoint{10.237139in}{5.186164in}}%
\pgfpathlineto{\pgfqpoint{10.514917in}{5.186164in}}%
\pgfpathlineto{\pgfqpoint{10.514917in}{5.283387in}}%
\pgfpathlineto{\pgfqpoint{10.237139in}{5.283387in}}%
\pgfpathlineto{\pgfqpoint{10.237139in}{5.186164in}}%
\pgfpathclose%
\pgfusepath{fill}%
\end{pgfscope}%
\begin{pgfscope}%
\definecolor{textcolor}{rgb}{0.000000,0.000000,0.000000}%
\pgfsetstrokecolor{textcolor}%
\pgfsetfillcolor{textcolor}%
\pgftext[x=10.626028in,y=5.186164in,left,base]{\color{textcolor}{\sffamily\fontsize{10.000000}{12.000000}\selectfont\catcode`\^=\active\def^{\ifmmode\sp\else\^{}\fi}\catcode`\%=\active\def%{\%}Competência II}}%
\end{pgfscope}%
\begin{pgfscope}%
\pgfsetbuttcap%
\pgfsetmiterjoin%
\definecolor{currentfill}{rgb}{0.229412,0.570588,0.229412}%
\pgfsetfillcolor{currentfill}%
\pgfsetlinewidth{0.000000pt}%
\definecolor{currentstroke}{rgb}{0.000000,0.000000,0.000000}%
\pgfsetstrokecolor{currentstroke}%
\pgfsetstrokeopacity{0.000000}%
\pgfsetdash{}{0pt}%
\pgfpathmoveto{\pgfqpoint{10.237139in}{4.976746in}}%
\pgfpathlineto{\pgfqpoint{10.514917in}{4.976746in}}%
\pgfpathlineto{\pgfqpoint{10.514917in}{5.073968in}}%
\pgfpathlineto{\pgfqpoint{10.237139in}{5.073968in}}%
\pgfpathlineto{\pgfqpoint{10.237139in}{4.976746in}}%
\pgfpathclose%
\pgfusepath{fill}%
\end{pgfscope}%
\begin{pgfscope}%
\definecolor{textcolor}{rgb}{0.000000,0.000000,0.000000}%
\pgfsetstrokecolor{textcolor}%
\pgfsetfillcolor{textcolor}%
\pgftext[x=10.626028in,y=4.976746in,left,base]{\color{textcolor}{\sffamily\fontsize{10.000000}{12.000000}\selectfont\catcode`\^=\active\def^{\ifmmode\sp\else\^{}\fi}\catcode`\%=\active\def%{\%}Competência III}}%
\end{pgfscope}%
\begin{pgfscope}%
\pgfsetbuttcap%
\pgfsetmiterjoin%
\definecolor{currentfill}{rgb}{0.753431,0.238725,0.241667}%
\pgfsetfillcolor{currentfill}%
\pgfsetlinewidth{0.000000pt}%
\definecolor{currentstroke}{rgb}{0.000000,0.000000,0.000000}%
\pgfsetstrokecolor{currentstroke}%
\pgfsetstrokeopacity{0.000000}%
\pgfsetdash{}{0pt}%
\pgfpathmoveto{\pgfqpoint{10.237139in}{4.767328in}}%
\pgfpathlineto{\pgfqpoint{10.514917in}{4.767328in}}%
\pgfpathlineto{\pgfqpoint{10.514917in}{4.864550in}}%
\pgfpathlineto{\pgfqpoint{10.237139in}{4.864550in}}%
\pgfpathlineto{\pgfqpoint{10.237139in}{4.767328in}}%
\pgfpathclose%
\pgfusepath{fill}%
\end{pgfscope}%
\begin{pgfscope}%
\definecolor{textcolor}{rgb}{0.000000,0.000000,0.000000}%
\pgfsetstrokecolor{textcolor}%
\pgfsetfillcolor{textcolor}%
\pgftext[x=10.626028in,y=4.767328in,left,base]{\color{textcolor}{\sffamily\fontsize{10.000000}{12.000000}\selectfont\catcode`\^=\active\def^{\ifmmode\sp\else\^{}\fi}\catcode`\%=\active\def%{\%}Competência IV}}%
\end{pgfscope}%
\begin{pgfscope}%
\pgfsetbuttcap%
\pgfsetmiterjoin%
\definecolor{currentfill}{rgb}{0.578431,0.446078,0.699020}%
\pgfsetfillcolor{currentfill}%
\pgfsetlinewidth{0.000000pt}%
\definecolor{currentstroke}{rgb}{0.000000,0.000000,0.000000}%
\pgfsetstrokecolor{currentstroke}%
\pgfsetstrokeopacity{0.000000}%
\pgfsetdash{}{0pt}%
\pgfpathmoveto{\pgfqpoint{10.237139in}{4.557910in}}%
\pgfpathlineto{\pgfqpoint{10.514917in}{4.557910in}}%
\pgfpathlineto{\pgfqpoint{10.514917in}{4.655132in}}%
\pgfpathlineto{\pgfqpoint{10.237139in}{4.655132in}}%
\pgfpathlineto{\pgfqpoint{10.237139in}{4.557910in}}%
\pgfpathclose%
\pgfusepath{fill}%
\end{pgfscope}%
\begin{pgfscope}%
\definecolor{textcolor}{rgb}{0.000000,0.000000,0.000000}%
\pgfsetstrokecolor{textcolor}%
\pgfsetfillcolor{textcolor}%
\pgftext[x=10.626028in,y=4.557910in,left,base]{\color{textcolor}{\sffamily\fontsize{10.000000}{12.000000}\selectfont\catcode`\^=\active\def^{\ifmmode\sp\else\^{}\fi}\catcode`\%=\active\def%{\%}Competência V}}%
\end{pgfscope}%
\end{pgfpicture}%
\makeatother%
\endgroup%
}
\end{figure}

Em geral, a partir da figura \ref{fig:essay-br-extended-analysis}, é possível notar o mesmo padrão da versão anterior, mas com um pequeno acréscimo na frequência de notas finais maiores. Observa-se que esse aumento ocorreu, principalmente, devido ao incremento de notas 160 para a primeira competência, dado o aumento excepcional nessa categoria.

Os autores \citet{marinho-et-al-22} também disponibilizaram uma biblioteca de manipulação de dados para a Essay-BR estendida, cujo código-fonte pode ser acessado pelo GitHub\footnote{\url{https://github.com/lplnufpi/essay-br}}. Entretanto, devido a pequenas diferenças na formatação e leitura dos dados, utilizamos, neste trabalho, um \textit{fork}\footnote{\url{https://github.com/josemayer/essay-br}} adaptado da dependência.

A biblioteca atualizada tem o principal intuito de transformar o \textit{dataframe} estendido no mesmo formato do básico, contraindo as competências a uma única coluna. O principal objetivo é manter uma consistência na \textit{pipeline} de preparação dos dados, abordada na seção \ref{sec:preprocessing}. Escolhemos a mesma convenção do Essay-BR básico devido a sua utilização mais frequente para experimentos iniciais do projeto.

% \subsection{Essay-BR Customizada}
% add-custom-info: falar sobre a base de dados extraídas também do UOL, mas adequadamente limpada para evitar os problemas da essay-br. falar sobre as professoras utilizadas para avaliar a consistência.

\section{Pré-processamento}
\label{sec:preprocessing}

Antes de iniciar o treinamento dos modelos, é crucial realizar o pré-processamento dos dados brutos para usá-los de maneira adequada. Essa etapa envolve transformações em sua formatação e estrutura, visando facilitar a extração eficiente de padrões. No contexto deste trabalho, diversos tratamentos foram implementados, com foco principal na remodelação das redações e das avaliações para a otimização do processo de aprendizado.

Como passo inicial, realizamos a junção dos textos em uma única sequência, adotando uma formatação adequada para a entrada do modelo. Para isso, concatenamos todos os elementos da lista de parágrafos da redação, separando-os por caracteres marcadores de nova linha. A Figura \ref{fig:preprocessing-1} ilustra um exemplo de redação antes e depois do processo, destacando a simplificação da estrutura textual.

\begin{figure}[H]
    \caption{Exemplo de redação antes e depois da concatenação dos parágrafos.}
    \label{fig:preprocessing-1}
    \centering
    \resizebox{0.9\textwidth}{!}{\usetikzlibrary{decorations.pathreplacing, calc, arrows.meta, shapes, positioning}

\begin{tikzpicture}[
    remember picture,
    newline/.style={draw, circle, dotted, gray, inner sep=0.4pt},
]
    \definecolor{snsblue}{RGB}{213,228,236}
    \definecolor{snsred}{RGB}{247,210,205}
    \definecolor{snsyellow}{RGB}{247,225,200}
    \definecolor{snsgreen}{RGB}{223,236,201}

    \node[rectangle, fill=snsblue, rounded corners, text width=5cm] (int1) {\tiny\baselineskip=3pt No jogo eletrônico “League of Legends”, Viktor é um ciborgue que mora na  cidade de Piltover, uma das mais tecnológicas e sofisticadas de seu entorno. Na trama  desse personagem, as condições do meio em que vivia o fizeram substituir partes de seu  corpo, que lhe garantiam a condição de humano, por estruturas mecanizadas, que lhe forneciam maior eficiência nas atividades realizadas. Fora de tal microcosmo ficcional, é  notório que o âmbito do trabalho brasileiro vem adotando o mesmo postulado de Viktor  ao promover a automação em massa da produção, o que gera impactos nos índices de emprego da população. Desse modo, é fato que a automatização da indústria suscita uma problemática cada vez mais crescente no país, uma vez que tanto diminui o grau de pleno emprego na sociedade quanto seleciona cognitivamente os profissionais do futuro.\par};

    \node[rectangle, fill=snsred, rounded corners, text width=5cm, below of=int1, yshift=-1em] (dev1) {\tiny\baselineskip=3pt Em primeiro lugar, convém ressaltar que a exigência de maior eficiência na cadeia produtiva faz com que as empresas optem pela utilização do maquinário na mão de obra, diminuindo os empregos disponíveis à população. No meio rural, por exemplo, a mudança abrupta do modelo extensivo de produção ficou conhecida como Revolução Verde e ocorreu em meados da década de 70. O novo sistema instaurado no campo, a partir desse período, passou a utilizar o maquinário pesado para as atividades que, antes, eram realizadas por trabalhadores assalariados. Assim, do mesmo modo que no âmbito rural, uma crise decorrente da falta de empregos passa a vigorar no meio urbano, já que a indústria adota, de modo crescente, novas tecnologias que possam reduzir a demanda de contratados.\par};

    \node[rectangle, fill=snsyellow, rounded corners, text width=5cm, below of=dev1, yshift=-2em] (dev2) {\tiny\baselineskip=3pt Além disso, para lidar com o novo modelo, é evidente que há uma maior exigência de mão de obra qualificada no mercado. Para o filósofo Jürgen Habermas, isso é fruto da racionalidade instrumental do modelo capitalista de produção, que condiciona a mão de obra com base na lucratividade que ela fornecerá ao empregador. O problema é que boa parte da população não tem o integral acesso ao ensino superior ou técnico, fazendo essa parcela ser excluída do perfil adequado à atual indústria. Logo, à medida que a demanda por profissionais qualificados aumenta, o desemprego nas classes com menos acesso à educação segue a mesma tendência, inviabilizando uma maior equidade na realidade brasileira.\par};

    \node[rectangle, fill=snsgreen, rounded corners, text width=5cm, below of=dev2, yshift=-3em] (conc1) {\tiny\baselineskip=3pt Sendo assim, medidas devem ser tomadas para reverter o quadro promovido pela automação produtiva no país. Para tanto, cabe à Secretaria do Trabalho do Ministério da Economia, por meio da criação de um fundo monetário específico, a elaboração de uma política de fomento ao emprego que possa conciliar não só incentivos fiscais às empresas, mas também a promoção de cursos técnicos de manuseamento de equipamentos e outros aparatos surgidos com a automação industrial aos trabalhadores já empregados nesses ramos. Isso deve ser realizado com o apoio das secretarias a nível estadual, que podem aplicar a política tanto nas capitais quanto no interior das unidades federativas, objetivando, com a ação, promover a diminuição do desemprego e dos impasses decorrentes da automatização aos trabalhadores. Somente assim, será possível reverter a crise causada desde a Revolução Verde no mundo do trabalho e, além disso, atender as demandas, como as de Viktor de “League of Legends”, da indústria do Brasil no século XXI.\par};

    \draw [-, thick] ($ (int1.north west) + (-0.5em,-0.5em) $) -- ($ (int1.north west) + (-0.5em, 0.5em) $) -- ($ (int1.north east) + (0.5em, 0.5em) $) -- ($ (int1.north east) + (0.5em, -0.5em) $);
    \draw [-, thick] ($ (conc1.south west) + (-0.5em,0.5em) $) -- ($ (conc1.south west) + (-0.5em,-0.5em) $) -- ($ (conc1.south east) + (0.5em,-0.5em) $) -- ($ (conc1.south east) + (0.5em, 0.5em) $);

    \node [rectangle, fill=snsblue, rounded corners, text width=7cm, right of=dev1, xshift=16em, yshift=-3.5em] (merged) {\tiny\baselineskip=3pt No jogo eletrônico “League of Legends”, Viktor é um ciborgue que mora na  cidade de Piltover, uma das mais tecnológicas e sofisticadas de seu entorno. Na trama  desse personagem, as condições do meio em que vivia o fizeram substituir partes de seu  corpo, que lhe garantiam a condição de humano, por estruturas mecanizadas, que lhe forneciam maior eficiência nas atividades realizadas. Fora de tal microcosmo ficcional, é  notório que o âmbito do trabalho brasileiro vem adotando o mesmo postulado de Viktor  ao promover a automação em massa da produção, o que gera impactos nos índices de emprego da população. Desse modo, é fato que a automatização da indústria suscita uma problemática cada vez mais crescente no país, uma vez que tanto diminui o grau de pleno emprego na sociedade quanto seleciona cognitivamente os profissionais do futuro. \subnode{sub1}{\textcolor{gray}{$\backslash$n}} \par\vspace{1.5pt}
        \baselineskip=3pt Em primeiro lugar, convém ressaltar que a exigência de maior eficiência na cadeia produtiva faz com que as empresas optem pela utilização do maquinário na mão de obra, diminuindo os empregos disponíveis à população. No meio rural, por exemplo, a mudança abrupta do modelo extensivo de produção ficou conhecida como Revolução Verde e ocorreu em meados da década de 70. O novo sistema instaurado no campo, a partir desse período, passou a utilizar o maquinário pesado para as atividades que, antes, eram realizadas por trabalhadores assalariados. Assim, do mesmo modo que no âmbito rural, uma crise decorrente da falta de empregos passa a vigorar no meio urbano, já que a indústria adota, de modo crescente, novas tecnologias que possam reduzir a demanda de contratados. \subnode{sub2}{\textcolor{gray}{$\backslash$n}} \par\vspace{1.5pt}
        \baselineskip=3pt Além disso, para lidar com o novo modelo, é evidente que há uma maior exigência de mão de obra qualificada no mercado. Para o filósofo Jurgën Habermas, isso é fruto da racionalidade instrumental do modelo capitalista de produção, que condiciona a mão de obra com base na lucratividade que ela fornecerá ao empregador. O problema é que boa parte da população não tem o integral acesso ao ensino superior ou técnico, fazendo essa parcela ser excluída do perfil adequado à atual indústria. Logo, à medida que a demanda por profissionais qualificados aumenta, o desemprego nas classes com menos acesso à educação segue a mesma tendência, inviabilizando uma maior equidade na realidade brasileira. \subnode{sub3}{\textcolor{gray}{$\backslash$n}} \par\vspace{1.5pt}
        \baselineskip=3pt Sendo assim, medidas devem ser tomadas para reverter o quadro promovido pela automação produtiva no país. Para tanto, cabe à Secretaria do Trabalho do Ministério da Economia, por meio da criação de um fundo monetário específico, a elaboração de uma política de fomento ao emprego que possa conciliar não só incentivos fiscais às empresas, mas também a promoção de cursos técnicos de manuseamento de equipamentos e outros aparatos surgidos com a automação industrial aos trabalhadores já empregados nesses ramos. Isso deve ser realizado com o apoio das secretarias a nível estadual, que podem aplicar a política tanto nas capitais quanto no interior das unidades federativas, objetivando, com a ação, promover a diminuição do desemprego e dos impasses decorrentes da automatização aos trabalhadores. Somente assim, será possível reverter a crise causada desde a Revolução Verde no mundo do trabalho e, além disso, atender as demandas, como as de Viktor de “League of Legends”, da indústria do Brasil no século XXI. \subnode{sub4}{\textcolor{gray}{$\backslash$n}} \par};

\node[newline] at (sub1) {\tiny $\backslash$n};
\node[newline] at (sub2) {\tiny $\backslash$n};
\node[newline] at (sub3) {\tiny $\backslash$n};
\node[newline] at (sub4) {\tiny $\backslash$n};

\draw[-latex, line width=0.7em, orange!25] ($ (dev2.east) + (0em,2em) $) -- ($ (merged.west) + (0,1.2em) $);

\end{tikzpicture}
}
\end{figure}

Em seguida, normalizamos as notas atribuídas às competências, de modo que elas estejam no intervalo de 0 a 5. Para isso, dividimos cada pontuação por 40, diferença máxima de duas categorias de avaliação. Assim, cada aluno pode ter uma nota final inteira de 0 a 25, considerando a soma dos valores padronizados. É importante ressaltar que não normalizamos o campo \texttt{score}, dado que ele não é utilizado no treinamento.

Por fim, separamos as notas das competências em 5 colunas distintas, de \texttt{compI}, a \texttt{compV}, de modo que cada uma corresponda a uma pontuação. O objetivo principal é facilitar a construção dos cinco sistemas avaliadores isolados, extraindo apenas as colunas do texto e da nota da competência correspondente a cada treinamento.

\section{Modelagem}

A concretização deste trabalho envolveu a aplicação de técnicas avançadas de aprendizado profundo e NLP, abordagem crucial para a construção de sistemas avaliadores capazes de capturar nuances complexas da língua portuguesa. Além disso, a execução do projeto foi pautada em escolhas de ferramentas adequadas para o desenvolvimento das redes especialistas, levando em consideração o repertório de recursos voltados ao aprendizado de máquina.

Os sistemas de avaliação foram desenvolvidos por meio do processo de \textit{fine-tuning} do BERT, utilizando a biblioteca de \textit{transformers} do HuggingFace\footnote{\url{https://huggingface.co}}, uma plataforma aberta de \textit{machine learning} e ciência de dados que oferece uma variedade de repositórios contendo, especialmente, modelos de linguagem abertos. A biblioteca disponibiliza uma interface de rápido acesso a arquiteturas pré-treinadas, como o BERTimbau.

Além disso, para desenvolver a estrutura do modelo especialista, utilizamos o Keras\footnote{\url{https://keras.io/}}, que fornece uma abordagem de alto nível para projetar redes neurais. A biblioteca permite a especificação de diversos hiperparâmetros relevantes para o processo de treinamento, como a função de otimização e o tamanho dos lotes utilizados no processo, por exemplo.

Dentre as técnicas para a atribuição de nota, escolhemos a regressão, que é um método de aprendizado que mapeia uma entrada para um valor numérico contínuo. A decisão foi guiada pelo fato de que a distribuição das notas das competências é relativamente normal e unimodal, como mostrado na análise exploratória realizada anteriormente.

Por fim, escolhemos o erro médio quadrático (MSE) como métrica de desempenho prinicipal, utilizando-a para avaliar e orientar o treinamento dos modelos. Algumas outras técnicas de verificação de concordância, como a taxa de correspondência exata e a visualização das matrizes de confusão, também foram utilizadas para um julgamento empírico da eficácia dos sistemas avaliadores na prática.

\subsection{Arquitetura}

A arquitetura do modelo foi projetada concentrando-se em duas etapas fundamentais: a de codificação dos textos e a de avaliação propriamente dita. Inicialmente, estruturamos uma camada de entrada baseada no BERTimbau, cuja função primordial é receber os textos de redações e, por meio de sua capacidade de processamento contextual, gerar \textit{embeddings} que capturam informações semânticas e sintáticas. Essa camada representa uma fase crucial da estrutura, permitindo ao modelo uma compreensão rica e contextualizada dos textos submetidos.

Posteriormente, projetamos uma camada de atribuição de notas, usando uma configuração de redes neurais com a saída associada a uma função de regressão linear. Essa camada é responsável por transformar as representações obtidas na etapa anterior em notas numéricas, refletindo a pontuação atribuída em relação à competência em foco.

É essencial notar que, independentemente da competência, a arquitetura do sistema é generalizável o suficiente para aprender a atribuir as notas. Dessa forma, definimos a estrutura dos cinco modelos avaliadores como a mesma, com diferença apenas na variável de interesse no processo de treinamento. A Figura \ref{fig:full_architecture} abaixo permite a visualização da arquitetura de maneira esquemática, evidenciando as duas camadas de alto nível e a conexão entre elas.

\begin{figure}[H]
    \caption{Arquitetura geral de um sistema avaliador.}
    \label{fig:full_architecture}
    \centering
    \resizebox{\textwidth}{!}{\usetikzlibrary{positioning, calc, shapes.misc, shapes.geometric, backgrounds}

\begin{tikzpicture}[
    >=latex,
    bertemb/.style={rectangle, rounded corners, draw, inner sep=0.3cm, minimum width=1.5cm},
    neuron/.style={circle, draw, minimum width=1cm, minimum height=1cm},
    every path/.style={thick},
]
    \definecolor{blue-bert}{RGB}{201,218,248}
    \definecolor{darkblue-bert}{RGB}{80, 129, 188}
    \node[label={north:Entrada}] (essay) {\includegraphics[width=3em]{document-icon.pdf}};

    \node[draw, right of=essay, rotate=-90, minimum width=8cm, yshift=2em] (token_layer) {\textit{Tokens} da redação};


    \node[bertemb, right of=token_layer, xshift=2em, yshift=3.05cm] (emb1) {$E_{\text{CLS}}$};
    \node[bertemb, below of=emb1, yshift=-2.45em] (emb2) {$E_{1}$};
    \node[bertemb, below of=emb2, yshift=-2.45em] (emb3) {$E_{2}$};
    \node[bertemb, below of=emb3, yshift=-2.45em] (emb4) {$E_{N}$};
    \node[below of=emb3, yshift=-0.2em] (dots) {$\vdots$};

    \node[draw, circle, right of=emb1, xshift=3em] (enc11) {\tiny Encoder};
    \node[draw, circle, right of=emb2, xshift=3em] (enc12) {\tiny Encoder};
    \node[draw, circle, right of=emb3, xshift=3em] (enc13) {\tiny Encoder};
    \node[draw, circle, right of=emb4, xshift=3em] (enc14) {\tiny Encoder};

    \node[draw, circle, right of=enc11, xshift=2em] (enc21) {\tiny Encoder};
    \node[draw, circle, right of=enc12, xshift=2em] (enc22) {\tiny Encoder};
    \node[draw, circle, right of=enc13, xshift=2em] (enc23) {\tiny Encoder};
    \node[draw, circle, right of=enc14, xshift=2em] (enc24) {\tiny Encoder};

    \node[bertemb, right of=enc21, xshift=3em] (t1) {$T_{\text{CLS}}$};
    \node[bertemb, right of=enc22, xshift=3em] (t2) {$T_{1}$};
    \node[bertemb, right of=enc23, xshift=3em] (t3) {$T_{2}$};
    \node[bertemb, right of=enc24, xshift=3em] (t4) {$T_{N}$};
    \node[below of=t3, yshift=-0.2em] (dots) {$\vdots$};

    % fully connect encs
    \foreach \i in {1,2,3,4} {
        \foreach \j in {1,2,3,4} {
            \draw[->] (enc1\i.east) -- (enc2\j.west);
        }
    }

    \draw[rounded corners] ($ (emb1.north west) + (-0.3cm, 0.5cm) $) rectangle ($ (emb4.south east) + (0.3cm, -0.5cm) $);
    \draw[rounded corners] ($ (t1.north west) + (-0.3cm, 0.5cm) $) rectangle ($ (t4.south east) + (0.3cm, -0.5cm) $);

    % fully connect embs to encs1
    \foreach \i in {1,2,3,4} {
        \foreach \j in {1,2,3,4} {
            \draw[->] (emb\i.east) -- (enc1\j.west);
        }
    }

    % connect each enc2 to its corresponding t
    \foreach \i in {1,2,3,4} {
        \draw[->] (enc2\i.east) -- (t\i.west);
    }

    \draw[->] (token_layer.north) -- ($ ($ 0.5*(emb1.west) + 0.5*(emb4.west) $) + (-0.3cm, 0) $);
    \draw[->] (essay.east) -- (token_layer.south);


    \draw[rounded corners, draw=darkblue-bert] ($ (token_layer.south west) + (-0.3cm, 0.3cm) $) rectangle ($ (t4.south east) + (0.6cm, -0.8cm) $);

    % internal middle vert node
    \node (middle-vert) at ($ 0.5*(t2.south) + 0.5*(t3.north) $) {};

    \draw[->, darkblue-bert] ($ (middle-vert) + (3.2em, 0) $) -- ($ (middle-vert) + (4.3em, 0) $);

    % first hidden layer
    \node[draw, neuron, right of=middle-vert, xshift=4em, yshift=3.5cm] (neuron11) {};
    \node[draw, neuron, below of=neuron11, yshift=-0.5em] (neuron12) {};
    \node[draw, neuron, below of=neuron12, yshift=-0.5em] (neuron13) {};
    \node[draw, neuron, right of=middle-vert, xshift=4em, yshift=-3.5cm] (neuron14) {};
    \node[draw, neuron, above of=neuron14, yshift=0.5em] (neuron15) {};
    \node[draw, neuron, above of=neuron15, yshift=0.5em] (neuron16) {};
    \node[below of=neuron13, yshift=-0.3em] (dots1) {$\vdots$};

    % second hidden layer
    \node[draw, neuron, right of=neuron13, xshift=5em, yshift=2em] (neuron21) {};
    \node[draw, neuron, right of=neuron16, xshift=5em, yshift=-2em] (neuron22) {};
    \node[right of=dots1, xshift=5em] (dots2) {$\vdots$};

    % third hidden layer
    \node[draw, neuron, right of=neuron21, xshift=5em, yshift=2em] (neuron31) {};
    \node[draw, neuron, right of=neuron22, xshift=5em, yshift=-2em] (neuron32) {};
    \node[draw, neuron, below of=neuron31, yshift=-0.5em] (neuron33) {};
    \node[draw, neuron, above of=neuron32, yshift=0.5em] (neuron34) {};
    \node[right of=dots2, xshift=5em] {$\vdots$};

    % fully connect first hidden layer with second
    \foreach \i in {1,2,3,4,5,6} {
        \foreach \j in {1,2} {
            \draw[->] (neuron1\i) -- (neuron2\j);
        }
    }

    % fully connect second hidden layer with third
    \foreach \i in {1,2} {
        \foreach \j in {1,2,3,4} {
            \draw[->] (neuron2\i) -- (neuron3\j);
        }
    }

    \node[draw, neuron, right of=middle-vert, xshift=25em, label={north:Saída}] (output) {};

    % fully connect third hidden layer with output
    \foreach \i in {1,2,3,4} {
        \draw[->] (neuron3\i) -- (output);
    }

    \draw[rounded corners, dashed] ($ (neuron11.north west) + (-0.5cm, 0.4cm) $) rectangle ($ (neuron14.south east) + (16em, -0.4cm) $);

    \node[above of=emb1, text=darkblue-bert, yshift=1.6em, xshift=-3em] (bert-label) {\sffamily \textbf{BERTimbau}};
    \node[above of=neuron11, yshift=0.3em, xshift=4em] (nn-label) {\sffamily \textbf{Rede neural especialista}};

\end{tikzpicture}
}
\end{figure}

Quanto à configuração da rede neural especialista, optamos por uma estrutura composta por três camadas ocultas, com dimensões definidas como 3000, 2000 e 2500 neurônios, respectivamente. A escolha dos tamanhos foi orientada pela necessidade de criar uma arquitetura suficientemente complexa para capturar as relações presentes nos textos das redações. Entretanto, uma quantidade grande de parâmetros pode ocasionar em problemas de sobreajustes do modelo.

Introduzimos, para mitigar esse risco, a função de regularização de \textit{dropout} entre todas as camadas da rede neural especialista, com fator de $10\%$. Essa técnica tem o intuito de agir aleatoriamente, ignorando uma quantidade de pesos proporcional ao fator nas transições da etapa de treinamento. Dessa forma, ela garante uma maior generalização do modelo, contribuindo para melhor desempenho em dados não vistos.

Quanto aos cálculos realizados em cada camada, escolhemos duas funções de ativação com caráter não linear para a efetiva extração de características das redações. A primeira delas, SeLU, é eficaz em lidar com problemas de gradiente desvanecente e em proporcionar uma convergência mais estável durante o treinamento. A outra, sigmoide (\ref{eq:sigmoid}), é uma escolha comum para modelagem de problemas de classificação binária e pode ser efetiva em lidar com esse tipo de características dos textos. Por fim, para a camada de saída, utilizamos a função linear com o intuito de obter o resultado da regressão final. A SeLU é representada pela equação \ref{eq:selu}, em que $\alpha = 1,67326324$ e $\lambda = 1,05070098$.

\begin{equation}
    \label{eq:selu}
    \text{f}(x) = \lambda
    \begin{cases}
        x, & \text{se } x > 0 \\
        \alpha (e^{x} - 1), & \text{se } x \leq 0
    \end{cases}
\end{equation}

\subsection{Implementação}
\label{subsec:implementation}

A implementação da arquitetura é feita a partir do Keras, com uma classe de hipermodelos. Essa classe é útil, em especial, para a otimização de parâmetros que devem ser escolhidos de forma antecipada pelos projetistas, como a dimensão das camadas ocultas, por exemplo. Nela, adicionamos um atributo especial para armazenar o modelo do BERTimbau e redefinimos os métodos de construção da rede e de treino. O trecho de código em Python \ref{alg:hypermodel} mostra, na prática, a implementação.

\begin{program}
    \index{EssayHyperModel}
    \centering
    \caption{Implementação em Python da classe \texttt{EssayHyperModel}}
    \label{alg:hypermodel}
\begin{lstlisting}[language=Python]
import tensorflow as tf
import keras_tuner as kt
from tf.keras.optimizers import Adam
from tf.keras.layers import Dense, Dropout, Input
from tf.keras.models import Model

class EssayHyperModel(kt.HyperModel):
    def __init__(self, bert):
      self.bert = bert

    def build(self, hp):
        input_ids = Input(shape=(None,), dtype=tf.int32, name="input_ids")
        embedding = self.bert({'input_ids': input_ids})['pooler_output']

        x = Dense(3000, activation=hp.Choice('a_l1', values=['selu', 'sigmoid']))(embedding)
        x = Dropout(0.1)(x)
        x = Dense(2000, activation=hp.Choice('a_l2', values=['selu', 'sigmoid']))(x)
        x = Dropout(0.1)(x)
        x = Dense(2500, activation=hp.Choice('a_l3', values=['selu', 'sigmoid']))(x)
        x = Dropout(0.1)(x)

        output = Dense(1, activation='linear')(x)

        model = Model(inputs=input_ids, outputs=output)

        optimizer = Adam(learning_rate=hp.Choice('lr', values=[2e-3, 2e-5]))
        model.compile(optimizer=optimizer, loss='mean_squared_error', metrics=['mse'])

        return model

    def fit(self, hp, model, *args, **kwargs):
        return model.fit(
            *args,
            batch_size=hp.Choice("bs", [2, 3, 4]),
            **kwargs,
        )
\end{lstlisting}
\end{program}

A primeira camada da rede consiste em uma entrada que recebe sequências de identificadores de \textit{tokens} dos textos, provenientes do processo de tokenização. Essa camada permite o processamento de sequências com comprimentos variáveis, sendo útil para lidar com redações de diferentes tamanhos. Os \textit{tokens} de entrada são então processados pelo BERTimbau, gerando \textit{embeddings} correspondentes à saída do \textit{pooler} do modelo.

Em seguida, na estruturação da rede especialista, define-se uma arquitetura de três camadas densas, cada uma seguida por uma etapa de \textit{dropout} para regularização. A escolha da função de ativação de cada nível é realizada de forma dinâmica, permitindo a personalização durante o processo de treinamento do modelo. Nas linhas 15, 17 e 19 são também definidas as dimensões das camadas densas: 3000, 2000 e 2500, respectivamente. A camada de saída, definida na linha 22, possui uma única unidade, utilizando ativação linear.

O otimizador escolhido para treinamento é o Adam, conforme a linha 26, e sua taxa de aprendizado é determinada como um dos parâmetros do espaço de busca hiperparamétrica. A função de perda adotada, explicitada na linha 27, é o erro quadrático médio (MSE), com o intuito de penalizar desvios significativos entre as predições do modelo e os valores de validação. O cálculo dessa métrica é dado pela equação \ref{eq:mse}, em que $y_{i}$ são os valores esperados e $\hat{y}_{i}$ são os valores obtidos pelo sistema avaliador, para $i \in \{1, \cdots, n\}$.

\begin{equation}
    \label{eq:mse}
    \text{MSE} = \frac{1}{n} \sum_{i=1}^{n} (y_i - \hat{y}_i)^2
\end{equation}

O processo de treinamento do modelo é configurado para ajustar-se aos parâmetros definidos dinamicamente pelo otimizador, técnica chamada de \textit{hypertuning}. O tamanho do lote, conforme a linha 34, também é tratado como um hiperparâmetro. Na seção \ref{subsec:hyperparameter-tuning}, explicaremos o processo de \textit{hypertuning} sobre um espaço de possibilidades limitado.

\subsection{Otimização de Hiperparâmetros}
\label{subsec:hyperparameter-tuning}

% falar sobre a biblioteca do keras para tuning (HyperModel, Tuner, etc.)

% explicar o hypertuning e qual foram os parametros escolhidos p tunar, qual range, qual a forma de busca (GridSearch), etc.

A otimização de hiperparâmetros é uma técnica utilizada para encontrar a melhor configuração de parâmetros pré-definidos de um modelo de aprendizado de máquina. Essa abordagem é essencial para a obtenção de resultados satisfatórios, já que alguns fatores definidos a nível de arquitetura são muito influentes no processo de treinamento.

Para realizar a otimização, utilizamos a biblioteca Keras Tuner, que fornece uma interface de alto nível para o processo de \textit{tuning}. Com ela, podemos definir um espaço de busca, que contém os parâmetros a serem otimizados, e um algoritmo de exploração, que define a estratégia de pesquisa nesse espaço. No caso do sistema especialista, optamos por realizar uma varredura completa nas possibilidades de hiperparâmetros, utilizando o algoritmo de busca em grade que analisa todas as combinações possíveis de variáveis.

O espaço de busca é definido ainda na criação da arquitetura dos hipermodelos, utilizando um método de escolha para definir as possibilidades desejadas. Na etapa de treino, cada iteração de configuração distinta é definida como uma \textit{trial}, em que a rede neural é otimizada normalmente em um dado número de épocas e, depois, avaliada em um conjunto de validação. Essa avaliação é a responsável por indicar se os hiperparâmetros da \textit{trial} melhoraram ou pioraram o resultado obtido. Ao fim de todo o processo, a melhor configuração é recuperada para a construção de uma nova rede.

No caso do sistema avaliador, os hiperparâmetros escolhidos foram a taxa de aprendizado (\texttt{lr}), o tamanho do lote (\texttt{bs}) e as funções de ativação das camadas ocultas (\texttt{a\_l1}, \texttt{a\_l2} e \texttt{a\_l3}). A taxa de aprendizado é o fator que determina o tamanho do passo dado pelo otimizador de treino em cada iteração, enquanto que o tamanho do lote é a quantidade de instâncias utilizadas para o ajuste dos pesos do modelo a cada época. A tabela \ref{tab:hyperparameters-to-tune} lista cada um dos hiperparâmetros da classe \texttt{EssayHyperModel} (\autoref{alg:hypermodel}) e suas possibilidades de escolha.

\begin{table}[H]
    \centering
    \caption{Hiperparâmetros da classe \texttt{EssayHyperModel} escolhidos para otimização, com seus respectivos espaços de escolha.}
    \label{tab:hyperparameters-to-tune}
    \begin{tabular}{ll}
        \toprule
        \textbf{Hiperparâmetro} & \textbf{Valores} \\
        \midrule
        Taxa de aprendizado (\texttt{lr}) & $2 \times 10^{-3}$ e $2 \times 10^{-5}$ \\
        Tamanho do lote (\texttt{bs}) & 2, 3 e 4 \\
        Função de ativação da camada 1 (\texttt{a\_l1}) & SeLU e Sigmoide \\
        Função de ativação da camada 2 (\texttt{a\_l2}) & SeLU e Sigmoide \\
        Função de ativação da camada 3 (\texttt{a\_l3}) & SeLU e Sigmoide \\
        \bottomrule
    \end{tabular}
\end{table}


\subsection{Avaliação}

No contexto do aprendizado de máquina, a avaliação de modelos é uma etapa crucial para a verificação de sua eficácia. Neste trabalho, utilizamos uma abordagem de validação cruzada, que consiste em dividir a base de dados em conjuntos de treino e teste, de modo que o modelo seja treinado em uma parcela dos dados e avaliado em outra independente. Essa técnica é fundamental para a verificação da capacidade de generalização dos sistemas avaliadores, já que permite a análise de seu desempenho em dados não vistos.

A divisão das entradas foi realizada a partir da biblioteca de manipulação disponibilizada pelos autores da Essay-BR (\cite{marinho-et-al-21}; \cite{marinho-et-al-22}), de modo que o conjunto de teste representa 15\% das instâncias totais, totalmente isoladas daquelas usadas no processo de treino. Além disso, como a saída da rede especialista retorna um valor contínuo, realizamos o tratamento das notas inferidas com uma função de arredondamento para um número inteiro mais próximo, a fim de adequá-lo ao formato da parcela de testes.

A avaliação dos modelos foi realizada com base nas métricas de MSE (\autoref{eq:mse}), proporção de correspondência exata, divergência e \textit{quadratic weighted kappa} (QWK). A seguir, detalharemos cada uma delas.

\subsubsection{Erro Quadrático Médio (MSE)}

O erro quadrático médio (MSE) é uma métrica de desempenho comumente utilizada para avaliar modelos de regressão. Ela é definida como a média dos erros quadráticos entre as predições e os valores esperados, conforme a equação \ref{eq:mse}.

\subsubsection{Proporção de Correspondência Exata}

A proporção de correspondência exata é uma métrica de desempenho utilizada para avaliar modelos de classificação com múltiplas categorias. Ela é definida como a taxa de predições corretas em relação ao total de predições. Considerando $\hat{y}_{i}$ como o $i$-ésimo valor inferido e $y_{i}$ como o $i$-ésimo valor real, definimos a proporção de correspondência exata pela equação \ref{eq:exact_match_ratio}, em que $I(\hat{y}_{i} = y_{i})$ vale 1 caso os dois valores sejam iguais e 0 em caso contrário.

\begin{equation}
    \label{eq:exact_match_ratio}
    \text{MR} = \frac{1}{n} \sum_{i=0}^{n} I(\hat{y}_{i} = y_{i})
\end{equation}

\subsubsection{Divergência}

A métrica de divergência avalia a consistência interna da nota de uma competência, inspirada no processo de correção das redações do ENEM. Ela é calculada como a proporção de avaliações que diferiram em mais de 2 pontos para a saída do sistema avaliador. Essa convenção é feita com base na definição de \citet{cartilha-redacao}, que classifica como divergente duas correções cujas notas de competência difiram em mais de 100 pontos (2 pontos, na abordagem normalizada). Considerando a função $I$:

\begin{equation}
    I(\hat{y}_{i} \lessgtr y_{i}) = \begin{cases}
        1, & \text{se } |\hat{y}_{i} - y_{i}| > 2 \\
        0, & \text{caso contrário}
    \end{cases}
\end{equation}

em que $\hat{y_{i}}$ é o $i$-ésimo valor inferido e $y_{i}$ é o $i$-ésimo valor real, a divergência pode ser calculada pela equação \ref{eq:divergence}.

\begin{equation}
    \label{eq:divergence}
    \text{D} = \frac{1}{n} \sum_{i=0}^{n} I(\hat{y}_{i} \lessgtr y_{i})
\end{equation}

\subsubsection{\textit{Quadratic Weighted Kappa} (QWK)}

O \textit{quadratic weighted kappa} (QWK) é uma métrica de desempenho usada para avaliar modelos de classificação multicategóricos, introduzida por \citet{cohen-1968-qwk}. Essa métrica é definida através da comparação entre a concordância observada $O$ entre as classificações e a concordância esperada ao acaso $E$, levando em consideração os pesos $w_{ij}$ que refletem a importância das discordâncias entre diferentes categorias.

A fórmula do QWK é expressa pela equação \ref{eq:qwk}, em que $O_{ij}$ é o número normalizado de pares de instâncias classificadas como $i$ por um avaliador e $j$ por outro, $E_{ij}$ é o número normalizado esperado de pares de instâncias que seriam classificadas como $i$ por um avaliador e como $j$ por outro, assumindo que não houvesse correspondência entre as avaliações, e $w_{ij}$ são os pesos atribuídos aos pares de classes $i$ e $j$, considerando a discordância quadrática relativa entre as categorias.

\begin{equation}
    \label{eq:qwk}
    QWK = 1 - \frac{\sum_{i,j}w_{ij}O_{ij}}{\sum_{i,j}w_{ij}E_{ij}}
\end{equation}

Na implementação do algoritmo de cálculo do QWK, utiliza-se matrizes $O$ e $E$, de dimensões $i \times j$, para armazenar as informações de $O_{ij}$ e $E_{ij}$. A matriz $O$ é calculada a partir da matriz de confusão, enquanto que a matriz $E$ é calculada pela distribuição marginal dos valores inferidos em relação aos valores reais.

Essa métrica varia no intervalo de -1 a 1, onde 1 indica concordância perfeita, 0 indica concordância ao acaso e valores negativos indicam discordância pior do que a esperada aleatoriamente. O QWK é particularmente útil em problemas onde as categorias são uma sequência, como no caso das notas das competências, considerando não apenas a correspondência direta entre as previsões e os rótulos reais, mas também a conformidade em termos de ordem.

De modo geral, podemos definir intervalos de interpretação para o QWK, conforme a tabela \ref{tab:qwk-interpretation}.

\begin{table}[H]
    \centering
    \caption{Intervalos de interpretação para valores do \textit{quadratic weighted kappa} (QWK). Tabela extraída de \url{https://www.kaggle.com/code/prashant111/simple-explanation-of-quadratic-weighted-kappa}.}
    \label{tab:qwk-interpretation}
    \begin{tabular}{ll}
        \toprule
        \textbf{QWK} & \textbf{Interpretação} \\
        \midrule
        $-1$ & Discordância completa \\
        $0$ & Concordância ao acaso \\
        $0 - 0,2$ & Concordância precária \\
        $0,2 - 0,4$ & Concordância moderada \\
        $0,4 - 0,6$ & Concordância boa \\
        $0,6 - 0,8$ & Concordância muito boa \\
        $0,8 - 1$ & Concordância quase perfeita \\
        $1$ & Concordância perfeita \\
        \bottomrule
    \end{tabular}
\end{table}

\section{Treinamento}

O treinamento é a etapa principal de ajuste da arquitetura, em que os sistemas aprendem os padrões de avaliação a partir da base de redações anotada. A biblioteca TensorFlow\footnote{\url{https://www.tensorflow.org}} foi a escolha principal para esse estágio, dado seu suporte eficiente para operações de álgebra linear e sua facilidade de utilização. Detalharemos, a seguir, o processo de otimização dos sistemas avaliadores, definindo melhor as configurações utilizadas e o ambiente de execução.

\subsection{Configurações}
\label{subsec:training-configurations}

No processo de treinamento, foi utilizado um conjunto de dados complementar à divisão de testes da Essay-BR, com 85\% do total de instâncias. Desse valor, 70\% da base original foi alocada para a fase de otimização do modelo, enquanto que os 15\% restantes foram reservados para a validação e ajuste da estrutura. No caso das redes especialistas, tal ajuste ocorreu por meio do uso do MSE como função de perda.

Alguns dos parâmetros de treinamento foram escolhidos de maneira automática, com o intuito de otimizar o desempenho geral dos avaliadores, como o tamanho do lote (\textit{batch size}) e a taxa de aprendizado. Além disso, o Adam (\textit{Adaptative Moment Estimation}), proposto por \citet{kingma2017adam}, foi definido como otimizador da rede neural, devido à sua eficácia em combinar as vantagens do método de gradiente estocástico (SGD) com a adaptação da taxa de aprendizado ao longo do treino.

Durante a busca dos hiperparâmetros, foram utilizados dois \textit{callbacks} personalizados para monitoramento e limpeza. O primeiro, \texttt{LogCallback}, tem o intuito de registrar as avaliações de cada \textit{trial} e o segundo, \texttt{DeleteCallback}, visa remover salvamentos parciais de redes que possam sobrecarregar o disco. Cada iteração treina um modelo distinto por 5 épocas, atualizando o conjunto ótimo de hiperparâmetros ao longo do processo.

Ao fim da busca, os melhores valores são usados para a construção de um novo modelo, que é treinado em 50 épocas. Nessa etapa, utilizamos somente o \textit{callback} \texttt{ModelCheckpoint} do Keras, a fim de salvar a rede com a melhor validação dentre todo o processo de otimização. Isso é feito para prevenir a ocorrência de sobreajustes, já que um número de épocas muito alto pode levar a tal risco. O trecho de código \ref{alg:search-and-train} abaixo demonstra a implementação das configurações abordadas, suprimindo algumas informações com ``\texttt{...}'' para fins de concisão.

\begin{program}
    \index{TrainHypertuning}
    \centering
    \caption{Algoritmo do treinamento e busca de hiperparâmetros}
    \label{alg:search-and-train}
\begin{lstlisting}[language=Python]
hypermodel = EssayHyperModel(bert)

tuner = kt.GridSearch(hypermodel, objective='val_loss', executions_per_trial=1, ...)
tuner.search(..., epochs=5, callbacks=[LogCallback(...), DeleteCallback()])

best_model_hps = tuner.get_best_hyperparameters(num_trials=1)[0]
best_model = tuner.hypermodel.build(best_model_hps)

history = best_model.fit(
    ...,
    epochs=50,
    batch_size=best_model_hps.get('batch_size'),
    callbacks=[ModelCheckpoint(filepath=(...), monitor='val_loss', save_best_only=True)]
)
\end{lstlisting}
\end{program}

\subsection{Ambiente}

O treinamento dos modelos foi realizado em quatro ambientes controlados, todos com \textit{hardwares} adequados para lidar com a tarefa de processamento massivo. O primeiro deles, o Google Colab\footnote{\url{https://colab.research.google.com/}}, foi utilizado no início do projeto para a realização de experimentos iniciais e para a construção de redes mais simples em \textit{notebooks}, documentos interativos que permitem execução de código. Já o segundo, a Rede GNU/Linux do IME-USP\footnote{\url{https://www.linux.ime.usp.br/}}, participou das etapas primitivas de \textit{hypertuning} das redes especialistas e foi crucial para a transição do projeto para uma versão puramente em Python.

Na versão mais madura do trabalho, utilizamos unidades com maior poder computacional para a realização dos treinamentos e otimização de hiperparâmetros definivos. Uma delas, chamada PGM, é uma máquina física da seção de pesquisas do Departamento de Ciência da Computação (DCC) do IME-USP. A outra é uma máquina virtual hospedada no serviço Google Cloud\footnote{\url{https://cloud.google.com/}}.

Os quatro ambientes dispunham de GPUs que aceleraram o processo de treinamento, reduzindo o tempo de execução das iterações de \textit{hypertuning}. Abaixo, são listadas as especificações de cada máquina usada.

\begin{itemize}
    \item \textbf{Google Colab}: GPU Tesla T4, 16GB de GPU RAM.
    \item \textbf{Rede GNU/Linux}: GPU NVIDIA GeForce RTX 3060, 12GB de GPU RAM.
    \item \textbf{PGM}: GPU NVIDIA GeForce GTX 1080 Ti, 12GB de GPU RAM.
    \item \textbf{Google Cloud}: GPU NVIDIA Tesla P100, 16GB de GPU RAM.
\end{itemize}

\chapter{Experimentos}
\label{chap:experiments}

Com base na metodologia detalhada no capítulo anterior, uma série de experimentos foi conduzida com o propósito de avaliar o desempenho dos sistemas de correção automática desenvolvidos. Durante essa etapa, exploramos diversas configurações treinando modelos não só com a versão estendida, mas também com a versão básica da Essay-BR --- dada a variação dos conjuntos de dados.

Além disso, para garantir uma comparação adequada, foram incluídos experimentos de controle que utilizam hiperparâmetros fixados e conhecidos para o treinamento dos modelos, provenientes de explorações iniciais do projeto. Essa abordagem permite metrificar o impacto do \textit{hypertuning} no desenvolvimento de redes especialistas de avaliação.

Por fim, visando explorar a capacidade de aprendizado dos modelos, foram utilizados gráficos que denotam a evolução de cada treinamento realizado. Isso proporciona uma compreensão mais abrangente da influência de escolhas estruturais no processo de otimização, além de indicar caminhos possíveis para contornar o subdesempenho dos sistemas de correção automática.

\section{Treinamento}

Nos experimentos de treinamento, optamos por manter o registro do histórico de desempenho apenas do conjunto de dados estendido, para fins de concisão. No entanto, ressaltamos que a versão básica da Essay-BR também foi utilizada para a criação de outras de redes, cujos resultados são usados para comparações gerais na seção \ref{sec:evaluate}.

A versão do BERTimbau escolhida como base para a implementação das redes especialistas foi a \texttt{bert-base-portuguese-cased} do HuggingFace, que possui o mesmo número de parâmetros do $\text{BERT}_{\textbf{\text{BASE}}}$, abordado na seção \ref{subsec:bert_architecture}, e é sensitiva a letras maíusculas e minúsculas nos textos de entrada.

Em geral, foram realizados experimentos que levaram em conta hiperparâmetros otimizados e fixos. A seguir, detalharemos os resultados obtidos no processo de \textit{hypertuning}, seguido da análise comparativa da função de perda para ambos os casos de treino.

\subsection{Otimização de Hiperparâmetros}

Na otimização dos hiperparâmetros, todas as combinações possíveis de valores foram testadas, em um total de 48 \textit{trials} ($2 \cdot 2 \cdot 2 \cdot 2 \cdot 3$). A melhor configuração das redes é representada pelo conjunto de variáveis que fazem a função de perda ter o menor valor possível entre as iterações.

Nas seções \ref{subsec:exp-hyp-c1}, \ref{subsec:exp-hyp-c2}, \ref{subsec:exp-hyp-c3}, \ref{subsec:exp-hyp-c4} e \ref{subsec:exp-hyp-c5} exploraremos melhor o processo de otimização para os sistemas avaliadores de cada competência.

\subsubsection{Competência I}
\label{subsec:exp-hyp-c1}

\begin{figure}[H]
    \resizebox{0.5\textwidth}{!}{%% Creator: Matplotlib, PGF backend
%%
%% To include the figure in your LaTeX document, write
%%   \input{<filename>.pgf}
%%
%% Make sure the required packages are loaded in your preamble
%%   \usepackage{pgf}
%%
%% Also ensure that all the required font packages are loaded; for instance,
%% the lmodern package is sometimes necessary when using math font.
%%   \usepackage{lmodern}
%%
%% Figures using additional raster images can only be included by \input if
%% they are in the same directory as the main LaTeX file. For loading figures
%% from other directories you can use the `import` package
%%   \usepackage{import}
%%
%% and then include the figures with
%%   \import{<path to file>}{<filename>.pgf}
%%
%% Matplotlib used the following preamble
%%   \def\mathdefault#1{#1}
%%   \everymath=\expandafter{\the\everymath\displaystyle}
%%   
%%   \usepackage{fontspec}
%%   \setmainfont{DejaVuSerif.ttf}[Path=\detokenize{/Users/josemayer/Documents/Pacotes/mambaforge/lib/python3.10/site-packages/matplotlib/mpl-data/fonts/ttf/}]
%%   \setsansfont{DejaVuSans.ttf}[Path=\detokenize{/Users/josemayer/Documents/Pacotes/mambaforge/lib/python3.10/site-packages/matplotlib/mpl-data/fonts/ttf/}]
%%   \setmonofont{DejaVuSansMono.ttf}[Path=\detokenize{/Users/josemayer/Documents/Pacotes/mambaforge/lib/python3.10/site-packages/matplotlib/mpl-data/fonts/ttf/}]
%%   \makeatletter\@ifpackageloaded{underscore}{}{\usepackage[strings]{underscore}}\makeatother
%%
\begingroup%
\makeatletter%
\begin{pgfpicture}%
\pgfpathrectangle{\pgfpointorigin}{\pgfqpoint{6.400000in}{4.800000in}}%
\pgfusepath{use as bounding box, clip}%
\begin{pgfscope}%
\pgfsetbuttcap%
\pgfsetmiterjoin%
\definecolor{currentfill}{rgb}{1.000000,1.000000,1.000000}%
\pgfsetfillcolor{currentfill}%
\pgfsetlinewidth{0.000000pt}%
\definecolor{currentstroke}{rgb}{1.000000,1.000000,1.000000}%
\pgfsetstrokecolor{currentstroke}%
\pgfsetdash{}{0pt}%
\pgfpathmoveto{\pgfqpoint{0.000000in}{0.000000in}}%
\pgfpathlineto{\pgfqpoint{6.400000in}{0.000000in}}%
\pgfpathlineto{\pgfqpoint{6.400000in}{4.800000in}}%
\pgfpathlineto{\pgfqpoint{0.000000in}{4.800000in}}%
\pgfpathlineto{\pgfqpoint{0.000000in}{0.000000in}}%
\pgfpathclose%
\pgfusepath{fill}%
\end{pgfscope}%
\begin{pgfscope}%
\pgfsetbuttcap%
\pgfsetmiterjoin%
\definecolor{currentfill}{rgb}{1.000000,1.000000,1.000000}%
\pgfsetfillcolor{currentfill}%
\pgfsetlinewidth{0.000000pt}%
\definecolor{currentstroke}{rgb}{0.000000,0.000000,0.000000}%
\pgfsetstrokecolor{currentstroke}%
\pgfsetstrokeopacity{0.000000}%
\pgfsetdash{}{0pt}%
\pgfpathmoveto{\pgfqpoint{0.800000in}{0.528000in}}%
\pgfpathlineto{\pgfqpoint{5.760000in}{0.528000in}}%
\pgfpathlineto{\pgfqpoint{5.760000in}{4.224000in}}%
\pgfpathlineto{\pgfqpoint{0.800000in}{4.224000in}}%
\pgfpathlineto{\pgfqpoint{0.800000in}{0.528000in}}%
\pgfpathclose%
\pgfusepath{fill}%
\end{pgfscope}%
\begin{pgfscope}%
\pgfsetbuttcap%
\pgfsetroundjoin%
\definecolor{currentfill}{rgb}{0.000000,0.000000,0.000000}%
\pgfsetfillcolor{currentfill}%
\pgfsetlinewidth{0.803000pt}%
\definecolor{currentstroke}{rgb}{0.000000,0.000000,0.000000}%
\pgfsetstrokecolor{currentstroke}%
\pgfsetdash{}{0pt}%
\pgfsys@defobject{currentmarker}{\pgfqpoint{0.000000in}{-0.048611in}}{\pgfqpoint{0.000000in}{0.000000in}}{%
\pgfpathmoveto{\pgfqpoint{0.000000in}{0.000000in}}%
\pgfpathlineto{\pgfqpoint{0.000000in}{-0.048611in}}%
\pgfusepath{stroke,fill}%
}%
\begin{pgfscope}%
\pgfsys@transformshift{0.929516in}{0.528000in}%
\pgfsys@useobject{currentmarker}{}%
\end{pgfscope}%
\end{pgfscope}%
\begin{pgfscope}%
\definecolor{textcolor}{rgb}{0.000000,0.000000,0.000000}%
\pgfsetstrokecolor{textcolor}%
\pgfsetfillcolor{textcolor}%
\pgftext[x=0.929516in,y=0.430778in,,top]{\color{textcolor}{\sffamily\fontsize{10.000000}{12.000000}\selectfont\catcode`\^=\active\def^{\ifmmode\sp\else\^{}\fi}\catcode`\%=\active\def%{\%}0}}%
\end{pgfscope}%
\begin{pgfscope}%
\pgfsetbuttcap%
\pgfsetroundjoin%
\definecolor{currentfill}{rgb}{0.000000,0.000000,0.000000}%
\pgfsetfillcolor{currentfill}%
\pgfsetlinewidth{0.803000pt}%
\definecolor{currentstroke}{rgb}{0.000000,0.000000,0.000000}%
\pgfsetstrokecolor{currentstroke}%
\pgfsetdash{}{0pt}%
\pgfsys@defobject{currentmarker}{\pgfqpoint{0.000000in}{-0.048611in}}{\pgfqpoint{0.000000in}{0.000000in}}{%
\pgfpathmoveto{\pgfqpoint{0.000000in}{0.000000in}}%
\pgfpathlineto{\pgfqpoint{0.000000in}{-0.048611in}}%
\pgfusepath{stroke,fill}%
}%
\begin{pgfscope}%
\pgfsys@transformshift{1.888897in}{0.528000in}%
\pgfsys@useobject{currentmarker}{}%
\end{pgfscope}%
\end{pgfscope}%
\begin{pgfscope}%
\definecolor{textcolor}{rgb}{0.000000,0.000000,0.000000}%
\pgfsetstrokecolor{textcolor}%
\pgfsetfillcolor{textcolor}%
\pgftext[x=1.888897in,y=0.430778in,,top]{\color{textcolor}{\sffamily\fontsize{10.000000}{12.000000}\selectfont\catcode`\^=\active\def^{\ifmmode\sp\else\^{}\fi}\catcode`\%=\active\def%{\%}10}}%
\end{pgfscope}%
\begin{pgfscope}%
\pgfsetbuttcap%
\pgfsetroundjoin%
\definecolor{currentfill}{rgb}{0.000000,0.000000,0.000000}%
\pgfsetfillcolor{currentfill}%
\pgfsetlinewidth{0.803000pt}%
\definecolor{currentstroke}{rgb}{0.000000,0.000000,0.000000}%
\pgfsetstrokecolor{currentstroke}%
\pgfsetdash{}{0pt}%
\pgfsys@defobject{currentmarker}{\pgfqpoint{0.000000in}{-0.048611in}}{\pgfqpoint{0.000000in}{0.000000in}}{%
\pgfpathmoveto{\pgfqpoint{0.000000in}{0.000000in}}%
\pgfpathlineto{\pgfqpoint{0.000000in}{-0.048611in}}%
\pgfusepath{stroke,fill}%
}%
\begin{pgfscope}%
\pgfsys@transformshift{2.848279in}{0.528000in}%
\pgfsys@useobject{currentmarker}{}%
\end{pgfscope}%
\end{pgfscope}%
\begin{pgfscope}%
\definecolor{textcolor}{rgb}{0.000000,0.000000,0.000000}%
\pgfsetstrokecolor{textcolor}%
\pgfsetfillcolor{textcolor}%
\pgftext[x=2.848279in,y=0.430778in,,top]{\color{textcolor}{\sffamily\fontsize{10.000000}{12.000000}\selectfont\catcode`\^=\active\def^{\ifmmode\sp\else\^{}\fi}\catcode`\%=\active\def%{\%}20}}%
\end{pgfscope}%
\begin{pgfscope}%
\pgfsetbuttcap%
\pgfsetroundjoin%
\definecolor{currentfill}{rgb}{0.000000,0.000000,0.000000}%
\pgfsetfillcolor{currentfill}%
\pgfsetlinewidth{0.803000pt}%
\definecolor{currentstroke}{rgb}{0.000000,0.000000,0.000000}%
\pgfsetstrokecolor{currentstroke}%
\pgfsetdash{}{0pt}%
\pgfsys@defobject{currentmarker}{\pgfqpoint{0.000000in}{-0.048611in}}{\pgfqpoint{0.000000in}{0.000000in}}{%
\pgfpathmoveto{\pgfqpoint{0.000000in}{0.000000in}}%
\pgfpathlineto{\pgfqpoint{0.000000in}{-0.048611in}}%
\pgfusepath{stroke,fill}%
}%
\begin{pgfscope}%
\pgfsys@transformshift{3.807660in}{0.528000in}%
\pgfsys@useobject{currentmarker}{}%
\end{pgfscope}%
\end{pgfscope}%
\begin{pgfscope}%
\definecolor{textcolor}{rgb}{0.000000,0.000000,0.000000}%
\pgfsetstrokecolor{textcolor}%
\pgfsetfillcolor{textcolor}%
\pgftext[x=3.807660in,y=0.430778in,,top]{\color{textcolor}{\sffamily\fontsize{10.000000}{12.000000}\selectfont\catcode`\^=\active\def^{\ifmmode\sp\else\^{}\fi}\catcode`\%=\active\def%{\%}30}}%
\end{pgfscope}%
\begin{pgfscope}%
\pgfsetbuttcap%
\pgfsetroundjoin%
\definecolor{currentfill}{rgb}{0.000000,0.000000,0.000000}%
\pgfsetfillcolor{currentfill}%
\pgfsetlinewidth{0.803000pt}%
\definecolor{currentstroke}{rgb}{0.000000,0.000000,0.000000}%
\pgfsetstrokecolor{currentstroke}%
\pgfsetdash{}{0pt}%
\pgfsys@defobject{currentmarker}{\pgfqpoint{0.000000in}{-0.048611in}}{\pgfqpoint{0.000000in}{0.000000in}}{%
\pgfpathmoveto{\pgfqpoint{0.000000in}{0.000000in}}%
\pgfpathlineto{\pgfqpoint{0.000000in}{-0.048611in}}%
\pgfusepath{stroke,fill}%
}%
\begin{pgfscope}%
\pgfsys@transformshift{4.767041in}{0.528000in}%
\pgfsys@useobject{currentmarker}{}%
\end{pgfscope}%
\end{pgfscope}%
\begin{pgfscope}%
\definecolor{textcolor}{rgb}{0.000000,0.000000,0.000000}%
\pgfsetstrokecolor{textcolor}%
\pgfsetfillcolor{textcolor}%
\pgftext[x=4.767041in,y=0.430778in,,top]{\color{textcolor}{\sffamily\fontsize{10.000000}{12.000000}\selectfont\catcode`\^=\active\def^{\ifmmode\sp\else\^{}\fi}\catcode`\%=\active\def%{\%}40}}%
\end{pgfscope}%
\begin{pgfscope}%
\pgfsetbuttcap%
\pgfsetroundjoin%
\definecolor{currentfill}{rgb}{0.000000,0.000000,0.000000}%
\pgfsetfillcolor{currentfill}%
\pgfsetlinewidth{0.803000pt}%
\definecolor{currentstroke}{rgb}{0.000000,0.000000,0.000000}%
\pgfsetstrokecolor{currentstroke}%
\pgfsetdash{}{0pt}%
\pgfsys@defobject{currentmarker}{\pgfqpoint{0.000000in}{-0.048611in}}{\pgfqpoint{0.000000in}{0.000000in}}{%
\pgfpathmoveto{\pgfqpoint{0.000000in}{0.000000in}}%
\pgfpathlineto{\pgfqpoint{0.000000in}{-0.048611in}}%
\pgfusepath{stroke,fill}%
}%
\begin{pgfscope}%
\pgfsys@transformshift{5.726422in}{0.528000in}%
\pgfsys@useobject{currentmarker}{}%
\end{pgfscope}%
\end{pgfscope}%
\begin{pgfscope}%
\definecolor{textcolor}{rgb}{0.000000,0.000000,0.000000}%
\pgfsetstrokecolor{textcolor}%
\pgfsetfillcolor{textcolor}%
\pgftext[x=5.726422in,y=0.430778in,,top]{\color{textcolor}{\sffamily\fontsize{10.000000}{12.000000}\selectfont\catcode`\^=\active\def^{\ifmmode\sp\else\^{}\fi}\catcode`\%=\active\def%{\%}50}}%
\end{pgfscope}%
\begin{pgfscope}%
\definecolor{textcolor}{rgb}{0.000000,0.000000,0.000000}%
\pgfsetstrokecolor{textcolor}%
\pgfsetfillcolor{textcolor}%
\pgftext[x=3.280000in,y=0.240809in,,top]{\color{textcolor}{\sffamily\fontsize{10.000000}{12.000000}\selectfont\catcode`\^=\active\def^{\ifmmode\sp\else\^{}\fi}\catcode`\%=\active\def%{\%}Trial}}%
\end{pgfscope}%
\begin{pgfscope}%
\pgfsetbuttcap%
\pgfsetroundjoin%
\definecolor{currentfill}{rgb}{0.000000,0.000000,0.000000}%
\pgfsetfillcolor{currentfill}%
\pgfsetlinewidth{0.803000pt}%
\definecolor{currentstroke}{rgb}{0.000000,0.000000,0.000000}%
\pgfsetstrokecolor{currentstroke}%
\pgfsetdash{}{0pt}%
\pgfsys@defobject{currentmarker}{\pgfqpoint{-0.048611in}{0.000000in}}{\pgfqpoint{-0.000000in}{0.000000in}}{%
\pgfpathmoveto{\pgfqpoint{-0.000000in}{0.000000in}}%
\pgfpathlineto{\pgfqpoint{-0.048611in}{0.000000in}}%
\pgfusepath{stroke,fill}%
}%
\begin{pgfscope}%
\pgfsys@transformshift{0.800000in}{0.919016in}%
\pgfsys@useobject{currentmarker}{}%
\end{pgfscope}%
\end{pgfscope}%
\begin{pgfscope}%
\definecolor{textcolor}{rgb}{0.000000,0.000000,0.000000}%
\pgfsetstrokecolor{textcolor}%
\pgfsetfillcolor{textcolor}%
\pgftext[x=0.393533in, y=0.866255in, left, base]{\color{textcolor}{\sffamily\fontsize{10.000000}{12.000000}\selectfont\catcode`\^=\active\def^{\ifmmode\sp\else\^{}\fi}\catcode`\%=\active\def%{\%}0.70}}%
\end{pgfscope}%
\begin{pgfscope}%
\pgfsetbuttcap%
\pgfsetroundjoin%
\definecolor{currentfill}{rgb}{0.000000,0.000000,0.000000}%
\pgfsetfillcolor{currentfill}%
\pgfsetlinewidth{0.803000pt}%
\definecolor{currentstroke}{rgb}{0.000000,0.000000,0.000000}%
\pgfsetstrokecolor{currentstroke}%
\pgfsetdash{}{0pt}%
\pgfsys@defobject{currentmarker}{\pgfqpoint{-0.048611in}{0.000000in}}{\pgfqpoint{-0.000000in}{0.000000in}}{%
\pgfpathmoveto{\pgfqpoint{-0.000000in}{0.000000in}}%
\pgfpathlineto{\pgfqpoint{-0.048611in}{0.000000in}}%
\pgfusepath{stroke,fill}%
}%
\begin{pgfscope}%
\pgfsys@transformshift{0.800000in}{1.491081in}%
\pgfsys@useobject{currentmarker}{}%
\end{pgfscope}%
\end{pgfscope}%
\begin{pgfscope}%
\definecolor{textcolor}{rgb}{0.000000,0.000000,0.000000}%
\pgfsetstrokecolor{textcolor}%
\pgfsetfillcolor{textcolor}%
\pgftext[x=0.393533in, y=1.438319in, left, base]{\color{textcolor}{\sffamily\fontsize{10.000000}{12.000000}\selectfont\catcode`\^=\active\def^{\ifmmode\sp\else\^{}\fi}\catcode`\%=\active\def%{\%}0.72}}%
\end{pgfscope}%
\begin{pgfscope}%
\pgfsetbuttcap%
\pgfsetroundjoin%
\definecolor{currentfill}{rgb}{0.000000,0.000000,0.000000}%
\pgfsetfillcolor{currentfill}%
\pgfsetlinewidth{0.803000pt}%
\definecolor{currentstroke}{rgb}{0.000000,0.000000,0.000000}%
\pgfsetstrokecolor{currentstroke}%
\pgfsetdash{}{0pt}%
\pgfsys@defobject{currentmarker}{\pgfqpoint{-0.048611in}{0.000000in}}{\pgfqpoint{-0.000000in}{0.000000in}}{%
\pgfpathmoveto{\pgfqpoint{-0.000000in}{0.000000in}}%
\pgfpathlineto{\pgfqpoint{-0.048611in}{0.000000in}}%
\pgfusepath{stroke,fill}%
}%
\begin{pgfscope}%
\pgfsys@transformshift{0.800000in}{2.063146in}%
\pgfsys@useobject{currentmarker}{}%
\end{pgfscope}%
\end{pgfscope}%
\begin{pgfscope}%
\definecolor{textcolor}{rgb}{0.000000,0.000000,0.000000}%
\pgfsetstrokecolor{textcolor}%
\pgfsetfillcolor{textcolor}%
\pgftext[x=0.393533in, y=2.010384in, left, base]{\color{textcolor}{\sffamily\fontsize{10.000000}{12.000000}\selectfont\catcode`\^=\active\def^{\ifmmode\sp\else\^{}\fi}\catcode`\%=\active\def%{\%}0.74}}%
\end{pgfscope}%
\begin{pgfscope}%
\pgfsetbuttcap%
\pgfsetroundjoin%
\definecolor{currentfill}{rgb}{0.000000,0.000000,0.000000}%
\pgfsetfillcolor{currentfill}%
\pgfsetlinewidth{0.803000pt}%
\definecolor{currentstroke}{rgb}{0.000000,0.000000,0.000000}%
\pgfsetstrokecolor{currentstroke}%
\pgfsetdash{}{0pt}%
\pgfsys@defobject{currentmarker}{\pgfqpoint{-0.048611in}{0.000000in}}{\pgfqpoint{-0.000000in}{0.000000in}}{%
\pgfpathmoveto{\pgfqpoint{-0.000000in}{0.000000in}}%
\pgfpathlineto{\pgfqpoint{-0.048611in}{0.000000in}}%
\pgfusepath{stroke,fill}%
}%
\begin{pgfscope}%
\pgfsys@transformshift{0.800000in}{2.635210in}%
\pgfsys@useobject{currentmarker}{}%
\end{pgfscope}%
\end{pgfscope}%
\begin{pgfscope}%
\definecolor{textcolor}{rgb}{0.000000,0.000000,0.000000}%
\pgfsetstrokecolor{textcolor}%
\pgfsetfillcolor{textcolor}%
\pgftext[x=0.393533in, y=2.582449in, left, base]{\color{textcolor}{\sffamily\fontsize{10.000000}{12.000000}\selectfont\catcode`\^=\active\def^{\ifmmode\sp\else\^{}\fi}\catcode`\%=\active\def%{\%}0.76}}%
\end{pgfscope}%
\begin{pgfscope}%
\pgfsetbuttcap%
\pgfsetroundjoin%
\definecolor{currentfill}{rgb}{0.000000,0.000000,0.000000}%
\pgfsetfillcolor{currentfill}%
\pgfsetlinewidth{0.803000pt}%
\definecolor{currentstroke}{rgb}{0.000000,0.000000,0.000000}%
\pgfsetstrokecolor{currentstroke}%
\pgfsetdash{}{0pt}%
\pgfsys@defobject{currentmarker}{\pgfqpoint{-0.048611in}{0.000000in}}{\pgfqpoint{-0.000000in}{0.000000in}}{%
\pgfpathmoveto{\pgfqpoint{-0.000000in}{0.000000in}}%
\pgfpathlineto{\pgfqpoint{-0.048611in}{0.000000in}}%
\pgfusepath{stroke,fill}%
}%
\begin{pgfscope}%
\pgfsys@transformshift{0.800000in}{3.207275in}%
\pgfsys@useobject{currentmarker}{}%
\end{pgfscope}%
\end{pgfscope}%
\begin{pgfscope}%
\definecolor{textcolor}{rgb}{0.000000,0.000000,0.000000}%
\pgfsetstrokecolor{textcolor}%
\pgfsetfillcolor{textcolor}%
\pgftext[x=0.393533in, y=3.154513in, left, base]{\color{textcolor}{\sffamily\fontsize{10.000000}{12.000000}\selectfont\catcode`\^=\active\def^{\ifmmode\sp\else\^{}\fi}\catcode`\%=\active\def%{\%}0.78}}%
\end{pgfscope}%
\begin{pgfscope}%
\pgfsetbuttcap%
\pgfsetroundjoin%
\definecolor{currentfill}{rgb}{0.000000,0.000000,0.000000}%
\pgfsetfillcolor{currentfill}%
\pgfsetlinewidth{0.803000pt}%
\definecolor{currentstroke}{rgb}{0.000000,0.000000,0.000000}%
\pgfsetstrokecolor{currentstroke}%
\pgfsetdash{}{0pt}%
\pgfsys@defobject{currentmarker}{\pgfqpoint{-0.048611in}{0.000000in}}{\pgfqpoint{-0.000000in}{0.000000in}}{%
\pgfpathmoveto{\pgfqpoint{-0.000000in}{0.000000in}}%
\pgfpathlineto{\pgfqpoint{-0.048611in}{0.000000in}}%
\pgfusepath{stroke,fill}%
}%
\begin{pgfscope}%
\pgfsys@transformshift{0.800000in}{3.779339in}%
\pgfsys@useobject{currentmarker}{}%
\end{pgfscope}%
\end{pgfscope}%
\begin{pgfscope}%
\definecolor{textcolor}{rgb}{0.000000,0.000000,0.000000}%
\pgfsetstrokecolor{textcolor}%
\pgfsetfillcolor{textcolor}%
\pgftext[x=0.393533in, y=3.726578in, left, base]{\color{textcolor}{\sffamily\fontsize{10.000000}{12.000000}\selectfont\catcode`\^=\active\def^{\ifmmode\sp\else\^{}\fi}\catcode`\%=\active\def%{\%}0.80}}%
\end{pgfscope}%
\begin{pgfscope}%
\definecolor{textcolor}{rgb}{0.000000,0.000000,0.000000}%
\pgfsetstrokecolor{textcolor}%
\pgfsetfillcolor{textcolor}%
\pgftext[x=0.337977in,y=2.376000in,,bottom,rotate=90.000000]{\color{textcolor}{\sffamily\fontsize{10.000000}{12.000000}\selectfont\catcode`\^=\active\def^{\ifmmode\sp\else\^{}\fi}\catcode`\%=\active\def%{\%}Perda em Validação}}%
\end{pgfscope}%
\begin{pgfscope}%
\pgfpathrectangle{\pgfqpoint{0.800000in}{0.528000in}}{\pgfqpoint{4.960000in}{3.696000in}}%
\pgfusepath{clip}%
\pgfsetrectcap%
\pgfsetroundjoin%
\pgfsetlinewidth{1.505625pt}%
\definecolor{currentstroke}{rgb}{0.121569,0.466667,0.705882}%
\pgfsetstrokecolor{currentstroke}%
\pgfsetdash{}{0pt}%
\pgfpathmoveto{\pgfqpoint{1.025455in}{0.786591in}}%
\pgfpathlineto{\pgfqpoint{1.121393in}{1.474826in}}%
\pgfpathlineto{\pgfqpoint{1.217331in}{0.796713in}}%
\pgfpathlineto{\pgfqpoint{1.313269in}{3.327305in}}%
\pgfpathlineto{\pgfqpoint{1.409207in}{1.364416in}}%
\pgfpathlineto{\pgfqpoint{1.505145in}{2.220993in}}%
\pgfpathlineto{\pgfqpoint{1.601083in}{0.849317in}}%
\pgfpathlineto{\pgfqpoint{1.697021in}{0.696199in}}%
\pgfpathlineto{\pgfqpoint{1.792959in}{1.298157in}}%
\pgfpathlineto{\pgfqpoint{1.888897in}{0.744932in}}%
\pgfpathlineto{\pgfqpoint{1.984836in}{1.805178in}}%
\pgfpathlineto{\pgfqpoint{2.080774in}{0.746869in}}%
\pgfpathlineto{\pgfqpoint{2.176712in}{1.323885in}}%
\pgfpathlineto{\pgfqpoint{2.272650in}{1.214176in}}%
\pgfpathlineto{\pgfqpoint{2.368588in}{0.773066in}}%
\pgfpathlineto{\pgfqpoint{2.464526in}{0.702288in}}%
\pgfpathlineto{\pgfqpoint{2.560464in}{0.696677in}}%
\pgfpathlineto{\pgfqpoint{2.656402in}{0.698129in}}%
\pgfpathlineto{\pgfqpoint{2.752340in}{0.696411in}}%
\pgfpathlineto{\pgfqpoint{2.848279in}{0.717898in}}%
\pgfpathlineto{\pgfqpoint{2.944217in}{0.696000in}}%
\pgfpathlineto{\pgfqpoint{3.040155in}{0.792254in}}%
\pgfpathlineto{\pgfqpoint{3.136093in}{0.810267in}}%
\pgfpathlineto{\pgfqpoint{3.232031in}{1.001669in}}%
\pgfpathlineto{\pgfqpoint{3.327969in}{0.701959in}}%
\pgfpathlineto{\pgfqpoint{3.423907in}{1.321178in}}%
\pgfpathlineto{\pgfqpoint{3.519845in}{0.703213in}}%
\pgfpathlineto{\pgfqpoint{3.615783in}{1.271644in}}%
\pgfpathlineto{\pgfqpoint{3.711721in}{3.173901in}}%
\pgfpathlineto{\pgfqpoint{3.807660in}{1.257310in}}%
\pgfpathlineto{\pgfqpoint{3.903598in}{0.743658in}}%
\pgfpathlineto{\pgfqpoint{3.999536in}{0.751624in}}%
\pgfpathlineto{\pgfqpoint{4.095474in}{4.056000in}}%
\pgfpathlineto{\pgfqpoint{4.191412in}{0.736387in}}%
\pgfpathlineto{\pgfqpoint{4.287350in}{0.829605in}}%
\pgfpathlineto{\pgfqpoint{4.383288in}{0.714237in}}%
\pgfpathlineto{\pgfqpoint{4.479226in}{2.981511in}}%
\pgfpathlineto{\pgfqpoint{4.575164in}{0.699193in}}%
\pgfpathlineto{\pgfqpoint{4.671103in}{0.723256in}}%
\pgfpathlineto{\pgfqpoint{4.767041in}{0.696598in}}%
\pgfpathlineto{\pgfqpoint{4.862979in}{0.779355in}}%
\pgfpathlineto{\pgfqpoint{4.958917in}{0.746088in}}%
\pgfpathlineto{\pgfqpoint{5.054855in}{0.782315in}}%
\pgfpathlineto{\pgfqpoint{5.150793in}{0.703125in}}%
\pgfpathlineto{\pgfqpoint{5.246731in}{0.696114in}}%
\pgfpathlineto{\pgfqpoint{5.342669in}{0.713040in}}%
\pgfpathlineto{\pgfqpoint{5.438607in}{0.711197in}}%
\pgfpathlineto{\pgfqpoint{5.534545in}{0.717981in}}%
\pgfusepath{stroke}%
\end{pgfscope}%
\begin{pgfscope}%
\pgfsetrectcap%
\pgfsetmiterjoin%
\pgfsetlinewidth{0.803000pt}%
\definecolor{currentstroke}{rgb}{0.000000,0.000000,0.000000}%
\pgfsetstrokecolor{currentstroke}%
\pgfsetdash{}{0pt}%
\pgfpathmoveto{\pgfqpoint{0.800000in}{0.528000in}}%
\pgfpathlineto{\pgfqpoint{0.800000in}{4.224000in}}%
\pgfusepath{stroke}%
\end{pgfscope}%
\begin{pgfscope}%
\pgfsetrectcap%
\pgfsetmiterjoin%
\pgfsetlinewidth{0.803000pt}%
\definecolor{currentstroke}{rgb}{0.000000,0.000000,0.000000}%
\pgfsetstrokecolor{currentstroke}%
\pgfsetdash{}{0pt}%
\pgfpathmoveto{\pgfqpoint{5.760000in}{0.528000in}}%
\pgfpathlineto{\pgfqpoint{5.760000in}{4.224000in}}%
\pgfusepath{stroke}%
\end{pgfscope}%
\begin{pgfscope}%
\pgfsetrectcap%
\pgfsetmiterjoin%
\pgfsetlinewidth{0.803000pt}%
\definecolor{currentstroke}{rgb}{0.000000,0.000000,0.000000}%
\pgfsetstrokecolor{currentstroke}%
\pgfsetdash{}{0pt}%
\pgfpathmoveto{\pgfqpoint{0.800000in}{0.528000in}}%
\pgfpathlineto{\pgfqpoint{5.760000in}{0.528000in}}%
\pgfusepath{stroke}%
\end{pgfscope}%
\begin{pgfscope}%
\pgfsetrectcap%
\pgfsetmiterjoin%
\pgfsetlinewidth{0.803000pt}%
\definecolor{currentstroke}{rgb}{0.000000,0.000000,0.000000}%
\pgfsetstrokecolor{currentstroke}%
\pgfsetdash{}{0pt}%
\pgfpathmoveto{\pgfqpoint{0.800000in}{4.224000in}}%
\pgfpathlineto{\pgfqpoint{5.760000in}{4.224000in}}%
\pgfusepath{stroke}%
\end{pgfscope}%
\begin{pgfscope}%
\definecolor{textcolor}{rgb}{0.000000,0.000000,0.000000}%
\pgfsetstrokecolor{textcolor}%
\pgfsetfillcolor{textcolor}%
\pgftext[x=3.280000in,y=4.307333in,,base]{\color{textcolor}{\sffamily\fontsize{12.000000}{14.400000}\selectfont\catcode`\^=\active\def^{\ifmmode\sp\else\^{}\fi}\catcode`\%=\active\def%{\%}Otimização de Hiperparâmetros (compI)}}%
\end{pgfscope}%
\end{pgfpicture}%
\makeatother%
\endgroup%
}
    \caption{Evolução da função de perda no \textit{hypertuning} do modelo da competência I.}
    \label{fig:exp-hyp-c1}
\end{figure}

A figura \ref{fig:exp-hyp-c1} mostra a evolução da função de perda ao longo das iterações da busca em grade para o modelo da competência I. A melhor configuração de hiperparâmetros obtida correspondeu a uma taxa de aprendizado de $2 \cdot 10^{-3}$, a um tamanho de lote de 4 e a funções de ativação de SeLU, Sigmoide e Sigmoide, respectivamente, nas três camadas ocultas. A perda de validação para esse conjunto foi de aproximadamente 0,692203.


\subsubsection{Competência II}
\label{subsec:exp-hyp-c2}

\begin{figure}[H]
    \resizebox{0.5\textwidth}{!}{%% Creator: Matplotlib, PGF backend
%%
%% To include the figure in your LaTeX document, write
%%   \input{<filename>.pgf}
%%
%% Make sure the required packages are loaded in your preamble
%%   \usepackage{pgf}
%%
%% Also ensure that all the required font packages are loaded; for instance,
%% the lmodern package is sometimes necessary when using math font.
%%   \usepackage{lmodern}
%%
%% Figures using additional raster images can only be included by \input if
%% they are in the same directory as the main LaTeX file. For loading figures
%% from other directories you can use the `import` package
%%   \usepackage{import}
%%
%% and then include the figures with
%%   \import{<path to file>}{<filename>.pgf}
%%
%% Matplotlib used the following preamble
%%   \def\mathdefault#1{#1}
%%   \everymath=\expandafter{\the\everymath\displaystyle}
%%   
%%   \usepackage{fontspec}
%%   \setmainfont{DejaVuSerif.ttf}[Path=\detokenize{/Users/josemayer/Documents/Pacotes/mambaforge/lib/python3.10/site-packages/matplotlib/mpl-data/fonts/ttf/}]
%%   \setsansfont{DejaVuSans.ttf}[Path=\detokenize{/Users/josemayer/Documents/Pacotes/mambaforge/lib/python3.10/site-packages/matplotlib/mpl-data/fonts/ttf/}]
%%   \setmonofont{DejaVuSansMono.ttf}[Path=\detokenize{/Users/josemayer/Documents/Pacotes/mambaforge/lib/python3.10/site-packages/matplotlib/mpl-data/fonts/ttf/}]
%%   \makeatletter\@ifpackageloaded{underscore}{}{\usepackage[strings]{underscore}}\makeatother
%%
\begingroup%
\makeatletter%
\begin{pgfpicture}%
\pgfpathrectangle{\pgfpointorigin}{\pgfqpoint{6.400000in}{4.800000in}}%
\pgfusepath{use as bounding box, clip}%
\begin{pgfscope}%
\pgfsetbuttcap%
\pgfsetmiterjoin%
\definecolor{currentfill}{rgb}{1.000000,1.000000,1.000000}%
\pgfsetfillcolor{currentfill}%
\pgfsetlinewidth{0.000000pt}%
\definecolor{currentstroke}{rgb}{1.000000,1.000000,1.000000}%
\pgfsetstrokecolor{currentstroke}%
\pgfsetdash{}{0pt}%
\pgfpathmoveto{\pgfqpoint{0.000000in}{0.000000in}}%
\pgfpathlineto{\pgfqpoint{6.400000in}{0.000000in}}%
\pgfpathlineto{\pgfqpoint{6.400000in}{4.800000in}}%
\pgfpathlineto{\pgfqpoint{0.000000in}{4.800000in}}%
\pgfpathlineto{\pgfqpoint{0.000000in}{0.000000in}}%
\pgfpathclose%
\pgfusepath{fill}%
\end{pgfscope}%
\begin{pgfscope}%
\pgfsetbuttcap%
\pgfsetmiterjoin%
\definecolor{currentfill}{rgb}{1.000000,1.000000,1.000000}%
\pgfsetfillcolor{currentfill}%
\pgfsetlinewidth{0.000000pt}%
\definecolor{currentstroke}{rgb}{0.000000,0.000000,0.000000}%
\pgfsetstrokecolor{currentstroke}%
\pgfsetstrokeopacity{0.000000}%
\pgfsetdash{}{0pt}%
\pgfpathmoveto{\pgfqpoint{0.800000in}{0.528000in}}%
\pgfpathlineto{\pgfqpoint{5.760000in}{0.528000in}}%
\pgfpathlineto{\pgfqpoint{5.760000in}{4.224000in}}%
\pgfpathlineto{\pgfqpoint{0.800000in}{4.224000in}}%
\pgfpathlineto{\pgfqpoint{0.800000in}{0.528000in}}%
\pgfpathclose%
\pgfusepath{fill}%
\end{pgfscope}%
\begin{pgfscope}%
\pgfsetbuttcap%
\pgfsetroundjoin%
\definecolor{currentfill}{rgb}{0.000000,0.000000,0.000000}%
\pgfsetfillcolor{currentfill}%
\pgfsetlinewidth{0.803000pt}%
\definecolor{currentstroke}{rgb}{0.000000,0.000000,0.000000}%
\pgfsetstrokecolor{currentstroke}%
\pgfsetdash{}{0pt}%
\pgfsys@defobject{currentmarker}{\pgfqpoint{0.000000in}{-0.048611in}}{\pgfqpoint{0.000000in}{0.000000in}}{%
\pgfpathmoveto{\pgfqpoint{0.000000in}{0.000000in}}%
\pgfpathlineto{\pgfqpoint{0.000000in}{-0.048611in}}%
\pgfusepath{stroke,fill}%
}%
\begin{pgfscope}%
\pgfsys@transformshift{0.929516in}{0.528000in}%
\pgfsys@useobject{currentmarker}{}%
\end{pgfscope}%
\end{pgfscope}%
\begin{pgfscope}%
\definecolor{textcolor}{rgb}{0.000000,0.000000,0.000000}%
\pgfsetstrokecolor{textcolor}%
\pgfsetfillcolor{textcolor}%
\pgftext[x=0.929516in,y=0.430778in,,top]{\color{textcolor}{\sffamily\fontsize{10.000000}{12.000000}\selectfont\catcode`\^=\active\def^{\ifmmode\sp\else\^{}\fi}\catcode`\%=\active\def%{\%}0}}%
\end{pgfscope}%
\begin{pgfscope}%
\pgfsetbuttcap%
\pgfsetroundjoin%
\definecolor{currentfill}{rgb}{0.000000,0.000000,0.000000}%
\pgfsetfillcolor{currentfill}%
\pgfsetlinewidth{0.803000pt}%
\definecolor{currentstroke}{rgb}{0.000000,0.000000,0.000000}%
\pgfsetstrokecolor{currentstroke}%
\pgfsetdash{}{0pt}%
\pgfsys@defobject{currentmarker}{\pgfqpoint{0.000000in}{-0.048611in}}{\pgfqpoint{0.000000in}{0.000000in}}{%
\pgfpathmoveto{\pgfqpoint{0.000000in}{0.000000in}}%
\pgfpathlineto{\pgfqpoint{0.000000in}{-0.048611in}}%
\pgfusepath{stroke,fill}%
}%
\begin{pgfscope}%
\pgfsys@transformshift{1.888897in}{0.528000in}%
\pgfsys@useobject{currentmarker}{}%
\end{pgfscope}%
\end{pgfscope}%
\begin{pgfscope}%
\definecolor{textcolor}{rgb}{0.000000,0.000000,0.000000}%
\pgfsetstrokecolor{textcolor}%
\pgfsetfillcolor{textcolor}%
\pgftext[x=1.888897in,y=0.430778in,,top]{\color{textcolor}{\sffamily\fontsize{10.000000}{12.000000}\selectfont\catcode`\^=\active\def^{\ifmmode\sp\else\^{}\fi}\catcode`\%=\active\def%{\%}10}}%
\end{pgfscope}%
\begin{pgfscope}%
\pgfsetbuttcap%
\pgfsetroundjoin%
\definecolor{currentfill}{rgb}{0.000000,0.000000,0.000000}%
\pgfsetfillcolor{currentfill}%
\pgfsetlinewidth{0.803000pt}%
\definecolor{currentstroke}{rgb}{0.000000,0.000000,0.000000}%
\pgfsetstrokecolor{currentstroke}%
\pgfsetdash{}{0pt}%
\pgfsys@defobject{currentmarker}{\pgfqpoint{0.000000in}{-0.048611in}}{\pgfqpoint{0.000000in}{0.000000in}}{%
\pgfpathmoveto{\pgfqpoint{0.000000in}{0.000000in}}%
\pgfpathlineto{\pgfqpoint{0.000000in}{-0.048611in}}%
\pgfusepath{stroke,fill}%
}%
\begin{pgfscope}%
\pgfsys@transformshift{2.848279in}{0.528000in}%
\pgfsys@useobject{currentmarker}{}%
\end{pgfscope}%
\end{pgfscope}%
\begin{pgfscope}%
\definecolor{textcolor}{rgb}{0.000000,0.000000,0.000000}%
\pgfsetstrokecolor{textcolor}%
\pgfsetfillcolor{textcolor}%
\pgftext[x=2.848279in,y=0.430778in,,top]{\color{textcolor}{\sffamily\fontsize{10.000000}{12.000000}\selectfont\catcode`\^=\active\def^{\ifmmode\sp\else\^{}\fi}\catcode`\%=\active\def%{\%}20}}%
\end{pgfscope}%
\begin{pgfscope}%
\pgfsetbuttcap%
\pgfsetroundjoin%
\definecolor{currentfill}{rgb}{0.000000,0.000000,0.000000}%
\pgfsetfillcolor{currentfill}%
\pgfsetlinewidth{0.803000pt}%
\definecolor{currentstroke}{rgb}{0.000000,0.000000,0.000000}%
\pgfsetstrokecolor{currentstroke}%
\pgfsetdash{}{0pt}%
\pgfsys@defobject{currentmarker}{\pgfqpoint{0.000000in}{-0.048611in}}{\pgfqpoint{0.000000in}{0.000000in}}{%
\pgfpathmoveto{\pgfqpoint{0.000000in}{0.000000in}}%
\pgfpathlineto{\pgfqpoint{0.000000in}{-0.048611in}}%
\pgfusepath{stroke,fill}%
}%
\begin{pgfscope}%
\pgfsys@transformshift{3.807660in}{0.528000in}%
\pgfsys@useobject{currentmarker}{}%
\end{pgfscope}%
\end{pgfscope}%
\begin{pgfscope}%
\definecolor{textcolor}{rgb}{0.000000,0.000000,0.000000}%
\pgfsetstrokecolor{textcolor}%
\pgfsetfillcolor{textcolor}%
\pgftext[x=3.807660in,y=0.430778in,,top]{\color{textcolor}{\sffamily\fontsize{10.000000}{12.000000}\selectfont\catcode`\^=\active\def^{\ifmmode\sp\else\^{}\fi}\catcode`\%=\active\def%{\%}30}}%
\end{pgfscope}%
\begin{pgfscope}%
\pgfsetbuttcap%
\pgfsetroundjoin%
\definecolor{currentfill}{rgb}{0.000000,0.000000,0.000000}%
\pgfsetfillcolor{currentfill}%
\pgfsetlinewidth{0.803000pt}%
\definecolor{currentstroke}{rgb}{0.000000,0.000000,0.000000}%
\pgfsetstrokecolor{currentstroke}%
\pgfsetdash{}{0pt}%
\pgfsys@defobject{currentmarker}{\pgfqpoint{0.000000in}{-0.048611in}}{\pgfqpoint{0.000000in}{0.000000in}}{%
\pgfpathmoveto{\pgfqpoint{0.000000in}{0.000000in}}%
\pgfpathlineto{\pgfqpoint{0.000000in}{-0.048611in}}%
\pgfusepath{stroke,fill}%
}%
\begin{pgfscope}%
\pgfsys@transformshift{4.767041in}{0.528000in}%
\pgfsys@useobject{currentmarker}{}%
\end{pgfscope}%
\end{pgfscope}%
\begin{pgfscope}%
\definecolor{textcolor}{rgb}{0.000000,0.000000,0.000000}%
\pgfsetstrokecolor{textcolor}%
\pgfsetfillcolor{textcolor}%
\pgftext[x=4.767041in,y=0.430778in,,top]{\color{textcolor}{\sffamily\fontsize{10.000000}{12.000000}\selectfont\catcode`\^=\active\def^{\ifmmode\sp\else\^{}\fi}\catcode`\%=\active\def%{\%}40}}%
\end{pgfscope}%
\begin{pgfscope}%
\pgfsetbuttcap%
\pgfsetroundjoin%
\definecolor{currentfill}{rgb}{0.000000,0.000000,0.000000}%
\pgfsetfillcolor{currentfill}%
\pgfsetlinewidth{0.803000pt}%
\definecolor{currentstroke}{rgb}{0.000000,0.000000,0.000000}%
\pgfsetstrokecolor{currentstroke}%
\pgfsetdash{}{0pt}%
\pgfsys@defobject{currentmarker}{\pgfqpoint{0.000000in}{-0.048611in}}{\pgfqpoint{0.000000in}{0.000000in}}{%
\pgfpathmoveto{\pgfqpoint{0.000000in}{0.000000in}}%
\pgfpathlineto{\pgfqpoint{0.000000in}{-0.048611in}}%
\pgfusepath{stroke,fill}%
}%
\begin{pgfscope}%
\pgfsys@transformshift{5.726422in}{0.528000in}%
\pgfsys@useobject{currentmarker}{}%
\end{pgfscope}%
\end{pgfscope}%
\begin{pgfscope}%
\definecolor{textcolor}{rgb}{0.000000,0.000000,0.000000}%
\pgfsetstrokecolor{textcolor}%
\pgfsetfillcolor{textcolor}%
\pgftext[x=5.726422in,y=0.430778in,,top]{\color{textcolor}{\sffamily\fontsize{10.000000}{12.000000}\selectfont\catcode`\^=\active\def^{\ifmmode\sp\else\^{}\fi}\catcode`\%=\active\def%{\%}50}}%
\end{pgfscope}%
\begin{pgfscope}%
\definecolor{textcolor}{rgb}{0.000000,0.000000,0.000000}%
\pgfsetstrokecolor{textcolor}%
\pgfsetfillcolor{textcolor}%
\pgftext[x=3.280000in,y=0.240809in,,top]{\color{textcolor}{\sffamily\fontsize{10.000000}{12.000000}\selectfont\catcode`\^=\active\def^{\ifmmode\sp\else\^{}\fi}\catcode`\%=\active\def%{\%}Trial}}%
\end{pgfscope}%
\begin{pgfscope}%
\pgfsetbuttcap%
\pgfsetroundjoin%
\definecolor{currentfill}{rgb}{0.000000,0.000000,0.000000}%
\pgfsetfillcolor{currentfill}%
\pgfsetlinewidth{0.803000pt}%
\definecolor{currentstroke}{rgb}{0.000000,0.000000,0.000000}%
\pgfsetstrokecolor{currentstroke}%
\pgfsetdash{}{0pt}%
\pgfsys@defobject{currentmarker}{\pgfqpoint{-0.048611in}{0.000000in}}{\pgfqpoint{-0.000000in}{0.000000in}}{%
\pgfpathmoveto{\pgfqpoint{-0.000000in}{0.000000in}}%
\pgfpathlineto{\pgfqpoint{-0.048611in}{0.000000in}}%
\pgfusepath{stroke,fill}%
}%
\begin{pgfscope}%
\pgfsys@transformshift{0.800000in}{0.568795in}%
\pgfsys@useobject{currentmarker}{}%
\end{pgfscope}%
\end{pgfscope}%
\begin{pgfscope}%
\definecolor{textcolor}{rgb}{0.000000,0.000000,0.000000}%
\pgfsetstrokecolor{textcolor}%
\pgfsetfillcolor{textcolor}%
\pgftext[x=0.393533in, y=0.516033in, left, base]{\color{textcolor}{\sffamily\fontsize{10.000000}{12.000000}\selectfont\catcode`\^=\active\def^{\ifmmode\sp\else\^{}\fi}\catcode`\%=\active\def%{\%}1.02}}%
\end{pgfscope}%
\begin{pgfscope}%
\pgfsetbuttcap%
\pgfsetroundjoin%
\definecolor{currentfill}{rgb}{0.000000,0.000000,0.000000}%
\pgfsetfillcolor{currentfill}%
\pgfsetlinewidth{0.803000pt}%
\definecolor{currentstroke}{rgb}{0.000000,0.000000,0.000000}%
\pgfsetstrokecolor{currentstroke}%
\pgfsetdash{}{0pt}%
\pgfsys@defobject{currentmarker}{\pgfqpoint{-0.048611in}{0.000000in}}{\pgfqpoint{-0.000000in}{0.000000in}}{%
\pgfpathmoveto{\pgfqpoint{-0.000000in}{0.000000in}}%
\pgfpathlineto{\pgfqpoint{-0.048611in}{0.000000in}}%
\pgfusepath{stroke,fill}%
}%
\begin{pgfscope}%
\pgfsys@transformshift{0.800000in}{1.108583in}%
\pgfsys@useobject{currentmarker}{}%
\end{pgfscope}%
\end{pgfscope}%
\begin{pgfscope}%
\definecolor{textcolor}{rgb}{0.000000,0.000000,0.000000}%
\pgfsetstrokecolor{textcolor}%
\pgfsetfillcolor{textcolor}%
\pgftext[x=0.393533in, y=1.055821in, left, base]{\color{textcolor}{\sffamily\fontsize{10.000000}{12.000000}\selectfont\catcode`\^=\active\def^{\ifmmode\sp\else\^{}\fi}\catcode`\%=\active\def%{\%}1.04}}%
\end{pgfscope}%
\begin{pgfscope}%
\pgfsetbuttcap%
\pgfsetroundjoin%
\definecolor{currentfill}{rgb}{0.000000,0.000000,0.000000}%
\pgfsetfillcolor{currentfill}%
\pgfsetlinewidth{0.803000pt}%
\definecolor{currentstroke}{rgb}{0.000000,0.000000,0.000000}%
\pgfsetstrokecolor{currentstroke}%
\pgfsetdash{}{0pt}%
\pgfsys@defobject{currentmarker}{\pgfqpoint{-0.048611in}{0.000000in}}{\pgfqpoint{-0.000000in}{0.000000in}}{%
\pgfpathmoveto{\pgfqpoint{-0.000000in}{0.000000in}}%
\pgfpathlineto{\pgfqpoint{-0.048611in}{0.000000in}}%
\pgfusepath{stroke,fill}%
}%
\begin{pgfscope}%
\pgfsys@transformshift{0.800000in}{1.648370in}%
\pgfsys@useobject{currentmarker}{}%
\end{pgfscope}%
\end{pgfscope}%
\begin{pgfscope}%
\definecolor{textcolor}{rgb}{0.000000,0.000000,0.000000}%
\pgfsetstrokecolor{textcolor}%
\pgfsetfillcolor{textcolor}%
\pgftext[x=0.393533in, y=1.595609in, left, base]{\color{textcolor}{\sffamily\fontsize{10.000000}{12.000000}\selectfont\catcode`\^=\active\def^{\ifmmode\sp\else\^{}\fi}\catcode`\%=\active\def%{\%}1.06}}%
\end{pgfscope}%
\begin{pgfscope}%
\pgfsetbuttcap%
\pgfsetroundjoin%
\definecolor{currentfill}{rgb}{0.000000,0.000000,0.000000}%
\pgfsetfillcolor{currentfill}%
\pgfsetlinewidth{0.803000pt}%
\definecolor{currentstroke}{rgb}{0.000000,0.000000,0.000000}%
\pgfsetstrokecolor{currentstroke}%
\pgfsetdash{}{0pt}%
\pgfsys@defobject{currentmarker}{\pgfqpoint{-0.048611in}{0.000000in}}{\pgfqpoint{-0.000000in}{0.000000in}}{%
\pgfpathmoveto{\pgfqpoint{-0.000000in}{0.000000in}}%
\pgfpathlineto{\pgfqpoint{-0.048611in}{0.000000in}}%
\pgfusepath{stroke,fill}%
}%
\begin{pgfscope}%
\pgfsys@transformshift{0.800000in}{2.188158in}%
\pgfsys@useobject{currentmarker}{}%
\end{pgfscope}%
\end{pgfscope}%
\begin{pgfscope}%
\definecolor{textcolor}{rgb}{0.000000,0.000000,0.000000}%
\pgfsetstrokecolor{textcolor}%
\pgfsetfillcolor{textcolor}%
\pgftext[x=0.393533in, y=2.135397in, left, base]{\color{textcolor}{\sffamily\fontsize{10.000000}{12.000000}\selectfont\catcode`\^=\active\def^{\ifmmode\sp\else\^{}\fi}\catcode`\%=\active\def%{\%}1.08}}%
\end{pgfscope}%
\begin{pgfscope}%
\pgfsetbuttcap%
\pgfsetroundjoin%
\definecolor{currentfill}{rgb}{0.000000,0.000000,0.000000}%
\pgfsetfillcolor{currentfill}%
\pgfsetlinewidth{0.803000pt}%
\definecolor{currentstroke}{rgb}{0.000000,0.000000,0.000000}%
\pgfsetstrokecolor{currentstroke}%
\pgfsetdash{}{0pt}%
\pgfsys@defobject{currentmarker}{\pgfqpoint{-0.048611in}{0.000000in}}{\pgfqpoint{-0.000000in}{0.000000in}}{%
\pgfpathmoveto{\pgfqpoint{-0.000000in}{0.000000in}}%
\pgfpathlineto{\pgfqpoint{-0.048611in}{0.000000in}}%
\pgfusepath{stroke,fill}%
}%
\begin{pgfscope}%
\pgfsys@transformshift{0.800000in}{2.727946in}%
\pgfsys@useobject{currentmarker}{}%
\end{pgfscope}%
\end{pgfscope}%
\begin{pgfscope}%
\definecolor{textcolor}{rgb}{0.000000,0.000000,0.000000}%
\pgfsetstrokecolor{textcolor}%
\pgfsetfillcolor{textcolor}%
\pgftext[x=0.393533in, y=2.675184in, left, base]{\color{textcolor}{\sffamily\fontsize{10.000000}{12.000000}\selectfont\catcode`\^=\active\def^{\ifmmode\sp\else\^{}\fi}\catcode`\%=\active\def%{\%}1.10}}%
\end{pgfscope}%
\begin{pgfscope}%
\pgfsetbuttcap%
\pgfsetroundjoin%
\definecolor{currentfill}{rgb}{0.000000,0.000000,0.000000}%
\pgfsetfillcolor{currentfill}%
\pgfsetlinewidth{0.803000pt}%
\definecolor{currentstroke}{rgb}{0.000000,0.000000,0.000000}%
\pgfsetstrokecolor{currentstroke}%
\pgfsetdash{}{0pt}%
\pgfsys@defobject{currentmarker}{\pgfqpoint{-0.048611in}{0.000000in}}{\pgfqpoint{-0.000000in}{0.000000in}}{%
\pgfpathmoveto{\pgfqpoint{-0.000000in}{0.000000in}}%
\pgfpathlineto{\pgfqpoint{-0.048611in}{0.000000in}}%
\pgfusepath{stroke,fill}%
}%
\begin{pgfscope}%
\pgfsys@transformshift{0.800000in}{3.267734in}%
\pgfsys@useobject{currentmarker}{}%
\end{pgfscope}%
\end{pgfscope}%
\begin{pgfscope}%
\definecolor{textcolor}{rgb}{0.000000,0.000000,0.000000}%
\pgfsetstrokecolor{textcolor}%
\pgfsetfillcolor{textcolor}%
\pgftext[x=0.393533in, y=3.214972in, left, base]{\color{textcolor}{\sffamily\fontsize{10.000000}{12.000000}\selectfont\catcode`\^=\active\def^{\ifmmode\sp\else\^{}\fi}\catcode`\%=\active\def%{\%}1.12}}%
\end{pgfscope}%
\begin{pgfscope}%
\pgfsetbuttcap%
\pgfsetroundjoin%
\definecolor{currentfill}{rgb}{0.000000,0.000000,0.000000}%
\pgfsetfillcolor{currentfill}%
\pgfsetlinewidth{0.803000pt}%
\definecolor{currentstroke}{rgb}{0.000000,0.000000,0.000000}%
\pgfsetstrokecolor{currentstroke}%
\pgfsetdash{}{0pt}%
\pgfsys@defobject{currentmarker}{\pgfqpoint{-0.048611in}{0.000000in}}{\pgfqpoint{-0.000000in}{0.000000in}}{%
\pgfpathmoveto{\pgfqpoint{-0.000000in}{0.000000in}}%
\pgfpathlineto{\pgfqpoint{-0.048611in}{0.000000in}}%
\pgfusepath{stroke,fill}%
}%
\begin{pgfscope}%
\pgfsys@transformshift{0.800000in}{3.807522in}%
\pgfsys@useobject{currentmarker}{}%
\end{pgfscope}%
\end{pgfscope}%
\begin{pgfscope}%
\definecolor{textcolor}{rgb}{0.000000,0.000000,0.000000}%
\pgfsetstrokecolor{textcolor}%
\pgfsetfillcolor{textcolor}%
\pgftext[x=0.393533in, y=3.754760in, left, base]{\color{textcolor}{\sffamily\fontsize{10.000000}{12.000000}\selectfont\catcode`\^=\active\def^{\ifmmode\sp\else\^{}\fi}\catcode`\%=\active\def%{\%}1.14}}%
\end{pgfscope}%
\begin{pgfscope}%
\definecolor{textcolor}{rgb}{0.000000,0.000000,0.000000}%
\pgfsetstrokecolor{textcolor}%
\pgfsetfillcolor{textcolor}%
\pgftext[x=0.337977in,y=2.376000in,,bottom,rotate=90.000000]{\color{textcolor}{\sffamily\fontsize{10.000000}{12.000000}\selectfont\catcode`\^=\active\def^{\ifmmode\sp\else\^{}\fi}\catcode`\%=\active\def%{\%}Perda em Validação}}%
\end{pgfscope}%
\begin{pgfscope}%
\pgfpathrectangle{\pgfqpoint{0.800000in}{0.528000in}}{\pgfqpoint{4.960000in}{3.696000in}}%
\pgfusepath{clip}%
\pgfsetrectcap%
\pgfsetroundjoin%
\pgfsetlinewidth{1.505625pt}%
\definecolor{currentstroke}{rgb}{0.121569,0.466667,0.705882}%
\pgfsetstrokecolor{currentstroke}%
\pgfsetdash{}{0pt}%
\pgfpathmoveto{\pgfqpoint{1.025455in}{0.935531in}}%
\pgfpathlineto{\pgfqpoint{1.121393in}{0.778552in}}%
\pgfpathlineto{\pgfqpoint{1.217331in}{0.735445in}}%
\pgfpathlineto{\pgfqpoint{1.313269in}{1.970175in}}%
\pgfpathlineto{\pgfqpoint{1.409207in}{0.699475in}}%
\pgfpathlineto{\pgfqpoint{1.505145in}{1.405009in}}%
\pgfpathlineto{\pgfqpoint{1.601083in}{0.873748in}}%
\pgfpathlineto{\pgfqpoint{1.697021in}{0.696000in}}%
\pgfpathlineto{\pgfqpoint{1.792959in}{0.708603in}}%
\pgfpathlineto{\pgfqpoint{1.888897in}{0.732437in}}%
\pgfpathlineto{\pgfqpoint{1.984836in}{0.712251in}}%
\pgfpathlineto{\pgfqpoint{2.080774in}{0.755705in}}%
\pgfpathlineto{\pgfqpoint{2.176712in}{0.816652in}}%
\pgfpathlineto{\pgfqpoint{2.272650in}{0.713998in}}%
\pgfpathlineto{\pgfqpoint{2.368588in}{0.701666in}}%
\pgfpathlineto{\pgfqpoint{2.464526in}{0.875804in}}%
\pgfpathlineto{\pgfqpoint{2.560464in}{0.712399in}}%
\pgfpathlineto{\pgfqpoint{2.656402in}{0.720259in}}%
\pgfpathlineto{\pgfqpoint{2.752340in}{0.748662in}}%
\pgfpathlineto{\pgfqpoint{2.848279in}{0.696097in}}%
\pgfpathlineto{\pgfqpoint{2.944217in}{0.786537in}}%
\pgfpathlineto{\pgfqpoint{3.040155in}{0.701798in}}%
\pgfpathlineto{\pgfqpoint{3.136093in}{0.749238in}}%
\pgfpathlineto{\pgfqpoint{3.232031in}{0.696174in}}%
\pgfpathlineto{\pgfqpoint{3.327969in}{4.056000in}}%
\pgfpathlineto{\pgfqpoint{3.423907in}{0.873510in}}%
\pgfpathlineto{\pgfqpoint{3.519845in}{0.730822in}}%
\pgfpathlineto{\pgfqpoint{3.615783in}{0.874758in}}%
\pgfpathlineto{\pgfqpoint{3.711721in}{0.785389in}}%
\pgfpathlineto{\pgfqpoint{3.807660in}{1.137210in}}%
\pgfpathlineto{\pgfqpoint{3.903598in}{0.850216in}}%
\pgfpathlineto{\pgfqpoint{3.999536in}{2.636702in}}%
\pgfpathlineto{\pgfqpoint{4.095474in}{0.771029in}}%
\pgfpathlineto{\pgfqpoint{4.191412in}{0.701277in}}%
\pgfpathlineto{\pgfqpoint{4.287350in}{0.696473in}}%
\pgfpathlineto{\pgfqpoint{4.383288in}{0.783204in}}%
\pgfpathlineto{\pgfqpoint{4.479226in}{0.714983in}}%
\pgfpathlineto{\pgfqpoint{4.575164in}{0.847976in}}%
\pgfpathlineto{\pgfqpoint{4.671103in}{0.744270in}}%
\pgfpathlineto{\pgfqpoint{4.767041in}{0.724136in}}%
\pgfpathlineto{\pgfqpoint{4.862979in}{0.701746in}}%
\pgfpathlineto{\pgfqpoint{4.958917in}{0.733129in}}%
\pgfpathlineto{\pgfqpoint{5.054855in}{0.910886in}}%
\pgfpathlineto{\pgfqpoint{5.150793in}{0.719358in}}%
\pgfpathlineto{\pgfqpoint{5.246731in}{0.709198in}}%
\pgfpathlineto{\pgfqpoint{5.342669in}{0.766432in}}%
\pgfpathlineto{\pgfqpoint{5.438607in}{0.712888in}}%
\pgfpathlineto{\pgfqpoint{5.534545in}{0.696637in}}%
\pgfusepath{stroke}%
\end{pgfscope}%
\begin{pgfscope}%
\pgfsetrectcap%
\pgfsetmiterjoin%
\pgfsetlinewidth{0.803000pt}%
\definecolor{currentstroke}{rgb}{0.000000,0.000000,0.000000}%
\pgfsetstrokecolor{currentstroke}%
\pgfsetdash{}{0pt}%
\pgfpathmoveto{\pgfqpoint{0.800000in}{0.528000in}}%
\pgfpathlineto{\pgfqpoint{0.800000in}{4.224000in}}%
\pgfusepath{stroke}%
\end{pgfscope}%
\begin{pgfscope}%
\pgfsetrectcap%
\pgfsetmiterjoin%
\pgfsetlinewidth{0.803000pt}%
\definecolor{currentstroke}{rgb}{0.000000,0.000000,0.000000}%
\pgfsetstrokecolor{currentstroke}%
\pgfsetdash{}{0pt}%
\pgfpathmoveto{\pgfqpoint{5.760000in}{0.528000in}}%
\pgfpathlineto{\pgfqpoint{5.760000in}{4.224000in}}%
\pgfusepath{stroke}%
\end{pgfscope}%
\begin{pgfscope}%
\pgfsetrectcap%
\pgfsetmiterjoin%
\pgfsetlinewidth{0.803000pt}%
\definecolor{currentstroke}{rgb}{0.000000,0.000000,0.000000}%
\pgfsetstrokecolor{currentstroke}%
\pgfsetdash{}{0pt}%
\pgfpathmoveto{\pgfqpoint{0.800000in}{0.528000in}}%
\pgfpathlineto{\pgfqpoint{5.760000in}{0.528000in}}%
\pgfusepath{stroke}%
\end{pgfscope}%
\begin{pgfscope}%
\pgfsetrectcap%
\pgfsetmiterjoin%
\pgfsetlinewidth{0.803000pt}%
\definecolor{currentstroke}{rgb}{0.000000,0.000000,0.000000}%
\pgfsetstrokecolor{currentstroke}%
\pgfsetdash{}{0pt}%
\pgfpathmoveto{\pgfqpoint{0.800000in}{4.224000in}}%
\pgfpathlineto{\pgfqpoint{5.760000in}{4.224000in}}%
\pgfusepath{stroke}%
\end{pgfscope}%
\begin{pgfscope}%
\definecolor{textcolor}{rgb}{0.000000,0.000000,0.000000}%
\pgfsetstrokecolor{textcolor}%
\pgfsetfillcolor{textcolor}%
\pgftext[x=3.280000in,y=4.307333in,,base]{\color{textcolor}{\sffamily\fontsize{12.000000}{14.400000}\selectfont\catcode`\^=\active\def^{\ifmmode\sp\else\^{}\fi}\catcode`\%=\active\def%{\%}Otimização de Hiperparâmetros (compII)}}%
\end{pgfscope}%
\end{pgfpicture}%
\makeatother%
\endgroup%
}
    \caption{Evolução da função de perda no \textit{hypertuning} do modelo da competência II.}
    \label{fig:exp-hyp-c2}
\end{figure}

A figura \ref{fig:exp-hyp-c2} evidencia a evolução da função de perda para a competência II. A melhor configuração de hiperparâmetros obtida correspondeu a uma taxa de aprendizado de $2 \cdot 10^{-3}$, a um tamanho de lote de $3$ e a funções de ativação de SeLU, SeLU e Sigmoide, respectivamente. A melhor perda de validação obtida foi de aproximadamente 1,024713.

\subsubsection{Competência III}
\label{subsec:exp-hyp-c3}

\begin{figure}[H]
    \resizebox{0.5\textwidth}{!}{%% Creator: Matplotlib, PGF backend
%%
%% To include the figure in your LaTeX document, write
%%   \input{<filename>.pgf}
%%
%% Make sure the required packages are loaded in your preamble
%%   \usepackage{pgf}
%%
%% Also ensure that all the required font packages are loaded; for instance,
%% the lmodern package is sometimes necessary when using math font.
%%   \usepackage{lmodern}
%%
%% Figures using additional raster images can only be included by \input if
%% they are in the same directory as the main LaTeX file. For loading figures
%% from other directories you can use the `import` package
%%   \usepackage{import}
%%
%% and then include the figures with
%%   \import{<path to file>}{<filename>.pgf}
%%
%% Matplotlib used the following preamble
%%   \def\mathdefault#1{#1}
%%   \everymath=\expandafter{\the\everymath\displaystyle}
%%   
%%   \usepackage{fontspec}
%%   \setmainfont{DejaVuSerif.ttf}[Path=\detokenize{/Users/josemayer/Documents/Pacotes/mambaforge/lib/python3.10/site-packages/matplotlib/mpl-data/fonts/ttf/}]
%%   \setsansfont{DejaVuSans.ttf}[Path=\detokenize{/Users/josemayer/Documents/Pacotes/mambaforge/lib/python3.10/site-packages/matplotlib/mpl-data/fonts/ttf/}]
%%   \setmonofont{DejaVuSansMono.ttf}[Path=\detokenize{/Users/josemayer/Documents/Pacotes/mambaforge/lib/python3.10/site-packages/matplotlib/mpl-data/fonts/ttf/}]
%%   \makeatletter\@ifpackageloaded{underscore}{}{\usepackage[strings]{underscore}}\makeatother
%%
\begingroup%
\makeatletter%
\begin{pgfpicture}%
\pgfpathrectangle{\pgfpointorigin}{\pgfqpoint{6.400000in}{4.800000in}}%
\pgfusepath{use as bounding box, clip}%
\begin{pgfscope}%
\pgfsetbuttcap%
\pgfsetmiterjoin%
\definecolor{currentfill}{rgb}{1.000000,1.000000,1.000000}%
\pgfsetfillcolor{currentfill}%
\pgfsetlinewidth{0.000000pt}%
\definecolor{currentstroke}{rgb}{1.000000,1.000000,1.000000}%
\pgfsetstrokecolor{currentstroke}%
\pgfsetdash{}{0pt}%
\pgfpathmoveto{\pgfqpoint{0.000000in}{0.000000in}}%
\pgfpathlineto{\pgfqpoint{6.400000in}{0.000000in}}%
\pgfpathlineto{\pgfqpoint{6.400000in}{4.800000in}}%
\pgfpathlineto{\pgfqpoint{0.000000in}{4.800000in}}%
\pgfpathlineto{\pgfqpoint{0.000000in}{0.000000in}}%
\pgfpathclose%
\pgfusepath{fill}%
\end{pgfscope}%
\begin{pgfscope}%
\pgfsetbuttcap%
\pgfsetmiterjoin%
\definecolor{currentfill}{rgb}{1.000000,1.000000,1.000000}%
\pgfsetfillcolor{currentfill}%
\pgfsetlinewidth{0.000000pt}%
\definecolor{currentstroke}{rgb}{0.000000,0.000000,0.000000}%
\pgfsetstrokecolor{currentstroke}%
\pgfsetstrokeopacity{0.000000}%
\pgfsetdash{}{0pt}%
\pgfpathmoveto{\pgfqpoint{0.800000in}{0.528000in}}%
\pgfpathlineto{\pgfqpoint{5.760000in}{0.528000in}}%
\pgfpathlineto{\pgfqpoint{5.760000in}{4.224000in}}%
\pgfpathlineto{\pgfqpoint{0.800000in}{4.224000in}}%
\pgfpathlineto{\pgfqpoint{0.800000in}{0.528000in}}%
\pgfpathclose%
\pgfusepath{fill}%
\end{pgfscope}%
\begin{pgfscope}%
\pgfsetbuttcap%
\pgfsetroundjoin%
\definecolor{currentfill}{rgb}{0.000000,0.000000,0.000000}%
\pgfsetfillcolor{currentfill}%
\pgfsetlinewidth{0.803000pt}%
\definecolor{currentstroke}{rgb}{0.000000,0.000000,0.000000}%
\pgfsetstrokecolor{currentstroke}%
\pgfsetdash{}{0pt}%
\pgfsys@defobject{currentmarker}{\pgfqpoint{0.000000in}{-0.048611in}}{\pgfqpoint{0.000000in}{0.000000in}}{%
\pgfpathmoveto{\pgfqpoint{0.000000in}{0.000000in}}%
\pgfpathlineto{\pgfqpoint{0.000000in}{-0.048611in}}%
\pgfusepath{stroke,fill}%
}%
\begin{pgfscope}%
\pgfsys@transformshift{0.929516in}{0.528000in}%
\pgfsys@useobject{currentmarker}{}%
\end{pgfscope}%
\end{pgfscope}%
\begin{pgfscope}%
\definecolor{textcolor}{rgb}{0.000000,0.000000,0.000000}%
\pgfsetstrokecolor{textcolor}%
\pgfsetfillcolor{textcolor}%
\pgftext[x=0.929516in,y=0.430778in,,top]{\color{textcolor}{\sffamily\fontsize{10.000000}{12.000000}\selectfont\catcode`\^=\active\def^{\ifmmode\sp\else\^{}\fi}\catcode`\%=\active\def%{\%}0}}%
\end{pgfscope}%
\begin{pgfscope}%
\pgfsetbuttcap%
\pgfsetroundjoin%
\definecolor{currentfill}{rgb}{0.000000,0.000000,0.000000}%
\pgfsetfillcolor{currentfill}%
\pgfsetlinewidth{0.803000pt}%
\definecolor{currentstroke}{rgb}{0.000000,0.000000,0.000000}%
\pgfsetstrokecolor{currentstroke}%
\pgfsetdash{}{0pt}%
\pgfsys@defobject{currentmarker}{\pgfqpoint{0.000000in}{-0.048611in}}{\pgfqpoint{0.000000in}{0.000000in}}{%
\pgfpathmoveto{\pgfqpoint{0.000000in}{0.000000in}}%
\pgfpathlineto{\pgfqpoint{0.000000in}{-0.048611in}}%
\pgfusepath{stroke,fill}%
}%
\begin{pgfscope}%
\pgfsys@transformshift{1.888897in}{0.528000in}%
\pgfsys@useobject{currentmarker}{}%
\end{pgfscope}%
\end{pgfscope}%
\begin{pgfscope}%
\definecolor{textcolor}{rgb}{0.000000,0.000000,0.000000}%
\pgfsetstrokecolor{textcolor}%
\pgfsetfillcolor{textcolor}%
\pgftext[x=1.888897in,y=0.430778in,,top]{\color{textcolor}{\sffamily\fontsize{10.000000}{12.000000}\selectfont\catcode`\^=\active\def^{\ifmmode\sp\else\^{}\fi}\catcode`\%=\active\def%{\%}10}}%
\end{pgfscope}%
\begin{pgfscope}%
\pgfsetbuttcap%
\pgfsetroundjoin%
\definecolor{currentfill}{rgb}{0.000000,0.000000,0.000000}%
\pgfsetfillcolor{currentfill}%
\pgfsetlinewidth{0.803000pt}%
\definecolor{currentstroke}{rgb}{0.000000,0.000000,0.000000}%
\pgfsetstrokecolor{currentstroke}%
\pgfsetdash{}{0pt}%
\pgfsys@defobject{currentmarker}{\pgfqpoint{0.000000in}{-0.048611in}}{\pgfqpoint{0.000000in}{0.000000in}}{%
\pgfpathmoveto{\pgfqpoint{0.000000in}{0.000000in}}%
\pgfpathlineto{\pgfqpoint{0.000000in}{-0.048611in}}%
\pgfusepath{stroke,fill}%
}%
\begin{pgfscope}%
\pgfsys@transformshift{2.848279in}{0.528000in}%
\pgfsys@useobject{currentmarker}{}%
\end{pgfscope}%
\end{pgfscope}%
\begin{pgfscope}%
\definecolor{textcolor}{rgb}{0.000000,0.000000,0.000000}%
\pgfsetstrokecolor{textcolor}%
\pgfsetfillcolor{textcolor}%
\pgftext[x=2.848279in,y=0.430778in,,top]{\color{textcolor}{\sffamily\fontsize{10.000000}{12.000000}\selectfont\catcode`\^=\active\def^{\ifmmode\sp\else\^{}\fi}\catcode`\%=\active\def%{\%}20}}%
\end{pgfscope}%
\begin{pgfscope}%
\pgfsetbuttcap%
\pgfsetroundjoin%
\definecolor{currentfill}{rgb}{0.000000,0.000000,0.000000}%
\pgfsetfillcolor{currentfill}%
\pgfsetlinewidth{0.803000pt}%
\definecolor{currentstroke}{rgb}{0.000000,0.000000,0.000000}%
\pgfsetstrokecolor{currentstroke}%
\pgfsetdash{}{0pt}%
\pgfsys@defobject{currentmarker}{\pgfqpoint{0.000000in}{-0.048611in}}{\pgfqpoint{0.000000in}{0.000000in}}{%
\pgfpathmoveto{\pgfqpoint{0.000000in}{0.000000in}}%
\pgfpathlineto{\pgfqpoint{0.000000in}{-0.048611in}}%
\pgfusepath{stroke,fill}%
}%
\begin{pgfscope}%
\pgfsys@transformshift{3.807660in}{0.528000in}%
\pgfsys@useobject{currentmarker}{}%
\end{pgfscope}%
\end{pgfscope}%
\begin{pgfscope}%
\definecolor{textcolor}{rgb}{0.000000,0.000000,0.000000}%
\pgfsetstrokecolor{textcolor}%
\pgfsetfillcolor{textcolor}%
\pgftext[x=3.807660in,y=0.430778in,,top]{\color{textcolor}{\sffamily\fontsize{10.000000}{12.000000}\selectfont\catcode`\^=\active\def^{\ifmmode\sp\else\^{}\fi}\catcode`\%=\active\def%{\%}30}}%
\end{pgfscope}%
\begin{pgfscope}%
\pgfsetbuttcap%
\pgfsetroundjoin%
\definecolor{currentfill}{rgb}{0.000000,0.000000,0.000000}%
\pgfsetfillcolor{currentfill}%
\pgfsetlinewidth{0.803000pt}%
\definecolor{currentstroke}{rgb}{0.000000,0.000000,0.000000}%
\pgfsetstrokecolor{currentstroke}%
\pgfsetdash{}{0pt}%
\pgfsys@defobject{currentmarker}{\pgfqpoint{0.000000in}{-0.048611in}}{\pgfqpoint{0.000000in}{0.000000in}}{%
\pgfpathmoveto{\pgfqpoint{0.000000in}{0.000000in}}%
\pgfpathlineto{\pgfqpoint{0.000000in}{-0.048611in}}%
\pgfusepath{stroke,fill}%
}%
\begin{pgfscope}%
\pgfsys@transformshift{4.767041in}{0.528000in}%
\pgfsys@useobject{currentmarker}{}%
\end{pgfscope}%
\end{pgfscope}%
\begin{pgfscope}%
\definecolor{textcolor}{rgb}{0.000000,0.000000,0.000000}%
\pgfsetstrokecolor{textcolor}%
\pgfsetfillcolor{textcolor}%
\pgftext[x=4.767041in,y=0.430778in,,top]{\color{textcolor}{\sffamily\fontsize{10.000000}{12.000000}\selectfont\catcode`\^=\active\def^{\ifmmode\sp\else\^{}\fi}\catcode`\%=\active\def%{\%}40}}%
\end{pgfscope}%
\begin{pgfscope}%
\pgfsetbuttcap%
\pgfsetroundjoin%
\definecolor{currentfill}{rgb}{0.000000,0.000000,0.000000}%
\pgfsetfillcolor{currentfill}%
\pgfsetlinewidth{0.803000pt}%
\definecolor{currentstroke}{rgb}{0.000000,0.000000,0.000000}%
\pgfsetstrokecolor{currentstroke}%
\pgfsetdash{}{0pt}%
\pgfsys@defobject{currentmarker}{\pgfqpoint{0.000000in}{-0.048611in}}{\pgfqpoint{0.000000in}{0.000000in}}{%
\pgfpathmoveto{\pgfqpoint{0.000000in}{0.000000in}}%
\pgfpathlineto{\pgfqpoint{0.000000in}{-0.048611in}}%
\pgfusepath{stroke,fill}%
}%
\begin{pgfscope}%
\pgfsys@transformshift{5.726422in}{0.528000in}%
\pgfsys@useobject{currentmarker}{}%
\end{pgfscope}%
\end{pgfscope}%
\begin{pgfscope}%
\definecolor{textcolor}{rgb}{0.000000,0.000000,0.000000}%
\pgfsetstrokecolor{textcolor}%
\pgfsetfillcolor{textcolor}%
\pgftext[x=5.726422in,y=0.430778in,,top]{\color{textcolor}{\sffamily\fontsize{10.000000}{12.000000}\selectfont\catcode`\^=\active\def^{\ifmmode\sp\else\^{}\fi}\catcode`\%=\active\def%{\%}50}}%
\end{pgfscope}%
\begin{pgfscope}%
\definecolor{textcolor}{rgb}{0.000000,0.000000,0.000000}%
\pgfsetstrokecolor{textcolor}%
\pgfsetfillcolor{textcolor}%
\pgftext[x=3.280000in,y=0.240809in,,top]{\color{textcolor}{\sffamily\fontsize{10.000000}{12.000000}\selectfont\catcode`\^=\active\def^{\ifmmode\sp\else\^{}\fi}\catcode`\%=\active\def%{\%}Trial}}%
\end{pgfscope}%
\begin{pgfscope}%
\pgfsetbuttcap%
\pgfsetroundjoin%
\definecolor{currentfill}{rgb}{0.000000,0.000000,0.000000}%
\pgfsetfillcolor{currentfill}%
\pgfsetlinewidth{0.803000pt}%
\definecolor{currentstroke}{rgb}{0.000000,0.000000,0.000000}%
\pgfsetstrokecolor{currentstroke}%
\pgfsetdash{}{0pt}%
\pgfsys@defobject{currentmarker}{\pgfqpoint{-0.048611in}{0.000000in}}{\pgfqpoint{-0.000000in}{0.000000in}}{%
\pgfpathmoveto{\pgfqpoint{-0.000000in}{0.000000in}}%
\pgfpathlineto{\pgfqpoint{-0.048611in}{0.000000in}}%
\pgfusepath{stroke,fill}%
}%
\begin{pgfscope}%
\pgfsys@transformshift{0.800000in}{0.687616in}%
\pgfsys@useobject{currentmarker}{}%
\end{pgfscope}%
\end{pgfscope}%
\begin{pgfscope}%
\definecolor{textcolor}{rgb}{0.000000,0.000000,0.000000}%
\pgfsetstrokecolor{textcolor}%
\pgfsetfillcolor{textcolor}%
\pgftext[x=0.393533in, y=0.634855in, left, base]{\color{textcolor}{\sffamily\fontsize{10.000000}{12.000000}\selectfont\catcode`\^=\active\def^{\ifmmode\sp\else\^{}\fi}\catcode`\%=\active\def%{\%}0.94}}%
\end{pgfscope}%
\begin{pgfscope}%
\pgfsetbuttcap%
\pgfsetroundjoin%
\definecolor{currentfill}{rgb}{0.000000,0.000000,0.000000}%
\pgfsetfillcolor{currentfill}%
\pgfsetlinewidth{0.803000pt}%
\definecolor{currentstroke}{rgb}{0.000000,0.000000,0.000000}%
\pgfsetstrokecolor{currentstroke}%
\pgfsetdash{}{0pt}%
\pgfsys@defobject{currentmarker}{\pgfqpoint{-0.048611in}{0.000000in}}{\pgfqpoint{-0.000000in}{0.000000in}}{%
\pgfpathmoveto{\pgfqpoint{-0.000000in}{0.000000in}}%
\pgfpathlineto{\pgfqpoint{-0.048611in}{0.000000in}}%
\pgfusepath{stroke,fill}%
}%
\begin{pgfscope}%
\pgfsys@transformshift{0.800000in}{1.270893in}%
\pgfsys@useobject{currentmarker}{}%
\end{pgfscope}%
\end{pgfscope}%
\begin{pgfscope}%
\definecolor{textcolor}{rgb}{0.000000,0.000000,0.000000}%
\pgfsetstrokecolor{textcolor}%
\pgfsetfillcolor{textcolor}%
\pgftext[x=0.393533in, y=1.218131in, left, base]{\color{textcolor}{\sffamily\fontsize{10.000000}{12.000000}\selectfont\catcode`\^=\active\def^{\ifmmode\sp\else\^{}\fi}\catcode`\%=\active\def%{\%}0.96}}%
\end{pgfscope}%
\begin{pgfscope}%
\pgfsetbuttcap%
\pgfsetroundjoin%
\definecolor{currentfill}{rgb}{0.000000,0.000000,0.000000}%
\pgfsetfillcolor{currentfill}%
\pgfsetlinewidth{0.803000pt}%
\definecolor{currentstroke}{rgb}{0.000000,0.000000,0.000000}%
\pgfsetstrokecolor{currentstroke}%
\pgfsetdash{}{0pt}%
\pgfsys@defobject{currentmarker}{\pgfqpoint{-0.048611in}{0.000000in}}{\pgfqpoint{-0.000000in}{0.000000in}}{%
\pgfpathmoveto{\pgfqpoint{-0.000000in}{0.000000in}}%
\pgfpathlineto{\pgfqpoint{-0.048611in}{0.000000in}}%
\pgfusepath{stroke,fill}%
}%
\begin{pgfscope}%
\pgfsys@transformshift{0.800000in}{1.854169in}%
\pgfsys@useobject{currentmarker}{}%
\end{pgfscope}%
\end{pgfscope}%
\begin{pgfscope}%
\definecolor{textcolor}{rgb}{0.000000,0.000000,0.000000}%
\pgfsetstrokecolor{textcolor}%
\pgfsetfillcolor{textcolor}%
\pgftext[x=0.393533in, y=1.801407in, left, base]{\color{textcolor}{\sffamily\fontsize{10.000000}{12.000000}\selectfont\catcode`\^=\active\def^{\ifmmode\sp\else\^{}\fi}\catcode`\%=\active\def%{\%}0.98}}%
\end{pgfscope}%
\begin{pgfscope}%
\pgfsetbuttcap%
\pgfsetroundjoin%
\definecolor{currentfill}{rgb}{0.000000,0.000000,0.000000}%
\pgfsetfillcolor{currentfill}%
\pgfsetlinewidth{0.803000pt}%
\definecolor{currentstroke}{rgb}{0.000000,0.000000,0.000000}%
\pgfsetstrokecolor{currentstroke}%
\pgfsetdash{}{0pt}%
\pgfsys@defobject{currentmarker}{\pgfqpoint{-0.048611in}{0.000000in}}{\pgfqpoint{-0.000000in}{0.000000in}}{%
\pgfpathmoveto{\pgfqpoint{-0.000000in}{0.000000in}}%
\pgfpathlineto{\pgfqpoint{-0.048611in}{0.000000in}}%
\pgfusepath{stroke,fill}%
}%
\begin{pgfscope}%
\pgfsys@transformshift{0.800000in}{2.437445in}%
\pgfsys@useobject{currentmarker}{}%
\end{pgfscope}%
\end{pgfscope}%
\begin{pgfscope}%
\definecolor{textcolor}{rgb}{0.000000,0.000000,0.000000}%
\pgfsetstrokecolor{textcolor}%
\pgfsetfillcolor{textcolor}%
\pgftext[x=0.393533in, y=2.384684in, left, base]{\color{textcolor}{\sffamily\fontsize{10.000000}{12.000000}\selectfont\catcode`\^=\active\def^{\ifmmode\sp\else\^{}\fi}\catcode`\%=\active\def%{\%}1.00}}%
\end{pgfscope}%
\begin{pgfscope}%
\pgfsetbuttcap%
\pgfsetroundjoin%
\definecolor{currentfill}{rgb}{0.000000,0.000000,0.000000}%
\pgfsetfillcolor{currentfill}%
\pgfsetlinewidth{0.803000pt}%
\definecolor{currentstroke}{rgb}{0.000000,0.000000,0.000000}%
\pgfsetstrokecolor{currentstroke}%
\pgfsetdash{}{0pt}%
\pgfsys@defobject{currentmarker}{\pgfqpoint{-0.048611in}{0.000000in}}{\pgfqpoint{-0.000000in}{0.000000in}}{%
\pgfpathmoveto{\pgfqpoint{-0.000000in}{0.000000in}}%
\pgfpathlineto{\pgfqpoint{-0.048611in}{0.000000in}}%
\pgfusepath{stroke,fill}%
}%
\begin{pgfscope}%
\pgfsys@transformshift{0.800000in}{3.020722in}%
\pgfsys@useobject{currentmarker}{}%
\end{pgfscope}%
\end{pgfscope}%
\begin{pgfscope}%
\definecolor{textcolor}{rgb}{0.000000,0.000000,0.000000}%
\pgfsetstrokecolor{textcolor}%
\pgfsetfillcolor{textcolor}%
\pgftext[x=0.393533in, y=2.967960in, left, base]{\color{textcolor}{\sffamily\fontsize{10.000000}{12.000000}\selectfont\catcode`\^=\active\def^{\ifmmode\sp\else\^{}\fi}\catcode`\%=\active\def%{\%}1.02}}%
\end{pgfscope}%
\begin{pgfscope}%
\pgfsetbuttcap%
\pgfsetroundjoin%
\definecolor{currentfill}{rgb}{0.000000,0.000000,0.000000}%
\pgfsetfillcolor{currentfill}%
\pgfsetlinewidth{0.803000pt}%
\definecolor{currentstroke}{rgb}{0.000000,0.000000,0.000000}%
\pgfsetstrokecolor{currentstroke}%
\pgfsetdash{}{0pt}%
\pgfsys@defobject{currentmarker}{\pgfqpoint{-0.048611in}{0.000000in}}{\pgfqpoint{-0.000000in}{0.000000in}}{%
\pgfpathmoveto{\pgfqpoint{-0.000000in}{0.000000in}}%
\pgfpathlineto{\pgfqpoint{-0.048611in}{0.000000in}}%
\pgfusepath{stroke,fill}%
}%
\begin{pgfscope}%
\pgfsys@transformshift{0.800000in}{3.603998in}%
\pgfsys@useobject{currentmarker}{}%
\end{pgfscope}%
\end{pgfscope}%
\begin{pgfscope}%
\definecolor{textcolor}{rgb}{0.000000,0.000000,0.000000}%
\pgfsetstrokecolor{textcolor}%
\pgfsetfillcolor{textcolor}%
\pgftext[x=0.393533in, y=3.551237in, left, base]{\color{textcolor}{\sffamily\fontsize{10.000000}{12.000000}\selectfont\catcode`\^=\active\def^{\ifmmode\sp\else\^{}\fi}\catcode`\%=\active\def%{\%}1.04}}%
\end{pgfscope}%
\begin{pgfscope}%
\pgfsetbuttcap%
\pgfsetroundjoin%
\definecolor{currentfill}{rgb}{0.000000,0.000000,0.000000}%
\pgfsetfillcolor{currentfill}%
\pgfsetlinewidth{0.803000pt}%
\definecolor{currentstroke}{rgb}{0.000000,0.000000,0.000000}%
\pgfsetstrokecolor{currentstroke}%
\pgfsetdash{}{0pt}%
\pgfsys@defobject{currentmarker}{\pgfqpoint{-0.048611in}{0.000000in}}{\pgfqpoint{-0.000000in}{0.000000in}}{%
\pgfpathmoveto{\pgfqpoint{-0.000000in}{0.000000in}}%
\pgfpathlineto{\pgfqpoint{-0.048611in}{0.000000in}}%
\pgfusepath{stroke,fill}%
}%
\begin{pgfscope}%
\pgfsys@transformshift{0.800000in}{4.187275in}%
\pgfsys@useobject{currentmarker}{}%
\end{pgfscope}%
\end{pgfscope}%
\begin{pgfscope}%
\definecolor{textcolor}{rgb}{0.000000,0.000000,0.000000}%
\pgfsetstrokecolor{textcolor}%
\pgfsetfillcolor{textcolor}%
\pgftext[x=0.393533in, y=4.134513in, left, base]{\color{textcolor}{\sffamily\fontsize{10.000000}{12.000000}\selectfont\catcode`\^=\active\def^{\ifmmode\sp\else\^{}\fi}\catcode`\%=\active\def%{\%}1.06}}%
\end{pgfscope}%
\begin{pgfscope}%
\definecolor{textcolor}{rgb}{0.000000,0.000000,0.000000}%
\pgfsetstrokecolor{textcolor}%
\pgfsetfillcolor{textcolor}%
\pgftext[x=0.337977in,y=2.376000in,,bottom,rotate=90.000000]{\color{textcolor}{\sffamily\fontsize{10.000000}{12.000000}\selectfont\catcode`\^=\active\def^{\ifmmode\sp\else\^{}\fi}\catcode`\%=\active\def%{\%}Perda em Validação}}%
\end{pgfscope}%
\begin{pgfscope}%
\pgfpathrectangle{\pgfqpoint{0.800000in}{0.528000in}}{\pgfqpoint{4.960000in}{3.696000in}}%
\pgfusepath{clip}%
\pgfsetrectcap%
\pgfsetroundjoin%
\pgfsetlinewidth{1.505625pt}%
\definecolor{currentstroke}{rgb}{0.121569,0.466667,0.705882}%
\pgfsetstrokecolor{currentstroke}%
\pgfsetdash{}{0pt}%
\pgfpathmoveto{\pgfqpoint{1.025455in}{0.842222in}}%
\pgfpathlineto{\pgfqpoint{1.121393in}{0.728145in}}%
\pgfpathlineto{\pgfqpoint{1.217331in}{0.882878in}}%
\pgfpathlineto{\pgfqpoint{1.313269in}{1.051901in}}%
\pgfpathlineto{\pgfqpoint{1.409207in}{0.861707in}}%
\pgfpathlineto{\pgfqpoint{1.505145in}{2.258067in}}%
\pgfpathlineto{\pgfqpoint{1.601083in}{2.181842in}}%
\pgfpathlineto{\pgfqpoint{1.697021in}{1.193767in}}%
\pgfpathlineto{\pgfqpoint{1.792959in}{2.341599in}}%
\pgfpathlineto{\pgfqpoint{1.888897in}{0.778738in}}%
\pgfpathlineto{\pgfqpoint{1.984836in}{0.729601in}}%
\pgfpathlineto{\pgfqpoint{2.080774in}{0.739292in}}%
\pgfpathlineto{\pgfqpoint{2.176712in}{0.788670in}}%
\pgfpathlineto{\pgfqpoint{2.272650in}{1.611364in}}%
\pgfpathlineto{\pgfqpoint{2.368588in}{0.696207in}}%
\pgfpathlineto{\pgfqpoint{2.464526in}{0.717857in}}%
\pgfpathlineto{\pgfqpoint{2.560464in}{0.856118in}}%
\pgfpathlineto{\pgfqpoint{2.656402in}{0.698070in}}%
\pgfpathlineto{\pgfqpoint{2.752340in}{1.184057in}}%
\pgfpathlineto{\pgfqpoint{2.848279in}{0.862868in}}%
\pgfpathlineto{\pgfqpoint{2.944217in}{0.924364in}}%
\pgfpathlineto{\pgfqpoint{3.040155in}{0.700026in}}%
\pgfpathlineto{\pgfqpoint{3.136093in}{0.699049in}}%
\pgfpathlineto{\pgfqpoint{3.232031in}{0.698628in}}%
\pgfpathlineto{\pgfqpoint{3.327969in}{0.696765in}}%
\pgfpathlineto{\pgfqpoint{3.423907in}{1.004934in}}%
\pgfpathlineto{\pgfqpoint{3.519845in}{0.696866in}}%
\pgfpathlineto{\pgfqpoint{3.615783in}{4.056000in}}%
\pgfpathlineto{\pgfqpoint{3.711721in}{0.699638in}}%
\pgfpathlineto{\pgfqpoint{3.807660in}{1.034007in}}%
\pgfpathlineto{\pgfqpoint{3.903598in}{3.268843in}}%
\pgfpathlineto{\pgfqpoint{3.999536in}{0.713008in}}%
\pgfpathlineto{\pgfqpoint{4.095474in}{0.806340in}}%
\pgfpathlineto{\pgfqpoint{4.191412in}{0.702221in}}%
\pgfpathlineto{\pgfqpoint{4.287350in}{0.718471in}}%
\pgfpathlineto{\pgfqpoint{4.383288in}{0.730568in}}%
\pgfpathlineto{\pgfqpoint{4.479226in}{0.703144in}}%
\pgfpathlineto{\pgfqpoint{4.575164in}{0.713791in}}%
\pgfpathlineto{\pgfqpoint{4.671103in}{0.698564in}}%
\pgfpathlineto{\pgfqpoint{4.767041in}{0.724800in}}%
\pgfpathlineto{\pgfqpoint{4.862979in}{0.696122in}}%
\pgfpathlineto{\pgfqpoint{4.958917in}{0.770910in}}%
\pgfpathlineto{\pgfqpoint{5.054855in}{0.710689in}}%
\pgfpathlineto{\pgfqpoint{5.150793in}{0.742884in}}%
\pgfpathlineto{\pgfqpoint{5.246731in}{0.704321in}}%
\pgfpathlineto{\pgfqpoint{5.342669in}{0.696066in}}%
\pgfpathlineto{\pgfqpoint{5.438607in}{0.721278in}}%
\pgfpathlineto{\pgfqpoint{5.534545in}{0.696000in}}%
\pgfusepath{stroke}%
\end{pgfscope}%
\begin{pgfscope}%
\pgfsetrectcap%
\pgfsetmiterjoin%
\pgfsetlinewidth{0.803000pt}%
\definecolor{currentstroke}{rgb}{0.000000,0.000000,0.000000}%
\pgfsetstrokecolor{currentstroke}%
\pgfsetdash{}{0pt}%
\pgfpathmoveto{\pgfqpoint{0.800000in}{0.528000in}}%
\pgfpathlineto{\pgfqpoint{0.800000in}{4.224000in}}%
\pgfusepath{stroke}%
\end{pgfscope}%
\begin{pgfscope}%
\pgfsetrectcap%
\pgfsetmiterjoin%
\pgfsetlinewidth{0.803000pt}%
\definecolor{currentstroke}{rgb}{0.000000,0.000000,0.000000}%
\pgfsetstrokecolor{currentstroke}%
\pgfsetdash{}{0pt}%
\pgfpathmoveto{\pgfqpoint{5.760000in}{0.528000in}}%
\pgfpathlineto{\pgfqpoint{5.760000in}{4.224000in}}%
\pgfusepath{stroke}%
\end{pgfscope}%
\begin{pgfscope}%
\pgfsetrectcap%
\pgfsetmiterjoin%
\pgfsetlinewidth{0.803000pt}%
\definecolor{currentstroke}{rgb}{0.000000,0.000000,0.000000}%
\pgfsetstrokecolor{currentstroke}%
\pgfsetdash{}{0pt}%
\pgfpathmoveto{\pgfqpoint{0.800000in}{0.528000in}}%
\pgfpathlineto{\pgfqpoint{5.760000in}{0.528000in}}%
\pgfusepath{stroke}%
\end{pgfscope}%
\begin{pgfscope}%
\pgfsetrectcap%
\pgfsetmiterjoin%
\pgfsetlinewidth{0.803000pt}%
\definecolor{currentstroke}{rgb}{0.000000,0.000000,0.000000}%
\pgfsetstrokecolor{currentstroke}%
\pgfsetdash{}{0pt}%
\pgfpathmoveto{\pgfqpoint{0.800000in}{4.224000in}}%
\pgfpathlineto{\pgfqpoint{5.760000in}{4.224000in}}%
\pgfusepath{stroke}%
\end{pgfscope}%
\begin{pgfscope}%
\definecolor{textcolor}{rgb}{0.000000,0.000000,0.000000}%
\pgfsetstrokecolor{textcolor}%
\pgfsetfillcolor{textcolor}%
\pgftext[x=3.280000in,y=4.307333in,,base]{\color{textcolor}{\sffamily\fontsize{12.000000}{14.400000}\selectfont\catcode`\^=\active\def^{\ifmmode\sp\else\^{}\fi}\catcode`\%=\active\def%{\%}Otimização de Hiperparâmetros (compIII)}}%
\end{pgfscope}%
\end{pgfpicture}%
\makeatother%
\endgroup%
}
    \caption{Evolução da função de perda no \textit{hypertuning} do modelo da competência III.}
    \label{fig:exp-hyp-c3}
\end{figure}

A figura \ref{fig:exp-hyp-c3} demonstra a evolução da função de perda para a competência III. A melhor configuração de hiperparâmetros obtida correspondeu a uma taxa de aprendizado de $2 \cdot 10^{-5}$, a um tamanho de lote de $4$ e a funções de ativação de Sigmoide, Sigmoide e Sigmoide, respectivamente. A melhor perda de validação registrada foi de aproximadamente 0,940287.

\subsubsection{Competência IV}
\label{subsec:exp-hyp-c4}

\begin{figure}[H]
    \resizebox{0.5\textwidth}{!}{%% Creator: Matplotlib, PGF backend
%%
%% To include the figure in your LaTeX document, write
%%   \input{<filename>.pgf}
%%
%% Make sure the required packages are loaded in your preamble
%%   \usepackage{pgf}
%%
%% Also ensure that all the required font packages are loaded; for instance,
%% the lmodern package is sometimes necessary when using math font.
%%   \usepackage{lmodern}
%%
%% Figures using additional raster images can only be included by \input if
%% they are in the same directory as the main LaTeX file. For loading figures
%% from other directories you can use the `import` package
%%   \usepackage{import}
%%
%% and then include the figures with
%%   \import{<path to file>}{<filename>.pgf}
%%
%% Matplotlib used the following preamble
%%   \def\mathdefault#1{#1}
%%   \everymath=\expandafter{\the\everymath\displaystyle}
%%   
%%   \usepackage{fontspec}
%%   \setmainfont{DejaVuSerif.ttf}[Path=\detokenize{/Users/josemayer/Documents/Pacotes/mambaforge/lib/python3.10/site-packages/matplotlib/mpl-data/fonts/ttf/}]
%%   \setsansfont{DejaVuSans.ttf}[Path=\detokenize{/Users/josemayer/Documents/Pacotes/mambaforge/lib/python3.10/site-packages/matplotlib/mpl-data/fonts/ttf/}]
%%   \setmonofont{DejaVuSansMono.ttf}[Path=\detokenize{/Users/josemayer/Documents/Pacotes/mambaforge/lib/python3.10/site-packages/matplotlib/mpl-data/fonts/ttf/}]
%%   \makeatletter\@ifpackageloaded{underscore}{}{\usepackage[strings]{underscore}}\makeatother
%%
\begingroup%
\makeatletter%
\begin{pgfpicture}%
\pgfpathrectangle{\pgfpointorigin}{\pgfqpoint{6.400000in}{4.800000in}}%
\pgfusepath{use as bounding box, clip}%
\begin{pgfscope}%
\pgfsetbuttcap%
\pgfsetmiterjoin%
\definecolor{currentfill}{rgb}{1.000000,1.000000,1.000000}%
\pgfsetfillcolor{currentfill}%
\pgfsetlinewidth{0.000000pt}%
\definecolor{currentstroke}{rgb}{1.000000,1.000000,1.000000}%
\pgfsetstrokecolor{currentstroke}%
\pgfsetdash{}{0pt}%
\pgfpathmoveto{\pgfqpoint{0.000000in}{0.000000in}}%
\pgfpathlineto{\pgfqpoint{6.400000in}{0.000000in}}%
\pgfpathlineto{\pgfqpoint{6.400000in}{4.800000in}}%
\pgfpathlineto{\pgfqpoint{0.000000in}{4.800000in}}%
\pgfpathlineto{\pgfqpoint{0.000000in}{0.000000in}}%
\pgfpathclose%
\pgfusepath{fill}%
\end{pgfscope}%
\begin{pgfscope}%
\pgfsetbuttcap%
\pgfsetmiterjoin%
\definecolor{currentfill}{rgb}{1.000000,1.000000,1.000000}%
\pgfsetfillcolor{currentfill}%
\pgfsetlinewidth{0.000000pt}%
\definecolor{currentstroke}{rgb}{0.000000,0.000000,0.000000}%
\pgfsetstrokecolor{currentstroke}%
\pgfsetstrokeopacity{0.000000}%
\pgfsetdash{}{0pt}%
\pgfpathmoveto{\pgfqpoint{0.800000in}{0.528000in}}%
\pgfpathlineto{\pgfqpoint{5.760000in}{0.528000in}}%
\pgfpathlineto{\pgfqpoint{5.760000in}{4.224000in}}%
\pgfpathlineto{\pgfqpoint{0.800000in}{4.224000in}}%
\pgfpathlineto{\pgfqpoint{0.800000in}{0.528000in}}%
\pgfpathclose%
\pgfusepath{fill}%
\end{pgfscope}%
\begin{pgfscope}%
\pgfsetbuttcap%
\pgfsetroundjoin%
\definecolor{currentfill}{rgb}{0.000000,0.000000,0.000000}%
\pgfsetfillcolor{currentfill}%
\pgfsetlinewidth{0.803000pt}%
\definecolor{currentstroke}{rgb}{0.000000,0.000000,0.000000}%
\pgfsetstrokecolor{currentstroke}%
\pgfsetdash{}{0pt}%
\pgfsys@defobject{currentmarker}{\pgfqpoint{0.000000in}{-0.048611in}}{\pgfqpoint{0.000000in}{0.000000in}}{%
\pgfpathmoveto{\pgfqpoint{0.000000in}{0.000000in}}%
\pgfpathlineto{\pgfqpoint{0.000000in}{-0.048611in}}%
\pgfusepath{stroke,fill}%
}%
\begin{pgfscope}%
\pgfsys@transformshift{0.929516in}{0.528000in}%
\pgfsys@useobject{currentmarker}{}%
\end{pgfscope}%
\end{pgfscope}%
\begin{pgfscope}%
\definecolor{textcolor}{rgb}{0.000000,0.000000,0.000000}%
\pgfsetstrokecolor{textcolor}%
\pgfsetfillcolor{textcolor}%
\pgftext[x=0.929516in,y=0.430778in,,top]{\color{textcolor}{\sffamily\fontsize{10.000000}{12.000000}\selectfont\catcode`\^=\active\def^{\ifmmode\sp\else\^{}\fi}\catcode`\%=\active\def%{\%}0}}%
\end{pgfscope}%
\begin{pgfscope}%
\pgfsetbuttcap%
\pgfsetroundjoin%
\definecolor{currentfill}{rgb}{0.000000,0.000000,0.000000}%
\pgfsetfillcolor{currentfill}%
\pgfsetlinewidth{0.803000pt}%
\definecolor{currentstroke}{rgb}{0.000000,0.000000,0.000000}%
\pgfsetstrokecolor{currentstroke}%
\pgfsetdash{}{0pt}%
\pgfsys@defobject{currentmarker}{\pgfqpoint{0.000000in}{-0.048611in}}{\pgfqpoint{0.000000in}{0.000000in}}{%
\pgfpathmoveto{\pgfqpoint{0.000000in}{0.000000in}}%
\pgfpathlineto{\pgfqpoint{0.000000in}{-0.048611in}}%
\pgfusepath{stroke,fill}%
}%
\begin{pgfscope}%
\pgfsys@transformshift{1.888897in}{0.528000in}%
\pgfsys@useobject{currentmarker}{}%
\end{pgfscope}%
\end{pgfscope}%
\begin{pgfscope}%
\definecolor{textcolor}{rgb}{0.000000,0.000000,0.000000}%
\pgfsetstrokecolor{textcolor}%
\pgfsetfillcolor{textcolor}%
\pgftext[x=1.888897in,y=0.430778in,,top]{\color{textcolor}{\sffamily\fontsize{10.000000}{12.000000}\selectfont\catcode`\^=\active\def^{\ifmmode\sp\else\^{}\fi}\catcode`\%=\active\def%{\%}10}}%
\end{pgfscope}%
\begin{pgfscope}%
\pgfsetbuttcap%
\pgfsetroundjoin%
\definecolor{currentfill}{rgb}{0.000000,0.000000,0.000000}%
\pgfsetfillcolor{currentfill}%
\pgfsetlinewidth{0.803000pt}%
\definecolor{currentstroke}{rgb}{0.000000,0.000000,0.000000}%
\pgfsetstrokecolor{currentstroke}%
\pgfsetdash{}{0pt}%
\pgfsys@defobject{currentmarker}{\pgfqpoint{0.000000in}{-0.048611in}}{\pgfqpoint{0.000000in}{0.000000in}}{%
\pgfpathmoveto{\pgfqpoint{0.000000in}{0.000000in}}%
\pgfpathlineto{\pgfqpoint{0.000000in}{-0.048611in}}%
\pgfusepath{stroke,fill}%
}%
\begin{pgfscope}%
\pgfsys@transformshift{2.848279in}{0.528000in}%
\pgfsys@useobject{currentmarker}{}%
\end{pgfscope}%
\end{pgfscope}%
\begin{pgfscope}%
\definecolor{textcolor}{rgb}{0.000000,0.000000,0.000000}%
\pgfsetstrokecolor{textcolor}%
\pgfsetfillcolor{textcolor}%
\pgftext[x=2.848279in,y=0.430778in,,top]{\color{textcolor}{\sffamily\fontsize{10.000000}{12.000000}\selectfont\catcode`\^=\active\def^{\ifmmode\sp\else\^{}\fi}\catcode`\%=\active\def%{\%}20}}%
\end{pgfscope}%
\begin{pgfscope}%
\pgfsetbuttcap%
\pgfsetroundjoin%
\definecolor{currentfill}{rgb}{0.000000,0.000000,0.000000}%
\pgfsetfillcolor{currentfill}%
\pgfsetlinewidth{0.803000pt}%
\definecolor{currentstroke}{rgb}{0.000000,0.000000,0.000000}%
\pgfsetstrokecolor{currentstroke}%
\pgfsetdash{}{0pt}%
\pgfsys@defobject{currentmarker}{\pgfqpoint{0.000000in}{-0.048611in}}{\pgfqpoint{0.000000in}{0.000000in}}{%
\pgfpathmoveto{\pgfqpoint{0.000000in}{0.000000in}}%
\pgfpathlineto{\pgfqpoint{0.000000in}{-0.048611in}}%
\pgfusepath{stroke,fill}%
}%
\begin{pgfscope}%
\pgfsys@transformshift{3.807660in}{0.528000in}%
\pgfsys@useobject{currentmarker}{}%
\end{pgfscope}%
\end{pgfscope}%
\begin{pgfscope}%
\definecolor{textcolor}{rgb}{0.000000,0.000000,0.000000}%
\pgfsetstrokecolor{textcolor}%
\pgfsetfillcolor{textcolor}%
\pgftext[x=3.807660in,y=0.430778in,,top]{\color{textcolor}{\sffamily\fontsize{10.000000}{12.000000}\selectfont\catcode`\^=\active\def^{\ifmmode\sp\else\^{}\fi}\catcode`\%=\active\def%{\%}30}}%
\end{pgfscope}%
\begin{pgfscope}%
\pgfsetbuttcap%
\pgfsetroundjoin%
\definecolor{currentfill}{rgb}{0.000000,0.000000,0.000000}%
\pgfsetfillcolor{currentfill}%
\pgfsetlinewidth{0.803000pt}%
\definecolor{currentstroke}{rgb}{0.000000,0.000000,0.000000}%
\pgfsetstrokecolor{currentstroke}%
\pgfsetdash{}{0pt}%
\pgfsys@defobject{currentmarker}{\pgfqpoint{0.000000in}{-0.048611in}}{\pgfqpoint{0.000000in}{0.000000in}}{%
\pgfpathmoveto{\pgfqpoint{0.000000in}{0.000000in}}%
\pgfpathlineto{\pgfqpoint{0.000000in}{-0.048611in}}%
\pgfusepath{stroke,fill}%
}%
\begin{pgfscope}%
\pgfsys@transformshift{4.767041in}{0.528000in}%
\pgfsys@useobject{currentmarker}{}%
\end{pgfscope}%
\end{pgfscope}%
\begin{pgfscope}%
\definecolor{textcolor}{rgb}{0.000000,0.000000,0.000000}%
\pgfsetstrokecolor{textcolor}%
\pgfsetfillcolor{textcolor}%
\pgftext[x=4.767041in,y=0.430778in,,top]{\color{textcolor}{\sffamily\fontsize{10.000000}{12.000000}\selectfont\catcode`\^=\active\def^{\ifmmode\sp\else\^{}\fi}\catcode`\%=\active\def%{\%}40}}%
\end{pgfscope}%
\begin{pgfscope}%
\pgfsetbuttcap%
\pgfsetroundjoin%
\definecolor{currentfill}{rgb}{0.000000,0.000000,0.000000}%
\pgfsetfillcolor{currentfill}%
\pgfsetlinewidth{0.803000pt}%
\definecolor{currentstroke}{rgb}{0.000000,0.000000,0.000000}%
\pgfsetstrokecolor{currentstroke}%
\pgfsetdash{}{0pt}%
\pgfsys@defobject{currentmarker}{\pgfqpoint{0.000000in}{-0.048611in}}{\pgfqpoint{0.000000in}{0.000000in}}{%
\pgfpathmoveto{\pgfqpoint{0.000000in}{0.000000in}}%
\pgfpathlineto{\pgfqpoint{0.000000in}{-0.048611in}}%
\pgfusepath{stroke,fill}%
}%
\begin{pgfscope}%
\pgfsys@transformshift{5.726422in}{0.528000in}%
\pgfsys@useobject{currentmarker}{}%
\end{pgfscope}%
\end{pgfscope}%
\begin{pgfscope}%
\definecolor{textcolor}{rgb}{0.000000,0.000000,0.000000}%
\pgfsetstrokecolor{textcolor}%
\pgfsetfillcolor{textcolor}%
\pgftext[x=5.726422in,y=0.430778in,,top]{\color{textcolor}{\sffamily\fontsize{10.000000}{12.000000}\selectfont\catcode`\^=\active\def^{\ifmmode\sp\else\^{}\fi}\catcode`\%=\active\def%{\%}50}}%
\end{pgfscope}%
\begin{pgfscope}%
\definecolor{textcolor}{rgb}{0.000000,0.000000,0.000000}%
\pgfsetstrokecolor{textcolor}%
\pgfsetfillcolor{textcolor}%
\pgftext[x=3.280000in,y=0.240809in,,top]{\color{textcolor}{\sffamily\fontsize{10.000000}{12.000000}\selectfont\catcode`\^=\active\def^{\ifmmode\sp\else\^{}\fi}\catcode`\%=\active\def%{\%}Trial}}%
\end{pgfscope}%
\begin{pgfscope}%
\pgfsetbuttcap%
\pgfsetroundjoin%
\definecolor{currentfill}{rgb}{0.000000,0.000000,0.000000}%
\pgfsetfillcolor{currentfill}%
\pgfsetlinewidth{0.803000pt}%
\definecolor{currentstroke}{rgb}{0.000000,0.000000,0.000000}%
\pgfsetstrokecolor{currentstroke}%
\pgfsetdash{}{0pt}%
\pgfsys@defobject{currentmarker}{\pgfqpoint{-0.048611in}{0.000000in}}{\pgfqpoint{-0.000000in}{0.000000in}}{%
\pgfpathmoveto{\pgfqpoint{-0.000000in}{0.000000in}}%
\pgfpathlineto{\pgfqpoint{-0.048611in}{0.000000in}}%
\pgfusepath{stroke,fill}%
}%
\begin{pgfscope}%
\pgfsys@transformshift{0.800000in}{0.883809in}%
\pgfsys@useobject{currentmarker}{}%
\end{pgfscope}%
\end{pgfscope}%
\begin{pgfscope}%
\definecolor{textcolor}{rgb}{0.000000,0.000000,0.000000}%
\pgfsetstrokecolor{textcolor}%
\pgfsetfillcolor{textcolor}%
\pgftext[x=0.393533in, y=0.831047in, left, base]{\color{textcolor}{\sffamily\fontsize{10.000000}{12.000000}\selectfont\catcode`\^=\active\def^{\ifmmode\sp\else\^{}\fi}\catcode`\%=\active\def%{\%}1.30}}%
\end{pgfscope}%
\begin{pgfscope}%
\pgfsetbuttcap%
\pgfsetroundjoin%
\definecolor{currentfill}{rgb}{0.000000,0.000000,0.000000}%
\pgfsetfillcolor{currentfill}%
\pgfsetlinewidth{0.803000pt}%
\definecolor{currentstroke}{rgb}{0.000000,0.000000,0.000000}%
\pgfsetstrokecolor{currentstroke}%
\pgfsetdash{}{0pt}%
\pgfsys@defobject{currentmarker}{\pgfqpoint{-0.048611in}{0.000000in}}{\pgfqpoint{-0.000000in}{0.000000in}}{%
\pgfpathmoveto{\pgfqpoint{-0.000000in}{0.000000in}}%
\pgfpathlineto{\pgfqpoint{-0.048611in}{0.000000in}}%
\pgfusepath{stroke,fill}%
}%
\begin{pgfscope}%
\pgfsys@transformshift{0.800000in}{1.470530in}%
\pgfsys@useobject{currentmarker}{}%
\end{pgfscope}%
\end{pgfscope}%
\begin{pgfscope}%
\definecolor{textcolor}{rgb}{0.000000,0.000000,0.000000}%
\pgfsetstrokecolor{textcolor}%
\pgfsetfillcolor{textcolor}%
\pgftext[x=0.393533in, y=1.417768in, left, base]{\color{textcolor}{\sffamily\fontsize{10.000000}{12.000000}\selectfont\catcode`\^=\active\def^{\ifmmode\sp\else\^{}\fi}\catcode`\%=\active\def%{\%}1.35}}%
\end{pgfscope}%
\begin{pgfscope}%
\pgfsetbuttcap%
\pgfsetroundjoin%
\definecolor{currentfill}{rgb}{0.000000,0.000000,0.000000}%
\pgfsetfillcolor{currentfill}%
\pgfsetlinewidth{0.803000pt}%
\definecolor{currentstroke}{rgb}{0.000000,0.000000,0.000000}%
\pgfsetstrokecolor{currentstroke}%
\pgfsetdash{}{0pt}%
\pgfsys@defobject{currentmarker}{\pgfqpoint{-0.048611in}{0.000000in}}{\pgfqpoint{-0.000000in}{0.000000in}}{%
\pgfpathmoveto{\pgfqpoint{-0.000000in}{0.000000in}}%
\pgfpathlineto{\pgfqpoint{-0.048611in}{0.000000in}}%
\pgfusepath{stroke,fill}%
}%
\begin{pgfscope}%
\pgfsys@transformshift{0.800000in}{2.057251in}%
\pgfsys@useobject{currentmarker}{}%
\end{pgfscope}%
\end{pgfscope}%
\begin{pgfscope}%
\definecolor{textcolor}{rgb}{0.000000,0.000000,0.000000}%
\pgfsetstrokecolor{textcolor}%
\pgfsetfillcolor{textcolor}%
\pgftext[x=0.393533in, y=2.004489in, left, base]{\color{textcolor}{\sffamily\fontsize{10.000000}{12.000000}\selectfont\catcode`\^=\active\def^{\ifmmode\sp\else\^{}\fi}\catcode`\%=\active\def%{\%}1.40}}%
\end{pgfscope}%
\begin{pgfscope}%
\pgfsetbuttcap%
\pgfsetroundjoin%
\definecolor{currentfill}{rgb}{0.000000,0.000000,0.000000}%
\pgfsetfillcolor{currentfill}%
\pgfsetlinewidth{0.803000pt}%
\definecolor{currentstroke}{rgb}{0.000000,0.000000,0.000000}%
\pgfsetstrokecolor{currentstroke}%
\pgfsetdash{}{0pt}%
\pgfsys@defobject{currentmarker}{\pgfqpoint{-0.048611in}{0.000000in}}{\pgfqpoint{-0.000000in}{0.000000in}}{%
\pgfpathmoveto{\pgfqpoint{-0.000000in}{0.000000in}}%
\pgfpathlineto{\pgfqpoint{-0.048611in}{0.000000in}}%
\pgfusepath{stroke,fill}%
}%
\begin{pgfscope}%
\pgfsys@transformshift{0.800000in}{2.643972in}%
\pgfsys@useobject{currentmarker}{}%
\end{pgfscope}%
\end{pgfscope}%
\begin{pgfscope}%
\definecolor{textcolor}{rgb}{0.000000,0.000000,0.000000}%
\pgfsetstrokecolor{textcolor}%
\pgfsetfillcolor{textcolor}%
\pgftext[x=0.393533in, y=2.591210in, left, base]{\color{textcolor}{\sffamily\fontsize{10.000000}{12.000000}\selectfont\catcode`\^=\active\def^{\ifmmode\sp\else\^{}\fi}\catcode`\%=\active\def%{\%}1.45}}%
\end{pgfscope}%
\begin{pgfscope}%
\pgfsetbuttcap%
\pgfsetroundjoin%
\definecolor{currentfill}{rgb}{0.000000,0.000000,0.000000}%
\pgfsetfillcolor{currentfill}%
\pgfsetlinewidth{0.803000pt}%
\definecolor{currentstroke}{rgb}{0.000000,0.000000,0.000000}%
\pgfsetstrokecolor{currentstroke}%
\pgfsetdash{}{0pt}%
\pgfsys@defobject{currentmarker}{\pgfqpoint{-0.048611in}{0.000000in}}{\pgfqpoint{-0.000000in}{0.000000in}}{%
\pgfpathmoveto{\pgfqpoint{-0.000000in}{0.000000in}}%
\pgfpathlineto{\pgfqpoint{-0.048611in}{0.000000in}}%
\pgfusepath{stroke,fill}%
}%
\begin{pgfscope}%
\pgfsys@transformshift{0.800000in}{3.230693in}%
\pgfsys@useobject{currentmarker}{}%
\end{pgfscope}%
\end{pgfscope}%
\begin{pgfscope}%
\definecolor{textcolor}{rgb}{0.000000,0.000000,0.000000}%
\pgfsetstrokecolor{textcolor}%
\pgfsetfillcolor{textcolor}%
\pgftext[x=0.393533in, y=3.177931in, left, base]{\color{textcolor}{\sffamily\fontsize{10.000000}{12.000000}\selectfont\catcode`\^=\active\def^{\ifmmode\sp\else\^{}\fi}\catcode`\%=\active\def%{\%}1.50}}%
\end{pgfscope}%
\begin{pgfscope}%
\pgfsetbuttcap%
\pgfsetroundjoin%
\definecolor{currentfill}{rgb}{0.000000,0.000000,0.000000}%
\pgfsetfillcolor{currentfill}%
\pgfsetlinewidth{0.803000pt}%
\definecolor{currentstroke}{rgb}{0.000000,0.000000,0.000000}%
\pgfsetstrokecolor{currentstroke}%
\pgfsetdash{}{0pt}%
\pgfsys@defobject{currentmarker}{\pgfqpoint{-0.048611in}{0.000000in}}{\pgfqpoint{-0.000000in}{0.000000in}}{%
\pgfpathmoveto{\pgfqpoint{-0.000000in}{0.000000in}}%
\pgfpathlineto{\pgfqpoint{-0.048611in}{0.000000in}}%
\pgfusepath{stroke,fill}%
}%
\begin{pgfscope}%
\pgfsys@transformshift{0.800000in}{3.817414in}%
\pgfsys@useobject{currentmarker}{}%
\end{pgfscope}%
\end{pgfscope}%
\begin{pgfscope}%
\definecolor{textcolor}{rgb}{0.000000,0.000000,0.000000}%
\pgfsetstrokecolor{textcolor}%
\pgfsetfillcolor{textcolor}%
\pgftext[x=0.393533in, y=3.764652in, left, base]{\color{textcolor}{\sffamily\fontsize{10.000000}{12.000000}\selectfont\catcode`\^=\active\def^{\ifmmode\sp\else\^{}\fi}\catcode`\%=\active\def%{\%}1.55}}%
\end{pgfscope}%
\begin{pgfscope}%
\definecolor{textcolor}{rgb}{0.000000,0.000000,0.000000}%
\pgfsetstrokecolor{textcolor}%
\pgfsetfillcolor{textcolor}%
\pgftext[x=0.337977in,y=2.376000in,,bottom,rotate=90.000000]{\color{textcolor}{\sffamily\fontsize{10.000000}{12.000000}\selectfont\catcode`\^=\active\def^{\ifmmode\sp\else\^{}\fi}\catcode`\%=\active\def%{\%}Perda em Validação}}%
\end{pgfscope}%
\begin{pgfscope}%
\pgfpathrectangle{\pgfqpoint{0.800000in}{0.528000in}}{\pgfqpoint{4.960000in}{3.696000in}}%
\pgfusepath{clip}%
\pgfsetrectcap%
\pgfsetroundjoin%
\pgfsetlinewidth{1.505625pt}%
\definecolor{currentstroke}{rgb}{0.121569,0.466667,0.705882}%
\pgfsetstrokecolor{currentstroke}%
\pgfsetdash{}{0pt}%
\pgfpathmoveto{\pgfqpoint{1.025455in}{0.712796in}}%
\pgfpathlineto{\pgfqpoint{1.121393in}{0.696490in}}%
\pgfpathlineto{\pgfqpoint{1.217331in}{0.696035in}}%
\pgfpathlineto{\pgfqpoint{1.313269in}{0.702053in}}%
\pgfpathlineto{\pgfqpoint{1.409207in}{1.321909in}}%
\pgfpathlineto{\pgfqpoint{1.505145in}{1.394742in}}%
\pgfpathlineto{\pgfqpoint{1.601083in}{0.737405in}}%
\pgfpathlineto{\pgfqpoint{1.697021in}{0.750169in}}%
\pgfpathlineto{\pgfqpoint{1.792959in}{1.486060in}}%
\pgfpathlineto{\pgfqpoint{1.888897in}{0.721158in}}%
\pgfpathlineto{\pgfqpoint{1.984836in}{0.703936in}}%
\pgfpathlineto{\pgfqpoint{2.080774in}{0.869176in}}%
\pgfpathlineto{\pgfqpoint{2.176712in}{0.724573in}}%
\pgfpathlineto{\pgfqpoint{2.272650in}{0.741449in}}%
\pgfpathlineto{\pgfqpoint{2.368588in}{0.850184in}}%
\pgfpathlineto{\pgfqpoint{2.464526in}{0.715520in}}%
\pgfpathlineto{\pgfqpoint{2.560464in}{0.715343in}}%
\pgfpathlineto{\pgfqpoint{2.656402in}{0.786888in}}%
\pgfpathlineto{\pgfqpoint{2.752340in}{0.696000in}}%
\pgfpathlineto{\pgfqpoint{2.848279in}{0.878345in}}%
\pgfpathlineto{\pgfqpoint{2.944217in}{0.722904in}}%
\pgfpathlineto{\pgfqpoint{3.040155in}{0.704057in}}%
\pgfpathlineto{\pgfqpoint{3.136093in}{0.872055in}}%
\pgfpathlineto{\pgfqpoint{3.232031in}{0.727014in}}%
\pgfpathlineto{\pgfqpoint{3.327969in}{2.545356in}}%
\pgfpathlineto{\pgfqpoint{3.423907in}{0.726133in}}%
\pgfpathlineto{\pgfqpoint{3.519845in}{0.711050in}}%
\pgfpathlineto{\pgfqpoint{3.615783in}{4.056000in}}%
\pgfpathlineto{\pgfqpoint{3.711721in}{0.897004in}}%
\pgfpathlineto{\pgfqpoint{3.807660in}{0.759603in}}%
\pgfpathlineto{\pgfqpoint{3.903598in}{0.814777in}}%
\pgfpathlineto{\pgfqpoint{3.999536in}{1.214319in}}%
\pgfpathlineto{\pgfqpoint{4.095474in}{0.800840in}}%
\pgfpathlineto{\pgfqpoint{4.191412in}{0.698367in}}%
\pgfpathlineto{\pgfqpoint{4.287350in}{0.765302in}}%
\pgfpathlineto{\pgfqpoint{4.383288in}{0.696344in}}%
\pgfpathlineto{\pgfqpoint{4.479226in}{0.711077in}}%
\pgfpathlineto{\pgfqpoint{4.575164in}{0.782878in}}%
\pgfpathlineto{\pgfqpoint{4.671103in}{0.703282in}}%
\pgfpathlineto{\pgfqpoint{4.767041in}{0.697588in}}%
\pgfpathlineto{\pgfqpoint{4.862979in}{0.986463in}}%
\pgfpathlineto{\pgfqpoint{4.958917in}{0.700830in}}%
\pgfpathlineto{\pgfqpoint{5.054855in}{1.428664in}}%
\pgfpathlineto{\pgfqpoint{5.150793in}{0.696133in}}%
\pgfpathlineto{\pgfqpoint{5.246731in}{0.790740in}}%
\pgfpathlineto{\pgfqpoint{5.342669in}{0.701397in}}%
\pgfpathlineto{\pgfqpoint{5.438607in}{0.720195in}}%
\pgfpathlineto{\pgfqpoint{5.534545in}{0.697225in}}%
\pgfusepath{stroke}%
\end{pgfscope}%
\begin{pgfscope}%
\pgfsetrectcap%
\pgfsetmiterjoin%
\pgfsetlinewidth{0.803000pt}%
\definecolor{currentstroke}{rgb}{0.000000,0.000000,0.000000}%
\pgfsetstrokecolor{currentstroke}%
\pgfsetdash{}{0pt}%
\pgfpathmoveto{\pgfqpoint{0.800000in}{0.528000in}}%
\pgfpathlineto{\pgfqpoint{0.800000in}{4.224000in}}%
\pgfusepath{stroke}%
\end{pgfscope}%
\begin{pgfscope}%
\pgfsetrectcap%
\pgfsetmiterjoin%
\pgfsetlinewidth{0.803000pt}%
\definecolor{currentstroke}{rgb}{0.000000,0.000000,0.000000}%
\pgfsetstrokecolor{currentstroke}%
\pgfsetdash{}{0pt}%
\pgfpathmoveto{\pgfqpoint{5.760000in}{0.528000in}}%
\pgfpathlineto{\pgfqpoint{5.760000in}{4.224000in}}%
\pgfusepath{stroke}%
\end{pgfscope}%
\begin{pgfscope}%
\pgfsetrectcap%
\pgfsetmiterjoin%
\pgfsetlinewidth{0.803000pt}%
\definecolor{currentstroke}{rgb}{0.000000,0.000000,0.000000}%
\pgfsetstrokecolor{currentstroke}%
\pgfsetdash{}{0pt}%
\pgfpathmoveto{\pgfqpoint{0.800000in}{0.528000in}}%
\pgfpathlineto{\pgfqpoint{5.760000in}{0.528000in}}%
\pgfusepath{stroke}%
\end{pgfscope}%
\begin{pgfscope}%
\pgfsetrectcap%
\pgfsetmiterjoin%
\pgfsetlinewidth{0.803000pt}%
\definecolor{currentstroke}{rgb}{0.000000,0.000000,0.000000}%
\pgfsetstrokecolor{currentstroke}%
\pgfsetdash{}{0pt}%
\pgfpathmoveto{\pgfqpoint{0.800000in}{4.224000in}}%
\pgfpathlineto{\pgfqpoint{5.760000in}{4.224000in}}%
\pgfusepath{stroke}%
\end{pgfscope}%
\begin{pgfscope}%
\definecolor{textcolor}{rgb}{0.000000,0.000000,0.000000}%
\pgfsetstrokecolor{textcolor}%
\pgfsetfillcolor{textcolor}%
\pgftext[x=3.280000in,y=4.307333in,,base]{\color{textcolor}{\sffamily\fontsize{12.000000}{14.400000}\selectfont\catcode`\^=\active\def^{\ifmmode\sp\else\^{}\fi}\catcode`\%=\active\def%{\%}Otimização de Hiperparâmetros (compIV)}}%
\end{pgfscope}%
\end{pgfpicture}%
\makeatother%
\endgroup%
}
    \caption{Evolução da função de perda no \textit{hypertuning} do modelo da competência IV.}
    \label{fig:exp-hyp-c4}
\end{figure}

A figura \ref{fig:exp-hyp-c4} ilustra a evolução da função de perda para a competência IV. A melhor configuração de hiperparâmetros obtida correspondeu a uma taxa de aprendizado de $2 \cdot 10^{-3}$, a um tamanho de lote de $2$ e a funções de ativação de SeLU, Sigmoide e Sigmoide, respectivamente. A melhor perda de validação obtida foi de aproximadamente 1,283995.

\subsubsection{Competência V}
\label{subsec:exp-hyp-c5}

\begin{figure}[H]
    \resizebox{0.5\textwidth}{!}{%% Creator: Matplotlib, PGF backend
%%
%% To include the figure in your LaTeX document, write
%%   \input{<filename>.pgf}
%%
%% Make sure the required packages are loaded in your preamble
%%   \usepackage{pgf}
%%
%% Also ensure that all the required font packages are loaded; for instance,
%% the lmodern package is sometimes necessary when using math font.
%%   \usepackage{lmodern}
%%
%% Figures using additional raster images can only be included by \input if
%% they are in the same directory as the main LaTeX file. For loading figures
%% from other directories you can use the `import` package
%%   \usepackage{import}
%%
%% and then include the figures with
%%   \import{<path to file>}{<filename>.pgf}
%%
%% Matplotlib used the following preamble
%%   \def\mathdefault#1{#1}
%%   \everymath=\expandafter{\the\everymath\displaystyle}
%%   
%%   \usepackage{fontspec}
%%   \setmainfont{DejaVuSerif.ttf}[Path=\detokenize{/Users/josemayer/Documents/Pacotes/mambaforge/lib/python3.10/site-packages/matplotlib/mpl-data/fonts/ttf/}]
%%   \setsansfont{DejaVuSans.ttf}[Path=\detokenize{/Users/josemayer/Documents/Pacotes/mambaforge/lib/python3.10/site-packages/matplotlib/mpl-data/fonts/ttf/}]
%%   \setmonofont{DejaVuSansMono.ttf}[Path=\detokenize{/Users/josemayer/Documents/Pacotes/mambaforge/lib/python3.10/site-packages/matplotlib/mpl-data/fonts/ttf/}]
%%   \makeatletter\@ifpackageloaded{underscore}{}{\usepackage[strings]{underscore}}\makeatother
%%
\begingroup%
\makeatletter%
\begin{pgfpicture}%
\pgfpathrectangle{\pgfpointorigin}{\pgfqpoint{6.400000in}{4.800000in}}%
\pgfusepath{use as bounding box, clip}%
\begin{pgfscope}%
\pgfsetbuttcap%
\pgfsetmiterjoin%
\definecolor{currentfill}{rgb}{1.000000,1.000000,1.000000}%
\pgfsetfillcolor{currentfill}%
\pgfsetlinewidth{0.000000pt}%
\definecolor{currentstroke}{rgb}{1.000000,1.000000,1.000000}%
\pgfsetstrokecolor{currentstroke}%
\pgfsetdash{}{0pt}%
\pgfpathmoveto{\pgfqpoint{0.000000in}{0.000000in}}%
\pgfpathlineto{\pgfqpoint{6.400000in}{0.000000in}}%
\pgfpathlineto{\pgfqpoint{6.400000in}{4.800000in}}%
\pgfpathlineto{\pgfqpoint{0.000000in}{4.800000in}}%
\pgfpathlineto{\pgfqpoint{0.000000in}{0.000000in}}%
\pgfpathclose%
\pgfusepath{fill}%
\end{pgfscope}%
\begin{pgfscope}%
\pgfsetbuttcap%
\pgfsetmiterjoin%
\definecolor{currentfill}{rgb}{1.000000,1.000000,1.000000}%
\pgfsetfillcolor{currentfill}%
\pgfsetlinewidth{0.000000pt}%
\definecolor{currentstroke}{rgb}{0.000000,0.000000,0.000000}%
\pgfsetstrokecolor{currentstroke}%
\pgfsetstrokeopacity{0.000000}%
\pgfsetdash{}{0pt}%
\pgfpathmoveto{\pgfqpoint{0.800000in}{0.528000in}}%
\pgfpathlineto{\pgfqpoint{5.760000in}{0.528000in}}%
\pgfpathlineto{\pgfqpoint{5.760000in}{4.224000in}}%
\pgfpathlineto{\pgfqpoint{0.800000in}{4.224000in}}%
\pgfpathlineto{\pgfqpoint{0.800000in}{0.528000in}}%
\pgfpathclose%
\pgfusepath{fill}%
\end{pgfscope}%
\begin{pgfscope}%
\pgfsetbuttcap%
\pgfsetroundjoin%
\definecolor{currentfill}{rgb}{0.000000,0.000000,0.000000}%
\pgfsetfillcolor{currentfill}%
\pgfsetlinewidth{0.803000pt}%
\definecolor{currentstroke}{rgb}{0.000000,0.000000,0.000000}%
\pgfsetstrokecolor{currentstroke}%
\pgfsetdash{}{0pt}%
\pgfsys@defobject{currentmarker}{\pgfqpoint{0.000000in}{-0.048611in}}{\pgfqpoint{0.000000in}{0.000000in}}{%
\pgfpathmoveto{\pgfqpoint{0.000000in}{0.000000in}}%
\pgfpathlineto{\pgfqpoint{0.000000in}{-0.048611in}}%
\pgfusepath{stroke,fill}%
}%
\begin{pgfscope}%
\pgfsys@transformshift{0.929516in}{0.528000in}%
\pgfsys@useobject{currentmarker}{}%
\end{pgfscope}%
\end{pgfscope}%
\begin{pgfscope}%
\definecolor{textcolor}{rgb}{0.000000,0.000000,0.000000}%
\pgfsetstrokecolor{textcolor}%
\pgfsetfillcolor{textcolor}%
\pgftext[x=0.929516in,y=0.430778in,,top]{\color{textcolor}{\sffamily\fontsize{10.000000}{12.000000}\selectfont\catcode`\^=\active\def^{\ifmmode\sp\else\^{}\fi}\catcode`\%=\active\def%{\%}0}}%
\end{pgfscope}%
\begin{pgfscope}%
\pgfsetbuttcap%
\pgfsetroundjoin%
\definecolor{currentfill}{rgb}{0.000000,0.000000,0.000000}%
\pgfsetfillcolor{currentfill}%
\pgfsetlinewidth{0.803000pt}%
\definecolor{currentstroke}{rgb}{0.000000,0.000000,0.000000}%
\pgfsetstrokecolor{currentstroke}%
\pgfsetdash{}{0pt}%
\pgfsys@defobject{currentmarker}{\pgfqpoint{0.000000in}{-0.048611in}}{\pgfqpoint{0.000000in}{0.000000in}}{%
\pgfpathmoveto{\pgfqpoint{0.000000in}{0.000000in}}%
\pgfpathlineto{\pgfqpoint{0.000000in}{-0.048611in}}%
\pgfusepath{stroke,fill}%
}%
\begin{pgfscope}%
\pgfsys@transformshift{1.888897in}{0.528000in}%
\pgfsys@useobject{currentmarker}{}%
\end{pgfscope}%
\end{pgfscope}%
\begin{pgfscope}%
\definecolor{textcolor}{rgb}{0.000000,0.000000,0.000000}%
\pgfsetstrokecolor{textcolor}%
\pgfsetfillcolor{textcolor}%
\pgftext[x=1.888897in,y=0.430778in,,top]{\color{textcolor}{\sffamily\fontsize{10.000000}{12.000000}\selectfont\catcode`\^=\active\def^{\ifmmode\sp\else\^{}\fi}\catcode`\%=\active\def%{\%}10}}%
\end{pgfscope}%
\begin{pgfscope}%
\pgfsetbuttcap%
\pgfsetroundjoin%
\definecolor{currentfill}{rgb}{0.000000,0.000000,0.000000}%
\pgfsetfillcolor{currentfill}%
\pgfsetlinewidth{0.803000pt}%
\definecolor{currentstroke}{rgb}{0.000000,0.000000,0.000000}%
\pgfsetstrokecolor{currentstroke}%
\pgfsetdash{}{0pt}%
\pgfsys@defobject{currentmarker}{\pgfqpoint{0.000000in}{-0.048611in}}{\pgfqpoint{0.000000in}{0.000000in}}{%
\pgfpathmoveto{\pgfqpoint{0.000000in}{0.000000in}}%
\pgfpathlineto{\pgfqpoint{0.000000in}{-0.048611in}}%
\pgfusepath{stroke,fill}%
}%
\begin{pgfscope}%
\pgfsys@transformshift{2.848279in}{0.528000in}%
\pgfsys@useobject{currentmarker}{}%
\end{pgfscope}%
\end{pgfscope}%
\begin{pgfscope}%
\definecolor{textcolor}{rgb}{0.000000,0.000000,0.000000}%
\pgfsetstrokecolor{textcolor}%
\pgfsetfillcolor{textcolor}%
\pgftext[x=2.848279in,y=0.430778in,,top]{\color{textcolor}{\sffamily\fontsize{10.000000}{12.000000}\selectfont\catcode`\^=\active\def^{\ifmmode\sp\else\^{}\fi}\catcode`\%=\active\def%{\%}20}}%
\end{pgfscope}%
\begin{pgfscope}%
\pgfsetbuttcap%
\pgfsetroundjoin%
\definecolor{currentfill}{rgb}{0.000000,0.000000,0.000000}%
\pgfsetfillcolor{currentfill}%
\pgfsetlinewidth{0.803000pt}%
\definecolor{currentstroke}{rgb}{0.000000,0.000000,0.000000}%
\pgfsetstrokecolor{currentstroke}%
\pgfsetdash{}{0pt}%
\pgfsys@defobject{currentmarker}{\pgfqpoint{0.000000in}{-0.048611in}}{\pgfqpoint{0.000000in}{0.000000in}}{%
\pgfpathmoveto{\pgfqpoint{0.000000in}{0.000000in}}%
\pgfpathlineto{\pgfqpoint{0.000000in}{-0.048611in}}%
\pgfusepath{stroke,fill}%
}%
\begin{pgfscope}%
\pgfsys@transformshift{3.807660in}{0.528000in}%
\pgfsys@useobject{currentmarker}{}%
\end{pgfscope}%
\end{pgfscope}%
\begin{pgfscope}%
\definecolor{textcolor}{rgb}{0.000000,0.000000,0.000000}%
\pgfsetstrokecolor{textcolor}%
\pgfsetfillcolor{textcolor}%
\pgftext[x=3.807660in,y=0.430778in,,top]{\color{textcolor}{\sffamily\fontsize{10.000000}{12.000000}\selectfont\catcode`\^=\active\def^{\ifmmode\sp\else\^{}\fi}\catcode`\%=\active\def%{\%}30}}%
\end{pgfscope}%
\begin{pgfscope}%
\pgfsetbuttcap%
\pgfsetroundjoin%
\definecolor{currentfill}{rgb}{0.000000,0.000000,0.000000}%
\pgfsetfillcolor{currentfill}%
\pgfsetlinewidth{0.803000pt}%
\definecolor{currentstroke}{rgb}{0.000000,0.000000,0.000000}%
\pgfsetstrokecolor{currentstroke}%
\pgfsetdash{}{0pt}%
\pgfsys@defobject{currentmarker}{\pgfqpoint{0.000000in}{-0.048611in}}{\pgfqpoint{0.000000in}{0.000000in}}{%
\pgfpathmoveto{\pgfqpoint{0.000000in}{0.000000in}}%
\pgfpathlineto{\pgfqpoint{0.000000in}{-0.048611in}}%
\pgfusepath{stroke,fill}%
}%
\begin{pgfscope}%
\pgfsys@transformshift{4.767041in}{0.528000in}%
\pgfsys@useobject{currentmarker}{}%
\end{pgfscope}%
\end{pgfscope}%
\begin{pgfscope}%
\definecolor{textcolor}{rgb}{0.000000,0.000000,0.000000}%
\pgfsetstrokecolor{textcolor}%
\pgfsetfillcolor{textcolor}%
\pgftext[x=4.767041in,y=0.430778in,,top]{\color{textcolor}{\sffamily\fontsize{10.000000}{12.000000}\selectfont\catcode`\^=\active\def^{\ifmmode\sp\else\^{}\fi}\catcode`\%=\active\def%{\%}40}}%
\end{pgfscope}%
\begin{pgfscope}%
\pgfsetbuttcap%
\pgfsetroundjoin%
\definecolor{currentfill}{rgb}{0.000000,0.000000,0.000000}%
\pgfsetfillcolor{currentfill}%
\pgfsetlinewidth{0.803000pt}%
\definecolor{currentstroke}{rgb}{0.000000,0.000000,0.000000}%
\pgfsetstrokecolor{currentstroke}%
\pgfsetdash{}{0pt}%
\pgfsys@defobject{currentmarker}{\pgfqpoint{0.000000in}{-0.048611in}}{\pgfqpoint{0.000000in}{0.000000in}}{%
\pgfpathmoveto{\pgfqpoint{0.000000in}{0.000000in}}%
\pgfpathlineto{\pgfqpoint{0.000000in}{-0.048611in}}%
\pgfusepath{stroke,fill}%
}%
\begin{pgfscope}%
\pgfsys@transformshift{5.726422in}{0.528000in}%
\pgfsys@useobject{currentmarker}{}%
\end{pgfscope}%
\end{pgfscope}%
\begin{pgfscope}%
\definecolor{textcolor}{rgb}{0.000000,0.000000,0.000000}%
\pgfsetstrokecolor{textcolor}%
\pgfsetfillcolor{textcolor}%
\pgftext[x=5.726422in,y=0.430778in,,top]{\color{textcolor}{\sffamily\fontsize{10.000000}{12.000000}\selectfont\catcode`\^=\active\def^{\ifmmode\sp\else\^{}\fi}\catcode`\%=\active\def%{\%}50}}%
\end{pgfscope}%
\begin{pgfscope}%
\definecolor{textcolor}{rgb}{0.000000,0.000000,0.000000}%
\pgfsetstrokecolor{textcolor}%
\pgfsetfillcolor{textcolor}%
\pgftext[x=3.280000in,y=0.240809in,,top]{\color{textcolor}{\sffamily\fontsize{10.000000}{12.000000}\selectfont\catcode`\^=\active\def^{\ifmmode\sp\else\^{}\fi}\catcode`\%=\active\def%{\%}Trial}}%
\end{pgfscope}%
\begin{pgfscope}%
\pgfsetbuttcap%
\pgfsetroundjoin%
\definecolor{currentfill}{rgb}{0.000000,0.000000,0.000000}%
\pgfsetfillcolor{currentfill}%
\pgfsetlinewidth{0.803000pt}%
\definecolor{currentstroke}{rgb}{0.000000,0.000000,0.000000}%
\pgfsetstrokecolor{currentstroke}%
\pgfsetdash{}{0pt}%
\pgfsys@defobject{currentmarker}{\pgfqpoint{-0.048611in}{0.000000in}}{\pgfqpoint{-0.000000in}{0.000000in}}{%
\pgfpathmoveto{\pgfqpoint{-0.000000in}{0.000000in}}%
\pgfpathlineto{\pgfqpoint{-0.048611in}{0.000000in}}%
\pgfusepath{stroke,fill}%
}%
\begin{pgfscope}%
\pgfsys@transformshift{0.800000in}{0.795162in}%
\pgfsys@useobject{currentmarker}{}%
\end{pgfscope}%
\end{pgfscope}%
\begin{pgfscope}%
\definecolor{textcolor}{rgb}{0.000000,0.000000,0.000000}%
\pgfsetstrokecolor{textcolor}%
\pgfsetfillcolor{textcolor}%
\pgftext[x=0.481898in, y=0.742401in, left, base]{\color{textcolor}{\sffamily\fontsize{10.000000}{12.000000}\selectfont\catcode`\^=\active\def^{\ifmmode\sp\else\^{}\fi}\catcode`\%=\active\def%{\%}1.8}}%
\end{pgfscope}%
\begin{pgfscope}%
\pgfsetbuttcap%
\pgfsetroundjoin%
\definecolor{currentfill}{rgb}{0.000000,0.000000,0.000000}%
\pgfsetfillcolor{currentfill}%
\pgfsetlinewidth{0.803000pt}%
\definecolor{currentstroke}{rgb}{0.000000,0.000000,0.000000}%
\pgfsetstrokecolor{currentstroke}%
\pgfsetdash{}{0pt}%
\pgfsys@defobject{currentmarker}{\pgfqpoint{-0.048611in}{0.000000in}}{\pgfqpoint{-0.000000in}{0.000000in}}{%
\pgfpathmoveto{\pgfqpoint{-0.000000in}{0.000000in}}%
\pgfpathlineto{\pgfqpoint{-0.048611in}{0.000000in}}%
\pgfusepath{stroke,fill}%
}%
\begin{pgfscope}%
\pgfsys@transformshift{0.800000in}{1.288186in}%
\pgfsys@useobject{currentmarker}{}%
\end{pgfscope}%
\end{pgfscope}%
\begin{pgfscope}%
\definecolor{textcolor}{rgb}{0.000000,0.000000,0.000000}%
\pgfsetstrokecolor{textcolor}%
\pgfsetfillcolor{textcolor}%
\pgftext[x=0.481898in, y=1.235424in, left, base]{\color{textcolor}{\sffamily\fontsize{10.000000}{12.000000}\selectfont\catcode`\^=\active\def^{\ifmmode\sp\else\^{}\fi}\catcode`\%=\active\def%{\%}1.9}}%
\end{pgfscope}%
\begin{pgfscope}%
\pgfsetbuttcap%
\pgfsetroundjoin%
\definecolor{currentfill}{rgb}{0.000000,0.000000,0.000000}%
\pgfsetfillcolor{currentfill}%
\pgfsetlinewidth{0.803000pt}%
\definecolor{currentstroke}{rgb}{0.000000,0.000000,0.000000}%
\pgfsetstrokecolor{currentstroke}%
\pgfsetdash{}{0pt}%
\pgfsys@defobject{currentmarker}{\pgfqpoint{-0.048611in}{0.000000in}}{\pgfqpoint{-0.000000in}{0.000000in}}{%
\pgfpathmoveto{\pgfqpoint{-0.000000in}{0.000000in}}%
\pgfpathlineto{\pgfqpoint{-0.048611in}{0.000000in}}%
\pgfusepath{stroke,fill}%
}%
\begin{pgfscope}%
\pgfsys@transformshift{0.800000in}{1.781210in}%
\pgfsys@useobject{currentmarker}{}%
\end{pgfscope}%
\end{pgfscope}%
\begin{pgfscope}%
\definecolor{textcolor}{rgb}{0.000000,0.000000,0.000000}%
\pgfsetstrokecolor{textcolor}%
\pgfsetfillcolor{textcolor}%
\pgftext[x=0.481898in, y=1.728448in, left, base]{\color{textcolor}{\sffamily\fontsize{10.000000}{12.000000}\selectfont\catcode`\^=\active\def^{\ifmmode\sp\else\^{}\fi}\catcode`\%=\active\def%{\%}2.0}}%
\end{pgfscope}%
\begin{pgfscope}%
\pgfsetbuttcap%
\pgfsetroundjoin%
\definecolor{currentfill}{rgb}{0.000000,0.000000,0.000000}%
\pgfsetfillcolor{currentfill}%
\pgfsetlinewidth{0.803000pt}%
\definecolor{currentstroke}{rgb}{0.000000,0.000000,0.000000}%
\pgfsetstrokecolor{currentstroke}%
\pgfsetdash{}{0pt}%
\pgfsys@defobject{currentmarker}{\pgfqpoint{-0.048611in}{0.000000in}}{\pgfqpoint{-0.000000in}{0.000000in}}{%
\pgfpathmoveto{\pgfqpoint{-0.000000in}{0.000000in}}%
\pgfpathlineto{\pgfqpoint{-0.048611in}{0.000000in}}%
\pgfusepath{stroke,fill}%
}%
\begin{pgfscope}%
\pgfsys@transformshift{0.800000in}{2.274234in}%
\pgfsys@useobject{currentmarker}{}%
\end{pgfscope}%
\end{pgfscope}%
\begin{pgfscope}%
\definecolor{textcolor}{rgb}{0.000000,0.000000,0.000000}%
\pgfsetstrokecolor{textcolor}%
\pgfsetfillcolor{textcolor}%
\pgftext[x=0.481898in, y=2.221472in, left, base]{\color{textcolor}{\sffamily\fontsize{10.000000}{12.000000}\selectfont\catcode`\^=\active\def^{\ifmmode\sp\else\^{}\fi}\catcode`\%=\active\def%{\%}2.1}}%
\end{pgfscope}%
\begin{pgfscope}%
\pgfsetbuttcap%
\pgfsetroundjoin%
\definecolor{currentfill}{rgb}{0.000000,0.000000,0.000000}%
\pgfsetfillcolor{currentfill}%
\pgfsetlinewidth{0.803000pt}%
\definecolor{currentstroke}{rgb}{0.000000,0.000000,0.000000}%
\pgfsetstrokecolor{currentstroke}%
\pgfsetdash{}{0pt}%
\pgfsys@defobject{currentmarker}{\pgfqpoint{-0.048611in}{0.000000in}}{\pgfqpoint{-0.000000in}{0.000000in}}{%
\pgfpathmoveto{\pgfqpoint{-0.000000in}{0.000000in}}%
\pgfpathlineto{\pgfqpoint{-0.048611in}{0.000000in}}%
\pgfusepath{stroke,fill}%
}%
\begin{pgfscope}%
\pgfsys@transformshift{0.800000in}{2.767258in}%
\pgfsys@useobject{currentmarker}{}%
\end{pgfscope}%
\end{pgfscope}%
\begin{pgfscope}%
\definecolor{textcolor}{rgb}{0.000000,0.000000,0.000000}%
\pgfsetstrokecolor{textcolor}%
\pgfsetfillcolor{textcolor}%
\pgftext[x=0.481898in, y=2.714496in, left, base]{\color{textcolor}{\sffamily\fontsize{10.000000}{12.000000}\selectfont\catcode`\^=\active\def^{\ifmmode\sp\else\^{}\fi}\catcode`\%=\active\def%{\%}2.2}}%
\end{pgfscope}%
\begin{pgfscope}%
\pgfsetbuttcap%
\pgfsetroundjoin%
\definecolor{currentfill}{rgb}{0.000000,0.000000,0.000000}%
\pgfsetfillcolor{currentfill}%
\pgfsetlinewidth{0.803000pt}%
\definecolor{currentstroke}{rgb}{0.000000,0.000000,0.000000}%
\pgfsetstrokecolor{currentstroke}%
\pgfsetdash{}{0pt}%
\pgfsys@defobject{currentmarker}{\pgfqpoint{-0.048611in}{0.000000in}}{\pgfqpoint{-0.000000in}{0.000000in}}{%
\pgfpathmoveto{\pgfqpoint{-0.000000in}{0.000000in}}%
\pgfpathlineto{\pgfqpoint{-0.048611in}{0.000000in}}%
\pgfusepath{stroke,fill}%
}%
\begin{pgfscope}%
\pgfsys@transformshift{0.800000in}{3.260282in}%
\pgfsys@useobject{currentmarker}{}%
\end{pgfscope}%
\end{pgfscope}%
\begin{pgfscope}%
\definecolor{textcolor}{rgb}{0.000000,0.000000,0.000000}%
\pgfsetstrokecolor{textcolor}%
\pgfsetfillcolor{textcolor}%
\pgftext[x=0.481898in, y=3.207520in, left, base]{\color{textcolor}{\sffamily\fontsize{10.000000}{12.000000}\selectfont\catcode`\^=\active\def^{\ifmmode\sp\else\^{}\fi}\catcode`\%=\active\def%{\%}2.3}}%
\end{pgfscope}%
\begin{pgfscope}%
\pgfsetbuttcap%
\pgfsetroundjoin%
\definecolor{currentfill}{rgb}{0.000000,0.000000,0.000000}%
\pgfsetfillcolor{currentfill}%
\pgfsetlinewidth{0.803000pt}%
\definecolor{currentstroke}{rgb}{0.000000,0.000000,0.000000}%
\pgfsetstrokecolor{currentstroke}%
\pgfsetdash{}{0pt}%
\pgfsys@defobject{currentmarker}{\pgfqpoint{-0.048611in}{0.000000in}}{\pgfqpoint{-0.000000in}{0.000000in}}{%
\pgfpathmoveto{\pgfqpoint{-0.000000in}{0.000000in}}%
\pgfpathlineto{\pgfqpoint{-0.048611in}{0.000000in}}%
\pgfusepath{stroke,fill}%
}%
\begin{pgfscope}%
\pgfsys@transformshift{0.800000in}{3.753306in}%
\pgfsys@useobject{currentmarker}{}%
\end{pgfscope}%
\end{pgfscope}%
\begin{pgfscope}%
\definecolor{textcolor}{rgb}{0.000000,0.000000,0.000000}%
\pgfsetstrokecolor{textcolor}%
\pgfsetfillcolor{textcolor}%
\pgftext[x=0.481898in, y=3.700544in, left, base]{\color{textcolor}{\sffamily\fontsize{10.000000}{12.000000}\selectfont\catcode`\^=\active\def^{\ifmmode\sp\else\^{}\fi}\catcode`\%=\active\def%{\%}2.4}}%
\end{pgfscope}%
\begin{pgfscope}%
\definecolor{textcolor}{rgb}{0.000000,0.000000,0.000000}%
\pgfsetstrokecolor{textcolor}%
\pgfsetfillcolor{textcolor}%
\pgftext[x=0.426343in,y=2.376000in,,bottom,rotate=90.000000]{\color{textcolor}{\sffamily\fontsize{10.000000}{12.000000}\selectfont\catcode`\^=\active\def^{\ifmmode\sp\else\^{}\fi}\catcode`\%=\active\def%{\%}Perda em Validação}}%
\end{pgfscope}%
\begin{pgfscope}%
\pgfpathrectangle{\pgfqpoint{0.800000in}{0.528000in}}{\pgfqpoint{4.960000in}{3.696000in}}%
\pgfusepath{clip}%
\pgfsetrectcap%
\pgfsetroundjoin%
\pgfsetlinewidth{1.505625pt}%
\definecolor{currentstroke}{rgb}{0.121569,0.466667,0.705882}%
\pgfsetstrokecolor{currentstroke}%
\pgfsetdash{}{0pt}%
\pgfpathmoveto{\pgfqpoint{1.025455in}{0.781853in}}%
\pgfpathlineto{\pgfqpoint{1.121393in}{0.720031in}}%
\pgfpathlineto{\pgfqpoint{1.217331in}{0.758660in}}%
\pgfpathlineto{\pgfqpoint{1.313269in}{0.819362in}}%
\pgfpathlineto{\pgfqpoint{1.409207in}{0.868647in}}%
\pgfpathlineto{\pgfqpoint{1.505145in}{0.701417in}}%
\pgfpathlineto{\pgfqpoint{1.601083in}{1.057968in}}%
\pgfpathlineto{\pgfqpoint{1.697021in}{1.215338in}}%
\pgfpathlineto{\pgfqpoint{1.792959in}{0.736424in}}%
\pgfpathlineto{\pgfqpoint{1.888897in}{0.700476in}}%
\pgfpathlineto{\pgfqpoint{1.984836in}{0.696422in}}%
\pgfpathlineto{\pgfqpoint{2.080774in}{0.740246in}}%
\pgfpathlineto{\pgfqpoint{2.176712in}{4.056000in}}%
\pgfpathlineto{\pgfqpoint{2.272650in}{0.745530in}}%
\pgfpathlineto{\pgfqpoint{2.368588in}{0.696000in}}%
\pgfpathlineto{\pgfqpoint{2.464526in}{0.701716in}}%
\pgfpathlineto{\pgfqpoint{2.560464in}{0.696009in}}%
\pgfpathlineto{\pgfqpoint{2.656402in}{0.853582in}}%
\pgfpathlineto{\pgfqpoint{2.752340in}{0.724302in}}%
\pgfpathlineto{\pgfqpoint{2.848279in}{0.702741in}}%
\pgfpathlineto{\pgfqpoint{2.944217in}{0.697431in}}%
\pgfpathlineto{\pgfqpoint{3.040155in}{0.700928in}}%
\pgfpathlineto{\pgfqpoint{3.136093in}{0.696138in}}%
\pgfpathlineto{\pgfqpoint{3.232031in}{0.714712in}}%
\pgfpathlineto{\pgfqpoint{3.327969in}{0.696006in}}%
\pgfpathlineto{\pgfqpoint{3.423907in}{0.705907in}}%
\pgfpathlineto{\pgfqpoint{3.519845in}{0.697073in}}%
\pgfpathlineto{\pgfqpoint{3.615783in}{1.125732in}}%
\pgfpathlineto{\pgfqpoint{3.711721in}{0.703158in}}%
\pgfpathlineto{\pgfqpoint{3.807660in}{0.827257in}}%
\pgfpathlineto{\pgfqpoint{3.903598in}{0.696142in}}%
\pgfpathlineto{\pgfqpoint{3.999536in}{0.729895in}}%
\pgfpathlineto{\pgfqpoint{4.095474in}{0.750014in}}%
\pgfpathlineto{\pgfqpoint{4.191412in}{0.826970in}}%
\pgfpathlineto{\pgfqpoint{4.287350in}{0.696030in}}%
\pgfpathlineto{\pgfqpoint{4.383288in}{0.753843in}}%
\pgfpathlineto{\pgfqpoint{4.479226in}{0.696799in}}%
\pgfpathlineto{\pgfqpoint{4.575164in}{0.697811in}}%
\pgfpathlineto{\pgfqpoint{4.671103in}{0.697170in}}%
\pgfpathlineto{\pgfqpoint{4.767041in}{0.714395in}}%
\pgfpathlineto{\pgfqpoint{4.862979in}{0.729923in}}%
\pgfpathlineto{\pgfqpoint{4.958917in}{0.698517in}}%
\pgfpathlineto{\pgfqpoint{5.054855in}{0.696000in}}%
\pgfpathlineto{\pgfqpoint{5.150793in}{0.758899in}}%
\pgfpathlineto{\pgfqpoint{5.246731in}{0.702852in}}%
\pgfpathlineto{\pgfqpoint{5.342669in}{0.704228in}}%
\pgfpathlineto{\pgfqpoint{5.438607in}{0.697474in}}%
\pgfpathlineto{\pgfqpoint{5.534545in}{0.716688in}}%
\pgfusepath{stroke}%
\end{pgfscope}%
\begin{pgfscope}%
\pgfsetrectcap%
\pgfsetmiterjoin%
\pgfsetlinewidth{0.803000pt}%
\definecolor{currentstroke}{rgb}{0.000000,0.000000,0.000000}%
\pgfsetstrokecolor{currentstroke}%
\pgfsetdash{}{0pt}%
\pgfpathmoveto{\pgfqpoint{0.800000in}{0.528000in}}%
\pgfpathlineto{\pgfqpoint{0.800000in}{4.224000in}}%
\pgfusepath{stroke}%
\end{pgfscope}%
\begin{pgfscope}%
\pgfsetrectcap%
\pgfsetmiterjoin%
\pgfsetlinewidth{0.803000pt}%
\definecolor{currentstroke}{rgb}{0.000000,0.000000,0.000000}%
\pgfsetstrokecolor{currentstroke}%
\pgfsetdash{}{0pt}%
\pgfpathmoveto{\pgfqpoint{5.760000in}{0.528000in}}%
\pgfpathlineto{\pgfqpoint{5.760000in}{4.224000in}}%
\pgfusepath{stroke}%
\end{pgfscope}%
\begin{pgfscope}%
\pgfsetrectcap%
\pgfsetmiterjoin%
\pgfsetlinewidth{0.803000pt}%
\definecolor{currentstroke}{rgb}{0.000000,0.000000,0.000000}%
\pgfsetstrokecolor{currentstroke}%
\pgfsetdash{}{0pt}%
\pgfpathmoveto{\pgfqpoint{0.800000in}{0.528000in}}%
\pgfpathlineto{\pgfqpoint{5.760000in}{0.528000in}}%
\pgfusepath{stroke}%
\end{pgfscope}%
\begin{pgfscope}%
\pgfsetrectcap%
\pgfsetmiterjoin%
\pgfsetlinewidth{0.803000pt}%
\definecolor{currentstroke}{rgb}{0.000000,0.000000,0.000000}%
\pgfsetstrokecolor{currentstroke}%
\pgfsetdash{}{0pt}%
\pgfpathmoveto{\pgfqpoint{0.800000in}{4.224000in}}%
\pgfpathlineto{\pgfqpoint{5.760000in}{4.224000in}}%
\pgfusepath{stroke}%
\end{pgfscope}%
\begin{pgfscope}%
\definecolor{textcolor}{rgb}{0.000000,0.000000,0.000000}%
\pgfsetstrokecolor{textcolor}%
\pgfsetfillcolor{textcolor}%
\pgftext[x=3.280000in,y=4.307333in,,base]{\color{textcolor}{\sffamily\fontsize{12.000000}{14.400000}\selectfont\catcode`\^=\active\def^{\ifmmode\sp\else\^{}\fi}\catcode`\%=\active\def%{\%}Otimização de Hiperparâmetros (compV)}}%
\end{pgfscope}%
\end{pgfpicture}%
\makeatother%
\endgroup%
}
    \caption{Evolução da função de perda no \textit{hypertuning} do modelo da competência V.}
    \label{fig:exp-hyp-c5}
\end{figure}

A figura \ref{fig:exp-hyp-c5} mostra a evolução da função de perda para a competência V. A melhor configuração de hiperparâmetros obtida correspondeu a uma taxa de aprendizado de $2 \cdot 10^{-3}$, a um tamanho de lote de $2$ e a funções de ativação de Sigmoide, Sigmoide e Sigmoide, respectivamente. A melhor perda de validação obtida foi de aproximadamente 1,779886.

\subsection{Análise Comparativa}

Além dos experimentos de otimização, realizamos experimentos de controle com hiperparâmetros fixados e conhecidos. Essa abordagem possibilita que se avalie a influência do processo de \textit{hypertuning} no desenvolvimento de redes especialistas de correção. O treinamento de controle foi feito nas duas bases de dados, mas a análise comparativa concentra-se nos resultados da base estendida.

Os mesmos hiperparâmetros foram escolhidos para os cinco modelos de avaliação: taxa de aprendizado de $2 \cdot 10^{-5}$, tamanho de lote de $4$ e funções de ativação SeLU, Sigmoide e SeLU, respectivamente. Essa configuração está dentro do espaço de possibilidades (\autoref{subsec:hyperparameter-tuning}) e foi selecionada experimentalmente no início do projeto por apresentar um desempenho razoável.

Os modelos de controle também foram treinados em 50 épocas, de modo que apenas a rede com melhor função de perda na base de validação foi salva. Isso é feito por meio do \textit{callback} \texttt{ModelCheckpoint}, apresentado anteriormente (\autoref{subsec:training-configurations}). Nas seções \ref{subsec:exp-fix-c1}, \ref{subsec:exp-fix-c2}, \ref{subsec:exp-fix-c3}, \ref{subsec:exp-fix-c4} e \ref{subsec:exp-fix-c5} confrontaremos, visualmente, a evolução dos treinamentos com \textit{hypertuning} e com hiperparâmetros fixados.

\subsubsection{Competência I}
\label{subsec:exp-fix-c1}

\begin{figure}[H]
    \begin{minipage}{0.45\textwidth}
        \resizebox{\textwidth}{!}{%% Creator: Matplotlib, PGF backend
%%
%% To include the figure in your LaTeX document, write
%%   \input{<filename>.pgf}
%%
%% Make sure the required packages are loaded in your preamble
%%   \usepackage{pgf}
%%
%% Also ensure that all the required font packages are loaded; for instance,
%% the lmodern package is sometimes necessary when using math font.
%%   \usepackage{lmodern}
%%
%% Figures using additional raster images can only be included by \input if
%% they are in the same directory as the main LaTeX file. For loading figures
%% from other directories you can use the `import` package
%%   \usepackage{import}
%%
%% and then include the figures with
%%   \import{<path to file>}{<filename>.pgf}
%%
%% Matplotlib used the following preamble
%%
%%   \usepackage{fontspec}
%%   \setmainfont{DejaVuSerif.ttf}[Path=\detokenize{/home/jose/anaconda3/envs/tf/lib/python3.9/site-packages/matplotlib/mpl-data/fonts/ttf/}]
%%   \setsansfont{DejaVuSans.ttf}[Path=\detokenize{/home/jose/anaconda3/envs/tf/lib/python3.9/site-packages/matplotlib/mpl-data/fonts/ttf/}]
%%   \setmonofont{DejaVuSansMono.ttf}[Path=\detokenize{/home/jose/anaconda3/envs/tf/lib/python3.9/site-packages/matplotlib/mpl-data/fonts/ttf/}]
%%   \makeatletter\@ifpackageloaded{underscore}{}{\usepackage[strings]{underscore}}\makeatother
%%
\begingroup%
\makeatletter%
\begin{pgfpicture}%
\pgfpathrectangle{\pgfpointorigin}{\pgfqpoint{6.400000in}{4.800000in}}%
\pgfusepath{use as bounding box, clip}%
\begin{pgfscope}%
\pgfsetbuttcap%
\pgfsetmiterjoin%
\definecolor{currentfill}{rgb}{1.000000,1.000000,1.000000}%
\pgfsetfillcolor{currentfill}%
\pgfsetlinewidth{0.000000pt}%
\definecolor{currentstroke}{rgb}{1.000000,1.000000,1.000000}%
\pgfsetstrokecolor{currentstroke}%
\pgfsetdash{}{0pt}%
\pgfpathmoveto{\pgfqpoint{0.000000in}{0.000000in}}%
\pgfpathlineto{\pgfqpoint{6.400000in}{0.000000in}}%
\pgfpathlineto{\pgfqpoint{6.400000in}{4.800000in}}%
\pgfpathlineto{\pgfqpoint{0.000000in}{4.800000in}}%
\pgfpathlineto{\pgfqpoint{0.000000in}{0.000000in}}%
\pgfpathclose%
\pgfusepath{fill}%
\end{pgfscope}%
\begin{pgfscope}%
\pgfsetbuttcap%
\pgfsetmiterjoin%
\definecolor{currentfill}{rgb}{1.000000,1.000000,1.000000}%
\pgfsetfillcolor{currentfill}%
\pgfsetlinewidth{0.000000pt}%
\definecolor{currentstroke}{rgb}{0.000000,0.000000,0.000000}%
\pgfsetstrokecolor{currentstroke}%
\pgfsetstrokeopacity{0.000000}%
\pgfsetdash{}{0pt}%
\pgfpathmoveto{\pgfqpoint{0.800000in}{0.528000in}}%
\pgfpathlineto{\pgfqpoint{5.760000in}{0.528000in}}%
\pgfpathlineto{\pgfqpoint{5.760000in}{4.224000in}}%
\pgfpathlineto{\pgfqpoint{0.800000in}{4.224000in}}%
\pgfpathlineto{\pgfqpoint{0.800000in}{0.528000in}}%
\pgfpathclose%
\pgfusepath{fill}%
\end{pgfscope}%
\begin{pgfscope}%
\pgfsetbuttcap%
\pgfsetroundjoin%
\definecolor{currentfill}{rgb}{0.000000,0.000000,0.000000}%
\pgfsetfillcolor{currentfill}%
\pgfsetlinewidth{0.803000pt}%
\definecolor{currentstroke}{rgb}{0.000000,0.000000,0.000000}%
\pgfsetstrokecolor{currentstroke}%
\pgfsetdash{}{0pt}%
\pgfsys@defobject{currentmarker}{\pgfqpoint{0.000000in}{-0.048611in}}{\pgfqpoint{0.000000in}{0.000000in}}{%
\pgfpathmoveto{\pgfqpoint{0.000000in}{0.000000in}}%
\pgfpathlineto{\pgfqpoint{0.000000in}{-0.048611in}}%
\pgfusepath{stroke,fill}%
}%
\begin{pgfscope}%
\pgfsys@transformshift{1.025455in}{0.528000in}%
\pgfsys@useobject{currentmarker}{}%
\end{pgfscope}%
\end{pgfscope}%
\begin{pgfscope}%
\definecolor{textcolor}{rgb}{0.000000,0.000000,0.000000}%
\pgfsetstrokecolor{textcolor}%
\pgfsetfillcolor{textcolor}%
\pgftext[x=1.025455in,y=0.430778in,,top]{\color{textcolor}\sffamily\fontsize{8.000000}{9.600000}\selectfont 0}%
\end{pgfscope}%
\begin{pgfscope}%
\pgfsetbuttcap%
\pgfsetroundjoin%
\definecolor{currentfill}{rgb}{0.000000,0.000000,0.000000}%
\pgfsetfillcolor{currentfill}%
\pgfsetlinewidth{0.803000pt}%
\definecolor{currentstroke}{rgb}{0.000000,0.000000,0.000000}%
\pgfsetstrokecolor{currentstroke}%
\pgfsetdash{}{0pt}%
\pgfsys@defobject{currentmarker}{\pgfqpoint{0.000000in}{-0.048611in}}{\pgfqpoint{0.000000in}{0.000000in}}{%
\pgfpathmoveto{\pgfqpoint{0.000000in}{0.000000in}}%
\pgfpathlineto{\pgfqpoint{0.000000in}{-0.048611in}}%
\pgfusepath{stroke,fill}%
}%
\begin{pgfscope}%
\pgfsys@transformshift{1.945677in}{0.528000in}%
\pgfsys@useobject{currentmarker}{}%
\end{pgfscope}%
\end{pgfscope}%
\begin{pgfscope}%
\definecolor{textcolor}{rgb}{0.000000,0.000000,0.000000}%
\pgfsetstrokecolor{textcolor}%
\pgfsetfillcolor{textcolor}%
\pgftext[x=1.945677in,y=0.430778in,,top]{\color{textcolor}\sffamily\fontsize{8.000000}{9.600000}\selectfont 10}%
\end{pgfscope}%
\begin{pgfscope}%
\pgfsetbuttcap%
\pgfsetroundjoin%
\definecolor{currentfill}{rgb}{0.000000,0.000000,0.000000}%
\pgfsetfillcolor{currentfill}%
\pgfsetlinewidth{0.803000pt}%
\definecolor{currentstroke}{rgb}{0.000000,0.000000,0.000000}%
\pgfsetstrokecolor{currentstroke}%
\pgfsetdash{}{0pt}%
\pgfsys@defobject{currentmarker}{\pgfqpoint{0.000000in}{-0.048611in}}{\pgfqpoint{0.000000in}{0.000000in}}{%
\pgfpathmoveto{\pgfqpoint{0.000000in}{0.000000in}}%
\pgfpathlineto{\pgfqpoint{0.000000in}{-0.048611in}}%
\pgfusepath{stroke,fill}%
}%
\begin{pgfscope}%
\pgfsys@transformshift{2.865900in}{0.528000in}%
\pgfsys@useobject{currentmarker}{}%
\end{pgfscope}%
\end{pgfscope}%
\begin{pgfscope}%
\definecolor{textcolor}{rgb}{0.000000,0.000000,0.000000}%
\pgfsetstrokecolor{textcolor}%
\pgfsetfillcolor{textcolor}%
\pgftext[x=2.865900in,y=0.430778in,,top]{\color{textcolor}\sffamily\fontsize{8.000000}{9.600000}\selectfont 20}%
\end{pgfscope}%
\begin{pgfscope}%
\pgfsetbuttcap%
\pgfsetroundjoin%
\definecolor{currentfill}{rgb}{0.000000,0.000000,0.000000}%
\pgfsetfillcolor{currentfill}%
\pgfsetlinewidth{0.803000pt}%
\definecolor{currentstroke}{rgb}{0.000000,0.000000,0.000000}%
\pgfsetstrokecolor{currentstroke}%
\pgfsetdash{}{0pt}%
\pgfsys@defobject{currentmarker}{\pgfqpoint{0.000000in}{-0.048611in}}{\pgfqpoint{0.000000in}{0.000000in}}{%
\pgfpathmoveto{\pgfqpoint{0.000000in}{0.000000in}}%
\pgfpathlineto{\pgfqpoint{0.000000in}{-0.048611in}}%
\pgfusepath{stroke,fill}%
}%
\begin{pgfscope}%
\pgfsys@transformshift{3.786122in}{0.528000in}%
\pgfsys@useobject{currentmarker}{}%
\end{pgfscope}%
\end{pgfscope}%
\begin{pgfscope}%
\definecolor{textcolor}{rgb}{0.000000,0.000000,0.000000}%
\pgfsetstrokecolor{textcolor}%
\pgfsetfillcolor{textcolor}%
\pgftext[x=3.786122in,y=0.430778in,,top]{\color{textcolor}\sffamily\fontsize{8.000000}{9.600000}\selectfont 30}%
\end{pgfscope}%
\begin{pgfscope}%
\pgfsetbuttcap%
\pgfsetroundjoin%
\definecolor{currentfill}{rgb}{0.000000,0.000000,0.000000}%
\pgfsetfillcolor{currentfill}%
\pgfsetlinewidth{0.803000pt}%
\definecolor{currentstroke}{rgb}{0.000000,0.000000,0.000000}%
\pgfsetstrokecolor{currentstroke}%
\pgfsetdash{}{0pt}%
\pgfsys@defobject{currentmarker}{\pgfqpoint{0.000000in}{-0.048611in}}{\pgfqpoint{0.000000in}{0.000000in}}{%
\pgfpathmoveto{\pgfqpoint{0.000000in}{0.000000in}}%
\pgfpathlineto{\pgfqpoint{0.000000in}{-0.048611in}}%
\pgfusepath{stroke,fill}%
}%
\begin{pgfscope}%
\pgfsys@transformshift{4.706345in}{0.528000in}%
\pgfsys@useobject{currentmarker}{}%
\end{pgfscope}%
\end{pgfscope}%
\begin{pgfscope}%
\definecolor{textcolor}{rgb}{0.000000,0.000000,0.000000}%
\pgfsetstrokecolor{textcolor}%
\pgfsetfillcolor{textcolor}%
\pgftext[x=4.706345in,y=0.430778in,,top]{\color{textcolor}\sffamily\fontsize{8.000000}{9.600000}\selectfont 40}%
\end{pgfscope}%
\begin{pgfscope}%
\pgfsetbuttcap%
\pgfsetroundjoin%
\definecolor{currentfill}{rgb}{0.000000,0.000000,0.000000}%
\pgfsetfillcolor{currentfill}%
\pgfsetlinewidth{0.803000pt}%
\definecolor{currentstroke}{rgb}{0.000000,0.000000,0.000000}%
\pgfsetstrokecolor{currentstroke}%
\pgfsetdash{}{0pt}%
\pgfsys@defobject{currentmarker}{\pgfqpoint{0.000000in}{-0.048611in}}{\pgfqpoint{0.000000in}{0.000000in}}{%
\pgfpathmoveto{\pgfqpoint{0.000000in}{0.000000in}}%
\pgfpathlineto{\pgfqpoint{0.000000in}{-0.048611in}}%
\pgfusepath{stroke,fill}%
}%
\begin{pgfscope}%
\pgfsys@transformshift{5.626568in}{0.528000in}%
\pgfsys@useobject{currentmarker}{}%
\end{pgfscope}%
\end{pgfscope}%
\begin{pgfscope}%
\definecolor{textcolor}{rgb}{0.000000,0.000000,0.000000}%
\pgfsetstrokecolor{textcolor}%
\pgfsetfillcolor{textcolor}%
\pgftext[x=5.626568in,y=0.430778in,,top]{\color{textcolor}\sffamily\fontsize{8.000000}{9.600000}\selectfont 50}%
\end{pgfscope}%
\begin{pgfscope}%
\definecolor{textcolor}{rgb}{0.000000,0.000000,0.000000}%
\pgfsetstrokecolor{textcolor}%
\pgfsetfillcolor{textcolor}%
\pgftext[x=3.280000in,y=0.267692in,,top]{\color{textcolor}\sffamily\fontsize{8.000000}{9.600000}\selectfont Época}%
\end{pgfscope}%
\begin{pgfscope}%
\pgfsetbuttcap%
\pgfsetroundjoin%
\definecolor{currentfill}{rgb}{0.000000,0.000000,0.000000}%
\pgfsetfillcolor{currentfill}%
\pgfsetlinewidth{0.803000pt}%
\definecolor{currentstroke}{rgb}{0.000000,0.000000,0.000000}%
\pgfsetstrokecolor{currentstroke}%
\pgfsetdash{}{0pt}%
\pgfsys@defobject{currentmarker}{\pgfqpoint{-0.048611in}{0.000000in}}{\pgfqpoint{-0.000000in}{0.000000in}}{%
\pgfpathmoveto{\pgfqpoint{-0.000000in}{0.000000in}}%
\pgfpathlineto{\pgfqpoint{-0.048611in}{0.000000in}}%
\pgfusepath{stroke,fill}%
}%
\begin{pgfscope}%
\pgfsys@transformshift{0.800000in}{0.729508in}%
\pgfsys@useobject{currentmarker}{}%
\end{pgfscope}%
\end{pgfscope}%
\begin{pgfscope}%
\definecolor{textcolor}{rgb}{0.000000,0.000000,0.000000}%
\pgfsetstrokecolor{textcolor}%
\pgfsetfillcolor{textcolor}%
\pgftext[x=0.526074in, y=0.687299in, left, base]{\color{textcolor}\sffamily\fontsize{8.000000}{9.600000}\selectfont 0.7}%
\end{pgfscope}%
\begin{pgfscope}%
\pgfsetbuttcap%
\pgfsetroundjoin%
\definecolor{currentfill}{rgb}{0.000000,0.000000,0.000000}%
\pgfsetfillcolor{currentfill}%
\pgfsetlinewidth{0.803000pt}%
\definecolor{currentstroke}{rgb}{0.000000,0.000000,0.000000}%
\pgfsetstrokecolor{currentstroke}%
\pgfsetdash{}{0pt}%
\pgfsys@defobject{currentmarker}{\pgfqpoint{-0.048611in}{0.000000in}}{\pgfqpoint{-0.000000in}{0.000000in}}{%
\pgfpathmoveto{\pgfqpoint{-0.000000in}{0.000000in}}%
\pgfpathlineto{\pgfqpoint{-0.048611in}{0.000000in}}%
\pgfusepath{stroke,fill}%
}%
\begin{pgfscope}%
\pgfsys@transformshift{0.800000in}{1.159373in}%
\pgfsys@useobject{currentmarker}{}%
\end{pgfscope}%
\end{pgfscope}%
\begin{pgfscope}%
\definecolor{textcolor}{rgb}{0.000000,0.000000,0.000000}%
\pgfsetstrokecolor{textcolor}%
\pgfsetfillcolor{textcolor}%
\pgftext[x=0.526074in, y=1.117164in, left, base]{\color{textcolor}\sffamily\fontsize{8.000000}{9.600000}\selectfont 0.8}%
\end{pgfscope}%
\begin{pgfscope}%
\pgfsetbuttcap%
\pgfsetroundjoin%
\definecolor{currentfill}{rgb}{0.000000,0.000000,0.000000}%
\pgfsetfillcolor{currentfill}%
\pgfsetlinewidth{0.803000pt}%
\definecolor{currentstroke}{rgb}{0.000000,0.000000,0.000000}%
\pgfsetstrokecolor{currentstroke}%
\pgfsetdash{}{0pt}%
\pgfsys@defobject{currentmarker}{\pgfqpoint{-0.048611in}{0.000000in}}{\pgfqpoint{-0.000000in}{0.000000in}}{%
\pgfpathmoveto{\pgfqpoint{-0.000000in}{0.000000in}}%
\pgfpathlineto{\pgfqpoint{-0.048611in}{0.000000in}}%
\pgfusepath{stroke,fill}%
}%
\begin{pgfscope}%
\pgfsys@transformshift{0.800000in}{1.589237in}%
\pgfsys@useobject{currentmarker}{}%
\end{pgfscope}%
\end{pgfscope}%
\begin{pgfscope}%
\definecolor{textcolor}{rgb}{0.000000,0.000000,0.000000}%
\pgfsetstrokecolor{textcolor}%
\pgfsetfillcolor{textcolor}%
\pgftext[x=0.526074in, y=1.547028in, left, base]{\color{textcolor}\sffamily\fontsize{8.000000}{9.600000}\selectfont 0.9}%
\end{pgfscope}%
\begin{pgfscope}%
\pgfsetbuttcap%
\pgfsetroundjoin%
\definecolor{currentfill}{rgb}{0.000000,0.000000,0.000000}%
\pgfsetfillcolor{currentfill}%
\pgfsetlinewidth{0.803000pt}%
\definecolor{currentstroke}{rgb}{0.000000,0.000000,0.000000}%
\pgfsetstrokecolor{currentstroke}%
\pgfsetdash{}{0pt}%
\pgfsys@defobject{currentmarker}{\pgfqpoint{-0.048611in}{0.000000in}}{\pgfqpoint{-0.000000in}{0.000000in}}{%
\pgfpathmoveto{\pgfqpoint{-0.000000in}{0.000000in}}%
\pgfpathlineto{\pgfqpoint{-0.048611in}{0.000000in}}%
\pgfusepath{stroke,fill}%
}%
\begin{pgfscope}%
\pgfsys@transformshift{0.800000in}{2.019102in}%
\pgfsys@useobject{currentmarker}{}%
\end{pgfscope}%
\end{pgfscope}%
\begin{pgfscope}%
\definecolor{textcolor}{rgb}{0.000000,0.000000,0.000000}%
\pgfsetstrokecolor{textcolor}%
\pgfsetfillcolor{textcolor}%
\pgftext[x=0.526074in, y=1.976893in, left, base]{\color{textcolor}\sffamily\fontsize{8.000000}{9.600000}\selectfont 1.0}%
\end{pgfscope}%
\begin{pgfscope}%
\pgfsetbuttcap%
\pgfsetroundjoin%
\definecolor{currentfill}{rgb}{0.000000,0.000000,0.000000}%
\pgfsetfillcolor{currentfill}%
\pgfsetlinewidth{0.803000pt}%
\definecolor{currentstroke}{rgb}{0.000000,0.000000,0.000000}%
\pgfsetstrokecolor{currentstroke}%
\pgfsetdash{}{0pt}%
\pgfsys@defobject{currentmarker}{\pgfqpoint{-0.048611in}{0.000000in}}{\pgfqpoint{-0.000000in}{0.000000in}}{%
\pgfpathmoveto{\pgfqpoint{-0.000000in}{0.000000in}}%
\pgfpathlineto{\pgfqpoint{-0.048611in}{0.000000in}}%
\pgfusepath{stroke,fill}%
}%
\begin{pgfscope}%
\pgfsys@transformshift{0.800000in}{2.448967in}%
\pgfsys@useobject{currentmarker}{}%
\end{pgfscope}%
\end{pgfscope}%
\begin{pgfscope}%
\definecolor{textcolor}{rgb}{0.000000,0.000000,0.000000}%
\pgfsetstrokecolor{textcolor}%
\pgfsetfillcolor{textcolor}%
\pgftext[x=0.526074in, y=2.406757in, left, base]{\color{textcolor}\sffamily\fontsize{8.000000}{9.600000}\selectfont 1.1}%
\end{pgfscope}%
\begin{pgfscope}%
\pgfsetbuttcap%
\pgfsetroundjoin%
\definecolor{currentfill}{rgb}{0.000000,0.000000,0.000000}%
\pgfsetfillcolor{currentfill}%
\pgfsetlinewidth{0.803000pt}%
\definecolor{currentstroke}{rgb}{0.000000,0.000000,0.000000}%
\pgfsetstrokecolor{currentstroke}%
\pgfsetdash{}{0pt}%
\pgfsys@defobject{currentmarker}{\pgfqpoint{-0.048611in}{0.000000in}}{\pgfqpoint{-0.000000in}{0.000000in}}{%
\pgfpathmoveto{\pgfqpoint{-0.000000in}{0.000000in}}%
\pgfpathlineto{\pgfqpoint{-0.048611in}{0.000000in}}%
\pgfusepath{stroke,fill}%
}%
\begin{pgfscope}%
\pgfsys@transformshift{0.800000in}{2.878831in}%
\pgfsys@useobject{currentmarker}{}%
\end{pgfscope}%
\end{pgfscope}%
\begin{pgfscope}%
\definecolor{textcolor}{rgb}{0.000000,0.000000,0.000000}%
\pgfsetstrokecolor{textcolor}%
\pgfsetfillcolor{textcolor}%
\pgftext[x=0.526074in, y=2.836622in, left, base]{\color{textcolor}\sffamily\fontsize{8.000000}{9.600000}\selectfont 1.2}%
\end{pgfscope}%
\begin{pgfscope}%
\pgfsetbuttcap%
\pgfsetroundjoin%
\definecolor{currentfill}{rgb}{0.000000,0.000000,0.000000}%
\pgfsetfillcolor{currentfill}%
\pgfsetlinewidth{0.803000pt}%
\definecolor{currentstroke}{rgb}{0.000000,0.000000,0.000000}%
\pgfsetstrokecolor{currentstroke}%
\pgfsetdash{}{0pt}%
\pgfsys@defobject{currentmarker}{\pgfqpoint{-0.048611in}{0.000000in}}{\pgfqpoint{-0.000000in}{0.000000in}}{%
\pgfpathmoveto{\pgfqpoint{-0.000000in}{0.000000in}}%
\pgfpathlineto{\pgfqpoint{-0.048611in}{0.000000in}}%
\pgfusepath{stroke,fill}%
}%
\begin{pgfscope}%
\pgfsys@transformshift{0.800000in}{3.308696in}%
\pgfsys@useobject{currentmarker}{}%
\end{pgfscope}%
\end{pgfscope}%
\begin{pgfscope}%
\definecolor{textcolor}{rgb}{0.000000,0.000000,0.000000}%
\pgfsetstrokecolor{textcolor}%
\pgfsetfillcolor{textcolor}%
\pgftext[x=0.526074in, y=3.266486in, left, base]{\color{textcolor}\sffamily\fontsize{8.000000}{9.600000}\selectfont 1.3}%
\end{pgfscope}%
\begin{pgfscope}%
\pgfsetbuttcap%
\pgfsetroundjoin%
\definecolor{currentfill}{rgb}{0.000000,0.000000,0.000000}%
\pgfsetfillcolor{currentfill}%
\pgfsetlinewidth{0.803000pt}%
\definecolor{currentstroke}{rgb}{0.000000,0.000000,0.000000}%
\pgfsetstrokecolor{currentstroke}%
\pgfsetdash{}{0pt}%
\pgfsys@defobject{currentmarker}{\pgfqpoint{-0.048611in}{0.000000in}}{\pgfqpoint{-0.000000in}{0.000000in}}{%
\pgfpathmoveto{\pgfqpoint{-0.000000in}{0.000000in}}%
\pgfpathlineto{\pgfqpoint{-0.048611in}{0.000000in}}%
\pgfusepath{stroke,fill}%
}%
\begin{pgfscope}%
\pgfsys@transformshift{0.800000in}{3.738560in}%
\pgfsys@useobject{currentmarker}{}%
\end{pgfscope}%
\end{pgfscope}%
\begin{pgfscope}%
\definecolor{textcolor}{rgb}{0.000000,0.000000,0.000000}%
\pgfsetstrokecolor{textcolor}%
\pgfsetfillcolor{textcolor}%
\pgftext[x=0.526074in, y=3.696351in, left, base]{\color{textcolor}\sffamily\fontsize{8.000000}{9.600000}\selectfont 1.4}%
\end{pgfscope}%
\begin{pgfscope}%
\pgfsetbuttcap%
\pgfsetroundjoin%
\definecolor{currentfill}{rgb}{0.000000,0.000000,0.000000}%
\pgfsetfillcolor{currentfill}%
\pgfsetlinewidth{0.803000pt}%
\definecolor{currentstroke}{rgb}{0.000000,0.000000,0.000000}%
\pgfsetstrokecolor{currentstroke}%
\pgfsetdash{}{0pt}%
\pgfsys@defobject{currentmarker}{\pgfqpoint{-0.048611in}{0.000000in}}{\pgfqpoint{-0.000000in}{0.000000in}}{%
\pgfpathmoveto{\pgfqpoint{-0.000000in}{0.000000in}}%
\pgfpathlineto{\pgfqpoint{-0.048611in}{0.000000in}}%
\pgfusepath{stroke,fill}%
}%
\begin{pgfscope}%
\pgfsys@transformshift{0.800000in}{4.168425in}%
\pgfsys@useobject{currentmarker}{}%
\end{pgfscope}%
\end{pgfscope}%
\begin{pgfscope}%
\definecolor{textcolor}{rgb}{0.000000,0.000000,0.000000}%
\pgfsetstrokecolor{textcolor}%
\pgfsetfillcolor{textcolor}%
\pgftext[x=0.526074in, y=4.126216in, left, base]{\color{textcolor}\sffamily\fontsize{8.000000}{9.600000}\selectfont 1.5}%
\end{pgfscope}%
\begin{pgfscope}%
\definecolor{textcolor}{rgb}{0.000000,0.000000,0.000000}%
\pgfsetstrokecolor{textcolor}%
\pgfsetfillcolor{textcolor}%
\pgftext[x=0.470519in,y=2.376000in,,bottom,rotate=90.000000]{\color{textcolor}\sffamily\fontsize{8.000000}{9.600000}\selectfont Perda}%
\end{pgfscope}%
\begin{pgfscope}%
\pgfpathrectangle{\pgfqpoint{0.800000in}{0.528000in}}{\pgfqpoint{4.960000in}{3.696000in}}%
\pgfusepath{clip}%
\pgfsetrectcap%
\pgfsetroundjoin%
\pgfsetlinewidth{1.505625pt}%
\definecolor{currentstroke}{rgb}{0.121569,0.466667,0.705882}%
\pgfsetstrokecolor{currentstroke}%
\pgfsetdash{}{0pt}%
\pgfpathmoveto{\pgfqpoint{1.025455in}{4.056000in}}%
\pgfpathlineto{\pgfqpoint{1.117477in}{1.047562in}}%
\pgfpathlineto{\pgfqpoint{1.209499in}{0.958176in}}%
\pgfpathlineto{\pgfqpoint{1.301521in}{0.921161in}}%
\pgfpathlineto{\pgfqpoint{1.393544in}{0.901513in}}%
\pgfpathlineto{\pgfqpoint{1.485566in}{0.911603in}}%
\pgfpathlineto{\pgfqpoint{1.577588in}{0.901772in}}%
\pgfpathlineto{\pgfqpoint{1.669610in}{0.928597in}}%
\pgfpathlineto{\pgfqpoint{1.761633in}{0.884825in}}%
\pgfpathlineto{\pgfqpoint{1.853655in}{0.929135in}}%
\pgfpathlineto{\pgfqpoint{1.945677in}{0.942752in}}%
\pgfpathlineto{\pgfqpoint{2.037699in}{0.884745in}}%
\pgfpathlineto{\pgfqpoint{2.129722in}{0.891837in}}%
\pgfpathlineto{\pgfqpoint{2.221744in}{0.871634in}}%
\pgfpathlineto{\pgfqpoint{2.313766in}{0.892207in}}%
\pgfpathlineto{\pgfqpoint{2.405788in}{0.865970in}}%
\pgfpathlineto{\pgfqpoint{2.497811in}{0.847707in}}%
\pgfpathlineto{\pgfqpoint{2.589833in}{0.864027in}}%
\pgfpathlineto{\pgfqpoint{2.681855in}{0.878973in}}%
\pgfpathlineto{\pgfqpoint{2.773878in}{0.869955in}}%
\pgfpathlineto{\pgfqpoint{2.865900in}{0.874514in}}%
\pgfpathlineto{\pgfqpoint{2.957922in}{0.882812in}}%
\pgfpathlineto{\pgfqpoint{3.049944in}{0.870360in}}%
\pgfpathlineto{\pgfqpoint{3.141967in}{0.851499in}}%
\pgfpathlineto{\pgfqpoint{3.233989in}{0.839868in}}%
\pgfpathlineto{\pgfqpoint{3.326011in}{0.862641in}}%
\pgfpathlineto{\pgfqpoint{3.418033in}{0.869071in}}%
\pgfpathlineto{\pgfqpoint{3.510056in}{0.826835in}}%
\pgfpathlineto{\pgfqpoint{3.602078in}{0.880820in}}%
\pgfpathlineto{\pgfqpoint{3.694100in}{0.852036in}}%
\pgfpathlineto{\pgfqpoint{3.786122in}{0.841208in}}%
\pgfpathlineto{\pgfqpoint{3.878145in}{0.896410in}}%
\pgfpathlineto{\pgfqpoint{3.970167in}{0.872094in}}%
\pgfpathlineto{\pgfqpoint{4.062189in}{0.846458in}}%
\pgfpathlineto{\pgfqpoint{4.154212in}{0.869610in}}%
\pgfpathlineto{\pgfqpoint{4.246234in}{0.850525in}}%
\pgfpathlineto{\pgfqpoint{4.338256in}{0.883259in}}%
\pgfpathlineto{\pgfqpoint{4.430278in}{0.843901in}}%
\pgfpathlineto{\pgfqpoint{4.522301in}{0.865635in}}%
\pgfpathlineto{\pgfqpoint{4.614323in}{0.827711in}}%
\pgfpathlineto{\pgfqpoint{4.706345in}{0.880810in}}%
\pgfpathlineto{\pgfqpoint{4.798367in}{0.829014in}}%
\pgfpathlineto{\pgfqpoint{4.890390in}{0.856239in}}%
\pgfpathlineto{\pgfqpoint{4.982412in}{0.808400in}}%
\pgfpathlineto{\pgfqpoint{5.074434in}{0.838446in}}%
\pgfpathlineto{\pgfqpoint{5.166456in}{0.833870in}}%
\pgfpathlineto{\pgfqpoint{5.258479in}{0.820048in}}%
\pgfpathlineto{\pgfqpoint{5.350501in}{0.839028in}}%
\pgfpathlineto{\pgfqpoint{5.442523in}{0.853466in}}%
\pgfpathlineto{\pgfqpoint{5.534545in}{0.779701in}}%
\pgfusepath{stroke}%
\end{pgfscope}%
\begin{pgfscope}%
\pgfpathrectangle{\pgfqpoint{0.800000in}{0.528000in}}{\pgfqpoint{4.960000in}{3.696000in}}%
\pgfusepath{clip}%
\pgfsetrectcap%
\pgfsetroundjoin%
\pgfsetlinewidth{1.505625pt}%
\definecolor{currentstroke}{rgb}{1.000000,0.498039,0.054902}%
\pgfsetstrokecolor{currentstroke}%
\pgfsetdash{}{0pt}%
\pgfpathmoveto{\pgfqpoint{1.025455in}{0.696000in}}%
\pgfpathlineto{\pgfqpoint{1.117477in}{1.155549in}}%
\pgfpathlineto{\pgfqpoint{1.209499in}{1.146423in}}%
\pgfpathlineto{\pgfqpoint{1.301521in}{0.770226in}}%
\pgfpathlineto{\pgfqpoint{1.393544in}{0.809789in}}%
\pgfpathlineto{\pgfqpoint{1.485566in}{0.710472in}}%
\pgfpathlineto{\pgfqpoint{1.577588in}{0.705492in}}%
\pgfpathlineto{\pgfqpoint{1.669610in}{1.382876in}}%
\pgfpathlineto{\pgfqpoint{1.761633in}{0.722961in}}%
\pgfpathlineto{\pgfqpoint{1.853655in}{0.743019in}}%
\pgfpathlineto{\pgfqpoint{1.945677in}{0.704366in}}%
\pgfpathlineto{\pgfqpoint{2.037699in}{1.279403in}}%
\pgfpathlineto{\pgfqpoint{2.129722in}{0.806500in}}%
\pgfpathlineto{\pgfqpoint{2.221744in}{0.752591in}}%
\pgfpathlineto{\pgfqpoint{2.313766in}{0.836259in}}%
\pgfpathlineto{\pgfqpoint{2.405788in}{0.780948in}}%
\pgfpathlineto{\pgfqpoint{2.497811in}{0.704831in}}%
\pgfpathlineto{\pgfqpoint{2.589833in}{0.705915in}}%
\pgfpathlineto{\pgfqpoint{2.681855in}{0.782699in}}%
\pgfpathlineto{\pgfqpoint{2.773878in}{0.836830in}}%
\pgfpathlineto{\pgfqpoint{2.865900in}{0.743298in}}%
\pgfpathlineto{\pgfqpoint{2.957922in}{0.696482in}}%
\pgfpathlineto{\pgfqpoint{3.049944in}{0.733582in}}%
\pgfpathlineto{\pgfqpoint{3.141967in}{0.766553in}}%
\pgfpathlineto{\pgfqpoint{3.233989in}{0.706063in}}%
\pgfpathlineto{\pgfqpoint{3.326011in}{0.942875in}}%
\pgfpathlineto{\pgfqpoint{3.418033in}{0.789961in}}%
\pgfpathlineto{\pgfqpoint{3.510056in}{0.775734in}}%
\pgfpathlineto{\pgfqpoint{3.602078in}{0.747750in}}%
\pgfpathlineto{\pgfqpoint{3.694100in}{0.712763in}}%
\pgfpathlineto{\pgfqpoint{3.786122in}{0.765040in}}%
\pgfpathlineto{\pgfqpoint{3.878145in}{0.789834in}}%
\pgfpathlineto{\pgfqpoint{3.970167in}{0.706722in}}%
\pgfpathlineto{\pgfqpoint{4.062189in}{0.805792in}}%
\pgfpathlineto{\pgfqpoint{4.154212in}{0.783465in}}%
\pgfpathlineto{\pgfqpoint{4.246234in}{0.800686in}}%
\pgfpathlineto{\pgfqpoint{4.338256in}{0.753120in}}%
\pgfpathlineto{\pgfqpoint{4.430278in}{0.747390in}}%
\pgfpathlineto{\pgfqpoint{4.522301in}{0.801855in}}%
\pgfpathlineto{\pgfqpoint{4.614323in}{0.702091in}}%
\pgfpathlineto{\pgfqpoint{4.706345in}{0.797953in}}%
\pgfpathlineto{\pgfqpoint{4.798367in}{0.696112in}}%
\pgfpathlineto{\pgfqpoint{4.890390in}{1.072882in}}%
\pgfpathlineto{\pgfqpoint{4.982412in}{0.714028in}}%
\pgfpathlineto{\pgfqpoint{5.074434in}{0.888070in}}%
\pgfpathlineto{\pgfqpoint{5.166456in}{0.732967in}}%
\pgfpathlineto{\pgfqpoint{5.258479in}{0.812858in}}%
\pgfpathlineto{\pgfqpoint{5.350501in}{0.696132in}}%
\pgfpathlineto{\pgfqpoint{5.442523in}{0.698440in}}%
\pgfpathlineto{\pgfqpoint{5.534545in}{0.715510in}}%
\pgfusepath{stroke}%
\end{pgfscope}%
\begin{pgfscope}%
\pgfsetrectcap%
\pgfsetmiterjoin%
\pgfsetlinewidth{0.803000pt}%
\definecolor{currentstroke}{rgb}{0.000000,0.000000,0.000000}%
\pgfsetstrokecolor{currentstroke}%
\pgfsetdash{}{0pt}%
\pgfpathmoveto{\pgfqpoint{0.800000in}{0.528000in}}%
\pgfpathlineto{\pgfqpoint{0.800000in}{4.224000in}}%
\pgfusepath{stroke}%
\end{pgfscope}%
\begin{pgfscope}%
\pgfsetrectcap%
\pgfsetmiterjoin%
\pgfsetlinewidth{0.803000pt}%
\definecolor{currentstroke}{rgb}{0.000000,0.000000,0.000000}%
\pgfsetstrokecolor{currentstroke}%
\pgfsetdash{}{0pt}%
\pgfpathmoveto{\pgfqpoint{5.760000in}{0.528000in}}%
\pgfpathlineto{\pgfqpoint{5.760000in}{4.224000in}}%
\pgfusepath{stroke}%
\end{pgfscope}%
\begin{pgfscope}%
\pgfsetrectcap%
\pgfsetmiterjoin%
\pgfsetlinewidth{0.803000pt}%
\definecolor{currentstroke}{rgb}{0.000000,0.000000,0.000000}%
\pgfsetstrokecolor{currentstroke}%
\pgfsetdash{}{0pt}%
\pgfpathmoveto{\pgfqpoint{0.800000in}{0.528000in}}%
\pgfpathlineto{\pgfqpoint{5.760000in}{0.528000in}}%
\pgfusepath{stroke}%
\end{pgfscope}%
\begin{pgfscope}%
\pgfsetrectcap%
\pgfsetmiterjoin%
\pgfsetlinewidth{0.803000pt}%
\definecolor{currentstroke}{rgb}{0.000000,0.000000,0.000000}%
\pgfsetstrokecolor{currentstroke}%
\pgfsetdash{}{0pt}%
\pgfpathmoveto{\pgfqpoint{0.800000in}{4.224000in}}%
\pgfpathlineto{\pgfqpoint{5.760000in}{4.224000in}}%
\pgfusepath{stroke}%
\end{pgfscope}%
\begin{pgfscope}%
\definecolor{textcolor}{rgb}{0.000000,0.000000,0.000000}%
\pgfsetstrokecolor{textcolor}%
\pgfsetfillcolor{textcolor}%
\pgftext[x=3.280000in,y=4.307333in,,base]{\color{textcolor}\sffamily\fontsize{9.600000}{11.520000}\selectfont Função de Perda em Treino (compI)}%
\end{pgfscope}%
\begin{pgfscope}%
\pgfsetbuttcap%
\pgfsetmiterjoin%
\definecolor{currentfill}{rgb}{1.000000,1.000000,1.000000}%
\pgfsetfillcolor{currentfill}%
\pgfsetfillopacity{0.800000}%
\pgfsetlinewidth{1.003750pt}%
\definecolor{currentstroke}{rgb}{0.800000,0.800000,0.800000}%
\pgfsetstrokecolor{currentstroke}%
\pgfsetstrokeopacity{0.800000}%
\pgfsetdash{}{0pt}%
\pgfpathmoveto{\pgfqpoint{4.635586in}{3.808723in}}%
\pgfpathlineto{\pgfqpoint{5.682222in}{3.808723in}}%
\pgfpathquadraticcurveto{\pgfqpoint{5.704444in}{3.808723in}}{\pgfqpoint{5.704444in}{3.830945in}}%
\pgfpathlineto{\pgfqpoint{5.704444in}{4.146222in}}%
\pgfpathquadraticcurveto{\pgfqpoint{5.704444in}{4.168444in}}{\pgfqpoint{5.682222in}{4.168444in}}%
\pgfpathlineto{\pgfqpoint{4.635586in}{4.168444in}}%
\pgfpathquadraticcurveto{\pgfqpoint{4.613364in}{4.168444in}}{\pgfqpoint{4.613364in}{4.146222in}}%
\pgfpathlineto{\pgfqpoint{4.613364in}{3.830945in}}%
\pgfpathquadraticcurveto{\pgfqpoint{4.613364in}{3.808723in}}{\pgfqpoint{4.635586in}{3.808723in}}%
\pgfpathlineto{\pgfqpoint{4.635586in}{3.808723in}}%
\pgfpathclose%
\pgfusepath{stroke,fill}%
\end{pgfscope}%
\begin{pgfscope}%
\pgfsetrectcap%
\pgfsetroundjoin%
\pgfsetlinewidth{1.505625pt}%
\definecolor{currentstroke}{rgb}{0.121569,0.466667,0.705882}%
\pgfsetstrokecolor{currentstroke}%
\pgfsetdash{}{0pt}%
\pgfpathmoveto{\pgfqpoint{4.657808in}{4.078470in}}%
\pgfpathlineto{\pgfqpoint{4.768919in}{4.078470in}}%
\pgfpathlineto{\pgfqpoint{4.880030in}{4.078470in}}%
\pgfusepath{stroke}%
\end{pgfscope}%
\begin{pgfscope}%
\definecolor{textcolor}{rgb}{0.000000,0.000000,0.000000}%
\pgfsetstrokecolor{textcolor}%
\pgfsetfillcolor{textcolor}%
\pgftext[x=4.968919in,y=4.039582in,left,base]{\color{textcolor}\sffamily\fontsize{8.000000}{9.600000}\selectfont Treinamento}%
\end{pgfscope}%
\begin{pgfscope}%
\pgfsetrectcap%
\pgfsetroundjoin%
\pgfsetlinewidth{1.505625pt}%
\definecolor{currentstroke}{rgb}{1.000000,0.498039,0.054902}%
\pgfsetstrokecolor{currentstroke}%
\pgfsetdash{}{0pt}%
\pgfpathmoveto{\pgfqpoint{4.657808in}{3.915168in}}%
\pgfpathlineto{\pgfqpoint{4.768919in}{3.915168in}}%
\pgfpathlineto{\pgfqpoint{4.880030in}{3.915168in}}%
\pgfusepath{stroke}%
\end{pgfscope}%
\begin{pgfscope}%
\definecolor{textcolor}{rgb}{0.000000,0.000000,0.000000}%
\pgfsetstrokecolor{textcolor}%
\pgfsetfillcolor{textcolor}%
\pgftext[x=4.968919in,y=3.876279in,left,base]{\color{textcolor}\sffamily\fontsize{8.000000}{9.600000}\selectfont Validação}%
\end{pgfscope}%
\end{pgfpicture}%
\makeatother%
\endgroup%
}
    \end{minipage}
    \begin{minipage}{0.45\textwidth}
        \resizebox{\textwidth}{!}{%% Creator: Matplotlib, PGF backend
%%
%% To include the figure in your LaTeX document, write
%%   \input{<filename>.pgf}
%%
%% Make sure the required packages are loaded in your preamble
%%   \usepackage{pgf}
%%
%% Also ensure that all the required font packages are loaded; for instance,
%% the lmodern package is sometimes necessary when using math font.
%%   \usepackage{lmodern}
%%
%% Figures using additional raster images can only be included by \input if
%% they are in the same directory as the main LaTeX file. For loading figures
%% from other directories you can use the `import` package
%%   \usepackage{import}
%%
%% and then include the figures with
%%   \import{<path to file>}{<filename>.pgf}
%%
%% Matplotlib used the following preamble
%%
%%   \usepackage{fontspec}
%%   \setmainfont{DejaVuSerif.ttf}[Path=\detokenize{/home/jose/anaconda3/envs/tf/lib/python3.9/site-packages/matplotlib/mpl-data/fonts/ttf/}]
%%   \setsansfont{DejaVuSans.ttf}[Path=\detokenize{/home/jose/anaconda3/envs/tf/lib/python3.9/site-packages/matplotlib/mpl-data/fonts/ttf/}]
%%   \setmonofont{DejaVuSansMono.ttf}[Path=\detokenize{/home/jose/anaconda3/envs/tf/lib/python3.9/site-packages/matplotlib/mpl-data/fonts/ttf/}]
%%   \makeatletter\@ifpackageloaded{underscore}{}{\usepackage[strings]{underscore}}\makeatother
%%
\begingroup%
\makeatletter%
\begin{pgfpicture}%
\pgfpathrectangle{\pgfpointorigin}{\pgfqpoint{6.400000in}{4.800000in}}%
\pgfusepath{use as bounding box, clip}%
\begin{pgfscope}%
\pgfsetbuttcap%
\pgfsetmiterjoin%
\definecolor{currentfill}{rgb}{1.000000,1.000000,1.000000}%
\pgfsetfillcolor{currentfill}%
\pgfsetlinewidth{0.000000pt}%
\definecolor{currentstroke}{rgb}{1.000000,1.000000,1.000000}%
\pgfsetstrokecolor{currentstroke}%
\pgfsetdash{}{0pt}%
\pgfpathmoveto{\pgfqpoint{0.000000in}{0.000000in}}%
\pgfpathlineto{\pgfqpoint{6.400000in}{0.000000in}}%
\pgfpathlineto{\pgfqpoint{6.400000in}{4.800000in}}%
\pgfpathlineto{\pgfqpoint{0.000000in}{4.800000in}}%
\pgfpathlineto{\pgfqpoint{0.000000in}{0.000000in}}%
\pgfpathclose%
\pgfusepath{fill}%
\end{pgfscope}%
\begin{pgfscope}%
\pgfsetbuttcap%
\pgfsetmiterjoin%
\definecolor{currentfill}{rgb}{1.000000,1.000000,1.000000}%
\pgfsetfillcolor{currentfill}%
\pgfsetlinewidth{0.000000pt}%
\definecolor{currentstroke}{rgb}{0.000000,0.000000,0.000000}%
\pgfsetstrokecolor{currentstroke}%
\pgfsetstrokeopacity{0.000000}%
\pgfsetdash{}{0pt}%
\pgfpathmoveto{\pgfqpoint{0.800000in}{0.528000in}}%
\pgfpathlineto{\pgfqpoint{5.760000in}{0.528000in}}%
\pgfpathlineto{\pgfqpoint{5.760000in}{4.224000in}}%
\pgfpathlineto{\pgfqpoint{0.800000in}{4.224000in}}%
\pgfpathlineto{\pgfqpoint{0.800000in}{0.528000in}}%
\pgfpathclose%
\pgfusepath{fill}%
\end{pgfscope}%
\begin{pgfscope}%
\pgfsetbuttcap%
\pgfsetroundjoin%
\definecolor{currentfill}{rgb}{0.000000,0.000000,0.000000}%
\pgfsetfillcolor{currentfill}%
\pgfsetlinewidth{0.803000pt}%
\definecolor{currentstroke}{rgb}{0.000000,0.000000,0.000000}%
\pgfsetstrokecolor{currentstroke}%
\pgfsetdash{}{0pt}%
\pgfsys@defobject{currentmarker}{\pgfqpoint{0.000000in}{-0.048611in}}{\pgfqpoint{0.000000in}{0.000000in}}{%
\pgfpathmoveto{\pgfqpoint{0.000000in}{0.000000in}}%
\pgfpathlineto{\pgfqpoint{0.000000in}{-0.048611in}}%
\pgfusepath{stroke,fill}%
}%
\begin{pgfscope}%
\pgfsys@transformshift{1.025455in}{0.528000in}%
\pgfsys@useobject{currentmarker}{}%
\end{pgfscope}%
\end{pgfscope}%
\begin{pgfscope}%
\definecolor{textcolor}{rgb}{0.000000,0.000000,0.000000}%
\pgfsetstrokecolor{textcolor}%
\pgfsetfillcolor{textcolor}%
\pgftext[x=1.025455in,y=0.430778in,,top]{\color{textcolor}\sffamily\fontsize{8.000000}{9.600000}\selectfont 0}%
\end{pgfscope}%
\begin{pgfscope}%
\pgfsetbuttcap%
\pgfsetroundjoin%
\definecolor{currentfill}{rgb}{0.000000,0.000000,0.000000}%
\pgfsetfillcolor{currentfill}%
\pgfsetlinewidth{0.803000pt}%
\definecolor{currentstroke}{rgb}{0.000000,0.000000,0.000000}%
\pgfsetstrokecolor{currentstroke}%
\pgfsetdash{}{0pt}%
\pgfsys@defobject{currentmarker}{\pgfqpoint{0.000000in}{-0.048611in}}{\pgfqpoint{0.000000in}{0.000000in}}{%
\pgfpathmoveto{\pgfqpoint{0.000000in}{0.000000in}}%
\pgfpathlineto{\pgfqpoint{0.000000in}{-0.048611in}}%
\pgfusepath{stroke,fill}%
}%
\begin{pgfscope}%
\pgfsys@transformshift{1.945677in}{0.528000in}%
\pgfsys@useobject{currentmarker}{}%
\end{pgfscope}%
\end{pgfscope}%
\begin{pgfscope}%
\definecolor{textcolor}{rgb}{0.000000,0.000000,0.000000}%
\pgfsetstrokecolor{textcolor}%
\pgfsetfillcolor{textcolor}%
\pgftext[x=1.945677in,y=0.430778in,,top]{\color{textcolor}\sffamily\fontsize{8.000000}{9.600000}\selectfont 10}%
\end{pgfscope}%
\begin{pgfscope}%
\pgfsetbuttcap%
\pgfsetroundjoin%
\definecolor{currentfill}{rgb}{0.000000,0.000000,0.000000}%
\pgfsetfillcolor{currentfill}%
\pgfsetlinewidth{0.803000pt}%
\definecolor{currentstroke}{rgb}{0.000000,0.000000,0.000000}%
\pgfsetstrokecolor{currentstroke}%
\pgfsetdash{}{0pt}%
\pgfsys@defobject{currentmarker}{\pgfqpoint{0.000000in}{-0.048611in}}{\pgfqpoint{0.000000in}{0.000000in}}{%
\pgfpathmoveto{\pgfqpoint{0.000000in}{0.000000in}}%
\pgfpathlineto{\pgfqpoint{0.000000in}{-0.048611in}}%
\pgfusepath{stroke,fill}%
}%
\begin{pgfscope}%
\pgfsys@transformshift{2.865900in}{0.528000in}%
\pgfsys@useobject{currentmarker}{}%
\end{pgfscope}%
\end{pgfscope}%
\begin{pgfscope}%
\definecolor{textcolor}{rgb}{0.000000,0.000000,0.000000}%
\pgfsetstrokecolor{textcolor}%
\pgfsetfillcolor{textcolor}%
\pgftext[x=2.865900in,y=0.430778in,,top]{\color{textcolor}\sffamily\fontsize{8.000000}{9.600000}\selectfont 20}%
\end{pgfscope}%
\begin{pgfscope}%
\pgfsetbuttcap%
\pgfsetroundjoin%
\definecolor{currentfill}{rgb}{0.000000,0.000000,0.000000}%
\pgfsetfillcolor{currentfill}%
\pgfsetlinewidth{0.803000pt}%
\definecolor{currentstroke}{rgb}{0.000000,0.000000,0.000000}%
\pgfsetstrokecolor{currentstroke}%
\pgfsetdash{}{0pt}%
\pgfsys@defobject{currentmarker}{\pgfqpoint{0.000000in}{-0.048611in}}{\pgfqpoint{0.000000in}{0.000000in}}{%
\pgfpathmoveto{\pgfqpoint{0.000000in}{0.000000in}}%
\pgfpathlineto{\pgfqpoint{0.000000in}{-0.048611in}}%
\pgfusepath{stroke,fill}%
}%
\begin{pgfscope}%
\pgfsys@transformshift{3.786122in}{0.528000in}%
\pgfsys@useobject{currentmarker}{}%
\end{pgfscope}%
\end{pgfscope}%
\begin{pgfscope}%
\definecolor{textcolor}{rgb}{0.000000,0.000000,0.000000}%
\pgfsetstrokecolor{textcolor}%
\pgfsetfillcolor{textcolor}%
\pgftext[x=3.786122in,y=0.430778in,,top]{\color{textcolor}\sffamily\fontsize{8.000000}{9.600000}\selectfont 30}%
\end{pgfscope}%
\begin{pgfscope}%
\pgfsetbuttcap%
\pgfsetroundjoin%
\definecolor{currentfill}{rgb}{0.000000,0.000000,0.000000}%
\pgfsetfillcolor{currentfill}%
\pgfsetlinewidth{0.803000pt}%
\definecolor{currentstroke}{rgb}{0.000000,0.000000,0.000000}%
\pgfsetstrokecolor{currentstroke}%
\pgfsetdash{}{0pt}%
\pgfsys@defobject{currentmarker}{\pgfqpoint{0.000000in}{-0.048611in}}{\pgfqpoint{0.000000in}{0.000000in}}{%
\pgfpathmoveto{\pgfqpoint{0.000000in}{0.000000in}}%
\pgfpathlineto{\pgfqpoint{0.000000in}{-0.048611in}}%
\pgfusepath{stroke,fill}%
}%
\begin{pgfscope}%
\pgfsys@transformshift{4.706345in}{0.528000in}%
\pgfsys@useobject{currentmarker}{}%
\end{pgfscope}%
\end{pgfscope}%
\begin{pgfscope}%
\definecolor{textcolor}{rgb}{0.000000,0.000000,0.000000}%
\pgfsetstrokecolor{textcolor}%
\pgfsetfillcolor{textcolor}%
\pgftext[x=4.706345in,y=0.430778in,,top]{\color{textcolor}\sffamily\fontsize{8.000000}{9.600000}\selectfont 40}%
\end{pgfscope}%
\begin{pgfscope}%
\pgfsetbuttcap%
\pgfsetroundjoin%
\definecolor{currentfill}{rgb}{0.000000,0.000000,0.000000}%
\pgfsetfillcolor{currentfill}%
\pgfsetlinewidth{0.803000pt}%
\definecolor{currentstroke}{rgb}{0.000000,0.000000,0.000000}%
\pgfsetstrokecolor{currentstroke}%
\pgfsetdash{}{0pt}%
\pgfsys@defobject{currentmarker}{\pgfqpoint{0.000000in}{-0.048611in}}{\pgfqpoint{0.000000in}{0.000000in}}{%
\pgfpathmoveto{\pgfqpoint{0.000000in}{0.000000in}}%
\pgfpathlineto{\pgfqpoint{0.000000in}{-0.048611in}}%
\pgfusepath{stroke,fill}%
}%
\begin{pgfscope}%
\pgfsys@transformshift{5.626568in}{0.528000in}%
\pgfsys@useobject{currentmarker}{}%
\end{pgfscope}%
\end{pgfscope}%
\begin{pgfscope}%
\definecolor{textcolor}{rgb}{0.000000,0.000000,0.000000}%
\pgfsetstrokecolor{textcolor}%
\pgfsetfillcolor{textcolor}%
\pgftext[x=5.626568in,y=0.430778in,,top]{\color{textcolor}\sffamily\fontsize{8.000000}{9.600000}\selectfont 50}%
\end{pgfscope}%
\begin{pgfscope}%
\definecolor{textcolor}{rgb}{0.000000,0.000000,0.000000}%
\pgfsetstrokecolor{textcolor}%
\pgfsetfillcolor{textcolor}%
\pgftext[x=3.280000in,y=0.267692in,,top]{\color{textcolor}\sffamily\fontsize{8.000000}{9.600000}\selectfont Época}%
\end{pgfscope}%
\begin{pgfscope}%
\pgfsetbuttcap%
\pgfsetroundjoin%
\definecolor{currentfill}{rgb}{0.000000,0.000000,0.000000}%
\pgfsetfillcolor{currentfill}%
\pgfsetlinewidth{0.803000pt}%
\definecolor{currentstroke}{rgb}{0.000000,0.000000,0.000000}%
\pgfsetstrokecolor{currentstroke}%
\pgfsetdash{}{0pt}%
\pgfsys@defobject{currentmarker}{\pgfqpoint{-0.048611in}{0.000000in}}{\pgfqpoint{-0.000000in}{0.000000in}}{%
\pgfpathmoveto{\pgfqpoint{-0.000000in}{0.000000in}}%
\pgfpathlineto{\pgfqpoint{-0.048611in}{0.000000in}}%
\pgfusepath{stroke,fill}%
}%
\begin{pgfscope}%
\pgfsys@transformshift{0.800000in}{0.628177in}%
\pgfsys@useobject{currentmarker}{}%
\end{pgfscope}%
\end{pgfscope}%
\begin{pgfscope}%
\definecolor{textcolor}{rgb}{0.000000,0.000000,0.000000}%
\pgfsetstrokecolor{textcolor}%
\pgfsetfillcolor{textcolor}%
\pgftext[x=0.526074in, y=0.585968in, left, base]{\color{textcolor}\sffamily\fontsize{8.000000}{9.600000}\selectfont 0.0}%
\end{pgfscope}%
\begin{pgfscope}%
\pgfsetbuttcap%
\pgfsetroundjoin%
\definecolor{currentfill}{rgb}{0.000000,0.000000,0.000000}%
\pgfsetfillcolor{currentfill}%
\pgfsetlinewidth{0.803000pt}%
\definecolor{currentstroke}{rgb}{0.000000,0.000000,0.000000}%
\pgfsetstrokecolor{currentstroke}%
\pgfsetdash{}{0pt}%
\pgfsys@defobject{currentmarker}{\pgfqpoint{-0.048611in}{0.000000in}}{\pgfqpoint{-0.000000in}{0.000000in}}{%
\pgfpathmoveto{\pgfqpoint{-0.000000in}{0.000000in}}%
\pgfpathlineto{\pgfqpoint{-0.048611in}{0.000000in}}%
\pgfusepath{stroke,fill}%
}%
\begin{pgfscope}%
\pgfsys@transformshift{0.800000in}{1.420130in}%
\pgfsys@useobject{currentmarker}{}%
\end{pgfscope}%
\end{pgfscope}%
\begin{pgfscope}%
\definecolor{textcolor}{rgb}{0.000000,0.000000,0.000000}%
\pgfsetstrokecolor{textcolor}%
\pgfsetfillcolor{textcolor}%
\pgftext[x=0.526074in, y=1.377921in, left, base]{\color{textcolor}\sffamily\fontsize{8.000000}{9.600000}\selectfont 0.2}%
\end{pgfscope}%
\begin{pgfscope}%
\pgfsetbuttcap%
\pgfsetroundjoin%
\definecolor{currentfill}{rgb}{0.000000,0.000000,0.000000}%
\pgfsetfillcolor{currentfill}%
\pgfsetlinewidth{0.803000pt}%
\definecolor{currentstroke}{rgb}{0.000000,0.000000,0.000000}%
\pgfsetstrokecolor{currentstroke}%
\pgfsetdash{}{0pt}%
\pgfsys@defobject{currentmarker}{\pgfqpoint{-0.048611in}{0.000000in}}{\pgfqpoint{-0.000000in}{0.000000in}}{%
\pgfpathmoveto{\pgfqpoint{-0.000000in}{0.000000in}}%
\pgfpathlineto{\pgfqpoint{-0.048611in}{0.000000in}}%
\pgfusepath{stroke,fill}%
}%
\begin{pgfscope}%
\pgfsys@transformshift{0.800000in}{2.212084in}%
\pgfsys@useobject{currentmarker}{}%
\end{pgfscope}%
\end{pgfscope}%
\begin{pgfscope}%
\definecolor{textcolor}{rgb}{0.000000,0.000000,0.000000}%
\pgfsetstrokecolor{textcolor}%
\pgfsetfillcolor{textcolor}%
\pgftext[x=0.526074in, y=2.169875in, left, base]{\color{textcolor}\sffamily\fontsize{8.000000}{9.600000}\selectfont 0.4}%
\end{pgfscope}%
\begin{pgfscope}%
\pgfsetbuttcap%
\pgfsetroundjoin%
\definecolor{currentfill}{rgb}{0.000000,0.000000,0.000000}%
\pgfsetfillcolor{currentfill}%
\pgfsetlinewidth{0.803000pt}%
\definecolor{currentstroke}{rgb}{0.000000,0.000000,0.000000}%
\pgfsetstrokecolor{currentstroke}%
\pgfsetdash{}{0pt}%
\pgfsys@defobject{currentmarker}{\pgfqpoint{-0.048611in}{0.000000in}}{\pgfqpoint{-0.000000in}{0.000000in}}{%
\pgfpathmoveto{\pgfqpoint{-0.000000in}{0.000000in}}%
\pgfpathlineto{\pgfqpoint{-0.048611in}{0.000000in}}%
\pgfusepath{stroke,fill}%
}%
\begin{pgfscope}%
\pgfsys@transformshift{0.800000in}{3.004037in}%
\pgfsys@useobject{currentmarker}{}%
\end{pgfscope}%
\end{pgfscope}%
\begin{pgfscope}%
\definecolor{textcolor}{rgb}{0.000000,0.000000,0.000000}%
\pgfsetstrokecolor{textcolor}%
\pgfsetfillcolor{textcolor}%
\pgftext[x=0.526074in, y=2.961828in, left, base]{\color{textcolor}\sffamily\fontsize{8.000000}{9.600000}\selectfont 0.6}%
\end{pgfscope}%
\begin{pgfscope}%
\pgfsetbuttcap%
\pgfsetroundjoin%
\definecolor{currentfill}{rgb}{0.000000,0.000000,0.000000}%
\pgfsetfillcolor{currentfill}%
\pgfsetlinewidth{0.803000pt}%
\definecolor{currentstroke}{rgb}{0.000000,0.000000,0.000000}%
\pgfsetstrokecolor{currentstroke}%
\pgfsetdash{}{0pt}%
\pgfsys@defobject{currentmarker}{\pgfqpoint{-0.048611in}{0.000000in}}{\pgfqpoint{-0.000000in}{0.000000in}}{%
\pgfpathmoveto{\pgfqpoint{-0.000000in}{0.000000in}}%
\pgfpathlineto{\pgfqpoint{-0.048611in}{0.000000in}}%
\pgfusepath{stroke,fill}%
}%
\begin{pgfscope}%
\pgfsys@transformshift{0.800000in}{3.795991in}%
\pgfsys@useobject{currentmarker}{}%
\end{pgfscope}%
\end{pgfscope}%
\begin{pgfscope}%
\definecolor{textcolor}{rgb}{0.000000,0.000000,0.000000}%
\pgfsetstrokecolor{textcolor}%
\pgfsetfillcolor{textcolor}%
\pgftext[x=0.526074in, y=3.753782in, left, base]{\color{textcolor}\sffamily\fontsize{8.000000}{9.600000}\selectfont 0.8}%
\end{pgfscope}%
\begin{pgfscope}%
\definecolor{textcolor}{rgb}{0.000000,0.000000,0.000000}%
\pgfsetstrokecolor{textcolor}%
\pgfsetfillcolor{textcolor}%
\pgftext[x=0.470519in,y=2.376000in,,bottom,rotate=90.000000]{\color{textcolor}\sffamily\fontsize{8.000000}{9.600000}\selectfont Perda}%
\end{pgfscope}%
\begin{pgfscope}%
\pgfpathrectangle{\pgfqpoint{0.800000in}{0.528000in}}{\pgfqpoint{4.960000in}{3.696000in}}%
\pgfusepath{clip}%
\pgfsetrectcap%
\pgfsetroundjoin%
\pgfsetlinewidth{1.505625pt}%
\definecolor{currentstroke}{rgb}{0.121569,0.466667,0.705882}%
\pgfsetstrokecolor{currentstroke}%
\pgfsetdash{}{0pt}%
\pgfpathmoveto{\pgfqpoint{1.025455in}{4.056000in}}%
\pgfpathlineto{\pgfqpoint{1.117477in}{3.094417in}}%
\pgfpathlineto{\pgfqpoint{1.209499in}{2.306043in}}%
\pgfpathlineto{\pgfqpoint{1.301521in}{1.602443in}}%
\pgfpathlineto{\pgfqpoint{1.393544in}{1.283534in}}%
\pgfpathlineto{\pgfqpoint{1.485566in}{1.115119in}}%
\pgfpathlineto{\pgfqpoint{1.577588in}{1.099821in}}%
\pgfpathlineto{\pgfqpoint{1.669610in}{1.038596in}}%
\pgfpathlineto{\pgfqpoint{1.761633in}{0.964588in}}%
\pgfpathlineto{\pgfqpoint{1.853655in}{0.932837in}}%
\pgfpathlineto{\pgfqpoint{1.945677in}{0.879214in}}%
\pgfpathlineto{\pgfqpoint{2.037699in}{0.898390in}}%
\pgfpathlineto{\pgfqpoint{2.129722in}{0.857296in}}%
\pgfpathlineto{\pgfqpoint{2.221744in}{0.821486in}}%
\pgfpathlineto{\pgfqpoint{2.313766in}{0.849553in}}%
\pgfpathlineto{\pgfqpoint{2.405788in}{0.826443in}}%
\pgfpathlineto{\pgfqpoint{2.497811in}{0.837711in}}%
\pgfpathlineto{\pgfqpoint{2.589833in}{0.785097in}}%
\pgfpathlineto{\pgfqpoint{2.681855in}{0.775750in}}%
\pgfpathlineto{\pgfqpoint{2.773878in}{0.788018in}}%
\pgfpathlineto{\pgfqpoint{2.865900in}{0.812963in}}%
\pgfpathlineto{\pgfqpoint{2.957922in}{0.744498in}}%
\pgfpathlineto{\pgfqpoint{3.049944in}{0.751441in}}%
\pgfpathlineto{\pgfqpoint{3.141967in}{0.745512in}}%
\pgfpathlineto{\pgfqpoint{3.233989in}{0.769421in}}%
\pgfpathlineto{\pgfqpoint{3.326011in}{0.753018in}}%
\pgfpathlineto{\pgfqpoint{3.418033in}{0.752095in}}%
\pgfpathlineto{\pgfqpoint{3.510056in}{0.740570in}}%
\pgfpathlineto{\pgfqpoint{3.602078in}{0.726826in}}%
\pgfpathlineto{\pgfqpoint{3.694100in}{0.739859in}}%
\pgfpathlineto{\pgfqpoint{3.786122in}{0.730942in}}%
\pgfpathlineto{\pgfqpoint{3.878145in}{0.736648in}}%
\pgfpathlineto{\pgfqpoint{3.970167in}{0.731313in}}%
\pgfpathlineto{\pgfqpoint{4.062189in}{0.834137in}}%
\pgfpathlineto{\pgfqpoint{4.154212in}{0.759668in}}%
\pgfpathlineto{\pgfqpoint{4.246234in}{0.744358in}}%
\pgfpathlineto{\pgfqpoint{4.338256in}{0.738536in}}%
\pgfpathlineto{\pgfqpoint{4.430278in}{0.723167in}}%
\pgfpathlineto{\pgfqpoint{4.522301in}{0.701003in}}%
\pgfpathlineto{\pgfqpoint{4.614323in}{0.742375in}}%
\pgfpathlineto{\pgfqpoint{4.706345in}{0.701772in}}%
\pgfpathlineto{\pgfqpoint{4.798367in}{0.698294in}}%
\pgfpathlineto{\pgfqpoint{4.890390in}{0.754929in}}%
\pgfpathlineto{\pgfqpoint{4.982412in}{0.734544in}}%
\pgfpathlineto{\pgfqpoint{5.074434in}{0.724136in}}%
\pgfpathlineto{\pgfqpoint{5.166456in}{0.746763in}}%
\pgfpathlineto{\pgfqpoint{5.258479in}{0.706010in}}%
\pgfpathlineto{\pgfqpoint{5.350501in}{0.696000in}}%
\pgfpathlineto{\pgfqpoint{5.442523in}{0.708650in}}%
\pgfpathlineto{\pgfqpoint{5.534545in}{0.728986in}}%
\pgfusepath{stroke}%
\end{pgfscope}%
\begin{pgfscope}%
\pgfpathrectangle{\pgfqpoint{0.800000in}{0.528000in}}{\pgfqpoint{4.960000in}{3.696000in}}%
\pgfusepath{clip}%
\pgfsetrectcap%
\pgfsetroundjoin%
\pgfsetlinewidth{1.505625pt}%
\definecolor{currentstroke}{rgb}{1.000000,0.498039,0.054902}%
\pgfsetstrokecolor{currentstroke}%
\pgfsetdash{}{0pt}%
\pgfpathmoveto{\pgfqpoint{1.025455in}{3.601966in}}%
\pgfpathlineto{\pgfqpoint{1.117477in}{2.712546in}}%
\pgfpathlineto{\pgfqpoint{1.209499in}{2.880721in}}%
\pgfpathlineto{\pgfqpoint{1.301521in}{2.377964in}}%
\pgfpathlineto{\pgfqpoint{1.393544in}{2.318877in}}%
\pgfpathlineto{\pgfqpoint{1.485566in}{2.442337in}}%
\pgfpathlineto{\pgfqpoint{1.577588in}{2.678586in}}%
\pgfpathlineto{\pgfqpoint{1.669610in}{2.456544in}}%
\pgfpathlineto{\pgfqpoint{1.761633in}{2.672211in}}%
\pgfpathlineto{\pgfqpoint{1.853655in}{2.359566in}}%
\pgfpathlineto{\pgfqpoint{1.945677in}{2.450242in}}%
\pgfpathlineto{\pgfqpoint{2.037699in}{2.506990in}}%
\pgfpathlineto{\pgfqpoint{2.129722in}{2.738225in}}%
\pgfpathlineto{\pgfqpoint{2.221744in}{2.400829in}}%
\pgfpathlineto{\pgfqpoint{2.313766in}{2.447314in}}%
\pgfpathlineto{\pgfqpoint{2.405788in}{2.397418in}}%
\pgfpathlineto{\pgfqpoint{2.497811in}{2.465580in}}%
\pgfpathlineto{\pgfqpoint{2.589833in}{2.560911in}}%
\pgfpathlineto{\pgfqpoint{2.681855in}{2.549400in}}%
\pgfpathlineto{\pgfqpoint{2.773878in}{2.621769in}}%
\pgfpathlineto{\pgfqpoint{2.865900in}{2.522333in}}%
\pgfpathlineto{\pgfqpoint{2.957922in}{2.617200in}}%
\pgfpathlineto{\pgfqpoint{3.049944in}{2.416674in}}%
\pgfpathlineto{\pgfqpoint{3.141967in}{2.507392in}}%
\pgfpathlineto{\pgfqpoint{3.233989in}{2.674313in}}%
\pgfpathlineto{\pgfqpoint{3.326011in}{2.521533in}}%
\pgfpathlineto{\pgfqpoint{3.418033in}{2.580523in}}%
\pgfpathlineto{\pgfqpoint{3.510056in}{2.521112in}}%
\pgfpathlineto{\pgfqpoint{3.602078in}{2.511791in}}%
\pgfpathlineto{\pgfqpoint{3.694100in}{2.546454in}}%
\pgfpathlineto{\pgfqpoint{3.786122in}{2.515249in}}%
\pgfpathlineto{\pgfqpoint{3.878145in}{2.495255in}}%
\pgfpathlineto{\pgfqpoint{3.970167in}{2.523452in}}%
\pgfpathlineto{\pgfqpoint{4.062189in}{2.623491in}}%
\pgfpathlineto{\pgfqpoint{4.154212in}{2.507945in}}%
\pgfpathlineto{\pgfqpoint{4.246234in}{2.506013in}}%
\pgfpathlineto{\pgfqpoint{4.338256in}{2.530722in}}%
\pgfpathlineto{\pgfqpoint{4.430278in}{2.450921in}}%
\pgfpathlineto{\pgfqpoint{4.522301in}{2.563834in}}%
\pgfpathlineto{\pgfqpoint{4.614323in}{2.555927in}}%
\pgfpathlineto{\pgfqpoint{4.706345in}{2.472654in}}%
\pgfpathlineto{\pgfqpoint{4.798367in}{2.531187in}}%
\pgfpathlineto{\pgfqpoint{4.890390in}{2.499527in}}%
\pgfpathlineto{\pgfqpoint{4.982412in}{2.495939in}}%
\pgfpathlineto{\pgfqpoint{5.074434in}{2.592099in}}%
\pgfpathlineto{\pgfqpoint{5.166456in}{2.495521in}}%
\pgfpathlineto{\pgfqpoint{5.258479in}{2.563293in}}%
\pgfpathlineto{\pgfqpoint{5.350501in}{2.593157in}}%
\pgfpathlineto{\pgfqpoint{5.442523in}{2.559708in}}%
\pgfpathlineto{\pgfqpoint{5.534545in}{2.617827in}}%
\pgfusepath{stroke}%
\end{pgfscope}%
\begin{pgfscope}%
\pgfsetrectcap%
\pgfsetmiterjoin%
\pgfsetlinewidth{0.803000pt}%
\definecolor{currentstroke}{rgb}{0.000000,0.000000,0.000000}%
\pgfsetstrokecolor{currentstroke}%
\pgfsetdash{}{0pt}%
\pgfpathmoveto{\pgfqpoint{0.800000in}{0.528000in}}%
\pgfpathlineto{\pgfqpoint{0.800000in}{4.224000in}}%
\pgfusepath{stroke}%
\end{pgfscope}%
\begin{pgfscope}%
\pgfsetrectcap%
\pgfsetmiterjoin%
\pgfsetlinewidth{0.803000pt}%
\definecolor{currentstroke}{rgb}{0.000000,0.000000,0.000000}%
\pgfsetstrokecolor{currentstroke}%
\pgfsetdash{}{0pt}%
\pgfpathmoveto{\pgfqpoint{5.760000in}{0.528000in}}%
\pgfpathlineto{\pgfqpoint{5.760000in}{4.224000in}}%
\pgfusepath{stroke}%
\end{pgfscope}%
\begin{pgfscope}%
\pgfsetrectcap%
\pgfsetmiterjoin%
\pgfsetlinewidth{0.803000pt}%
\definecolor{currentstroke}{rgb}{0.000000,0.000000,0.000000}%
\pgfsetstrokecolor{currentstroke}%
\pgfsetdash{}{0pt}%
\pgfpathmoveto{\pgfqpoint{0.800000in}{0.528000in}}%
\pgfpathlineto{\pgfqpoint{5.760000in}{0.528000in}}%
\pgfusepath{stroke}%
\end{pgfscope}%
\begin{pgfscope}%
\pgfsetrectcap%
\pgfsetmiterjoin%
\pgfsetlinewidth{0.803000pt}%
\definecolor{currentstroke}{rgb}{0.000000,0.000000,0.000000}%
\pgfsetstrokecolor{currentstroke}%
\pgfsetdash{}{0pt}%
\pgfpathmoveto{\pgfqpoint{0.800000in}{4.224000in}}%
\pgfpathlineto{\pgfqpoint{5.760000in}{4.224000in}}%
\pgfusepath{stroke}%
\end{pgfscope}%
\begin{pgfscope}%
\definecolor{textcolor}{rgb}{0.000000,0.000000,0.000000}%
\pgfsetstrokecolor{textcolor}%
\pgfsetfillcolor{textcolor}%
\pgftext[x=3.280000in,y=4.307333in,,base]{\color{textcolor}\sffamily\fontsize{9.600000}{11.520000}\selectfont Função de Perda em Treino (compI)}%
\end{pgfscope}%
\begin{pgfscope}%
\pgfsetbuttcap%
\pgfsetmiterjoin%
\definecolor{currentfill}{rgb}{1.000000,1.000000,1.000000}%
\pgfsetfillcolor{currentfill}%
\pgfsetfillopacity{0.800000}%
\pgfsetlinewidth{1.003750pt}%
\definecolor{currentstroke}{rgb}{0.800000,0.800000,0.800000}%
\pgfsetstrokecolor{currentstroke}%
\pgfsetstrokeopacity{0.800000}%
\pgfsetdash{}{0pt}%
\pgfpathmoveto{\pgfqpoint{4.635586in}{3.808723in}}%
\pgfpathlineto{\pgfqpoint{5.682222in}{3.808723in}}%
\pgfpathquadraticcurveto{\pgfqpoint{5.704444in}{3.808723in}}{\pgfqpoint{5.704444in}{3.830945in}}%
\pgfpathlineto{\pgfqpoint{5.704444in}{4.146222in}}%
\pgfpathquadraticcurveto{\pgfqpoint{5.704444in}{4.168444in}}{\pgfqpoint{5.682222in}{4.168444in}}%
\pgfpathlineto{\pgfqpoint{4.635586in}{4.168444in}}%
\pgfpathquadraticcurveto{\pgfqpoint{4.613364in}{4.168444in}}{\pgfqpoint{4.613364in}{4.146222in}}%
\pgfpathlineto{\pgfqpoint{4.613364in}{3.830945in}}%
\pgfpathquadraticcurveto{\pgfqpoint{4.613364in}{3.808723in}}{\pgfqpoint{4.635586in}{3.808723in}}%
\pgfpathlineto{\pgfqpoint{4.635586in}{3.808723in}}%
\pgfpathclose%
\pgfusepath{stroke,fill}%
\end{pgfscope}%
\begin{pgfscope}%
\pgfsetrectcap%
\pgfsetroundjoin%
\pgfsetlinewidth{1.505625pt}%
\definecolor{currentstroke}{rgb}{0.121569,0.466667,0.705882}%
\pgfsetstrokecolor{currentstroke}%
\pgfsetdash{}{0pt}%
\pgfpathmoveto{\pgfqpoint{4.657808in}{4.078470in}}%
\pgfpathlineto{\pgfqpoint{4.768919in}{4.078470in}}%
\pgfpathlineto{\pgfqpoint{4.880030in}{4.078470in}}%
\pgfusepath{stroke}%
\end{pgfscope}%
\begin{pgfscope}%
\definecolor{textcolor}{rgb}{0.000000,0.000000,0.000000}%
\pgfsetstrokecolor{textcolor}%
\pgfsetfillcolor{textcolor}%
\pgftext[x=4.968919in,y=4.039582in,left,base]{\color{textcolor}\sffamily\fontsize{8.000000}{9.600000}\selectfont Treinamento}%
\end{pgfscope}%
\begin{pgfscope}%
\pgfsetrectcap%
\pgfsetroundjoin%
\pgfsetlinewidth{1.505625pt}%
\definecolor{currentstroke}{rgb}{1.000000,0.498039,0.054902}%
\pgfsetstrokecolor{currentstroke}%
\pgfsetdash{}{0pt}%
\pgfpathmoveto{\pgfqpoint{4.657808in}{3.915168in}}%
\pgfpathlineto{\pgfqpoint{4.768919in}{3.915168in}}%
\pgfpathlineto{\pgfqpoint{4.880030in}{3.915168in}}%
\pgfusepath{stroke}%
\end{pgfscope}%
\begin{pgfscope}%
\definecolor{textcolor}{rgb}{0.000000,0.000000,0.000000}%
\pgfsetstrokecolor{textcolor}%
\pgfsetfillcolor{textcolor}%
\pgftext[x=4.968919in,y=3.876279in,left,base]{\color{textcolor}\sffamily\fontsize{8.000000}{9.600000}\selectfont Validação}%
\end{pgfscope}%
\end{pgfpicture}%
\makeatother%
\endgroup%
}
    \end{minipage}

    \caption{Evolução do treinamento com \textit{hypertuning} (à esquerda) e do treinamento com hiperparâmetros fixados (à direita) do sistema especialista da competência I.}
    \label{fig:exp-fix-c1}
\end{figure}

Pela figura \ref{fig:exp-fix-c1}, é possível notar que o treinamento com hiperparâmetros fixados apresentou uma evolução mais suave da função de perda, de modo que, em torno da 20ª época, o valor para a base de validação estagnou em cerca de 0,5 e o valor para a base de treino oscilou entre 0,05. No caso do \textit{hypertuning}, a função de perda registrou números mais altos tanto para a base de treino quanto de validação, convergindo para um valor de aproximadamente 0,7 já nas primeiras épocas. Constata-se, assim, que o processo de otimização não surtiu efeitos para a competência I.

\subsubsection{Competência II}
\label{subsec:exp-fix-c2}

\begin{figure}[H]
    \begin{minipage}{0.45\textwidth}
        \resizebox{\textwidth}{!}{%% Creator: Matplotlib, PGF backend
%%
%% To include the figure in your LaTeX document, write
%%   \input{<filename>.pgf}
%%
%% Make sure the required packages are loaded in your preamble
%%   \usepackage{pgf}
%%
%% Also ensure that all the required font packages are loaded; for instance,
%% the lmodern package is sometimes necessary when using math font.
%%   \usepackage{lmodern}
%%
%% Figures using additional raster images can only be included by \input if
%% they are in the same directory as the main LaTeX file. For loading figures
%% from other directories you can use the `import` package
%%   \usepackage{import}
%%
%% and then include the figures with
%%   \import{<path to file>}{<filename>.pgf}
%%
%% Matplotlib used the following preamble
%%
%%   \usepackage{fontspec}
%%   \setmainfont{DejaVuSerif.ttf}[Path=\detokenize{/home/jose/anaconda3/envs/tf/lib/python3.9/site-packages/matplotlib/mpl-data/fonts/ttf/}]
%%   \setsansfont{DejaVuSans.ttf}[Path=\detokenize{/home/jose/anaconda3/envs/tf/lib/python3.9/site-packages/matplotlib/mpl-data/fonts/ttf/}]
%%   \setmonofont{DejaVuSansMono.ttf}[Path=\detokenize{/home/jose/anaconda3/envs/tf/lib/python3.9/site-packages/matplotlib/mpl-data/fonts/ttf/}]
%%   \makeatletter\@ifpackageloaded{underscore}{}{\usepackage[strings]{underscore}}\makeatother
%%
\begingroup%
\makeatletter%
\begin{pgfpicture}%
\pgfpathrectangle{\pgfpointorigin}{\pgfqpoint{6.400000in}{4.800000in}}%
\pgfusepath{use as bounding box, clip}%
\begin{pgfscope}%
\pgfsetbuttcap%
\pgfsetmiterjoin%
\definecolor{currentfill}{rgb}{1.000000,1.000000,1.000000}%
\pgfsetfillcolor{currentfill}%
\pgfsetlinewidth{0.000000pt}%
\definecolor{currentstroke}{rgb}{1.000000,1.000000,1.000000}%
\pgfsetstrokecolor{currentstroke}%
\pgfsetdash{}{0pt}%
\pgfpathmoveto{\pgfqpoint{0.000000in}{0.000000in}}%
\pgfpathlineto{\pgfqpoint{6.400000in}{0.000000in}}%
\pgfpathlineto{\pgfqpoint{6.400000in}{4.800000in}}%
\pgfpathlineto{\pgfqpoint{0.000000in}{4.800000in}}%
\pgfpathlineto{\pgfqpoint{0.000000in}{0.000000in}}%
\pgfpathclose%
\pgfusepath{fill}%
\end{pgfscope}%
\begin{pgfscope}%
\pgfsetbuttcap%
\pgfsetmiterjoin%
\definecolor{currentfill}{rgb}{1.000000,1.000000,1.000000}%
\pgfsetfillcolor{currentfill}%
\pgfsetlinewidth{0.000000pt}%
\definecolor{currentstroke}{rgb}{0.000000,0.000000,0.000000}%
\pgfsetstrokecolor{currentstroke}%
\pgfsetstrokeopacity{0.000000}%
\pgfsetdash{}{0pt}%
\pgfpathmoveto{\pgfqpoint{0.800000in}{0.528000in}}%
\pgfpathlineto{\pgfqpoint{5.760000in}{0.528000in}}%
\pgfpathlineto{\pgfqpoint{5.760000in}{4.224000in}}%
\pgfpathlineto{\pgfqpoint{0.800000in}{4.224000in}}%
\pgfpathlineto{\pgfqpoint{0.800000in}{0.528000in}}%
\pgfpathclose%
\pgfusepath{fill}%
\end{pgfscope}%
\begin{pgfscope}%
\pgfsetbuttcap%
\pgfsetroundjoin%
\definecolor{currentfill}{rgb}{0.000000,0.000000,0.000000}%
\pgfsetfillcolor{currentfill}%
\pgfsetlinewidth{0.803000pt}%
\definecolor{currentstroke}{rgb}{0.000000,0.000000,0.000000}%
\pgfsetstrokecolor{currentstroke}%
\pgfsetdash{}{0pt}%
\pgfsys@defobject{currentmarker}{\pgfqpoint{0.000000in}{-0.048611in}}{\pgfqpoint{0.000000in}{0.000000in}}{%
\pgfpathmoveto{\pgfqpoint{0.000000in}{0.000000in}}%
\pgfpathlineto{\pgfqpoint{0.000000in}{-0.048611in}}%
\pgfusepath{stroke,fill}%
}%
\begin{pgfscope}%
\pgfsys@transformshift{1.025455in}{0.528000in}%
\pgfsys@useobject{currentmarker}{}%
\end{pgfscope}%
\end{pgfscope}%
\begin{pgfscope}%
\definecolor{textcolor}{rgb}{0.000000,0.000000,0.000000}%
\pgfsetstrokecolor{textcolor}%
\pgfsetfillcolor{textcolor}%
\pgftext[x=1.025455in,y=0.430778in,,top]{\color{textcolor}\sffamily\fontsize{8.000000}{9.600000}\selectfont 0}%
\end{pgfscope}%
\begin{pgfscope}%
\pgfsetbuttcap%
\pgfsetroundjoin%
\definecolor{currentfill}{rgb}{0.000000,0.000000,0.000000}%
\pgfsetfillcolor{currentfill}%
\pgfsetlinewidth{0.803000pt}%
\definecolor{currentstroke}{rgb}{0.000000,0.000000,0.000000}%
\pgfsetstrokecolor{currentstroke}%
\pgfsetdash{}{0pt}%
\pgfsys@defobject{currentmarker}{\pgfqpoint{0.000000in}{-0.048611in}}{\pgfqpoint{0.000000in}{0.000000in}}{%
\pgfpathmoveto{\pgfqpoint{0.000000in}{0.000000in}}%
\pgfpathlineto{\pgfqpoint{0.000000in}{-0.048611in}}%
\pgfusepath{stroke,fill}%
}%
\begin{pgfscope}%
\pgfsys@transformshift{1.945677in}{0.528000in}%
\pgfsys@useobject{currentmarker}{}%
\end{pgfscope}%
\end{pgfscope}%
\begin{pgfscope}%
\definecolor{textcolor}{rgb}{0.000000,0.000000,0.000000}%
\pgfsetstrokecolor{textcolor}%
\pgfsetfillcolor{textcolor}%
\pgftext[x=1.945677in,y=0.430778in,,top]{\color{textcolor}\sffamily\fontsize{8.000000}{9.600000}\selectfont 10}%
\end{pgfscope}%
\begin{pgfscope}%
\pgfsetbuttcap%
\pgfsetroundjoin%
\definecolor{currentfill}{rgb}{0.000000,0.000000,0.000000}%
\pgfsetfillcolor{currentfill}%
\pgfsetlinewidth{0.803000pt}%
\definecolor{currentstroke}{rgb}{0.000000,0.000000,0.000000}%
\pgfsetstrokecolor{currentstroke}%
\pgfsetdash{}{0pt}%
\pgfsys@defobject{currentmarker}{\pgfqpoint{0.000000in}{-0.048611in}}{\pgfqpoint{0.000000in}{0.000000in}}{%
\pgfpathmoveto{\pgfqpoint{0.000000in}{0.000000in}}%
\pgfpathlineto{\pgfqpoint{0.000000in}{-0.048611in}}%
\pgfusepath{stroke,fill}%
}%
\begin{pgfscope}%
\pgfsys@transformshift{2.865900in}{0.528000in}%
\pgfsys@useobject{currentmarker}{}%
\end{pgfscope}%
\end{pgfscope}%
\begin{pgfscope}%
\definecolor{textcolor}{rgb}{0.000000,0.000000,0.000000}%
\pgfsetstrokecolor{textcolor}%
\pgfsetfillcolor{textcolor}%
\pgftext[x=2.865900in,y=0.430778in,,top]{\color{textcolor}\sffamily\fontsize{8.000000}{9.600000}\selectfont 20}%
\end{pgfscope}%
\begin{pgfscope}%
\pgfsetbuttcap%
\pgfsetroundjoin%
\definecolor{currentfill}{rgb}{0.000000,0.000000,0.000000}%
\pgfsetfillcolor{currentfill}%
\pgfsetlinewidth{0.803000pt}%
\definecolor{currentstroke}{rgb}{0.000000,0.000000,0.000000}%
\pgfsetstrokecolor{currentstroke}%
\pgfsetdash{}{0pt}%
\pgfsys@defobject{currentmarker}{\pgfqpoint{0.000000in}{-0.048611in}}{\pgfqpoint{0.000000in}{0.000000in}}{%
\pgfpathmoveto{\pgfqpoint{0.000000in}{0.000000in}}%
\pgfpathlineto{\pgfqpoint{0.000000in}{-0.048611in}}%
\pgfusepath{stroke,fill}%
}%
\begin{pgfscope}%
\pgfsys@transformshift{3.786122in}{0.528000in}%
\pgfsys@useobject{currentmarker}{}%
\end{pgfscope}%
\end{pgfscope}%
\begin{pgfscope}%
\definecolor{textcolor}{rgb}{0.000000,0.000000,0.000000}%
\pgfsetstrokecolor{textcolor}%
\pgfsetfillcolor{textcolor}%
\pgftext[x=3.786122in,y=0.430778in,,top]{\color{textcolor}\sffamily\fontsize{8.000000}{9.600000}\selectfont 30}%
\end{pgfscope}%
\begin{pgfscope}%
\pgfsetbuttcap%
\pgfsetroundjoin%
\definecolor{currentfill}{rgb}{0.000000,0.000000,0.000000}%
\pgfsetfillcolor{currentfill}%
\pgfsetlinewidth{0.803000pt}%
\definecolor{currentstroke}{rgb}{0.000000,0.000000,0.000000}%
\pgfsetstrokecolor{currentstroke}%
\pgfsetdash{}{0pt}%
\pgfsys@defobject{currentmarker}{\pgfqpoint{0.000000in}{-0.048611in}}{\pgfqpoint{0.000000in}{0.000000in}}{%
\pgfpathmoveto{\pgfqpoint{0.000000in}{0.000000in}}%
\pgfpathlineto{\pgfqpoint{0.000000in}{-0.048611in}}%
\pgfusepath{stroke,fill}%
}%
\begin{pgfscope}%
\pgfsys@transformshift{4.706345in}{0.528000in}%
\pgfsys@useobject{currentmarker}{}%
\end{pgfscope}%
\end{pgfscope}%
\begin{pgfscope}%
\definecolor{textcolor}{rgb}{0.000000,0.000000,0.000000}%
\pgfsetstrokecolor{textcolor}%
\pgfsetfillcolor{textcolor}%
\pgftext[x=4.706345in,y=0.430778in,,top]{\color{textcolor}\sffamily\fontsize{8.000000}{9.600000}\selectfont 40}%
\end{pgfscope}%
\begin{pgfscope}%
\pgfsetbuttcap%
\pgfsetroundjoin%
\definecolor{currentfill}{rgb}{0.000000,0.000000,0.000000}%
\pgfsetfillcolor{currentfill}%
\pgfsetlinewidth{0.803000pt}%
\definecolor{currentstroke}{rgb}{0.000000,0.000000,0.000000}%
\pgfsetstrokecolor{currentstroke}%
\pgfsetdash{}{0pt}%
\pgfsys@defobject{currentmarker}{\pgfqpoint{0.000000in}{-0.048611in}}{\pgfqpoint{0.000000in}{0.000000in}}{%
\pgfpathmoveto{\pgfqpoint{0.000000in}{0.000000in}}%
\pgfpathlineto{\pgfqpoint{0.000000in}{-0.048611in}}%
\pgfusepath{stroke,fill}%
}%
\begin{pgfscope}%
\pgfsys@transformshift{5.626568in}{0.528000in}%
\pgfsys@useobject{currentmarker}{}%
\end{pgfscope}%
\end{pgfscope}%
\begin{pgfscope}%
\definecolor{textcolor}{rgb}{0.000000,0.000000,0.000000}%
\pgfsetstrokecolor{textcolor}%
\pgfsetfillcolor{textcolor}%
\pgftext[x=5.626568in,y=0.430778in,,top]{\color{textcolor}\sffamily\fontsize{8.000000}{9.600000}\selectfont 50}%
\end{pgfscope}%
\begin{pgfscope}%
\definecolor{textcolor}{rgb}{0.000000,0.000000,0.000000}%
\pgfsetstrokecolor{textcolor}%
\pgfsetfillcolor{textcolor}%
\pgftext[x=3.280000in,y=0.267692in,,top]{\color{textcolor}\sffamily\fontsize{8.000000}{9.600000}\selectfont Época}%
\end{pgfscope}%
\begin{pgfscope}%
\pgfsetbuttcap%
\pgfsetroundjoin%
\definecolor{currentfill}{rgb}{0.000000,0.000000,0.000000}%
\pgfsetfillcolor{currentfill}%
\pgfsetlinewidth{0.803000pt}%
\definecolor{currentstroke}{rgb}{0.000000,0.000000,0.000000}%
\pgfsetstrokecolor{currentstroke}%
\pgfsetdash{}{0pt}%
\pgfsys@defobject{currentmarker}{\pgfqpoint{-0.048611in}{0.000000in}}{\pgfqpoint{-0.000000in}{0.000000in}}{%
\pgfpathmoveto{\pgfqpoint{-0.000000in}{0.000000in}}%
\pgfpathlineto{\pgfqpoint{-0.048611in}{0.000000in}}%
\pgfusepath{stroke,fill}%
}%
\begin{pgfscope}%
\pgfsys@transformshift{0.800000in}{0.676899in}%
\pgfsys@useobject{currentmarker}{}%
\end{pgfscope}%
\end{pgfscope}%
\begin{pgfscope}%
\definecolor{textcolor}{rgb}{0.000000,0.000000,0.000000}%
\pgfsetstrokecolor{textcolor}%
\pgfsetfillcolor{textcolor}%
\pgftext[x=0.632086in, y=0.634689in, left, base]{\color{textcolor}\sffamily\fontsize{8.000000}{9.600000}\selectfont 1}%
\end{pgfscope}%
\begin{pgfscope}%
\pgfsetbuttcap%
\pgfsetroundjoin%
\definecolor{currentfill}{rgb}{0.000000,0.000000,0.000000}%
\pgfsetfillcolor{currentfill}%
\pgfsetlinewidth{0.803000pt}%
\definecolor{currentstroke}{rgb}{0.000000,0.000000,0.000000}%
\pgfsetstrokecolor{currentstroke}%
\pgfsetdash{}{0pt}%
\pgfsys@defobject{currentmarker}{\pgfqpoint{-0.048611in}{0.000000in}}{\pgfqpoint{-0.000000in}{0.000000in}}{%
\pgfpathmoveto{\pgfqpoint{-0.000000in}{0.000000in}}%
\pgfpathlineto{\pgfqpoint{-0.048611in}{0.000000in}}%
\pgfusepath{stroke,fill}%
}%
\begin{pgfscope}%
\pgfsys@transformshift{0.800000in}{1.449678in}%
\pgfsys@useobject{currentmarker}{}%
\end{pgfscope}%
\end{pgfscope}%
\begin{pgfscope}%
\definecolor{textcolor}{rgb}{0.000000,0.000000,0.000000}%
\pgfsetstrokecolor{textcolor}%
\pgfsetfillcolor{textcolor}%
\pgftext[x=0.632086in, y=1.407469in, left, base]{\color{textcolor}\sffamily\fontsize{8.000000}{9.600000}\selectfont 2}%
\end{pgfscope}%
\begin{pgfscope}%
\pgfsetbuttcap%
\pgfsetroundjoin%
\definecolor{currentfill}{rgb}{0.000000,0.000000,0.000000}%
\pgfsetfillcolor{currentfill}%
\pgfsetlinewidth{0.803000pt}%
\definecolor{currentstroke}{rgb}{0.000000,0.000000,0.000000}%
\pgfsetstrokecolor{currentstroke}%
\pgfsetdash{}{0pt}%
\pgfsys@defobject{currentmarker}{\pgfqpoint{-0.048611in}{0.000000in}}{\pgfqpoint{-0.000000in}{0.000000in}}{%
\pgfpathmoveto{\pgfqpoint{-0.000000in}{0.000000in}}%
\pgfpathlineto{\pgfqpoint{-0.048611in}{0.000000in}}%
\pgfusepath{stroke,fill}%
}%
\begin{pgfscope}%
\pgfsys@transformshift{0.800000in}{2.222458in}%
\pgfsys@useobject{currentmarker}{}%
\end{pgfscope}%
\end{pgfscope}%
\begin{pgfscope}%
\definecolor{textcolor}{rgb}{0.000000,0.000000,0.000000}%
\pgfsetstrokecolor{textcolor}%
\pgfsetfillcolor{textcolor}%
\pgftext[x=0.632086in, y=2.180249in, left, base]{\color{textcolor}\sffamily\fontsize{8.000000}{9.600000}\selectfont 3}%
\end{pgfscope}%
\begin{pgfscope}%
\pgfsetbuttcap%
\pgfsetroundjoin%
\definecolor{currentfill}{rgb}{0.000000,0.000000,0.000000}%
\pgfsetfillcolor{currentfill}%
\pgfsetlinewidth{0.803000pt}%
\definecolor{currentstroke}{rgb}{0.000000,0.000000,0.000000}%
\pgfsetstrokecolor{currentstroke}%
\pgfsetdash{}{0pt}%
\pgfsys@defobject{currentmarker}{\pgfqpoint{-0.048611in}{0.000000in}}{\pgfqpoint{-0.000000in}{0.000000in}}{%
\pgfpathmoveto{\pgfqpoint{-0.000000in}{0.000000in}}%
\pgfpathlineto{\pgfqpoint{-0.048611in}{0.000000in}}%
\pgfusepath{stroke,fill}%
}%
\begin{pgfscope}%
\pgfsys@transformshift{0.800000in}{2.995237in}%
\pgfsys@useobject{currentmarker}{}%
\end{pgfscope}%
\end{pgfscope}%
\begin{pgfscope}%
\definecolor{textcolor}{rgb}{0.000000,0.000000,0.000000}%
\pgfsetstrokecolor{textcolor}%
\pgfsetfillcolor{textcolor}%
\pgftext[x=0.632086in, y=2.953028in, left, base]{\color{textcolor}\sffamily\fontsize{8.000000}{9.600000}\selectfont 4}%
\end{pgfscope}%
\begin{pgfscope}%
\pgfsetbuttcap%
\pgfsetroundjoin%
\definecolor{currentfill}{rgb}{0.000000,0.000000,0.000000}%
\pgfsetfillcolor{currentfill}%
\pgfsetlinewidth{0.803000pt}%
\definecolor{currentstroke}{rgb}{0.000000,0.000000,0.000000}%
\pgfsetstrokecolor{currentstroke}%
\pgfsetdash{}{0pt}%
\pgfsys@defobject{currentmarker}{\pgfqpoint{-0.048611in}{0.000000in}}{\pgfqpoint{-0.000000in}{0.000000in}}{%
\pgfpathmoveto{\pgfqpoint{-0.000000in}{0.000000in}}%
\pgfpathlineto{\pgfqpoint{-0.048611in}{0.000000in}}%
\pgfusepath{stroke,fill}%
}%
\begin{pgfscope}%
\pgfsys@transformshift{0.800000in}{3.768017in}%
\pgfsys@useobject{currentmarker}{}%
\end{pgfscope}%
\end{pgfscope}%
\begin{pgfscope}%
\definecolor{textcolor}{rgb}{0.000000,0.000000,0.000000}%
\pgfsetstrokecolor{textcolor}%
\pgfsetfillcolor{textcolor}%
\pgftext[x=0.632086in, y=3.725808in, left, base]{\color{textcolor}\sffamily\fontsize{8.000000}{9.600000}\selectfont 5}%
\end{pgfscope}%
\begin{pgfscope}%
\definecolor{textcolor}{rgb}{0.000000,0.000000,0.000000}%
\pgfsetstrokecolor{textcolor}%
\pgfsetfillcolor{textcolor}%
\pgftext[x=0.576530in,y=2.376000in,,bottom,rotate=90.000000]{\color{textcolor}\sffamily\fontsize{8.000000}{9.600000}\selectfont Perda}%
\end{pgfscope}%
\begin{pgfscope}%
\pgfpathrectangle{\pgfqpoint{0.800000in}{0.528000in}}{\pgfqpoint{4.960000in}{3.696000in}}%
\pgfusepath{clip}%
\pgfsetrectcap%
\pgfsetroundjoin%
\pgfsetlinewidth{1.505625pt}%
\definecolor{currentstroke}{rgb}{0.121569,0.466667,0.705882}%
\pgfsetstrokecolor{currentstroke}%
\pgfsetdash{}{0pt}%
\pgfpathmoveto{\pgfqpoint{1.025455in}{4.056000in}}%
\pgfpathlineto{\pgfqpoint{1.117477in}{0.983108in}}%
\pgfpathlineto{\pgfqpoint{1.209499in}{1.039818in}}%
\pgfpathlineto{\pgfqpoint{1.301521in}{1.074617in}}%
\pgfpathlineto{\pgfqpoint{1.393544in}{1.011455in}}%
\pgfpathlineto{\pgfqpoint{1.485566in}{1.065534in}}%
\pgfpathlineto{\pgfqpoint{1.577588in}{1.160304in}}%
\pgfpathlineto{\pgfqpoint{1.669610in}{1.072629in}}%
\pgfpathlineto{\pgfqpoint{1.761633in}{1.120518in}}%
\pgfpathlineto{\pgfqpoint{1.853655in}{1.087053in}}%
\pgfpathlineto{\pgfqpoint{1.945677in}{1.134572in}}%
\pgfpathlineto{\pgfqpoint{2.037699in}{1.052943in}}%
\pgfpathlineto{\pgfqpoint{2.129722in}{1.091558in}}%
\pgfpathlineto{\pgfqpoint{2.221744in}{1.205355in}}%
\pgfpathlineto{\pgfqpoint{2.313766in}{1.057138in}}%
\pgfpathlineto{\pgfqpoint{2.405788in}{1.025101in}}%
\pgfpathlineto{\pgfqpoint{2.497811in}{1.055093in}}%
\pgfpathlineto{\pgfqpoint{2.589833in}{1.095439in}}%
\pgfpathlineto{\pgfqpoint{2.681855in}{1.056839in}}%
\pgfpathlineto{\pgfqpoint{2.773878in}{1.110564in}}%
\pgfpathlineto{\pgfqpoint{2.865900in}{1.057658in}}%
\pgfpathlineto{\pgfqpoint{2.957922in}{1.112522in}}%
\pgfpathlineto{\pgfqpoint{3.049944in}{1.026851in}}%
\pgfpathlineto{\pgfqpoint{3.141967in}{1.042205in}}%
\pgfpathlineto{\pgfqpoint{3.233989in}{1.079096in}}%
\pgfpathlineto{\pgfqpoint{3.326011in}{1.085561in}}%
\pgfpathlineto{\pgfqpoint{3.418033in}{1.040315in}}%
\pgfpathlineto{\pgfqpoint{3.510056in}{1.058561in}}%
\pgfpathlineto{\pgfqpoint{3.602078in}{1.131634in}}%
\pgfpathlineto{\pgfqpoint{3.694100in}{1.011108in}}%
\pgfpathlineto{\pgfqpoint{3.786122in}{1.130675in}}%
\pgfpathlineto{\pgfqpoint{3.878145in}{1.043490in}}%
\pgfpathlineto{\pgfqpoint{3.970167in}{1.102697in}}%
\pgfpathlineto{\pgfqpoint{4.062189in}{1.160469in}}%
\pgfpathlineto{\pgfqpoint{4.154212in}{1.105001in}}%
\pgfpathlineto{\pgfqpoint{4.246234in}{1.003980in}}%
\pgfpathlineto{\pgfqpoint{4.338256in}{1.059315in}}%
\pgfpathlineto{\pgfqpoint{4.430278in}{1.020170in}}%
\pgfpathlineto{\pgfqpoint{4.522301in}{1.087933in}}%
\pgfpathlineto{\pgfqpoint{4.614323in}{1.123919in}}%
\pgfpathlineto{\pgfqpoint{4.706345in}{1.076661in}}%
\pgfpathlineto{\pgfqpoint{4.798367in}{1.069210in}}%
\pgfpathlineto{\pgfqpoint{4.890390in}{1.027789in}}%
\pgfpathlineto{\pgfqpoint{4.982412in}{1.184502in}}%
\pgfpathlineto{\pgfqpoint{5.074434in}{1.015765in}}%
\pgfpathlineto{\pgfqpoint{5.166456in}{1.057410in}}%
\pgfpathlineto{\pgfqpoint{5.258479in}{1.112148in}}%
\pgfpathlineto{\pgfqpoint{5.350501in}{1.014236in}}%
\pgfpathlineto{\pgfqpoint{5.442523in}{1.064415in}}%
\pgfpathlineto{\pgfqpoint{5.534545in}{1.031047in}}%
\pgfusepath{stroke}%
\end{pgfscope}%
\begin{pgfscope}%
\pgfpathrectangle{\pgfqpoint{0.800000in}{0.528000in}}{\pgfqpoint{4.960000in}{3.696000in}}%
\pgfusepath{clip}%
\pgfsetrectcap%
\pgfsetroundjoin%
\pgfsetlinewidth{1.505625pt}%
\definecolor{currentstroke}{rgb}{1.000000,0.498039,0.054902}%
\pgfsetstrokecolor{currentstroke}%
\pgfsetdash{}{0pt}%
\pgfpathmoveto{\pgfqpoint{1.025455in}{1.009501in}}%
\pgfpathlineto{\pgfqpoint{1.117477in}{0.944776in}}%
\pgfpathlineto{\pgfqpoint{1.209499in}{0.811871in}}%
\pgfpathlineto{\pgfqpoint{1.301521in}{2.495892in}}%
\pgfpathlineto{\pgfqpoint{1.393544in}{1.858759in}}%
\pgfpathlineto{\pgfqpoint{1.485566in}{0.736555in}}%
\pgfpathlineto{\pgfqpoint{1.577588in}{0.759280in}}%
\pgfpathlineto{\pgfqpoint{1.669610in}{1.010117in}}%
\pgfpathlineto{\pgfqpoint{1.761633in}{1.649354in}}%
\pgfpathlineto{\pgfqpoint{1.853655in}{0.696000in}}%
\pgfpathlineto{\pgfqpoint{1.945677in}{0.697704in}}%
\pgfpathlineto{\pgfqpoint{2.037699in}{1.534661in}}%
\pgfpathlineto{\pgfqpoint{2.129722in}{0.855294in}}%
\pgfpathlineto{\pgfqpoint{2.221744in}{0.766402in}}%
\pgfpathlineto{\pgfqpoint{2.313766in}{1.153408in}}%
\pgfpathlineto{\pgfqpoint{2.405788in}{0.741607in}}%
\pgfpathlineto{\pgfqpoint{2.497811in}{0.696126in}}%
\pgfpathlineto{\pgfqpoint{2.589833in}{0.805264in}}%
\pgfpathlineto{\pgfqpoint{2.681855in}{0.732027in}}%
\pgfpathlineto{\pgfqpoint{2.773878in}{1.900204in}}%
\pgfpathlineto{\pgfqpoint{2.865900in}{0.959406in}}%
\pgfpathlineto{\pgfqpoint{2.957922in}{0.947015in}}%
\pgfpathlineto{\pgfqpoint{3.049944in}{0.740041in}}%
\pgfpathlineto{\pgfqpoint{3.141967in}{0.762173in}}%
\pgfpathlineto{\pgfqpoint{3.233989in}{2.163460in}}%
\pgfpathlineto{\pgfqpoint{3.326011in}{0.727893in}}%
\pgfpathlineto{\pgfqpoint{3.418033in}{0.820098in}}%
\pgfpathlineto{\pgfqpoint{3.510056in}{0.709167in}}%
\pgfpathlineto{\pgfqpoint{3.602078in}{0.701479in}}%
\pgfpathlineto{\pgfqpoint{3.694100in}{1.351581in}}%
\pgfpathlineto{\pgfqpoint{3.786122in}{0.812915in}}%
\pgfpathlineto{\pgfqpoint{3.878145in}{1.064208in}}%
\pgfpathlineto{\pgfqpoint{3.970167in}{0.852082in}}%
\pgfpathlineto{\pgfqpoint{4.062189in}{0.858300in}}%
\pgfpathlineto{\pgfqpoint{4.154212in}{0.697792in}}%
\pgfpathlineto{\pgfqpoint{4.246234in}{0.969703in}}%
\pgfpathlineto{\pgfqpoint{4.338256in}{0.796044in}}%
\pgfpathlineto{\pgfqpoint{4.430278in}{0.822572in}}%
\pgfpathlineto{\pgfqpoint{4.522301in}{1.165678in}}%
\pgfpathlineto{\pgfqpoint{4.614323in}{0.750686in}}%
\pgfpathlineto{\pgfqpoint{4.706345in}{0.913881in}}%
\pgfpathlineto{\pgfqpoint{4.798367in}{2.098281in}}%
\pgfpathlineto{\pgfqpoint{4.890390in}{0.724375in}}%
\pgfpathlineto{\pgfqpoint{4.982412in}{2.267733in}}%
\pgfpathlineto{\pgfqpoint{5.074434in}{2.044071in}}%
\pgfpathlineto{\pgfqpoint{5.166456in}{0.846737in}}%
\pgfpathlineto{\pgfqpoint{5.258479in}{0.759719in}}%
\pgfpathlineto{\pgfqpoint{5.350501in}{0.769983in}}%
\pgfpathlineto{\pgfqpoint{5.442523in}{0.742173in}}%
\pgfpathlineto{\pgfqpoint{5.534545in}{0.818491in}}%
\pgfusepath{stroke}%
\end{pgfscope}%
\begin{pgfscope}%
\pgfsetrectcap%
\pgfsetmiterjoin%
\pgfsetlinewidth{0.803000pt}%
\definecolor{currentstroke}{rgb}{0.000000,0.000000,0.000000}%
\pgfsetstrokecolor{currentstroke}%
\pgfsetdash{}{0pt}%
\pgfpathmoveto{\pgfqpoint{0.800000in}{0.528000in}}%
\pgfpathlineto{\pgfqpoint{0.800000in}{4.224000in}}%
\pgfusepath{stroke}%
\end{pgfscope}%
\begin{pgfscope}%
\pgfsetrectcap%
\pgfsetmiterjoin%
\pgfsetlinewidth{0.803000pt}%
\definecolor{currentstroke}{rgb}{0.000000,0.000000,0.000000}%
\pgfsetstrokecolor{currentstroke}%
\pgfsetdash{}{0pt}%
\pgfpathmoveto{\pgfqpoint{5.760000in}{0.528000in}}%
\pgfpathlineto{\pgfqpoint{5.760000in}{4.224000in}}%
\pgfusepath{stroke}%
\end{pgfscope}%
\begin{pgfscope}%
\pgfsetrectcap%
\pgfsetmiterjoin%
\pgfsetlinewidth{0.803000pt}%
\definecolor{currentstroke}{rgb}{0.000000,0.000000,0.000000}%
\pgfsetstrokecolor{currentstroke}%
\pgfsetdash{}{0pt}%
\pgfpathmoveto{\pgfqpoint{0.800000in}{0.528000in}}%
\pgfpathlineto{\pgfqpoint{5.760000in}{0.528000in}}%
\pgfusepath{stroke}%
\end{pgfscope}%
\begin{pgfscope}%
\pgfsetrectcap%
\pgfsetmiterjoin%
\pgfsetlinewidth{0.803000pt}%
\definecolor{currentstroke}{rgb}{0.000000,0.000000,0.000000}%
\pgfsetstrokecolor{currentstroke}%
\pgfsetdash{}{0pt}%
\pgfpathmoveto{\pgfqpoint{0.800000in}{4.224000in}}%
\pgfpathlineto{\pgfqpoint{5.760000in}{4.224000in}}%
\pgfusepath{stroke}%
\end{pgfscope}%
\begin{pgfscope}%
\definecolor{textcolor}{rgb}{0.000000,0.000000,0.000000}%
\pgfsetstrokecolor{textcolor}%
\pgfsetfillcolor{textcolor}%
\pgftext[x=3.280000in,y=4.307333in,,base]{\color{textcolor}\sffamily\fontsize{9.600000}{11.520000}\selectfont Função de Perda em Treino (compII)}%
\end{pgfscope}%
\begin{pgfscope}%
\pgfsetbuttcap%
\pgfsetmiterjoin%
\definecolor{currentfill}{rgb}{1.000000,1.000000,1.000000}%
\pgfsetfillcolor{currentfill}%
\pgfsetfillopacity{0.800000}%
\pgfsetlinewidth{1.003750pt}%
\definecolor{currentstroke}{rgb}{0.800000,0.800000,0.800000}%
\pgfsetstrokecolor{currentstroke}%
\pgfsetstrokeopacity{0.800000}%
\pgfsetdash{}{0pt}%
\pgfpathmoveto{\pgfqpoint{4.635586in}{3.808723in}}%
\pgfpathlineto{\pgfqpoint{5.682222in}{3.808723in}}%
\pgfpathquadraticcurveto{\pgfqpoint{5.704444in}{3.808723in}}{\pgfqpoint{5.704444in}{3.830945in}}%
\pgfpathlineto{\pgfqpoint{5.704444in}{4.146222in}}%
\pgfpathquadraticcurveto{\pgfqpoint{5.704444in}{4.168444in}}{\pgfqpoint{5.682222in}{4.168444in}}%
\pgfpathlineto{\pgfqpoint{4.635586in}{4.168444in}}%
\pgfpathquadraticcurveto{\pgfqpoint{4.613364in}{4.168444in}}{\pgfqpoint{4.613364in}{4.146222in}}%
\pgfpathlineto{\pgfqpoint{4.613364in}{3.830945in}}%
\pgfpathquadraticcurveto{\pgfqpoint{4.613364in}{3.808723in}}{\pgfqpoint{4.635586in}{3.808723in}}%
\pgfpathlineto{\pgfqpoint{4.635586in}{3.808723in}}%
\pgfpathclose%
\pgfusepath{stroke,fill}%
\end{pgfscope}%
\begin{pgfscope}%
\pgfsetrectcap%
\pgfsetroundjoin%
\pgfsetlinewidth{1.505625pt}%
\definecolor{currentstroke}{rgb}{0.121569,0.466667,0.705882}%
\pgfsetstrokecolor{currentstroke}%
\pgfsetdash{}{0pt}%
\pgfpathmoveto{\pgfqpoint{4.657808in}{4.078470in}}%
\pgfpathlineto{\pgfqpoint{4.768919in}{4.078470in}}%
\pgfpathlineto{\pgfqpoint{4.880030in}{4.078470in}}%
\pgfusepath{stroke}%
\end{pgfscope}%
\begin{pgfscope}%
\definecolor{textcolor}{rgb}{0.000000,0.000000,0.000000}%
\pgfsetstrokecolor{textcolor}%
\pgfsetfillcolor{textcolor}%
\pgftext[x=4.968919in,y=4.039582in,left,base]{\color{textcolor}\sffamily\fontsize{8.000000}{9.600000}\selectfont Treinamento}%
\end{pgfscope}%
\begin{pgfscope}%
\pgfsetrectcap%
\pgfsetroundjoin%
\pgfsetlinewidth{1.505625pt}%
\definecolor{currentstroke}{rgb}{1.000000,0.498039,0.054902}%
\pgfsetstrokecolor{currentstroke}%
\pgfsetdash{}{0pt}%
\pgfpathmoveto{\pgfqpoint{4.657808in}{3.915168in}}%
\pgfpathlineto{\pgfqpoint{4.768919in}{3.915168in}}%
\pgfpathlineto{\pgfqpoint{4.880030in}{3.915168in}}%
\pgfusepath{stroke}%
\end{pgfscope}%
\begin{pgfscope}%
\definecolor{textcolor}{rgb}{0.000000,0.000000,0.000000}%
\pgfsetstrokecolor{textcolor}%
\pgfsetfillcolor{textcolor}%
\pgftext[x=4.968919in,y=3.876279in,left,base]{\color{textcolor}\sffamily\fontsize{8.000000}{9.600000}\selectfont Validação}%
\end{pgfscope}%
\end{pgfpicture}%
\makeatother%
\endgroup%
}
    \end{minipage}
    \begin{minipage}{0.45\textwidth}
        \resizebox{\textwidth}{!}{%% Creator: Matplotlib, PGF backend
%%
%% To include the figure in your LaTeX document, write
%%   \input{<filename>.pgf}
%%
%% Make sure the required packages are loaded in your preamble
%%   \usepackage{pgf}
%%
%% Also ensure that all the required font packages are loaded; for instance,
%% the lmodern package is sometimes necessary when using math font.
%%   \usepackage{lmodern}
%%
%% Figures using additional raster images can only be included by \input if
%% they are in the same directory as the main LaTeX file. For loading figures
%% from other directories you can use the `import` package
%%   \usepackage{import}
%%
%% and then include the figures with
%%   \import{<path to file>}{<filename>.pgf}
%%
%% Matplotlib used the following preamble
%%   \def\mathdefault#1{#1}
%%   \everymath=\expandafter{\the\everymath\displaystyle}
%%
%%   \usepackage{fontspec}
%%   \setmainfont{DejaVuSerif.ttf}[Path=\detokenize{/home/josemayer/.local/lib/python3.9/site-packages/matplotlib/mpl-data/fonts/ttf/}]
%%   \setsansfont{DejaVuSans.ttf}[Path=\detokenize{/home/josemayer/.local/lib/python3.9/site-packages/matplotlib/mpl-data/fonts/ttf/}]
%%   \setmonofont{DejaVuSansMono.ttf}[Path=\detokenize{/home/josemayer/.local/lib/python3.9/site-packages/matplotlib/mpl-data/fonts/ttf/}]
%%   \makeatletter\@ifpackageloaded{underscore}{}{\usepackage[strings]{underscore}}\makeatother
%%
\begingroup%
\makeatletter%
\begin{pgfpicture}%
\pgfpathrectangle{\pgfpointorigin}{\pgfqpoint{6.400000in}{4.800000in}}%
\pgfusepath{use as bounding box, clip}%
\begin{pgfscope}%
\pgfsetbuttcap%
\pgfsetmiterjoin%
\definecolor{currentfill}{rgb}{1.000000,1.000000,1.000000}%
\pgfsetfillcolor{currentfill}%
\pgfsetlinewidth{0.000000pt}%
\definecolor{currentstroke}{rgb}{1.000000,1.000000,1.000000}%
\pgfsetstrokecolor{currentstroke}%
\pgfsetdash{}{0pt}%
\pgfpathmoveto{\pgfqpoint{0.000000in}{0.000000in}}%
\pgfpathlineto{\pgfqpoint{6.400000in}{0.000000in}}%
\pgfpathlineto{\pgfqpoint{6.400000in}{4.800000in}}%
\pgfpathlineto{\pgfqpoint{0.000000in}{4.800000in}}%
\pgfpathlineto{\pgfqpoint{0.000000in}{0.000000in}}%
\pgfpathclose%
\pgfusepath{fill}%
\end{pgfscope}%
\begin{pgfscope}%
\pgfsetbuttcap%
\pgfsetmiterjoin%
\definecolor{currentfill}{rgb}{1.000000,1.000000,1.000000}%
\pgfsetfillcolor{currentfill}%
\pgfsetlinewidth{0.000000pt}%
\definecolor{currentstroke}{rgb}{0.000000,0.000000,0.000000}%
\pgfsetstrokecolor{currentstroke}%
\pgfsetstrokeopacity{0.000000}%
\pgfsetdash{}{0pt}%
\pgfpathmoveto{\pgfqpoint{0.800000in}{0.528000in}}%
\pgfpathlineto{\pgfqpoint{5.760000in}{0.528000in}}%
\pgfpathlineto{\pgfqpoint{5.760000in}{4.224000in}}%
\pgfpathlineto{\pgfqpoint{0.800000in}{4.224000in}}%
\pgfpathlineto{\pgfqpoint{0.800000in}{0.528000in}}%
\pgfpathclose%
\pgfusepath{fill}%
\end{pgfscope}%
\begin{pgfscope}%
\pgfsetbuttcap%
\pgfsetroundjoin%
\definecolor{currentfill}{rgb}{0.000000,0.000000,0.000000}%
\pgfsetfillcolor{currentfill}%
\pgfsetlinewidth{0.803000pt}%
\definecolor{currentstroke}{rgb}{0.000000,0.000000,0.000000}%
\pgfsetstrokecolor{currentstroke}%
\pgfsetdash{}{0pt}%
\pgfsys@defobject{currentmarker}{\pgfqpoint{0.000000in}{-0.048611in}}{\pgfqpoint{0.000000in}{0.000000in}}{%
\pgfpathmoveto{\pgfqpoint{0.000000in}{0.000000in}}%
\pgfpathlineto{\pgfqpoint{0.000000in}{-0.048611in}}%
\pgfusepath{stroke,fill}%
}%
\begin{pgfscope}%
\pgfsys@transformshift{1.025455in}{0.528000in}%
\pgfsys@useobject{currentmarker}{}%
\end{pgfscope}%
\end{pgfscope}%
\begin{pgfscope}%
\definecolor{textcolor}{rgb}{0.000000,0.000000,0.000000}%
\pgfsetstrokecolor{textcolor}%
\pgfsetfillcolor{textcolor}%
\pgftext[x=1.025455in,y=0.430778in,,top]{\color{textcolor}{\sffamily\fontsize{8.000000}{9.600000}\selectfont\catcode`\^=\active\def^{\ifmmode\sp\else\^{}\fi}\catcode`\%=\active\def%{\%}0}}%
\end{pgfscope}%
\begin{pgfscope}%
\pgfsetbuttcap%
\pgfsetroundjoin%
\definecolor{currentfill}{rgb}{0.000000,0.000000,0.000000}%
\pgfsetfillcolor{currentfill}%
\pgfsetlinewidth{0.803000pt}%
\definecolor{currentstroke}{rgb}{0.000000,0.000000,0.000000}%
\pgfsetstrokecolor{currentstroke}%
\pgfsetdash{}{0pt}%
\pgfsys@defobject{currentmarker}{\pgfqpoint{0.000000in}{-0.048611in}}{\pgfqpoint{0.000000in}{0.000000in}}{%
\pgfpathmoveto{\pgfqpoint{0.000000in}{0.000000in}}%
\pgfpathlineto{\pgfqpoint{0.000000in}{-0.048611in}}%
\pgfusepath{stroke,fill}%
}%
\begin{pgfscope}%
\pgfsys@transformshift{1.945677in}{0.528000in}%
\pgfsys@useobject{currentmarker}{}%
\end{pgfscope}%
\end{pgfscope}%
\begin{pgfscope}%
\definecolor{textcolor}{rgb}{0.000000,0.000000,0.000000}%
\pgfsetstrokecolor{textcolor}%
\pgfsetfillcolor{textcolor}%
\pgftext[x=1.945677in,y=0.430778in,,top]{\color{textcolor}{\sffamily\fontsize{8.000000}{9.600000}\selectfont\catcode`\^=\active\def^{\ifmmode\sp\else\^{}\fi}\catcode`\%=\active\def%{\%}10}}%
\end{pgfscope}%
\begin{pgfscope}%
\pgfsetbuttcap%
\pgfsetroundjoin%
\definecolor{currentfill}{rgb}{0.000000,0.000000,0.000000}%
\pgfsetfillcolor{currentfill}%
\pgfsetlinewidth{0.803000pt}%
\definecolor{currentstroke}{rgb}{0.000000,0.000000,0.000000}%
\pgfsetstrokecolor{currentstroke}%
\pgfsetdash{}{0pt}%
\pgfsys@defobject{currentmarker}{\pgfqpoint{0.000000in}{-0.048611in}}{\pgfqpoint{0.000000in}{0.000000in}}{%
\pgfpathmoveto{\pgfqpoint{0.000000in}{0.000000in}}%
\pgfpathlineto{\pgfqpoint{0.000000in}{-0.048611in}}%
\pgfusepath{stroke,fill}%
}%
\begin{pgfscope}%
\pgfsys@transformshift{2.865900in}{0.528000in}%
\pgfsys@useobject{currentmarker}{}%
\end{pgfscope}%
\end{pgfscope}%
\begin{pgfscope}%
\definecolor{textcolor}{rgb}{0.000000,0.000000,0.000000}%
\pgfsetstrokecolor{textcolor}%
\pgfsetfillcolor{textcolor}%
\pgftext[x=2.865900in,y=0.430778in,,top]{\color{textcolor}{\sffamily\fontsize{8.000000}{9.600000}\selectfont\catcode`\^=\active\def^{\ifmmode\sp\else\^{}\fi}\catcode`\%=\active\def%{\%}20}}%
\end{pgfscope}%
\begin{pgfscope}%
\pgfsetbuttcap%
\pgfsetroundjoin%
\definecolor{currentfill}{rgb}{0.000000,0.000000,0.000000}%
\pgfsetfillcolor{currentfill}%
\pgfsetlinewidth{0.803000pt}%
\definecolor{currentstroke}{rgb}{0.000000,0.000000,0.000000}%
\pgfsetstrokecolor{currentstroke}%
\pgfsetdash{}{0pt}%
\pgfsys@defobject{currentmarker}{\pgfqpoint{0.000000in}{-0.048611in}}{\pgfqpoint{0.000000in}{0.000000in}}{%
\pgfpathmoveto{\pgfqpoint{0.000000in}{0.000000in}}%
\pgfpathlineto{\pgfqpoint{0.000000in}{-0.048611in}}%
\pgfusepath{stroke,fill}%
}%
\begin{pgfscope}%
\pgfsys@transformshift{3.786122in}{0.528000in}%
\pgfsys@useobject{currentmarker}{}%
\end{pgfscope}%
\end{pgfscope}%
\begin{pgfscope}%
\definecolor{textcolor}{rgb}{0.000000,0.000000,0.000000}%
\pgfsetstrokecolor{textcolor}%
\pgfsetfillcolor{textcolor}%
\pgftext[x=3.786122in,y=0.430778in,,top]{\color{textcolor}{\sffamily\fontsize{8.000000}{9.600000}\selectfont\catcode`\^=\active\def^{\ifmmode\sp\else\^{}\fi}\catcode`\%=\active\def%{\%}30}}%
\end{pgfscope}%
\begin{pgfscope}%
\pgfsetbuttcap%
\pgfsetroundjoin%
\definecolor{currentfill}{rgb}{0.000000,0.000000,0.000000}%
\pgfsetfillcolor{currentfill}%
\pgfsetlinewidth{0.803000pt}%
\definecolor{currentstroke}{rgb}{0.000000,0.000000,0.000000}%
\pgfsetstrokecolor{currentstroke}%
\pgfsetdash{}{0pt}%
\pgfsys@defobject{currentmarker}{\pgfqpoint{0.000000in}{-0.048611in}}{\pgfqpoint{0.000000in}{0.000000in}}{%
\pgfpathmoveto{\pgfqpoint{0.000000in}{0.000000in}}%
\pgfpathlineto{\pgfqpoint{0.000000in}{-0.048611in}}%
\pgfusepath{stroke,fill}%
}%
\begin{pgfscope}%
\pgfsys@transformshift{4.706345in}{0.528000in}%
\pgfsys@useobject{currentmarker}{}%
\end{pgfscope}%
\end{pgfscope}%
\begin{pgfscope}%
\definecolor{textcolor}{rgb}{0.000000,0.000000,0.000000}%
\pgfsetstrokecolor{textcolor}%
\pgfsetfillcolor{textcolor}%
\pgftext[x=4.706345in,y=0.430778in,,top]{\color{textcolor}{\sffamily\fontsize{8.000000}{9.600000}\selectfont\catcode`\^=\active\def^{\ifmmode\sp\else\^{}\fi}\catcode`\%=\active\def%{\%}40}}%
\end{pgfscope}%
\begin{pgfscope}%
\pgfsetbuttcap%
\pgfsetroundjoin%
\definecolor{currentfill}{rgb}{0.000000,0.000000,0.000000}%
\pgfsetfillcolor{currentfill}%
\pgfsetlinewidth{0.803000pt}%
\definecolor{currentstroke}{rgb}{0.000000,0.000000,0.000000}%
\pgfsetstrokecolor{currentstroke}%
\pgfsetdash{}{0pt}%
\pgfsys@defobject{currentmarker}{\pgfqpoint{0.000000in}{-0.048611in}}{\pgfqpoint{0.000000in}{0.000000in}}{%
\pgfpathmoveto{\pgfqpoint{0.000000in}{0.000000in}}%
\pgfpathlineto{\pgfqpoint{0.000000in}{-0.048611in}}%
\pgfusepath{stroke,fill}%
}%
\begin{pgfscope}%
\pgfsys@transformshift{5.626568in}{0.528000in}%
\pgfsys@useobject{currentmarker}{}%
\end{pgfscope}%
\end{pgfscope}%
\begin{pgfscope}%
\definecolor{textcolor}{rgb}{0.000000,0.000000,0.000000}%
\pgfsetstrokecolor{textcolor}%
\pgfsetfillcolor{textcolor}%
\pgftext[x=5.626568in,y=0.430778in,,top]{\color{textcolor}{\sffamily\fontsize{8.000000}{9.600000}\selectfont\catcode`\^=\active\def^{\ifmmode\sp\else\^{}\fi}\catcode`\%=\active\def%{\%}50}}%
\end{pgfscope}%
\begin{pgfscope}%
\definecolor{textcolor}{rgb}{0.000000,0.000000,0.000000}%
\pgfsetstrokecolor{textcolor}%
\pgfsetfillcolor{textcolor}%
\pgftext[x=3.280000in,y=0.267692in,,top]{\color{textcolor}{\sffamily\fontsize{8.000000}{9.600000}\selectfont\catcode`\^=\active\def^{\ifmmode\sp\else\^{}\fi}\catcode`\%=\active\def%{\%}Época}}%
\end{pgfscope}%
\begin{pgfscope}%
\pgfsetbuttcap%
\pgfsetroundjoin%
\definecolor{currentfill}{rgb}{0.000000,0.000000,0.000000}%
\pgfsetfillcolor{currentfill}%
\pgfsetlinewidth{0.803000pt}%
\definecolor{currentstroke}{rgb}{0.000000,0.000000,0.000000}%
\pgfsetstrokecolor{currentstroke}%
\pgfsetdash{}{0pt}%
\pgfsys@defobject{currentmarker}{\pgfqpoint{-0.048611in}{0.000000in}}{\pgfqpoint{-0.000000in}{0.000000in}}{%
\pgfpathmoveto{\pgfqpoint{-0.000000in}{0.000000in}}%
\pgfpathlineto{\pgfqpoint{-0.048611in}{0.000000in}}%
\pgfusepath{stroke,fill}%
}%
\begin{pgfscope}%
\pgfsys@transformshift{0.800000in}{0.627995in}%
\pgfsys@useobject{currentmarker}{}%
\end{pgfscope}%
\end{pgfscope}%
\begin{pgfscope}%
\definecolor{textcolor}{rgb}{0.000000,0.000000,0.000000}%
\pgfsetstrokecolor{textcolor}%
\pgfsetfillcolor{textcolor}%
\pgftext[x=0.526074in, y=0.585786in, left, base]{\color{textcolor}{\sffamily\fontsize{8.000000}{9.600000}\selectfont\catcode`\^=\active\def^{\ifmmode\sp\else\^{}\fi}\catcode`\%=\active\def%{\%}0.0}}%
\end{pgfscope}%
\begin{pgfscope}%
\pgfsetbuttcap%
\pgfsetroundjoin%
\definecolor{currentfill}{rgb}{0.000000,0.000000,0.000000}%
\pgfsetfillcolor{currentfill}%
\pgfsetlinewidth{0.803000pt}%
\definecolor{currentstroke}{rgb}{0.000000,0.000000,0.000000}%
\pgfsetstrokecolor{currentstroke}%
\pgfsetdash{}{0pt}%
\pgfsys@defobject{currentmarker}{\pgfqpoint{-0.048611in}{0.000000in}}{\pgfqpoint{-0.000000in}{0.000000in}}{%
\pgfpathmoveto{\pgfqpoint{-0.000000in}{0.000000in}}%
\pgfpathlineto{\pgfqpoint{-0.048611in}{0.000000in}}%
\pgfusepath{stroke,fill}%
}%
\begin{pgfscope}%
\pgfsys@transformshift{0.800000in}{1.160771in}%
\pgfsys@useobject{currentmarker}{}%
\end{pgfscope}%
\end{pgfscope}%
\begin{pgfscope}%
\definecolor{textcolor}{rgb}{0.000000,0.000000,0.000000}%
\pgfsetstrokecolor{textcolor}%
\pgfsetfillcolor{textcolor}%
\pgftext[x=0.526074in, y=1.118561in, left, base]{\color{textcolor}{\sffamily\fontsize{8.000000}{9.600000}\selectfont\catcode`\^=\active\def^{\ifmmode\sp\else\^{}\fi}\catcode`\%=\active\def%{\%}0.2}}%
\end{pgfscope}%
\begin{pgfscope}%
\pgfsetbuttcap%
\pgfsetroundjoin%
\definecolor{currentfill}{rgb}{0.000000,0.000000,0.000000}%
\pgfsetfillcolor{currentfill}%
\pgfsetlinewidth{0.803000pt}%
\definecolor{currentstroke}{rgb}{0.000000,0.000000,0.000000}%
\pgfsetstrokecolor{currentstroke}%
\pgfsetdash{}{0pt}%
\pgfsys@defobject{currentmarker}{\pgfqpoint{-0.048611in}{0.000000in}}{\pgfqpoint{-0.000000in}{0.000000in}}{%
\pgfpathmoveto{\pgfqpoint{-0.000000in}{0.000000in}}%
\pgfpathlineto{\pgfqpoint{-0.048611in}{0.000000in}}%
\pgfusepath{stroke,fill}%
}%
\begin{pgfscope}%
\pgfsys@transformshift{0.800000in}{1.693546in}%
\pgfsys@useobject{currentmarker}{}%
\end{pgfscope}%
\end{pgfscope}%
\begin{pgfscope}%
\definecolor{textcolor}{rgb}{0.000000,0.000000,0.000000}%
\pgfsetstrokecolor{textcolor}%
\pgfsetfillcolor{textcolor}%
\pgftext[x=0.526074in, y=1.651337in, left, base]{\color{textcolor}{\sffamily\fontsize{8.000000}{9.600000}\selectfont\catcode`\^=\active\def^{\ifmmode\sp\else\^{}\fi}\catcode`\%=\active\def%{\%}0.4}}%
\end{pgfscope}%
\begin{pgfscope}%
\pgfsetbuttcap%
\pgfsetroundjoin%
\definecolor{currentfill}{rgb}{0.000000,0.000000,0.000000}%
\pgfsetfillcolor{currentfill}%
\pgfsetlinewidth{0.803000pt}%
\definecolor{currentstroke}{rgb}{0.000000,0.000000,0.000000}%
\pgfsetstrokecolor{currentstroke}%
\pgfsetdash{}{0pt}%
\pgfsys@defobject{currentmarker}{\pgfqpoint{-0.048611in}{0.000000in}}{\pgfqpoint{-0.000000in}{0.000000in}}{%
\pgfpathmoveto{\pgfqpoint{-0.000000in}{0.000000in}}%
\pgfpathlineto{\pgfqpoint{-0.048611in}{0.000000in}}%
\pgfusepath{stroke,fill}%
}%
\begin{pgfscope}%
\pgfsys@transformshift{0.800000in}{2.226321in}%
\pgfsys@useobject{currentmarker}{}%
\end{pgfscope}%
\end{pgfscope}%
\begin{pgfscope}%
\definecolor{textcolor}{rgb}{0.000000,0.000000,0.000000}%
\pgfsetstrokecolor{textcolor}%
\pgfsetfillcolor{textcolor}%
\pgftext[x=0.526074in, y=2.184112in, left, base]{\color{textcolor}{\sffamily\fontsize{8.000000}{9.600000}\selectfont\catcode`\^=\active\def^{\ifmmode\sp\else\^{}\fi}\catcode`\%=\active\def%{\%}0.6}}%
\end{pgfscope}%
\begin{pgfscope}%
\pgfsetbuttcap%
\pgfsetroundjoin%
\definecolor{currentfill}{rgb}{0.000000,0.000000,0.000000}%
\pgfsetfillcolor{currentfill}%
\pgfsetlinewidth{0.803000pt}%
\definecolor{currentstroke}{rgb}{0.000000,0.000000,0.000000}%
\pgfsetstrokecolor{currentstroke}%
\pgfsetdash{}{0pt}%
\pgfsys@defobject{currentmarker}{\pgfqpoint{-0.048611in}{0.000000in}}{\pgfqpoint{-0.000000in}{0.000000in}}{%
\pgfpathmoveto{\pgfqpoint{-0.000000in}{0.000000in}}%
\pgfpathlineto{\pgfqpoint{-0.048611in}{0.000000in}}%
\pgfusepath{stroke,fill}%
}%
\begin{pgfscope}%
\pgfsys@transformshift{0.800000in}{2.759097in}%
\pgfsys@useobject{currentmarker}{}%
\end{pgfscope}%
\end{pgfscope}%
\begin{pgfscope}%
\definecolor{textcolor}{rgb}{0.000000,0.000000,0.000000}%
\pgfsetstrokecolor{textcolor}%
\pgfsetfillcolor{textcolor}%
\pgftext[x=0.526074in, y=2.716887in, left, base]{\color{textcolor}{\sffamily\fontsize{8.000000}{9.600000}\selectfont\catcode`\^=\active\def^{\ifmmode\sp\else\^{}\fi}\catcode`\%=\active\def%{\%}0.8}}%
\end{pgfscope}%
\begin{pgfscope}%
\pgfsetbuttcap%
\pgfsetroundjoin%
\definecolor{currentfill}{rgb}{0.000000,0.000000,0.000000}%
\pgfsetfillcolor{currentfill}%
\pgfsetlinewidth{0.803000pt}%
\definecolor{currentstroke}{rgb}{0.000000,0.000000,0.000000}%
\pgfsetstrokecolor{currentstroke}%
\pgfsetdash{}{0pt}%
\pgfsys@defobject{currentmarker}{\pgfqpoint{-0.048611in}{0.000000in}}{\pgfqpoint{-0.000000in}{0.000000in}}{%
\pgfpathmoveto{\pgfqpoint{-0.000000in}{0.000000in}}%
\pgfpathlineto{\pgfqpoint{-0.048611in}{0.000000in}}%
\pgfusepath{stroke,fill}%
}%
\begin{pgfscope}%
\pgfsys@transformshift{0.800000in}{3.291872in}%
\pgfsys@useobject{currentmarker}{}%
\end{pgfscope}%
\end{pgfscope}%
\begin{pgfscope}%
\definecolor{textcolor}{rgb}{0.000000,0.000000,0.000000}%
\pgfsetstrokecolor{textcolor}%
\pgfsetfillcolor{textcolor}%
\pgftext[x=0.526074in, y=3.249663in, left, base]{\color{textcolor}{\sffamily\fontsize{8.000000}{9.600000}\selectfont\catcode`\^=\active\def^{\ifmmode\sp\else\^{}\fi}\catcode`\%=\active\def%{\%}1.0}}%
\end{pgfscope}%
\begin{pgfscope}%
\pgfsetbuttcap%
\pgfsetroundjoin%
\definecolor{currentfill}{rgb}{0.000000,0.000000,0.000000}%
\pgfsetfillcolor{currentfill}%
\pgfsetlinewidth{0.803000pt}%
\definecolor{currentstroke}{rgb}{0.000000,0.000000,0.000000}%
\pgfsetstrokecolor{currentstroke}%
\pgfsetdash{}{0pt}%
\pgfsys@defobject{currentmarker}{\pgfqpoint{-0.048611in}{0.000000in}}{\pgfqpoint{-0.000000in}{0.000000in}}{%
\pgfpathmoveto{\pgfqpoint{-0.000000in}{0.000000in}}%
\pgfpathlineto{\pgfqpoint{-0.048611in}{0.000000in}}%
\pgfusepath{stroke,fill}%
}%
\begin{pgfscope}%
\pgfsys@transformshift{0.800000in}{3.824647in}%
\pgfsys@useobject{currentmarker}{}%
\end{pgfscope}%
\end{pgfscope}%
\begin{pgfscope}%
\definecolor{textcolor}{rgb}{0.000000,0.000000,0.000000}%
\pgfsetstrokecolor{textcolor}%
\pgfsetfillcolor{textcolor}%
\pgftext[x=0.526074in, y=3.782438in, left, base]{\color{textcolor}{\sffamily\fontsize{8.000000}{9.600000}\selectfont\catcode`\^=\active\def^{\ifmmode\sp\else\^{}\fi}\catcode`\%=\active\def%{\%}1.2}}%
\end{pgfscope}%
\begin{pgfscope}%
\definecolor{textcolor}{rgb}{0.000000,0.000000,0.000000}%
\pgfsetstrokecolor{textcolor}%
\pgfsetfillcolor{textcolor}%
\pgftext[x=0.470519in,y=2.376000in,,bottom,rotate=90.000000]{\color{textcolor}{\sffamily\fontsize{8.000000}{9.600000}\selectfont\catcode`\^=\active\def^{\ifmmode\sp\else\^{}\fi}\catcode`\%=\active\def%{\%}Perda}}%
\end{pgfscope}%
\begin{pgfscope}%
\pgfpathrectangle{\pgfqpoint{0.800000in}{0.528000in}}{\pgfqpoint{4.960000in}{3.696000in}}%
\pgfusepath{clip}%
\pgfsetrectcap%
\pgfsetroundjoin%
\pgfsetlinewidth{1.505625pt}%
\definecolor{currentstroke}{rgb}{0.121569,0.466667,0.705882}%
\pgfsetstrokecolor{currentstroke}%
\pgfsetdash{}{0pt}%
\pgfpathmoveto{\pgfqpoint{1.025455in}{3.793056in}}%
\pgfpathlineto{\pgfqpoint{1.117477in}{3.444711in}}%
\pgfpathlineto{\pgfqpoint{1.209499in}{3.138253in}}%
\pgfpathlineto{\pgfqpoint{1.301521in}{2.786236in}}%
\pgfpathlineto{\pgfqpoint{1.393544in}{2.537856in}}%
\pgfpathlineto{\pgfqpoint{1.485566in}{2.284212in}}%
\pgfpathlineto{\pgfqpoint{1.577588in}{2.078223in}}%
\pgfpathlineto{\pgfqpoint{1.669610in}{1.833899in}}%
\pgfpathlineto{\pgfqpoint{1.761633in}{1.651733in}}%
\pgfpathlineto{\pgfqpoint{1.853655in}{1.490025in}}%
\pgfpathlineto{\pgfqpoint{1.945677in}{1.394150in}}%
\pgfpathlineto{\pgfqpoint{2.037699in}{1.332655in}}%
\pgfpathlineto{\pgfqpoint{2.129722in}{1.196446in}}%
\pgfpathlineto{\pgfqpoint{2.221744in}{1.204404in}}%
\pgfpathlineto{\pgfqpoint{2.313766in}{1.109526in}}%
\pgfpathlineto{\pgfqpoint{2.405788in}{1.009371in}}%
\pgfpathlineto{\pgfqpoint{2.497811in}{0.999161in}}%
\pgfpathlineto{\pgfqpoint{2.589833in}{0.923752in}}%
\pgfpathlineto{\pgfqpoint{2.681855in}{0.900239in}}%
\pgfpathlineto{\pgfqpoint{2.773878in}{0.948409in}}%
\pgfpathlineto{\pgfqpoint{2.865900in}{0.818972in}}%
\pgfpathlineto{\pgfqpoint{2.957922in}{1.047486in}}%
\pgfpathlineto{\pgfqpoint{3.049944in}{0.796528in}}%
\pgfpathlineto{\pgfqpoint{3.141967in}{0.797438in}}%
\pgfpathlineto{\pgfqpoint{3.233989in}{0.783375in}}%
\pgfpathlineto{\pgfqpoint{3.326011in}{0.801394in}}%
\pgfpathlineto{\pgfqpoint{3.418033in}{0.787001in}}%
\pgfpathlineto{\pgfqpoint{3.510056in}{0.782815in}}%
\pgfpathlineto{\pgfqpoint{3.602078in}{0.795799in}}%
\pgfpathlineto{\pgfqpoint{3.694100in}{0.835435in}}%
\pgfpathlineto{\pgfqpoint{3.786122in}{0.826750in}}%
\pgfpathlineto{\pgfqpoint{3.878145in}{0.763143in}}%
\pgfpathlineto{\pgfqpoint{3.970167in}{0.820615in}}%
\pgfpathlineto{\pgfqpoint{4.062189in}{1.134001in}}%
\pgfpathlineto{\pgfqpoint{4.154212in}{0.920153in}}%
\pgfpathlineto{\pgfqpoint{4.246234in}{0.781518in}}%
\pgfpathlineto{\pgfqpoint{4.338256in}{0.863123in}}%
\pgfpathlineto{\pgfqpoint{4.430278in}{0.809563in}}%
\pgfpathlineto{\pgfqpoint{4.522301in}{0.749155in}}%
\pgfpathlineto{\pgfqpoint{4.614323in}{0.884156in}}%
\pgfpathlineto{\pgfqpoint{4.706345in}{0.839661in}}%
\pgfpathlineto{\pgfqpoint{4.798367in}{0.797738in}}%
\pgfpathlineto{\pgfqpoint{4.890390in}{2.264068in}}%
\pgfpathlineto{\pgfqpoint{4.982412in}{0.832307in}}%
\pgfpathlineto{\pgfqpoint{5.074434in}{0.734264in}}%
\pgfpathlineto{\pgfqpoint{5.166456in}{0.708059in}}%
\pgfpathlineto{\pgfqpoint{5.258479in}{0.696000in}}%
\pgfpathlineto{\pgfqpoint{5.350501in}{0.720159in}}%
\pgfpathlineto{\pgfqpoint{5.442523in}{1.230588in}}%
\pgfpathlineto{\pgfqpoint{5.534545in}{0.752070in}}%
\pgfusepath{stroke}%
\end{pgfscope}%
\begin{pgfscope}%
\pgfpathrectangle{\pgfqpoint{0.800000in}{0.528000in}}{\pgfqpoint{4.960000in}{3.696000in}}%
\pgfusepath{clip}%
\pgfsetrectcap%
\pgfsetroundjoin%
\pgfsetlinewidth{1.505625pt}%
\definecolor{currentstroke}{rgb}{1.000000,0.498039,0.054902}%
\pgfsetstrokecolor{currentstroke}%
\pgfsetdash{}{0pt}%
\pgfpathmoveto{\pgfqpoint{1.025455in}{3.610011in}}%
\pgfpathlineto{\pgfqpoint{1.117477in}{4.056000in}}%
\pgfpathlineto{\pgfqpoint{1.209499in}{3.791695in}}%
\pgfpathlineto{\pgfqpoint{1.301521in}{2.677161in}}%
\pgfpathlineto{\pgfqpoint{1.393544in}{2.914180in}}%
\pgfpathlineto{\pgfqpoint{1.485566in}{2.680927in}}%
\pgfpathlineto{\pgfqpoint{1.577588in}{2.741678in}}%
\pgfpathlineto{\pgfqpoint{1.669610in}{2.778923in}}%
\pgfpathlineto{\pgfqpoint{1.761633in}{2.777323in}}%
\pgfpathlineto{\pgfqpoint{1.853655in}{2.978918in}}%
\pgfpathlineto{\pgfqpoint{1.945677in}{2.796339in}}%
\pgfpathlineto{\pgfqpoint{2.037699in}{2.845445in}}%
\pgfpathlineto{\pgfqpoint{2.129722in}{2.828284in}}%
\pgfpathlineto{\pgfqpoint{2.221744in}{2.818410in}}%
\pgfpathlineto{\pgfqpoint{2.313766in}{3.025168in}}%
\pgfpathlineto{\pgfqpoint{2.405788in}{3.200111in}}%
\pgfpathlineto{\pgfqpoint{2.497811in}{2.861368in}}%
\pgfpathlineto{\pgfqpoint{2.589833in}{3.000731in}}%
\pgfpathlineto{\pgfqpoint{2.681855in}{2.938614in}}%
\pgfpathlineto{\pgfqpoint{2.773878in}{3.005043in}}%
\pgfpathlineto{\pgfqpoint{2.865900in}{2.907710in}}%
\pgfpathlineto{\pgfqpoint{2.957922in}{2.849404in}}%
\pgfpathlineto{\pgfqpoint{3.049944in}{2.921622in}}%
\pgfpathlineto{\pgfqpoint{3.141967in}{2.911304in}}%
\pgfpathlineto{\pgfqpoint{3.233989in}{2.833881in}}%
\pgfpathlineto{\pgfqpoint{3.326011in}{2.827427in}}%
\pgfpathlineto{\pgfqpoint{3.418033in}{3.017271in}}%
\pgfpathlineto{\pgfqpoint{3.510056in}{2.907281in}}%
\pgfpathlineto{\pgfqpoint{3.602078in}{2.867407in}}%
\pgfpathlineto{\pgfqpoint{3.694100in}{2.905509in}}%
\pgfpathlineto{\pgfqpoint{3.786122in}{2.859753in}}%
\pgfpathlineto{\pgfqpoint{3.878145in}{2.852721in}}%
\pgfpathlineto{\pgfqpoint{3.970167in}{2.886844in}}%
\pgfpathlineto{\pgfqpoint{4.062189in}{3.067301in}}%
\pgfpathlineto{\pgfqpoint{4.154212in}{2.907835in}}%
\pgfpathlineto{\pgfqpoint{4.246234in}{2.881300in}}%
\pgfpathlineto{\pgfqpoint{4.338256in}{3.046739in}}%
\pgfpathlineto{\pgfqpoint{4.430278in}{2.752573in}}%
\pgfpathlineto{\pgfqpoint{4.522301in}{2.771622in}}%
\pgfpathlineto{\pgfqpoint{4.614323in}{2.821879in}}%
\pgfpathlineto{\pgfqpoint{4.706345in}{2.771266in}}%
\pgfpathlineto{\pgfqpoint{4.798367in}{2.849081in}}%
\pgfpathlineto{\pgfqpoint{4.890390in}{2.824132in}}%
\pgfpathlineto{\pgfqpoint{4.982412in}{2.804736in}}%
\pgfpathlineto{\pgfqpoint{5.074434in}{2.822710in}}%
\pgfpathlineto{\pgfqpoint{5.166456in}{2.800458in}}%
\pgfpathlineto{\pgfqpoint{5.258479in}{3.069247in}}%
\pgfpathlineto{\pgfqpoint{5.350501in}{2.831382in}}%
\pgfpathlineto{\pgfqpoint{5.442523in}{2.960456in}}%
\pgfpathlineto{\pgfqpoint{5.534545in}{2.820238in}}%
\pgfusepath{stroke}%
\end{pgfscope}%
\begin{pgfscope}%
\pgfsetrectcap%
\pgfsetmiterjoin%
\pgfsetlinewidth{0.803000pt}%
\definecolor{currentstroke}{rgb}{0.000000,0.000000,0.000000}%
\pgfsetstrokecolor{currentstroke}%
\pgfsetdash{}{0pt}%
\pgfpathmoveto{\pgfqpoint{0.800000in}{0.528000in}}%
\pgfpathlineto{\pgfqpoint{0.800000in}{4.224000in}}%
\pgfusepath{stroke}%
\end{pgfscope}%
\begin{pgfscope}%
\pgfsetrectcap%
\pgfsetmiterjoin%
\pgfsetlinewidth{0.803000pt}%
\definecolor{currentstroke}{rgb}{0.000000,0.000000,0.000000}%
\pgfsetstrokecolor{currentstroke}%
\pgfsetdash{}{0pt}%
\pgfpathmoveto{\pgfqpoint{5.760000in}{0.528000in}}%
\pgfpathlineto{\pgfqpoint{5.760000in}{4.224000in}}%
\pgfusepath{stroke}%
\end{pgfscope}%
\begin{pgfscope}%
\pgfsetrectcap%
\pgfsetmiterjoin%
\pgfsetlinewidth{0.803000pt}%
\definecolor{currentstroke}{rgb}{0.000000,0.000000,0.000000}%
\pgfsetstrokecolor{currentstroke}%
\pgfsetdash{}{0pt}%
\pgfpathmoveto{\pgfqpoint{0.800000in}{0.528000in}}%
\pgfpathlineto{\pgfqpoint{5.760000in}{0.528000in}}%
\pgfusepath{stroke}%
\end{pgfscope}%
\begin{pgfscope}%
\pgfsetrectcap%
\pgfsetmiterjoin%
\pgfsetlinewidth{0.803000pt}%
\definecolor{currentstroke}{rgb}{0.000000,0.000000,0.000000}%
\pgfsetstrokecolor{currentstroke}%
\pgfsetdash{}{0pt}%
\pgfpathmoveto{\pgfqpoint{0.800000in}{4.224000in}}%
\pgfpathlineto{\pgfqpoint{5.760000in}{4.224000in}}%
\pgfusepath{stroke}%
\end{pgfscope}%
\begin{pgfscope}%
\definecolor{textcolor}{rgb}{0.000000,0.000000,0.000000}%
\pgfsetstrokecolor{textcolor}%
\pgfsetfillcolor{textcolor}%
\pgftext[x=3.280000in,y=4.307333in,,base]{\color{textcolor}{\sffamily\fontsize{9.600000}{11.520000}\selectfont\catcode`\^=\active\def^{\ifmmode\sp\else\^{}\fi}\catcode`\%=\active\def%{\%}Função de Perda em Treino (compII)}}%
\end{pgfscope}%
\begin{pgfscope}%
\pgfsetbuttcap%
\pgfsetmiterjoin%
\definecolor{currentfill}{rgb}{1.000000,1.000000,1.000000}%
\pgfsetfillcolor{currentfill}%
\pgfsetfillopacity{0.800000}%
\pgfsetlinewidth{1.003750pt}%
\definecolor{currentstroke}{rgb}{0.800000,0.800000,0.800000}%
\pgfsetstrokecolor{currentstroke}%
\pgfsetstrokeopacity{0.800000}%
\pgfsetdash{}{0pt}%
\pgfpathmoveto{\pgfqpoint{4.635586in}{3.808723in}}%
\pgfpathlineto{\pgfqpoint{5.682222in}{3.808723in}}%
\pgfpathquadraticcurveto{\pgfqpoint{5.704444in}{3.808723in}}{\pgfqpoint{5.704444in}{3.830945in}}%
\pgfpathlineto{\pgfqpoint{5.704444in}{4.146222in}}%
\pgfpathquadraticcurveto{\pgfqpoint{5.704444in}{4.168444in}}{\pgfqpoint{5.682222in}{4.168444in}}%
\pgfpathlineto{\pgfqpoint{4.635586in}{4.168444in}}%
\pgfpathquadraticcurveto{\pgfqpoint{4.613364in}{4.168444in}}{\pgfqpoint{4.613364in}{4.146222in}}%
\pgfpathlineto{\pgfqpoint{4.613364in}{3.830945in}}%
\pgfpathquadraticcurveto{\pgfqpoint{4.613364in}{3.808723in}}{\pgfqpoint{4.635586in}{3.808723in}}%
\pgfpathlineto{\pgfqpoint{4.635586in}{3.808723in}}%
\pgfpathclose%
\pgfusepath{stroke,fill}%
\end{pgfscope}%
\begin{pgfscope}%
\pgfsetrectcap%
\pgfsetroundjoin%
\pgfsetlinewidth{1.505625pt}%
\definecolor{currentstroke}{rgb}{0.121569,0.466667,0.705882}%
\pgfsetstrokecolor{currentstroke}%
\pgfsetdash{}{0pt}%
\pgfpathmoveto{\pgfqpoint{4.657808in}{4.078470in}}%
\pgfpathlineto{\pgfqpoint{4.768919in}{4.078470in}}%
\pgfpathlineto{\pgfqpoint{4.880030in}{4.078470in}}%
\pgfusepath{stroke}%
\end{pgfscope}%
\begin{pgfscope}%
\definecolor{textcolor}{rgb}{0.000000,0.000000,0.000000}%
\pgfsetstrokecolor{textcolor}%
\pgfsetfillcolor{textcolor}%
\pgftext[x=4.968919in,y=4.039582in,left,base]{\color{textcolor}{\sffamily\fontsize{8.000000}{9.600000}\selectfont\catcode`\^=\active\def^{\ifmmode\sp\else\^{}\fi}\catcode`\%=\active\def%{\%}Treinamento}}%
\end{pgfscope}%
\begin{pgfscope}%
\pgfsetrectcap%
\pgfsetroundjoin%
\pgfsetlinewidth{1.505625pt}%
\definecolor{currentstroke}{rgb}{1.000000,0.498039,0.054902}%
\pgfsetstrokecolor{currentstroke}%
\pgfsetdash{}{0pt}%
\pgfpathmoveto{\pgfqpoint{4.657808in}{3.915168in}}%
\pgfpathlineto{\pgfqpoint{4.768919in}{3.915168in}}%
\pgfpathlineto{\pgfqpoint{4.880030in}{3.915168in}}%
\pgfusepath{stroke}%
\end{pgfscope}%
\begin{pgfscope}%
\definecolor{textcolor}{rgb}{0.000000,0.000000,0.000000}%
\pgfsetstrokecolor{textcolor}%
\pgfsetfillcolor{textcolor}%
\pgftext[x=4.968919in,y=3.876279in,left,base]{\color{textcolor}{\sffamily\fontsize{8.000000}{9.600000}\selectfont\catcode`\^=\active\def^{\ifmmode\sp\else\^{}\fi}\catcode`\%=\active\def%{\%}Validação}}%
\end{pgfscope}%
\end{pgfpicture}%
\makeatother%
\endgroup%
}
    \end{minipage}

    \caption{Evolução do treinamento com \textit{hypertuning} (à esquerda) e do treinamento com hiperparâmetros fixados (à direita) do sistema especialista da competência II.}
    \label{fig:exp-fix-c2}
\end{figure}

Pela figura \ref{fig:exp-fix-c2}, é possível notar que o treinamento com hiperparâmetros fixados também apresentou uma evolução mais suave da função de perda, de modo que, em torno da 10ª época, o valor para a base de validação estagnou em cerca de 0,8. Além disso, a partir da 20ª época, a perda para a base de treino convergiu para aproximadamente 0,08, com alguns picos próximos das 40 iterações. No caso do \textit{hypertuning}, a função também registrou números mais altos para a base de treino e de validação. Na primeira situação, o valor estagnou em 1,5 ainda no início, enquanto que, na segunda, a perda oscilou entre 1 e 3 ao longo de todo o processo. Nota-se, assim, que o processo de otimização não foi efetivo para a competência II.


\subsubsection{Competência III}
\label{subsec:exp-fix-c3}

\begin{figure}[H]
    \begin{minipage}{0.45\textwidth}
        \resizebox{\textwidth}{!}{%% Creator: Matplotlib, PGF backend
%%
%% To include the figure in your LaTeX document, write
%%   \input{<filename>.pgf}
%%
%% Make sure the required packages are loaded in your preamble
%%   \usepackage{pgf}
%%
%% Also ensure that all the required font packages are loaded; for instance,
%% the lmodern package is sometimes necessary when using math font.
%%   \usepackage{lmodern}
%%
%% Figures using additional raster images can only be included by \input if
%% they are in the same directory as the main LaTeX file. For loading figures
%% from other directories you can use the `import` package
%%   \usepackage{import}
%%
%% and then include the figures with
%%   \import{<path to file>}{<filename>.pgf}
%%
%% Matplotlib used the following preamble
%%   \def\mathdefault#1{#1}
%%   \everymath=\expandafter{\the\everymath\displaystyle}
%%   
%%   \usepackage{fontspec}
%%   \setmainfont{DejaVuSerif.ttf}[Path=\detokenize{/home/josemayer/.local/lib/python3.9/site-packages/matplotlib/mpl-data/fonts/ttf/}]
%%   \setsansfont{DejaVuSans.ttf}[Path=\detokenize{/home/josemayer/.local/lib/python3.9/site-packages/matplotlib/mpl-data/fonts/ttf/}]
%%   \setmonofont{DejaVuSansMono.ttf}[Path=\detokenize{/home/josemayer/.local/lib/python3.9/site-packages/matplotlib/mpl-data/fonts/ttf/}]
%%   \makeatletter\@ifpackageloaded{underscore}{}{\usepackage[strings]{underscore}}\makeatother
%%
\begingroup%
\makeatletter%
\begin{pgfpicture}%
\pgfpathrectangle{\pgfpointorigin}{\pgfqpoint{6.400000in}{4.800000in}}%
\pgfusepath{use as bounding box, clip}%
\begin{pgfscope}%
\pgfsetbuttcap%
\pgfsetmiterjoin%
\definecolor{currentfill}{rgb}{1.000000,1.000000,1.000000}%
\pgfsetfillcolor{currentfill}%
\pgfsetlinewidth{0.000000pt}%
\definecolor{currentstroke}{rgb}{1.000000,1.000000,1.000000}%
\pgfsetstrokecolor{currentstroke}%
\pgfsetdash{}{0pt}%
\pgfpathmoveto{\pgfqpoint{0.000000in}{0.000000in}}%
\pgfpathlineto{\pgfqpoint{6.400000in}{0.000000in}}%
\pgfpathlineto{\pgfqpoint{6.400000in}{4.800000in}}%
\pgfpathlineto{\pgfqpoint{0.000000in}{4.800000in}}%
\pgfpathlineto{\pgfqpoint{0.000000in}{0.000000in}}%
\pgfpathclose%
\pgfusepath{fill}%
\end{pgfscope}%
\begin{pgfscope}%
\pgfsetbuttcap%
\pgfsetmiterjoin%
\definecolor{currentfill}{rgb}{1.000000,1.000000,1.000000}%
\pgfsetfillcolor{currentfill}%
\pgfsetlinewidth{0.000000pt}%
\definecolor{currentstroke}{rgb}{0.000000,0.000000,0.000000}%
\pgfsetstrokecolor{currentstroke}%
\pgfsetstrokeopacity{0.000000}%
\pgfsetdash{}{0pt}%
\pgfpathmoveto{\pgfqpoint{0.800000in}{0.528000in}}%
\pgfpathlineto{\pgfqpoint{5.760000in}{0.528000in}}%
\pgfpathlineto{\pgfqpoint{5.760000in}{4.224000in}}%
\pgfpathlineto{\pgfqpoint{0.800000in}{4.224000in}}%
\pgfpathlineto{\pgfqpoint{0.800000in}{0.528000in}}%
\pgfpathclose%
\pgfusepath{fill}%
\end{pgfscope}%
\begin{pgfscope}%
\pgfsetbuttcap%
\pgfsetroundjoin%
\definecolor{currentfill}{rgb}{0.000000,0.000000,0.000000}%
\pgfsetfillcolor{currentfill}%
\pgfsetlinewidth{0.803000pt}%
\definecolor{currentstroke}{rgb}{0.000000,0.000000,0.000000}%
\pgfsetstrokecolor{currentstroke}%
\pgfsetdash{}{0pt}%
\pgfsys@defobject{currentmarker}{\pgfqpoint{0.000000in}{-0.048611in}}{\pgfqpoint{0.000000in}{0.000000in}}{%
\pgfpathmoveto{\pgfqpoint{0.000000in}{0.000000in}}%
\pgfpathlineto{\pgfqpoint{0.000000in}{-0.048611in}}%
\pgfusepath{stroke,fill}%
}%
\begin{pgfscope}%
\pgfsys@transformshift{1.025455in}{0.528000in}%
\pgfsys@useobject{currentmarker}{}%
\end{pgfscope}%
\end{pgfscope}%
\begin{pgfscope}%
\definecolor{textcolor}{rgb}{0.000000,0.000000,0.000000}%
\pgfsetstrokecolor{textcolor}%
\pgfsetfillcolor{textcolor}%
\pgftext[x=1.025455in,y=0.430778in,,top]{\color{textcolor}{\sffamily\fontsize{8.000000}{9.600000}\selectfont\catcode`\^=\active\def^{\ifmmode\sp\else\^{}\fi}\catcode`\%=\active\def%{\%}0}}%
\end{pgfscope}%
\begin{pgfscope}%
\pgfsetbuttcap%
\pgfsetroundjoin%
\definecolor{currentfill}{rgb}{0.000000,0.000000,0.000000}%
\pgfsetfillcolor{currentfill}%
\pgfsetlinewidth{0.803000pt}%
\definecolor{currentstroke}{rgb}{0.000000,0.000000,0.000000}%
\pgfsetstrokecolor{currentstroke}%
\pgfsetdash{}{0pt}%
\pgfsys@defobject{currentmarker}{\pgfqpoint{0.000000in}{-0.048611in}}{\pgfqpoint{0.000000in}{0.000000in}}{%
\pgfpathmoveto{\pgfqpoint{0.000000in}{0.000000in}}%
\pgfpathlineto{\pgfqpoint{0.000000in}{-0.048611in}}%
\pgfusepath{stroke,fill}%
}%
\begin{pgfscope}%
\pgfsys@transformshift{1.945677in}{0.528000in}%
\pgfsys@useobject{currentmarker}{}%
\end{pgfscope}%
\end{pgfscope}%
\begin{pgfscope}%
\definecolor{textcolor}{rgb}{0.000000,0.000000,0.000000}%
\pgfsetstrokecolor{textcolor}%
\pgfsetfillcolor{textcolor}%
\pgftext[x=1.945677in,y=0.430778in,,top]{\color{textcolor}{\sffamily\fontsize{8.000000}{9.600000}\selectfont\catcode`\^=\active\def^{\ifmmode\sp\else\^{}\fi}\catcode`\%=\active\def%{\%}10}}%
\end{pgfscope}%
\begin{pgfscope}%
\pgfsetbuttcap%
\pgfsetroundjoin%
\definecolor{currentfill}{rgb}{0.000000,0.000000,0.000000}%
\pgfsetfillcolor{currentfill}%
\pgfsetlinewidth{0.803000pt}%
\definecolor{currentstroke}{rgb}{0.000000,0.000000,0.000000}%
\pgfsetstrokecolor{currentstroke}%
\pgfsetdash{}{0pt}%
\pgfsys@defobject{currentmarker}{\pgfqpoint{0.000000in}{-0.048611in}}{\pgfqpoint{0.000000in}{0.000000in}}{%
\pgfpathmoveto{\pgfqpoint{0.000000in}{0.000000in}}%
\pgfpathlineto{\pgfqpoint{0.000000in}{-0.048611in}}%
\pgfusepath{stroke,fill}%
}%
\begin{pgfscope}%
\pgfsys@transformshift{2.865900in}{0.528000in}%
\pgfsys@useobject{currentmarker}{}%
\end{pgfscope}%
\end{pgfscope}%
\begin{pgfscope}%
\definecolor{textcolor}{rgb}{0.000000,0.000000,0.000000}%
\pgfsetstrokecolor{textcolor}%
\pgfsetfillcolor{textcolor}%
\pgftext[x=2.865900in,y=0.430778in,,top]{\color{textcolor}{\sffamily\fontsize{8.000000}{9.600000}\selectfont\catcode`\^=\active\def^{\ifmmode\sp\else\^{}\fi}\catcode`\%=\active\def%{\%}20}}%
\end{pgfscope}%
\begin{pgfscope}%
\pgfsetbuttcap%
\pgfsetroundjoin%
\definecolor{currentfill}{rgb}{0.000000,0.000000,0.000000}%
\pgfsetfillcolor{currentfill}%
\pgfsetlinewidth{0.803000pt}%
\definecolor{currentstroke}{rgb}{0.000000,0.000000,0.000000}%
\pgfsetstrokecolor{currentstroke}%
\pgfsetdash{}{0pt}%
\pgfsys@defobject{currentmarker}{\pgfqpoint{0.000000in}{-0.048611in}}{\pgfqpoint{0.000000in}{0.000000in}}{%
\pgfpathmoveto{\pgfqpoint{0.000000in}{0.000000in}}%
\pgfpathlineto{\pgfqpoint{0.000000in}{-0.048611in}}%
\pgfusepath{stroke,fill}%
}%
\begin{pgfscope}%
\pgfsys@transformshift{3.786122in}{0.528000in}%
\pgfsys@useobject{currentmarker}{}%
\end{pgfscope}%
\end{pgfscope}%
\begin{pgfscope}%
\definecolor{textcolor}{rgb}{0.000000,0.000000,0.000000}%
\pgfsetstrokecolor{textcolor}%
\pgfsetfillcolor{textcolor}%
\pgftext[x=3.786122in,y=0.430778in,,top]{\color{textcolor}{\sffamily\fontsize{8.000000}{9.600000}\selectfont\catcode`\^=\active\def^{\ifmmode\sp\else\^{}\fi}\catcode`\%=\active\def%{\%}30}}%
\end{pgfscope}%
\begin{pgfscope}%
\pgfsetbuttcap%
\pgfsetroundjoin%
\definecolor{currentfill}{rgb}{0.000000,0.000000,0.000000}%
\pgfsetfillcolor{currentfill}%
\pgfsetlinewidth{0.803000pt}%
\definecolor{currentstroke}{rgb}{0.000000,0.000000,0.000000}%
\pgfsetstrokecolor{currentstroke}%
\pgfsetdash{}{0pt}%
\pgfsys@defobject{currentmarker}{\pgfqpoint{0.000000in}{-0.048611in}}{\pgfqpoint{0.000000in}{0.000000in}}{%
\pgfpathmoveto{\pgfqpoint{0.000000in}{0.000000in}}%
\pgfpathlineto{\pgfqpoint{0.000000in}{-0.048611in}}%
\pgfusepath{stroke,fill}%
}%
\begin{pgfscope}%
\pgfsys@transformshift{4.706345in}{0.528000in}%
\pgfsys@useobject{currentmarker}{}%
\end{pgfscope}%
\end{pgfscope}%
\begin{pgfscope}%
\definecolor{textcolor}{rgb}{0.000000,0.000000,0.000000}%
\pgfsetstrokecolor{textcolor}%
\pgfsetfillcolor{textcolor}%
\pgftext[x=4.706345in,y=0.430778in,,top]{\color{textcolor}{\sffamily\fontsize{8.000000}{9.600000}\selectfont\catcode`\^=\active\def^{\ifmmode\sp\else\^{}\fi}\catcode`\%=\active\def%{\%}40}}%
\end{pgfscope}%
\begin{pgfscope}%
\pgfsetbuttcap%
\pgfsetroundjoin%
\definecolor{currentfill}{rgb}{0.000000,0.000000,0.000000}%
\pgfsetfillcolor{currentfill}%
\pgfsetlinewidth{0.803000pt}%
\definecolor{currentstroke}{rgb}{0.000000,0.000000,0.000000}%
\pgfsetstrokecolor{currentstroke}%
\pgfsetdash{}{0pt}%
\pgfsys@defobject{currentmarker}{\pgfqpoint{0.000000in}{-0.048611in}}{\pgfqpoint{0.000000in}{0.000000in}}{%
\pgfpathmoveto{\pgfqpoint{0.000000in}{0.000000in}}%
\pgfpathlineto{\pgfqpoint{0.000000in}{-0.048611in}}%
\pgfusepath{stroke,fill}%
}%
\begin{pgfscope}%
\pgfsys@transformshift{5.626568in}{0.528000in}%
\pgfsys@useobject{currentmarker}{}%
\end{pgfscope}%
\end{pgfscope}%
\begin{pgfscope}%
\definecolor{textcolor}{rgb}{0.000000,0.000000,0.000000}%
\pgfsetstrokecolor{textcolor}%
\pgfsetfillcolor{textcolor}%
\pgftext[x=5.626568in,y=0.430778in,,top]{\color{textcolor}{\sffamily\fontsize{8.000000}{9.600000}\selectfont\catcode`\^=\active\def^{\ifmmode\sp\else\^{}\fi}\catcode`\%=\active\def%{\%}50}}%
\end{pgfscope}%
\begin{pgfscope}%
\definecolor{textcolor}{rgb}{0.000000,0.000000,0.000000}%
\pgfsetstrokecolor{textcolor}%
\pgfsetfillcolor{textcolor}%
\pgftext[x=3.280000in,y=0.267692in,,top]{\color{textcolor}{\sffamily\fontsize{8.000000}{9.600000}\selectfont\catcode`\^=\active\def^{\ifmmode\sp\else\^{}\fi}\catcode`\%=\active\def%{\%}Época}}%
\end{pgfscope}%
\begin{pgfscope}%
\pgfsetbuttcap%
\pgfsetroundjoin%
\definecolor{currentfill}{rgb}{0.000000,0.000000,0.000000}%
\pgfsetfillcolor{currentfill}%
\pgfsetlinewidth{0.803000pt}%
\definecolor{currentstroke}{rgb}{0.000000,0.000000,0.000000}%
\pgfsetstrokecolor{currentstroke}%
\pgfsetdash{}{0pt}%
\pgfsys@defobject{currentmarker}{\pgfqpoint{-0.048611in}{0.000000in}}{\pgfqpoint{-0.000000in}{0.000000in}}{%
\pgfpathmoveto{\pgfqpoint{-0.000000in}{0.000000in}}%
\pgfpathlineto{\pgfqpoint{-0.048611in}{0.000000in}}%
\pgfusepath{stroke,fill}%
}%
\begin{pgfscope}%
\pgfsys@transformshift{0.800000in}{0.858983in}%
\pgfsys@useobject{currentmarker}{}%
\end{pgfscope}%
\end{pgfscope}%
\begin{pgfscope}%
\definecolor{textcolor}{rgb}{0.000000,0.000000,0.000000}%
\pgfsetstrokecolor{textcolor}%
\pgfsetfillcolor{textcolor}%
\pgftext[x=0.455382in, y=0.816774in, left, base]{\color{textcolor}{\sffamily\fontsize{8.000000}{9.600000}\selectfont\catcode`\^=\active\def^{\ifmmode\sp\else\^{}\fi}\catcode`\%=\active\def%{\%}0.94}}%
\end{pgfscope}%
\begin{pgfscope}%
\pgfsetbuttcap%
\pgfsetroundjoin%
\definecolor{currentfill}{rgb}{0.000000,0.000000,0.000000}%
\pgfsetfillcolor{currentfill}%
\pgfsetlinewidth{0.803000pt}%
\definecolor{currentstroke}{rgb}{0.000000,0.000000,0.000000}%
\pgfsetstrokecolor{currentstroke}%
\pgfsetdash{}{0pt}%
\pgfsys@defobject{currentmarker}{\pgfqpoint{-0.048611in}{0.000000in}}{\pgfqpoint{-0.000000in}{0.000000in}}{%
\pgfpathmoveto{\pgfqpoint{-0.000000in}{0.000000in}}%
\pgfpathlineto{\pgfqpoint{-0.048611in}{0.000000in}}%
\pgfusepath{stroke,fill}%
}%
\begin{pgfscope}%
\pgfsys@transformshift{0.800000in}{1.425135in}%
\pgfsys@useobject{currentmarker}{}%
\end{pgfscope}%
\end{pgfscope}%
\begin{pgfscope}%
\definecolor{textcolor}{rgb}{0.000000,0.000000,0.000000}%
\pgfsetstrokecolor{textcolor}%
\pgfsetfillcolor{textcolor}%
\pgftext[x=0.455382in, y=1.382926in, left, base]{\color{textcolor}{\sffamily\fontsize{8.000000}{9.600000}\selectfont\catcode`\^=\active\def^{\ifmmode\sp\else\^{}\fi}\catcode`\%=\active\def%{\%}0.96}}%
\end{pgfscope}%
\begin{pgfscope}%
\pgfsetbuttcap%
\pgfsetroundjoin%
\definecolor{currentfill}{rgb}{0.000000,0.000000,0.000000}%
\pgfsetfillcolor{currentfill}%
\pgfsetlinewidth{0.803000pt}%
\definecolor{currentstroke}{rgb}{0.000000,0.000000,0.000000}%
\pgfsetstrokecolor{currentstroke}%
\pgfsetdash{}{0pt}%
\pgfsys@defobject{currentmarker}{\pgfqpoint{-0.048611in}{0.000000in}}{\pgfqpoint{-0.000000in}{0.000000in}}{%
\pgfpathmoveto{\pgfqpoint{-0.000000in}{0.000000in}}%
\pgfpathlineto{\pgfqpoint{-0.048611in}{0.000000in}}%
\pgfusepath{stroke,fill}%
}%
\begin{pgfscope}%
\pgfsys@transformshift{0.800000in}{1.991287in}%
\pgfsys@useobject{currentmarker}{}%
\end{pgfscope}%
\end{pgfscope}%
\begin{pgfscope}%
\definecolor{textcolor}{rgb}{0.000000,0.000000,0.000000}%
\pgfsetstrokecolor{textcolor}%
\pgfsetfillcolor{textcolor}%
\pgftext[x=0.455382in, y=1.949078in, left, base]{\color{textcolor}{\sffamily\fontsize{8.000000}{9.600000}\selectfont\catcode`\^=\active\def^{\ifmmode\sp\else\^{}\fi}\catcode`\%=\active\def%{\%}0.98}}%
\end{pgfscope}%
\begin{pgfscope}%
\pgfsetbuttcap%
\pgfsetroundjoin%
\definecolor{currentfill}{rgb}{0.000000,0.000000,0.000000}%
\pgfsetfillcolor{currentfill}%
\pgfsetlinewidth{0.803000pt}%
\definecolor{currentstroke}{rgb}{0.000000,0.000000,0.000000}%
\pgfsetstrokecolor{currentstroke}%
\pgfsetdash{}{0pt}%
\pgfsys@defobject{currentmarker}{\pgfqpoint{-0.048611in}{0.000000in}}{\pgfqpoint{-0.000000in}{0.000000in}}{%
\pgfpathmoveto{\pgfqpoint{-0.000000in}{0.000000in}}%
\pgfpathlineto{\pgfqpoint{-0.048611in}{0.000000in}}%
\pgfusepath{stroke,fill}%
}%
\begin{pgfscope}%
\pgfsys@transformshift{0.800000in}{2.557439in}%
\pgfsys@useobject{currentmarker}{}%
\end{pgfscope}%
\end{pgfscope}%
\begin{pgfscope}%
\definecolor{textcolor}{rgb}{0.000000,0.000000,0.000000}%
\pgfsetstrokecolor{textcolor}%
\pgfsetfillcolor{textcolor}%
\pgftext[x=0.455382in, y=2.515230in, left, base]{\color{textcolor}{\sffamily\fontsize{8.000000}{9.600000}\selectfont\catcode`\^=\active\def^{\ifmmode\sp\else\^{}\fi}\catcode`\%=\active\def%{\%}1.00}}%
\end{pgfscope}%
\begin{pgfscope}%
\pgfsetbuttcap%
\pgfsetroundjoin%
\definecolor{currentfill}{rgb}{0.000000,0.000000,0.000000}%
\pgfsetfillcolor{currentfill}%
\pgfsetlinewidth{0.803000pt}%
\definecolor{currentstroke}{rgb}{0.000000,0.000000,0.000000}%
\pgfsetstrokecolor{currentstroke}%
\pgfsetdash{}{0pt}%
\pgfsys@defobject{currentmarker}{\pgfqpoint{-0.048611in}{0.000000in}}{\pgfqpoint{-0.000000in}{0.000000in}}{%
\pgfpathmoveto{\pgfqpoint{-0.000000in}{0.000000in}}%
\pgfpathlineto{\pgfqpoint{-0.048611in}{0.000000in}}%
\pgfusepath{stroke,fill}%
}%
\begin{pgfscope}%
\pgfsys@transformshift{0.800000in}{3.123591in}%
\pgfsys@useobject{currentmarker}{}%
\end{pgfscope}%
\end{pgfscope}%
\begin{pgfscope}%
\definecolor{textcolor}{rgb}{0.000000,0.000000,0.000000}%
\pgfsetstrokecolor{textcolor}%
\pgfsetfillcolor{textcolor}%
\pgftext[x=0.455382in, y=3.081382in, left, base]{\color{textcolor}{\sffamily\fontsize{8.000000}{9.600000}\selectfont\catcode`\^=\active\def^{\ifmmode\sp\else\^{}\fi}\catcode`\%=\active\def%{\%}1.02}}%
\end{pgfscope}%
\begin{pgfscope}%
\pgfsetbuttcap%
\pgfsetroundjoin%
\definecolor{currentfill}{rgb}{0.000000,0.000000,0.000000}%
\pgfsetfillcolor{currentfill}%
\pgfsetlinewidth{0.803000pt}%
\definecolor{currentstroke}{rgb}{0.000000,0.000000,0.000000}%
\pgfsetstrokecolor{currentstroke}%
\pgfsetdash{}{0pt}%
\pgfsys@defobject{currentmarker}{\pgfqpoint{-0.048611in}{0.000000in}}{\pgfqpoint{-0.000000in}{0.000000in}}{%
\pgfpathmoveto{\pgfqpoint{-0.000000in}{0.000000in}}%
\pgfpathlineto{\pgfqpoint{-0.048611in}{0.000000in}}%
\pgfusepath{stroke,fill}%
}%
\begin{pgfscope}%
\pgfsys@transformshift{0.800000in}{3.689743in}%
\pgfsys@useobject{currentmarker}{}%
\end{pgfscope}%
\end{pgfscope}%
\begin{pgfscope}%
\definecolor{textcolor}{rgb}{0.000000,0.000000,0.000000}%
\pgfsetstrokecolor{textcolor}%
\pgfsetfillcolor{textcolor}%
\pgftext[x=0.455382in, y=3.647534in, left, base]{\color{textcolor}{\sffamily\fontsize{8.000000}{9.600000}\selectfont\catcode`\^=\active\def^{\ifmmode\sp\else\^{}\fi}\catcode`\%=\active\def%{\%}1.04}}%
\end{pgfscope}%
\begin{pgfscope}%
\definecolor{textcolor}{rgb}{0.000000,0.000000,0.000000}%
\pgfsetstrokecolor{textcolor}%
\pgfsetfillcolor{textcolor}%
\pgftext[x=0.399826in,y=2.376000in,,bottom,rotate=90.000000]{\color{textcolor}{\sffamily\fontsize{8.000000}{9.600000}\selectfont\catcode`\^=\active\def^{\ifmmode\sp\else\^{}\fi}\catcode`\%=\active\def%{\%}Perda}}%
\end{pgfscope}%
\begin{pgfscope}%
\pgfpathrectangle{\pgfqpoint{0.800000in}{0.528000in}}{\pgfqpoint{4.960000in}{3.696000in}}%
\pgfusepath{clip}%
\pgfsetrectcap%
\pgfsetroundjoin%
\pgfsetlinewidth{1.505625pt}%
\definecolor{currentstroke}{rgb}{0.121569,0.466667,0.705882}%
\pgfsetstrokecolor{currentstroke}%
\pgfsetdash{}{0pt}%
\pgfpathmoveto{\pgfqpoint{1.025455in}{3.375118in}}%
\pgfpathlineto{\pgfqpoint{1.117477in}{2.813019in}}%
\pgfpathlineto{\pgfqpoint{1.209499in}{2.366110in}}%
\pgfpathlineto{\pgfqpoint{1.301521in}{2.002376in}}%
\pgfpathlineto{\pgfqpoint{1.393544in}{1.949111in}}%
\pgfpathlineto{\pgfqpoint{1.485566in}{1.773035in}}%
\pgfpathlineto{\pgfqpoint{1.577588in}{2.303639in}}%
\pgfpathlineto{\pgfqpoint{1.669610in}{2.052286in}}%
\pgfpathlineto{\pgfqpoint{1.761633in}{1.923456in}}%
\pgfpathlineto{\pgfqpoint{1.853655in}{1.437713in}}%
\pgfpathlineto{\pgfqpoint{1.945677in}{1.568088in}}%
\pgfpathlineto{\pgfqpoint{2.037699in}{1.446902in}}%
\pgfpathlineto{\pgfqpoint{2.129722in}{1.489057in}}%
\pgfpathlineto{\pgfqpoint{2.221744in}{1.389580in}}%
\pgfpathlineto{\pgfqpoint{2.313766in}{1.851682in}}%
\pgfpathlineto{\pgfqpoint{2.405788in}{1.499755in}}%
\pgfpathlineto{\pgfqpoint{2.497811in}{1.360684in}}%
\pgfpathlineto{\pgfqpoint{2.589833in}{1.522159in}}%
\pgfpathlineto{\pgfqpoint{2.681855in}{1.565270in}}%
\pgfpathlineto{\pgfqpoint{2.773878in}{1.203350in}}%
\pgfpathlineto{\pgfqpoint{2.865900in}{1.138561in}}%
\pgfpathlineto{\pgfqpoint{2.957922in}{1.422057in}}%
\pgfpathlineto{\pgfqpoint{3.049944in}{1.407997in}}%
\pgfpathlineto{\pgfqpoint{3.141967in}{1.098524in}}%
\pgfpathlineto{\pgfqpoint{3.233989in}{1.197006in}}%
\pgfpathlineto{\pgfqpoint{3.326011in}{1.146960in}}%
\pgfpathlineto{\pgfqpoint{3.418033in}{1.296580in}}%
\pgfpathlineto{\pgfqpoint{3.510056in}{1.221833in}}%
\pgfpathlineto{\pgfqpoint{3.602078in}{1.181787in}}%
\pgfpathlineto{\pgfqpoint{3.694100in}{1.008404in}}%
\pgfpathlineto{\pgfqpoint{3.786122in}{0.940581in}}%
\pgfpathlineto{\pgfqpoint{3.878145in}{1.312882in}}%
\pgfpathlineto{\pgfqpoint{3.970167in}{0.984527in}}%
\pgfpathlineto{\pgfqpoint{4.062189in}{1.249489in}}%
\pgfpathlineto{\pgfqpoint{4.154212in}{1.049993in}}%
\pgfpathlineto{\pgfqpoint{4.246234in}{1.207266in}}%
\pgfpathlineto{\pgfqpoint{4.338256in}{0.696000in}}%
\pgfpathlineto{\pgfqpoint{4.430278in}{1.093293in}}%
\pgfpathlineto{\pgfqpoint{4.522301in}{0.995741in}}%
\pgfpathlineto{\pgfqpoint{4.614323in}{1.045645in}}%
\pgfpathlineto{\pgfqpoint{4.706345in}{0.774816in}}%
\pgfpathlineto{\pgfqpoint{4.798367in}{0.717705in}}%
\pgfpathlineto{\pgfqpoint{4.890390in}{0.984789in}}%
\pgfpathlineto{\pgfqpoint{4.982412in}{0.988828in}}%
\pgfpathlineto{\pgfqpoint{5.074434in}{0.805051in}}%
\pgfpathlineto{\pgfqpoint{5.166456in}{0.929035in}}%
\pgfpathlineto{\pgfqpoint{5.258479in}{0.871158in}}%
\pgfpathlineto{\pgfqpoint{5.350501in}{1.087086in}}%
\pgfpathlineto{\pgfqpoint{5.442523in}{0.880212in}}%
\pgfpathlineto{\pgfqpoint{5.534545in}{0.778013in}}%
\pgfusepath{stroke}%
\end{pgfscope}%
\begin{pgfscope}%
\pgfpathrectangle{\pgfqpoint{0.800000in}{0.528000in}}{\pgfqpoint{4.960000in}{3.696000in}}%
\pgfusepath{clip}%
\pgfsetrectcap%
\pgfsetroundjoin%
\pgfsetlinewidth{1.505625pt}%
\definecolor{currentstroke}{rgb}{1.000000,0.498039,0.054902}%
\pgfsetstrokecolor{currentstroke}%
\pgfsetdash{}{0pt}%
\pgfpathmoveto{\pgfqpoint{1.025455in}{1.800838in}}%
\pgfpathlineto{\pgfqpoint{1.117477in}{0.868302in}}%
\pgfpathlineto{\pgfqpoint{1.209499in}{3.232699in}}%
\pgfpathlineto{\pgfqpoint{1.301521in}{0.877288in}}%
\pgfpathlineto{\pgfqpoint{1.393544in}{2.997896in}}%
\pgfpathlineto{\pgfqpoint{1.485566in}{1.354826in}}%
\pgfpathlineto{\pgfqpoint{1.577588in}{0.894623in}}%
\pgfpathlineto{\pgfqpoint{1.669610in}{0.890663in}}%
\pgfpathlineto{\pgfqpoint{1.761633in}{1.108858in}}%
\pgfpathlineto{\pgfqpoint{1.853655in}{0.867300in}}%
\pgfpathlineto{\pgfqpoint{1.945677in}{1.294739in}}%
\pgfpathlineto{\pgfqpoint{2.037699in}{1.510456in}}%
\pgfpathlineto{\pgfqpoint{2.129722in}{0.941269in}}%
\pgfpathlineto{\pgfqpoint{2.221744in}{1.266601in}}%
\pgfpathlineto{\pgfqpoint{2.313766in}{1.560676in}}%
\pgfpathlineto{\pgfqpoint{2.405788in}{4.056000in}}%
\pgfpathlineto{\pgfqpoint{2.497811in}{1.815235in}}%
\pgfpathlineto{\pgfqpoint{2.589833in}{1.496686in}}%
\pgfpathlineto{\pgfqpoint{2.681855in}{0.928016in}}%
\pgfpathlineto{\pgfqpoint{2.773878in}{0.867644in}}%
\pgfpathlineto{\pgfqpoint{2.865900in}{1.473080in}}%
\pgfpathlineto{\pgfqpoint{2.957922in}{1.595977in}}%
\pgfpathlineto{\pgfqpoint{3.049944in}{1.348796in}}%
\pgfpathlineto{\pgfqpoint{3.141967in}{0.953824in}}%
\pgfpathlineto{\pgfqpoint{3.233989in}{0.971527in}}%
\pgfpathlineto{\pgfqpoint{3.326011in}{1.003509in}}%
\pgfpathlineto{\pgfqpoint{3.418033in}{1.321091in}}%
\pgfpathlineto{\pgfqpoint{3.510056in}{1.104008in}}%
\pgfpathlineto{\pgfqpoint{3.602078in}{0.886529in}}%
\pgfpathlineto{\pgfqpoint{3.694100in}{0.907310in}}%
\pgfpathlineto{\pgfqpoint{3.786122in}{1.310922in}}%
\pgfpathlineto{\pgfqpoint{3.878145in}{0.867085in}}%
\pgfpathlineto{\pgfqpoint{3.970167in}{0.927823in}}%
\pgfpathlineto{\pgfqpoint{4.062189in}{1.410923in}}%
\pgfpathlineto{\pgfqpoint{4.154212in}{0.868626in}}%
\pgfpathlineto{\pgfqpoint{4.246234in}{0.870100in}}%
\pgfpathlineto{\pgfqpoint{4.338256in}{1.332735in}}%
\pgfpathlineto{\pgfqpoint{4.430278in}{1.032268in}}%
\pgfpathlineto{\pgfqpoint{4.522301in}{0.876532in}}%
\pgfpathlineto{\pgfqpoint{4.614323in}{1.748068in}}%
\pgfpathlineto{\pgfqpoint{4.706345in}{0.873168in}}%
\pgfpathlineto{\pgfqpoint{4.798367in}{0.975278in}}%
\pgfpathlineto{\pgfqpoint{4.890390in}{0.873313in}}%
\pgfpathlineto{\pgfqpoint{4.982412in}{0.955943in}}%
\pgfpathlineto{\pgfqpoint{5.074434in}{1.276966in}}%
\pgfpathlineto{\pgfqpoint{5.166456in}{2.103813in}}%
\pgfpathlineto{\pgfqpoint{5.258479in}{1.173352in}}%
\pgfpathlineto{\pgfqpoint{5.350501in}{1.169053in}}%
\pgfpathlineto{\pgfqpoint{5.442523in}{2.019659in}}%
\pgfpathlineto{\pgfqpoint{5.534545in}{0.936798in}}%
\pgfusepath{stroke}%
\end{pgfscope}%
\begin{pgfscope}%
\pgfsetrectcap%
\pgfsetmiterjoin%
\pgfsetlinewidth{0.803000pt}%
\definecolor{currentstroke}{rgb}{0.000000,0.000000,0.000000}%
\pgfsetstrokecolor{currentstroke}%
\pgfsetdash{}{0pt}%
\pgfpathmoveto{\pgfqpoint{0.800000in}{0.528000in}}%
\pgfpathlineto{\pgfqpoint{0.800000in}{4.224000in}}%
\pgfusepath{stroke}%
\end{pgfscope}%
\begin{pgfscope}%
\pgfsetrectcap%
\pgfsetmiterjoin%
\pgfsetlinewidth{0.803000pt}%
\definecolor{currentstroke}{rgb}{0.000000,0.000000,0.000000}%
\pgfsetstrokecolor{currentstroke}%
\pgfsetdash{}{0pt}%
\pgfpathmoveto{\pgfqpoint{5.760000in}{0.528000in}}%
\pgfpathlineto{\pgfqpoint{5.760000in}{4.224000in}}%
\pgfusepath{stroke}%
\end{pgfscope}%
\begin{pgfscope}%
\pgfsetrectcap%
\pgfsetmiterjoin%
\pgfsetlinewidth{0.803000pt}%
\definecolor{currentstroke}{rgb}{0.000000,0.000000,0.000000}%
\pgfsetstrokecolor{currentstroke}%
\pgfsetdash{}{0pt}%
\pgfpathmoveto{\pgfqpoint{0.800000in}{0.528000in}}%
\pgfpathlineto{\pgfqpoint{5.760000in}{0.528000in}}%
\pgfusepath{stroke}%
\end{pgfscope}%
\begin{pgfscope}%
\pgfsetrectcap%
\pgfsetmiterjoin%
\pgfsetlinewidth{0.803000pt}%
\definecolor{currentstroke}{rgb}{0.000000,0.000000,0.000000}%
\pgfsetstrokecolor{currentstroke}%
\pgfsetdash{}{0pt}%
\pgfpathmoveto{\pgfqpoint{0.800000in}{4.224000in}}%
\pgfpathlineto{\pgfqpoint{5.760000in}{4.224000in}}%
\pgfusepath{stroke}%
\end{pgfscope}%
\begin{pgfscope}%
\definecolor{textcolor}{rgb}{0.000000,0.000000,0.000000}%
\pgfsetstrokecolor{textcolor}%
\pgfsetfillcolor{textcolor}%
\pgftext[x=3.280000in,y=4.307333in,,base]{\color{textcolor}{\sffamily\fontsize{9.600000}{11.520000}\selectfont\catcode`\^=\active\def^{\ifmmode\sp\else\^{}\fi}\catcode`\%=\active\def%{\%}Função de Perda em Treino (compIII)}}%
\end{pgfscope}%
\begin{pgfscope}%
\pgfsetbuttcap%
\pgfsetmiterjoin%
\definecolor{currentfill}{rgb}{1.000000,1.000000,1.000000}%
\pgfsetfillcolor{currentfill}%
\pgfsetfillopacity{0.800000}%
\pgfsetlinewidth{1.003750pt}%
\definecolor{currentstroke}{rgb}{0.800000,0.800000,0.800000}%
\pgfsetstrokecolor{currentstroke}%
\pgfsetstrokeopacity{0.800000}%
\pgfsetdash{}{0pt}%
\pgfpathmoveto{\pgfqpoint{4.635586in}{3.808723in}}%
\pgfpathlineto{\pgfqpoint{5.682222in}{3.808723in}}%
\pgfpathquadraticcurveto{\pgfqpoint{5.704444in}{3.808723in}}{\pgfqpoint{5.704444in}{3.830945in}}%
\pgfpathlineto{\pgfqpoint{5.704444in}{4.146222in}}%
\pgfpathquadraticcurveto{\pgfqpoint{5.704444in}{4.168444in}}{\pgfqpoint{5.682222in}{4.168444in}}%
\pgfpathlineto{\pgfqpoint{4.635586in}{4.168444in}}%
\pgfpathquadraticcurveto{\pgfqpoint{4.613364in}{4.168444in}}{\pgfqpoint{4.613364in}{4.146222in}}%
\pgfpathlineto{\pgfqpoint{4.613364in}{3.830945in}}%
\pgfpathquadraticcurveto{\pgfqpoint{4.613364in}{3.808723in}}{\pgfqpoint{4.635586in}{3.808723in}}%
\pgfpathlineto{\pgfqpoint{4.635586in}{3.808723in}}%
\pgfpathclose%
\pgfusepath{stroke,fill}%
\end{pgfscope}%
\begin{pgfscope}%
\pgfsetrectcap%
\pgfsetroundjoin%
\pgfsetlinewidth{1.505625pt}%
\definecolor{currentstroke}{rgb}{0.121569,0.466667,0.705882}%
\pgfsetstrokecolor{currentstroke}%
\pgfsetdash{}{0pt}%
\pgfpathmoveto{\pgfqpoint{4.657808in}{4.078470in}}%
\pgfpathlineto{\pgfqpoint{4.768919in}{4.078470in}}%
\pgfpathlineto{\pgfqpoint{4.880030in}{4.078470in}}%
\pgfusepath{stroke}%
\end{pgfscope}%
\begin{pgfscope}%
\definecolor{textcolor}{rgb}{0.000000,0.000000,0.000000}%
\pgfsetstrokecolor{textcolor}%
\pgfsetfillcolor{textcolor}%
\pgftext[x=4.968919in,y=4.039582in,left,base]{\color{textcolor}{\sffamily\fontsize{8.000000}{9.600000}\selectfont\catcode`\^=\active\def^{\ifmmode\sp\else\^{}\fi}\catcode`\%=\active\def%{\%}Treinamento}}%
\end{pgfscope}%
\begin{pgfscope}%
\pgfsetrectcap%
\pgfsetroundjoin%
\pgfsetlinewidth{1.505625pt}%
\definecolor{currentstroke}{rgb}{1.000000,0.498039,0.054902}%
\pgfsetstrokecolor{currentstroke}%
\pgfsetdash{}{0pt}%
\pgfpathmoveto{\pgfqpoint{4.657808in}{3.915168in}}%
\pgfpathlineto{\pgfqpoint{4.768919in}{3.915168in}}%
\pgfpathlineto{\pgfqpoint{4.880030in}{3.915168in}}%
\pgfusepath{stroke}%
\end{pgfscope}%
\begin{pgfscope}%
\definecolor{textcolor}{rgb}{0.000000,0.000000,0.000000}%
\pgfsetstrokecolor{textcolor}%
\pgfsetfillcolor{textcolor}%
\pgftext[x=4.968919in,y=3.876279in,left,base]{\color{textcolor}{\sffamily\fontsize{8.000000}{9.600000}\selectfont\catcode`\^=\active\def^{\ifmmode\sp\else\^{}\fi}\catcode`\%=\active\def%{\%}Validação}}%
\end{pgfscope}%
\end{pgfpicture}%
\makeatother%
\endgroup%
}
    \end{minipage}
    \begin{minipage}{0.45\textwidth}
        \resizebox{\textwidth}{!}{%% Creator: Matplotlib, PGF backend
%%
%% To include the figure in your LaTeX document, write
%%   \input{<filename>.pgf}
%%
%% Make sure the required packages are loaded in your preamble
%%   \usepackage{pgf}
%%
%% Also ensure that all the required font packages are loaded; for instance,
%% the lmodern package is sometimes necessary when using math font.
%%   \usepackage{lmodern}
%%
%% Figures using additional raster images can only be included by \input if
%% they are in the same directory as the main LaTeX file. For loading figures
%% from other directories you can use the `import` package
%%   \usepackage{import}
%%
%% and then include the figures with
%%   \import{<path to file>}{<filename>.pgf}
%%
%% Matplotlib used the following preamble
%%   \def\mathdefault#1{#1}
%%   \everymath=\expandafter{\the\everymath\displaystyle}
%%
%%   \usepackage{fontspec}
%%   \setmainfont{DejaVuSerif.ttf}[Path=\detokenize{/home/josemayer/.local/lib/python3.9/site-packages/matplotlib/mpl-data/fonts/ttf/}]
%%   \setsansfont{DejaVuSans.ttf}[Path=\detokenize{/home/josemayer/.local/lib/python3.9/site-packages/matplotlib/mpl-data/fonts/ttf/}]
%%   \setmonofont{DejaVuSansMono.ttf}[Path=\detokenize{/home/josemayer/.local/lib/python3.9/site-packages/matplotlib/mpl-data/fonts/ttf/}]
%%   \makeatletter\@ifpackageloaded{underscore}{}{\usepackage[strings]{underscore}}\makeatother
%%
\begingroup%
\makeatletter%
\begin{pgfpicture}%
\pgfpathrectangle{\pgfpointorigin}{\pgfqpoint{6.400000in}{4.800000in}}%
\pgfusepath{use as bounding box, clip}%
\begin{pgfscope}%
\pgfsetbuttcap%
\pgfsetmiterjoin%
\definecolor{currentfill}{rgb}{1.000000,1.000000,1.000000}%
\pgfsetfillcolor{currentfill}%
\pgfsetlinewidth{0.000000pt}%
\definecolor{currentstroke}{rgb}{1.000000,1.000000,1.000000}%
\pgfsetstrokecolor{currentstroke}%
\pgfsetdash{}{0pt}%
\pgfpathmoveto{\pgfqpoint{0.000000in}{0.000000in}}%
\pgfpathlineto{\pgfqpoint{6.400000in}{0.000000in}}%
\pgfpathlineto{\pgfqpoint{6.400000in}{4.800000in}}%
\pgfpathlineto{\pgfqpoint{0.000000in}{4.800000in}}%
\pgfpathlineto{\pgfqpoint{0.000000in}{0.000000in}}%
\pgfpathclose%
\pgfusepath{fill}%
\end{pgfscope}%
\begin{pgfscope}%
\pgfsetbuttcap%
\pgfsetmiterjoin%
\definecolor{currentfill}{rgb}{1.000000,1.000000,1.000000}%
\pgfsetfillcolor{currentfill}%
\pgfsetlinewidth{0.000000pt}%
\definecolor{currentstroke}{rgb}{0.000000,0.000000,0.000000}%
\pgfsetstrokecolor{currentstroke}%
\pgfsetstrokeopacity{0.000000}%
\pgfsetdash{}{0pt}%
\pgfpathmoveto{\pgfqpoint{0.800000in}{0.528000in}}%
\pgfpathlineto{\pgfqpoint{5.760000in}{0.528000in}}%
\pgfpathlineto{\pgfqpoint{5.760000in}{4.224000in}}%
\pgfpathlineto{\pgfqpoint{0.800000in}{4.224000in}}%
\pgfpathlineto{\pgfqpoint{0.800000in}{0.528000in}}%
\pgfpathclose%
\pgfusepath{fill}%
\end{pgfscope}%
\begin{pgfscope}%
\pgfsetbuttcap%
\pgfsetroundjoin%
\definecolor{currentfill}{rgb}{0.000000,0.000000,0.000000}%
\pgfsetfillcolor{currentfill}%
\pgfsetlinewidth{0.803000pt}%
\definecolor{currentstroke}{rgb}{0.000000,0.000000,0.000000}%
\pgfsetstrokecolor{currentstroke}%
\pgfsetdash{}{0pt}%
\pgfsys@defobject{currentmarker}{\pgfqpoint{0.000000in}{-0.048611in}}{\pgfqpoint{0.000000in}{0.000000in}}{%
\pgfpathmoveto{\pgfqpoint{0.000000in}{0.000000in}}%
\pgfpathlineto{\pgfqpoint{0.000000in}{-0.048611in}}%
\pgfusepath{stroke,fill}%
}%
\begin{pgfscope}%
\pgfsys@transformshift{1.025455in}{0.528000in}%
\pgfsys@useobject{currentmarker}{}%
\end{pgfscope}%
\end{pgfscope}%
\begin{pgfscope}%
\definecolor{textcolor}{rgb}{0.000000,0.000000,0.000000}%
\pgfsetstrokecolor{textcolor}%
\pgfsetfillcolor{textcolor}%
\pgftext[x=1.025455in,y=0.430778in,,top]{\color{textcolor}{\sffamily\fontsize{8.000000}{9.600000}\selectfont\catcode`\^=\active\def^{\ifmmode\sp\else\^{}\fi}\catcode`\%=\active\def%{\%}0}}%
\end{pgfscope}%
\begin{pgfscope}%
\pgfsetbuttcap%
\pgfsetroundjoin%
\definecolor{currentfill}{rgb}{0.000000,0.000000,0.000000}%
\pgfsetfillcolor{currentfill}%
\pgfsetlinewidth{0.803000pt}%
\definecolor{currentstroke}{rgb}{0.000000,0.000000,0.000000}%
\pgfsetstrokecolor{currentstroke}%
\pgfsetdash{}{0pt}%
\pgfsys@defobject{currentmarker}{\pgfqpoint{0.000000in}{-0.048611in}}{\pgfqpoint{0.000000in}{0.000000in}}{%
\pgfpathmoveto{\pgfqpoint{0.000000in}{0.000000in}}%
\pgfpathlineto{\pgfqpoint{0.000000in}{-0.048611in}}%
\pgfusepath{stroke,fill}%
}%
\begin{pgfscope}%
\pgfsys@transformshift{1.945677in}{0.528000in}%
\pgfsys@useobject{currentmarker}{}%
\end{pgfscope}%
\end{pgfscope}%
\begin{pgfscope}%
\definecolor{textcolor}{rgb}{0.000000,0.000000,0.000000}%
\pgfsetstrokecolor{textcolor}%
\pgfsetfillcolor{textcolor}%
\pgftext[x=1.945677in,y=0.430778in,,top]{\color{textcolor}{\sffamily\fontsize{8.000000}{9.600000}\selectfont\catcode`\^=\active\def^{\ifmmode\sp\else\^{}\fi}\catcode`\%=\active\def%{\%}10}}%
\end{pgfscope}%
\begin{pgfscope}%
\pgfsetbuttcap%
\pgfsetroundjoin%
\definecolor{currentfill}{rgb}{0.000000,0.000000,0.000000}%
\pgfsetfillcolor{currentfill}%
\pgfsetlinewidth{0.803000pt}%
\definecolor{currentstroke}{rgb}{0.000000,0.000000,0.000000}%
\pgfsetstrokecolor{currentstroke}%
\pgfsetdash{}{0pt}%
\pgfsys@defobject{currentmarker}{\pgfqpoint{0.000000in}{-0.048611in}}{\pgfqpoint{0.000000in}{0.000000in}}{%
\pgfpathmoveto{\pgfqpoint{0.000000in}{0.000000in}}%
\pgfpathlineto{\pgfqpoint{0.000000in}{-0.048611in}}%
\pgfusepath{stroke,fill}%
}%
\begin{pgfscope}%
\pgfsys@transformshift{2.865900in}{0.528000in}%
\pgfsys@useobject{currentmarker}{}%
\end{pgfscope}%
\end{pgfscope}%
\begin{pgfscope}%
\definecolor{textcolor}{rgb}{0.000000,0.000000,0.000000}%
\pgfsetstrokecolor{textcolor}%
\pgfsetfillcolor{textcolor}%
\pgftext[x=2.865900in,y=0.430778in,,top]{\color{textcolor}{\sffamily\fontsize{8.000000}{9.600000}\selectfont\catcode`\^=\active\def^{\ifmmode\sp\else\^{}\fi}\catcode`\%=\active\def%{\%}20}}%
\end{pgfscope}%
\begin{pgfscope}%
\pgfsetbuttcap%
\pgfsetroundjoin%
\definecolor{currentfill}{rgb}{0.000000,0.000000,0.000000}%
\pgfsetfillcolor{currentfill}%
\pgfsetlinewidth{0.803000pt}%
\definecolor{currentstroke}{rgb}{0.000000,0.000000,0.000000}%
\pgfsetstrokecolor{currentstroke}%
\pgfsetdash{}{0pt}%
\pgfsys@defobject{currentmarker}{\pgfqpoint{0.000000in}{-0.048611in}}{\pgfqpoint{0.000000in}{0.000000in}}{%
\pgfpathmoveto{\pgfqpoint{0.000000in}{0.000000in}}%
\pgfpathlineto{\pgfqpoint{0.000000in}{-0.048611in}}%
\pgfusepath{stroke,fill}%
}%
\begin{pgfscope}%
\pgfsys@transformshift{3.786122in}{0.528000in}%
\pgfsys@useobject{currentmarker}{}%
\end{pgfscope}%
\end{pgfscope}%
\begin{pgfscope}%
\definecolor{textcolor}{rgb}{0.000000,0.000000,0.000000}%
\pgfsetstrokecolor{textcolor}%
\pgfsetfillcolor{textcolor}%
\pgftext[x=3.786122in,y=0.430778in,,top]{\color{textcolor}{\sffamily\fontsize{8.000000}{9.600000}\selectfont\catcode`\^=\active\def^{\ifmmode\sp\else\^{}\fi}\catcode`\%=\active\def%{\%}30}}%
\end{pgfscope}%
\begin{pgfscope}%
\pgfsetbuttcap%
\pgfsetroundjoin%
\definecolor{currentfill}{rgb}{0.000000,0.000000,0.000000}%
\pgfsetfillcolor{currentfill}%
\pgfsetlinewidth{0.803000pt}%
\definecolor{currentstroke}{rgb}{0.000000,0.000000,0.000000}%
\pgfsetstrokecolor{currentstroke}%
\pgfsetdash{}{0pt}%
\pgfsys@defobject{currentmarker}{\pgfqpoint{0.000000in}{-0.048611in}}{\pgfqpoint{0.000000in}{0.000000in}}{%
\pgfpathmoveto{\pgfqpoint{0.000000in}{0.000000in}}%
\pgfpathlineto{\pgfqpoint{0.000000in}{-0.048611in}}%
\pgfusepath{stroke,fill}%
}%
\begin{pgfscope}%
\pgfsys@transformshift{4.706345in}{0.528000in}%
\pgfsys@useobject{currentmarker}{}%
\end{pgfscope}%
\end{pgfscope}%
\begin{pgfscope}%
\definecolor{textcolor}{rgb}{0.000000,0.000000,0.000000}%
\pgfsetstrokecolor{textcolor}%
\pgfsetfillcolor{textcolor}%
\pgftext[x=4.706345in,y=0.430778in,,top]{\color{textcolor}{\sffamily\fontsize{8.000000}{9.600000}\selectfont\catcode`\^=\active\def^{\ifmmode\sp\else\^{}\fi}\catcode`\%=\active\def%{\%}40}}%
\end{pgfscope}%
\begin{pgfscope}%
\pgfsetbuttcap%
\pgfsetroundjoin%
\definecolor{currentfill}{rgb}{0.000000,0.000000,0.000000}%
\pgfsetfillcolor{currentfill}%
\pgfsetlinewidth{0.803000pt}%
\definecolor{currentstroke}{rgb}{0.000000,0.000000,0.000000}%
\pgfsetstrokecolor{currentstroke}%
\pgfsetdash{}{0pt}%
\pgfsys@defobject{currentmarker}{\pgfqpoint{0.000000in}{-0.048611in}}{\pgfqpoint{0.000000in}{0.000000in}}{%
\pgfpathmoveto{\pgfqpoint{0.000000in}{0.000000in}}%
\pgfpathlineto{\pgfqpoint{0.000000in}{-0.048611in}}%
\pgfusepath{stroke,fill}%
}%
\begin{pgfscope}%
\pgfsys@transformshift{5.626568in}{0.528000in}%
\pgfsys@useobject{currentmarker}{}%
\end{pgfscope}%
\end{pgfscope}%
\begin{pgfscope}%
\definecolor{textcolor}{rgb}{0.000000,0.000000,0.000000}%
\pgfsetstrokecolor{textcolor}%
\pgfsetfillcolor{textcolor}%
\pgftext[x=5.626568in,y=0.430778in,,top]{\color{textcolor}{\sffamily\fontsize{8.000000}{9.600000}\selectfont\catcode`\^=\active\def^{\ifmmode\sp\else\^{}\fi}\catcode`\%=\active\def%{\%}50}}%
\end{pgfscope}%
\begin{pgfscope}%
\definecolor{textcolor}{rgb}{0.000000,0.000000,0.000000}%
\pgfsetstrokecolor{textcolor}%
\pgfsetfillcolor{textcolor}%
\pgftext[x=3.280000in,y=0.267692in,,top]{\color{textcolor}{\sffamily\fontsize{8.000000}{9.600000}\selectfont\catcode`\^=\active\def^{\ifmmode\sp\else\^{}\fi}\catcode`\%=\active\def%{\%}Época}}%
\end{pgfscope}%
\begin{pgfscope}%
\pgfsetbuttcap%
\pgfsetroundjoin%
\definecolor{currentfill}{rgb}{0.000000,0.000000,0.000000}%
\pgfsetfillcolor{currentfill}%
\pgfsetlinewidth{0.803000pt}%
\definecolor{currentstroke}{rgb}{0.000000,0.000000,0.000000}%
\pgfsetstrokecolor{currentstroke}%
\pgfsetdash{}{0pt}%
\pgfsys@defobject{currentmarker}{\pgfqpoint{-0.048611in}{0.000000in}}{\pgfqpoint{-0.000000in}{0.000000in}}{%
\pgfpathmoveto{\pgfqpoint{-0.000000in}{0.000000in}}%
\pgfpathlineto{\pgfqpoint{-0.048611in}{0.000000in}}%
\pgfusepath{stroke,fill}%
}%
\begin{pgfscope}%
\pgfsys@transformshift{0.800000in}{0.558083in}%
\pgfsys@useobject{currentmarker}{}%
\end{pgfscope}%
\end{pgfscope}%
\begin{pgfscope}%
\definecolor{textcolor}{rgb}{0.000000,0.000000,0.000000}%
\pgfsetstrokecolor{textcolor}%
\pgfsetfillcolor{textcolor}%
\pgftext[x=0.526074in, y=0.515874in, left, base]{\color{textcolor}{\sffamily\fontsize{8.000000}{9.600000}\selectfont\catcode`\^=\active\def^{\ifmmode\sp\else\^{}\fi}\catcode`\%=\active\def%{\%}0.0}}%
\end{pgfscope}%
\begin{pgfscope}%
\pgfsetbuttcap%
\pgfsetroundjoin%
\definecolor{currentfill}{rgb}{0.000000,0.000000,0.000000}%
\pgfsetfillcolor{currentfill}%
\pgfsetlinewidth{0.803000pt}%
\definecolor{currentstroke}{rgb}{0.000000,0.000000,0.000000}%
\pgfsetstrokecolor{currentstroke}%
\pgfsetdash{}{0pt}%
\pgfsys@defobject{currentmarker}{\pgfqpoint{-0.048611in}{0.000000in}}{\pgfqpoint{-0.000000in}{0.000000in}}{%
\pgfpathmoveto{\pgfqpoint{-0.000000in}{0.000000in}}%
\pgfpathlineto{\pgfqpoint{-0.048611in}{0.000000in}}%
\pgfusepath{stroke,fill}%
}%
\begin{pgfscope}%
\pgfsys@transformshift{0.800000in}{1.240295in}%
\pgfsys@useobject{currentmarker}{}%
\end{pgfscope}%
\end{pgfscope}%
\begin{pgfscope}%
\definecolor{textcolor}{rgb}{0.000000,0.000000,0.000000}%
\pgfsetstrokecolor{textcolor}%
\pgfsetfillcolor{textcolor}%
\pgftext[x=0.526074in, y=1.198086in, left, base]{\color{textcolor}{\sffamily\fontsize{8.000000}{9.600000}\selectfont\catcode`\^=\active\def^{\ifmmode\sp\else\^{}\fi}\catcode`\%=\active\def%{\%}0.2}}%
\end{pgfscope}%
\begin{pgfscope}%
\pgfsetbuttcap%
\pgfsetroundjoin%
\definecolor{currentfill}{rgb}{0.000000,0.000000,0.000000}%
\pgfsetfillcolor{currentfill}%
\pgfsetlinewidth{0.803000pt}%
\definecolor{currentstroke}{rgb}{0.000000,0.000000,0.000000}%
\pgfsetstrokecolor{currentstroke}%
\pgfsetdash{}{0pt}%
\pgfsys@defobject{currentmarker}{\pgfqpoint{-0.048611in}{0.000000in}}{\pgfqpoint{-0.000000in}{0.000000in}}{%
\pgfpathmoveto{\pgfqpoint{-0.000000in}{0.000000in}}%
\pgfpathlineto{\pgfqpoint{-0.048611in}{0.000000in}}%
\pgfusepath{stroke,fill}%
}%
\begin{pgfscope}%
\pgfsys@transformshift{0.800000in}{1.922507in}%
\pgfsys@useobject{currentmarker}{}%
\end{pgfscope}%
\end{pgfscope}%
\begin{pgfscope}%
\definecolor{textcolor}{rgb}{0.000000,0.000000,0.000000}%
\pgfsetstrokecolor{textcolor}%
\pgfsetfillcolor{textcolor}%
\pgftext[x=0.526074in, y=1.880298in, left, base]{\color{textcolor}{\sffamily\fontsize{8.000000}{9.600000}\selectfont\catcode`\^=\active\def^{\ifmmode\sp\else\^{}\fi}\catcode`\%=\active\def%{\%}0.4}}%
\end{pgfscope}%
\begin{pgfscope}%
\pgfsetbuttcap%
\pgfsetroundjoin%
\definecolor{currentfill}{rgb}{0.000000,0.000000,0.000000}%
\pgfsetfillcolor{currentfill}%
\pgfsetlinewidth{0.803000pt}%
\definecolor{currentstroke}{rgb}{0.000000,0.000000,0.000000}%
\pgfsetstrokecolor{currentstroke}%
\pgfsetdash{}{0pt}%
\pgfsys@defobject{currentmarker}{\pgfqpoint{-0.048611in}{0.000000in}}{\pgfqpoint{-0.000000in}{0.000000in}}{%
\pgfpathmoveto{\pgfqpoint{-0.000000in}{0.000000in}}%
\pgfpathlineto{\pgfqpoint{-0.048611in}{0.000000in}}%
\pgfusepath{stroke,fill}%
}%
\begin{pgfscope}%
\pgfsys@transformshift{0.800000in}{2.604719in}%
\pgfsys@useobject{currentmarker}{}%
\end{pgfscope}%
\end{pgfscope}%
\begin{pgfscope}%
\definecolor{textcolor}{rgb}{0.000000,0.000000,0.000000}%
\pgfsetstrokecolor{textcolor}%
\pgfsetfillcolor{textcolor}%
\pgftext[x=0.526074in, y=2.562510in, left, base]{\color{textcolor}{\sffamily\fontsize{8.000000}{9.600000}\selectfont\catcode`\^=\active\def^{\ifmmode\sp\else\^{}\fi}\catcode`\%=\active\def%{\%}0.6}}%
\end{pgfscope}%
\begin{pgfscope}%
\pgfsetbuttcap%
\pgfsetroundjoin%
\definecolor{currentfill}{rgb}{0.000000,0.000000,0.000000}%
\pgfsetfillcolor{currentfill}%
\pgfsetlinewidth{0.803000pt}%
\definecolor{currentstroke}{rgb}{0.000000,0.000000,0.000000}%
\pgfsetstrokecolor{currentstroke}%
\pgfsetdash{}{0pt}%
\pgfsys@defobject{currentmarker}{\pgfqpoint{-0.048611in}{0.000000in}}{\pgfqpoint{-0.000000in}{0.000000in}}{%
\pgfpathmoveto{\pgfqpoint{-0.000000in}{0.000000in}}%
\pgfpathlineto{\pgfqpoint{-0.048611in}{0.000000in}}%
\pgfusepath{stroke,fill}%
}%
\begin{pgfscope}%
\pgfsys@transformshift{0.800000in}{3.286931in}%
\pgfsys@useobject{currentmarker}{}%
\end{pgfscope}%
\end{pgfscope}%
\begin{pgfscope}%
\definecolor{textcolor}{rgb}{0.000000,0.000000,0.000000}%
\pgfsetstrokecolor{textcolor}%
\pgfsetfillcolor{textcolor}%
\pgftext[x=0.526074in, y=3.244722in, left, base]{\color{textcolor}{\sffamily\fontsize{8.000000}{9.600000}\selectfont\catcode`\^=\active\def^{\ifmmode\sp\else\^{}\fi}\catcode`\%=\active\def%{\%}0.8}}%
\end{pgfscope}%
\begin{pgfscope}%
\pgfsetbuttcap%
\pgfsetroundjoin%
\definecolor{currentfill}{rgb}{0.000000,0.000000,0.000000}%
\pgfsetfillcolor{currentfill}%
\pgfsetlinewidth{0.803000pt}%
\definecolor{currentstroke}{rgb}{0.000000,0.000000,0.000000}%
\pgfsetstrokecolor{currentstroke}%
\pgfsetdash{}{0pt}%
\pgfsys@defobject{currentmarker}{\pgfqpoint{-0.048611in}{0.000000in}}{\pgfqpoint{-0.000000in}{0.000000in}}{%
\pgfpathmoveto{\pgfqpoint{-0.000000in}{0.000000in}}%
\pgfpathlineto{\pgfqpoint{-0.048611in}{0.000000in}}%
\pgfusepath{stroke,fill}%
}%
\begin{pgfscope}%
\pgfsys@transformshift{0.800000in}{3.969143in}%
\pgfsys@useobject{currentmarker}{}%
\end{pgfscope}%
\end{pgfscope}%
\begin{pgfscope}%
\definecolor{textcolor}{rgb}{0.000000,0.000000,0.000000}%
\pgfsetstrokecolor{textcolor}%
\pgfsetfillcolor{textcolor}%
\pgftext[x=0.526074in, y=3.926933in, left, base]{\color{textcolor}{\sffamily\fontsize{8.000000}{9.600000}\selectfont\catcode`\^=\active\def^{\ifmmode\sp\else\^{}\fi}\catcode`\%=\active\def%{\%}1.0}}%
\end{pgfscope}%
\begin{pgfscope}%
\definecolor{textcolor}{rgb}{0.000000,0.000000,0.000000}%
\pgfsetstrokecolor{textcolor}%
\pgfsetfillcolor{textcolor}%
\pgftext[x=0.470519in,y=2.376000in,,bottom,rotate=90.000000]{\color{textcolor}{\sffamily\fontsize{8.000000}{9.600000}\selectfont\catcode`\^=\active\def^{\ifmmode\sp\else\^{}\fi}\catcode`\%=\active\def%{\%}Perda}}%
\end{pgfscope}%
\begin{pgfscope}%
\pgfpathrectangle{\pgfqpoint{0.800000in}{0.528000in}}{\pgfqpoint{4.960000in}{3.696000in}}%
\pgfusepath{clip}%
\pgfsetrectcap%
\pgfsetroundjoin%
\pgfsetlinewidth{1.505625pt}%
\definecolor{currentstroke}{rgb}{0.121569,0.466667,0.705882}%
\pgfsetstrokecolor{currentstroke}%
\pgfsetdash{}{0pt}%
\pgfpathmoveto{\pgfqpoint{1.025455in}{3.223815in}}%
\pgfpathlineto{\pgfqpoint{1.117477in}{3.170692in}}%
\pgfpathlineto{\pgfqpoint{1.209499in}{2.651556in}}%
\pgfpathlineto{\pgfqpoint{1.301521in}{2.372106in}}%
\pgfpathlineto{\pgfqpoint{1.393544in}{2.117576in}}%
\pgfpathlineto{\pgfqpoint{1.485566in}{2.216117in}}%
\pgfpathlineto{\pgfqpoint{1.577588in}{1.962034in}}%
\pgfpathlineto{\pgfqpoint{1.669610in}{1.790948in}}%
\pgfpathlineto{\pgfqpoint{1.761633in}{1.655106in}}%
\pgfpathlineto{\pgfqpoint{1.853655in}{1.650282in}}%
\pgfpathlineto{\pgfqpoint{1.945677in}{1.618384in}}%
\pgfpathlineto{\pgfqpoint{2.037699in}{1.385918in}}%
\pgfpathlineto{\pgfqpoint{2.129722in}{1.246623in}}%
\pgfpathlineto{\pgfqpoint{2.221744in}{1.167908in}}%
\pgfpathlineto{\pgfqpoint{2.313766in}{1.043589in}}%
\pgfpathlineto{\pgfqpoint{2.405788in}{0.974392in}}%
\pgfpathlineto{\pgfqpoint{2.497811in}{1.108480in}}%
\pgfpathlineto{\pgfqpoint{2.589833in}{1.012664in}}%
\pgfpathlineto{\pgfqpoint{2.681855in}{0.913290in}}%
\pgfpathlineto{\pgfqpoint{2.773878in}{0.871578in}}%
\pgfpathlineto{\pgfqpoint{2.865900in}{0.895382in}}%
\pgfpathlineto{\pgfqpoint{2.957922in}{0.932291in}}%
\pgfpathlineto{\pgfqpoint{3.049944in}{0.806645in}}%
\pgfpathlineto{\pgfqpoint{3.141967in}{0.820166in}}%
\pgfpathlineto{\pgfqpoint{3.233989in}{0.816772in}}%
\pgfpathlineto{\pgfqpoint{3.326011in}{0.792673in}}%
\pgfpathlineto{\pgfqpoint{3.418033in}{0.758451in}}%
\pgfpathlineto{\pgfqpoint{3.510056in}{0.794093in}}%
\pgfpathlineto{\pgfqpoint{3.602078in}{0.776938in}}%
\pgfpathlineto{\pgfqpoint{3.694100in}{0.763949in}}%
\pgfpathlineto{\pgfqpoint{3.786122in}{0.729052in}}%
\pgfpathlineto{\pgfqpoint{3.878145in}{0.712038in}}%
\pgfpathlineto{\pgfqpoint{3.970167in}{0.776637in}}%
\pgfpathlineto{\pgfqpoint{4.062189in}{0.750194in}}%
\pgfpathlineto{\pgfqpoint{4.154212in}{0.716202in}}%
\pgfpathlineto{\pgfqpoint{4.246234in}{0.892506in}}%
\pgfpathlineto{\pgfqpoint{4.338256in}{0.915710in}}%
\pgfpathlineto{\pgfqpoint{4.430278in}{0.875491in}}%
\pgfpathlineto{\pgfqpoint{4.522301in}{0.978969in}}%
\pgfpathlineto{\pgfqpoint{4.614323in}{0.853682in}}%
\pgfpathlineto{\pgfqpoint{4.706345in}{0.831640in}}%
\pgfpathlineto{\pgfqpoint{4.798367in}{0.740051in}}%
\pgfpathlineto{\pgfqpoint{4.890390in}{0.696000in}}%
\pgfpathlineto{\pgfqpoint{4.982412in}{0.813444in}}%
\pgfpathlineto{\pgfqpoint{5.074434in}{0.870241in}}%
\pgfpathlineto{\pgfqpoint{5.166456in}{0.739422in}}%
\pgfpathlineto{\pgfqpoint{5.258479in}{1.075621in}}%
\pgfpathlineto{\pgfqpoint{5.350501in}{2.153903in}}%
\pgfpathlineto{\pgfqpoint{5.442523in}{3.558905in}}%
\pgfpathlineto{\pgfqpoint{5.534545in}{3.434488in}}%
\pgfusepath{stroke}%
\end{pgfscope}%
\begin{pgfscope}%
\pgfpathrectangle{\pgfqpoint{0.800000in}{0.528000in}}{\pgfqpoint{4.960000in}{3.696000in}}%
\pgfusepath{clip}%
\pgfsetrectcap%
\pgfsetroundjoin%
\pgfsetlinewidth{1.505625pt}%
\definecolor{currentstroke}{rgb}{1.000000,0.498039,0.054902}%
\pgfsetstrokecolor{currentstroke}%
\pgfsetdash{}{0pt}%
\pgfpathmoveto{\pgfqpoint{1.025455in}{3.075671in}}%
\pgfpathlineto{\pgfqpoint{1.117477in}{3.204339in}}%
\pgfpathlineto{\pgfqpoint{1.209499in}{3.835038in}}%
\pgfpathlineto{\pgfqpoint{1.301521in}{3.891845in}}%
\pgfpathlineto{\pgfqpoint{1.393544in}{3.326927in}}%
\pgfpathlineto{\pgfqpoint{1.485566in}{3.165522in}}%
\pgfpathlineto{\pgfqpoint{1.577588in}{3.190405in}}%
\pgfpathlineto{\pgfqpoint{1.669610in}{3.111174in}}%
\pgfpathlineto{\pgfqpoint{1.761633in}{3.208249in}}%
\pgfpathlineto{\pgfqpoint{1.853655in}{3.209452in}}%
\pgfpathlineto{\pgfqpoint{1.945677in}{3.138908in}}%
\pgfpathlineto{\pgfqpoint{2.037699in}{3.182390in}}%
\pgfpathlineto{\pgfqpoint{2.129722in}{3.349845in}}%
\pgfpathlineto{\pgfqpoint{2.221744in}{3.231409in}}%
\pgfpathlineto{\pgfqpoint{2.313766in}{3.052129in}}%
\pgfpathlineto{\pgfqpoint{2.405788in}{3.398299in}}%
\pgfpathlineto{\pgfqpoint{2.497811in}{3.167916in}}%
\pgfpathlineto{\pgfqpoint{2.589833in}{3.172325in}}%
\pgfpathlineto{\pgfqpoint{2.681855in}{3.080009in}}%
\pgfpathlineto{\pgfqpoint{2.773878in}{3.310758in}}%
\pgfpathlineto{\pgfqpoint{2.865900in}{3.249555in}}%
\pgfpathlineto{\pgfqpoint{2.957922in}{3.182619in}}%
\pgfpathlineto{\pgfqpoint{3.049944in}{3.548885in}}%
\pgfpathlineto{\pgfqpoint{3.141967in}{3.149709in}}%
\pgfpathlineto{\pgfqpoint{3.233989in}{3.444017in}}%
\pgfpathlineto{\pgfqpoint{3.326011in}{3.394716in}}%
\pgfpathlineto{\pgfqpoint{3.418033in}{3.249557in}}%
\pgfpathlineto{\pgfqpoint{3.510056in}{3.242247in}}%
\pgfpathlineto{\pgfqpoint{3.602078in}{3.176817in}}%
\pgfpathlineto{\pgfqpoint{3.694100in}{3.513580in}}%
\pgfpathlineto{\pgfqpoint{3.786122in}{3.290303in}}%
\pgfpathlineto{\pgfqpoint{3.878145in}{3.238668in}}%
\pgfpathlineto{\pgfqpoint{3.970167in}{3.404663in}}%
\pgfpathlineto{\pgfqpoint{4.062189in}{3.195129in}}%
\pgfpathlineto{\pgfqpoint{4.154212in}{3.225684in}}%
\pgfpathlineto{\pgfqpoint{4.246234in}{3.219001in}}%
\pgfpathlineto{\pgfqpoint{4.338256in}{3.448035in}}%
\pgfpathlineto{\pgfqpoint{4.430278in}{3.258771in}}%
\pgfpathlineto{\pgfqpoint{4.522301in}{3.414142in}}%
\pgfpathlineto{\pgfqpoint{4.614323in}{3.376813in}}%
\pgfpathlineto{\pgfqpoint{4.706345in}{3.302021in}}%
\pgfpathlineto{\pgfqpoint{4.798367in}{3.189638in}}%
\pgfpathlineto{\pgfqpoint{4.890390in}{3.386421in}}%
\pgfpathlineto{\pgfqpoint{4.982412in}{3.341070in}}%
\pgfpathlineto{\pgfqpoint{5.074434in}{3.307080in}}%
\pgfpathlineto{\pgfqpoint{5.166456in}{3.339596in}}%
\pgfpathlineto{\pgfqpoint{5.258479in}{3.477043in}}%
\pgfpathlineto{\pgfqpoint{5.350501in}{4.056000in}}%
\pgfpathlineto{\pgfqpoint{5.442523in}{3.775419in}}%
\pgfpathlineto{\pgfqpoint{5.534545in}{3.819010in}}%
\pgfusepath{stroke}%
\end{pgfscope}%
\begin{pgfscope}%
\pgfsetrectcap%
\pgfsetmiterjoin%
\pgfsetlinewidth{0.803000pt}%
\definecolor{currentstroke}{rgb}{0.000000,0.000000,0.000000}%
\pgfsetstrokecolor{currentstroke}%
\pgfsetdash{}{0pt}%
\pgfpathmoveto{\pgfqpoint{0.800000in}{0.528000in}}%
\pgfpathlineto{\pgfqpoint{0.800000in}{4.224000in}}%
\pgfusepath{stroke}%
\end{pgfscope}%
\begin{pgfscope}%
\pgfsetrectcap%
\pgfsetmiterjoin%
\pgfsetlinewidth{0.803000pt}%
\definecolor{currentstroke}{rgb}{0.000000,0.000000,0.000000}%
\pgfsetstrokecolor{currentstroke}%
\pgfsetdash{}{0pt}%
\pgfpathmoveto{\pgfqpoint{5.760000in}{0.528000in}}%
\pgfpathlineto{\pgfqpoint{5.760000in}{4.224000in}}%
\pgfusepath{stroke}%
\end{pgfscope}%
\begin{pgfscope}%
\pgfsetrectcap%
\pgfsetmiterjoin%
\pgfsetlinewidth{0.803000pt}%
\definecolor{currentstroke}{rgb}{0.000000,0.000000,0.000000}%
\pgfsetstrokecolor{currentstroke}%
\pgfsetdash{}{0pt}%
\pgfpathmoveto{\pgfqpoint{0.800000in}{0.528000in}}%
\pgfpathlineto{\pgfqpoint{5.760000in}{0.528000in}}%
\pgfusepath{stroke}%
\end{pgfscope}%
\begin{pgfscope}%
\pgfsetrectcap%
\pgfsetmiterjoin%
\pgfsetlinewidth{0.803000pt}%
\definecolor{currentstroke}{rgb}{0.000000,0.000000,0.000000}%
\pgfsetstrokecolor{currentstroke}%
\pgfsetdash{}{0pt}%
\pgfpathmoveto{\pgfqpoint{0.800000in}{4.224000in}}%
\pgfpathlineto{\pgfqpoint{5.760000in}{4.224000in}}%
\pgfusepath{stroke}%
\end{pgfscope}%
\begin{pgfscope}%
\definecolor{textcolor}{rgb}{0.000000,0.000000,0.000000}%
\pgfsetstrokecolor{textcolor}%
\pgfsetfillcolor{textcolor}%
\pgftext[x=3.280000in,y=4.307333in,,base]{\color{textcolor}{\sffamily\fontsize{9.600000}{11.520000}\selectfont\catcode`\^=\active\def^{\ifmmode\sp\else\^{}\fi}\catcode`\%=\active\def%{\%}Função de Perda em Treino (compIII)}}%
\end{pgfscope}%
\begin{pgfscope}%
\pgfsetbuttcap%
\pgfsetmiterjoin%
\definecolor{currentfill}{rgb}{1.000000,1.000000,1.000000}%
\pgfsetfillcolor{currentfill}%
\pgfsetfillopacity{0.800000}%
\pgfsetlinewidth{1.003750pt}%
\definecolor{currentstroke}{rgb}{0.800000,0.800000,0.800000}%
\pgfsetstrokecolor{currentstroke}%
\pgfsetstrokeopacity{0.800000}%
\pgfsetdash{}{0pt}%
\pgfpathmoveto{\pgfqpoint{4.635586in}{3.808723in}}%
\pgfpathlineto{\pgfqpoint{5.682222in}{3.808723in}}%
\pgfpathquadraticcurveto{\pgfqpoint{5.704444in}{3.808723in}}{\pgfqpoint{5.704444in}{3.830945in}}%
\pgfpathlineto{\pgfqpoint{5.704444in}{4.146222in}}%
\pgfpathquadraticcurveto{\pgfqpoint{5.704444in}{4.168444in}}{\pgfqpoint{5.682222in}{4.168444in}}%
\pgfpathlineto{\pgfqpoint{4.635586in}{4.168444in}}%
\pgfpathquadraticcurveto{\pgfqpoint{4.613364in}{4.168444in}}{\pgfqpoint{4.613364in}{4.146222in}}%
\pgfpathlineto{\pgfqpoint{4.613364in}{3.830945in}}%
\pgfpathquadraticcurveto{\pgfqpoint{4.613364in}{3.808723in}}{\pgfqpoint{4.635586in}{3.808723in}}%
\pgfpathlineto{\pgfqpoint{4.635586in}{3.808723in}}%
\pgfpathclose%
\pgfusepath{stroke,fill}%
\end{pgfscope}%
\begin{pgfscope}%
\pgfsetrectcap%
\pgfsetroundjoin%
\pgfsetlinewidth{1.505625pt}%
\definecolor{currentstroke}{rgb}{0.121569,0.466667,0.705882}%
\pgfsetstrokecolor{currentstroke}%
\pgfsetdash{}{0pt}%
\pgfpathmoveto{\pgfqpoint{4.657808in}{4.078470in}}%
\pgfpathlineto{\pgfqpoint{4.768919in}{4.078470in}}%
\pgfpathlineto{\pgfqpoint{4.880030in}{4.078470in}}%
\pgfusepath{stroke}%
\end{pgfscope}%
\begin{pgfscope}%
\definecolor{textcolor}{rgb}{0.000000,0.000000,0.000000}%
\pgfsetstrokecolor{textcolor}%
\pgfsetfillcolor{textcolor}%
\pgftext[x=4.968919in,y=4.039582in,left,base]{\color{textcolor}{\sffamily\fontsize{8.000000}{9.600000}\selectfont\catcode`\^=\active\def^{\ifmmode\sp\else\^{}\fi}\catcode`\%=\active\def%{\%}Treinamento}}%
\end{pgfscope}%
\begin{pgfscope}%
\pgfsetrectcap%
\pgfsetroundjoin%
\pgfsetlinewidth{1.505625pt}%
\definecolor{currentstroke}{rgb}{1.000000,0.498039,0.054902}%
\pgfsetstrokecolor{currentstroke}%
\pgfsetdash{}{0pt}%
\pgfpathmoveto{\pgfqpoint{4.657808in}{3.915168in}}%
\pgfpathlineto{\pgfqpoint{4.768919in}{3.915168in}}%
\pgfpathlineto{\pgfqpoint{4.880030in}{3.915168in}}%
\pgfusepath{stroke}%
\end{pgfscope}%
\begin{pgfscope}%
\definecolor{textcolor}{rgb}{0.000000,0.000000,0.000000}%
\pgfsetstrokecolor{textcolor}%
\pgfsetfillcolor{textcolor}%
\pgftext[x=4.968919in,y=3.876279in,left,base]{\color{textcolor}{\sffamily\fontsize{8.000000}{9.600000}\selectfont\catcode`\^=\active\def^{\ifmmode\sp\else\^{}\fi}\catcode`\%=\active\def%{\%}Validação}}%
\end{pgfscope}%
\end{pgfpicture}%
\makeatother%
\endgroup%
}
    \end{minipage}

    \caption{Evolução do treinamento com \textit{hypertuning} (à esquerda) e do treinamento com hiperparâmetros fixados (à direita) do sistema especialista da competência III.}
    \label{fig:exp-fix-c3}
\end{figure}

Pela figura \ref{fig:exp-fix-c3}, é possível notar que o treinamento com hiperparâmetros fixados apresentou uma evolução levemente melhor da função de perda, de modo que, em torno das primeiras épocas, o valor para a base de validação oscilou entre 0,8 e, entre a 30ª e a 45ª época, o valor para a base de treino ficou em torno de 0,1. No final das iterações, houve, ainda, um aumento significativo da perda tanto para a base de treino quanto para a base de validação, que é irrelevante devido ao \textit{callback} \texttt{ModelCheckpoint}. No caso do \textit{hypertuning}, a função aplicada aos dados de treino apresentou um decaimento ao longo das épocas, mas a perda para os dados de validação teve uma oscilação entre 1,04 e 0,94 entre todo o processo. Desse modo, observa-se que os parâmetros fixados apresentam uma leve vantagem em relação ao processo de otimização para a competência III.

\subsubsection{Competência IV}
\label{subsec:exp-fix-c4}

\begin{figure}[H]
    \begin{minipage}{0.45\textwidth}
        \resizebox{\textwidth}{!}{%% Creator: Matplotlib, PGF backend
%%
%% To include the figure in your LaTeX document, write
%%   \input{<filename>.pgf}
%%
%% Make sure the required packages are loaded in your preamble
%%   \usepackage{pgf}
%%
%% Also ensure that all the required font packages are loaded; for instance,
%% the lmodern package is sometimes necessary when using math font.
%%   \usepackage{lmodern}
%%
%% Figures using additional raster images can only be included by \input if
%% they are in the same directory as the main LaTeX file. For loading figures
%% from other directories you can use the `import` package
%%   \usepackage{import}
%%
%% and then include the figures with
%%   \import{<path to file>}{<filename>.pgf}
%%
%% Matplotlib used the following preamble
%%
%%   \usepackage{fontspec}
%%   \setmainfont{DejaVuSerif.ttf}[Path=\detokenize{/home/jose/anaconda3/envs/tf/lib/python3.9/site-packages/matplotlib/mpl-data/fonts/ttf/}]
%%   \setsansfont{DejaVuSans.ttf}[Path=\detokenize{/home/jose/anaconda3/envs/tf/lib/python3.9/site-packages/matplotlib/mpl-data/fonts/ttf/}]
%%   \setmonofont{DejaVuSansMono.ttf}[Path=\detokenize{/home/jose/anaconda3/envs/tf/lib/python3.9/site-packages/matplotlib/mpl-data/fonts/ttf/}]
%%   \makeatletter\@ifpackageloaded{underscore}{}{\usepackage[strings]{underscore}}\makeatother
%%
\begingroup%
\makeatletter%
\begin{pgfpicture}%
\pgfpathrectangle{\pgfpointorigin}{\pgfqpoint{6.400000in}{4.800000in}}%
\pgfusepath{use as bounding box, clip}%
\begin{pgfscope}%
\pgfsetbuttcap%
\pgfsetmiterjoin%
\definecolor{currentfill}{rgb}{1.000000,1.000000,1.000000}%
\pgfsetfillcolor{currentfill}%
\pgfsetlinewidth{0.000000pt}%
\definecolor{currentstroke}{rgb}{1.000000,1.000000,1.000000}%
\pgfsetstrokecolor{currentstroke}%
\pgfsetdash{}{0pt}%
\pgfpathmoveto{\pgfqpoint{0.000000in}{0.000000in}}%
\pgfpathlineto{\pgfqpoint{6.400000in}{0.000000in}}%
\pgfpathlineto{\pgfqpoint{6.400000in}{4.800000in}}%
\pgfpathlineto{\pgfqpoint{0.000000in}{4.800000in}}%
\pgfpathlineto{\pgfqpoint{0.000000in}{0.000000in}}%
\pgfpathclose%
\pgfusepath{fill}%
\end{pgfscope}%
\begin{pgfscope}%
\pgfsetbuttcap%
\pgfsetmiterjoin%
\definecolor{currentfill}{rgb}{1.000000,1.000000,1.000000}%
\pgfsetfillcolor{currentfill}%
\pgfsetlinewidth{0.000000pt}%
\definecolor{currentstroke}{rgb}{0.000000,0.000000,0.000000}%
\pgfsetstrokecolor{currentstroke}%
\pgfsetstrokeopacity{0.000000}%
\pgfsetdash{}{0pt}%
\pgfpathmoveto{\pgfqpoint{0.800000in}{0.528000in}}%
\pgfpathlineto{\pgfqpoint{5.760000in}{0.528000in}}%
\pgfpathlineto{\pgfqpoint{5.760000in}{4.224000in}}%
\pgfpathlineto{\pgfqpoint{0.800000in}{4.224000in}}%
\pgfpathlineto{\pgfqpoint{0.800000in}{0.528000in}}%
\pgfpathclose%
\pgfusepath{fill}%
\end{pgfscope}%
\begin{pgfscope}%
\pgfsetbuttcap%
\pgfsetroundjoin%
\definecolor{currentfill}{rgb}{0.000000,0.000000,0.000000}%
\pgfsetfillcolor{currentfill}%
\pgfsetlinewidth{0.803000pt}%
\definecolor{currentstroke}{rgb}{0.000000,0.000000,0.000000}%
\pgfsetstrokecolor{currentstroke}%
\pgfsetdash{}{0pt}%
\pgfsys@defobject{currentmarker}{\pgfqpoint{0.000000in}{-0.048611in}}{\pgfqpoint{0.000000in}{0.000000in}}{%
\pgfpathmoveto{\pgfqpoint{0.000000in}{0.000000in}}%
\pgfpathlineto{\pgfqpoint{0.000000in}{-0.048611in}}%
\pgfusepath{stroke,fill}%
}%
\begin{pgfscope}%
\pgfsys@transformshift{1.025455in}{0.528000in}%
\pgfsys@useobject{currentmarker}{}%
\end{pgfscope}%
\end{pgfscope}%
\begin{pgfscope}%
\definecolor{textcolor}{rgb}{0.000000,0.000000,0.000000}%
\pgfsetstrokecolor{textcolor}%
\pgfsetfillcolor{textcolor}%
\pgftext[x=1.025455in,y=0.430778in,,top]{\color{textcolor}\sffamily\fontsize{8.000000}{9.600000}\selectfont 0}%
\end{pgfscope}%
\begin{pgfscope}%
\pgfsetbuttcap%
\pgfsetroundjoin%
\definecolor{currentfill}{rgb}{0.000000,0.000000,0.000000}%
\pgfsetfillcolor{currentfill}%
\pgfsetlinewidth{0.803000pt}%
\definecolor{currentstroke}{rgb}{0.000000,0.000000,0.000000}%
\pgfsetstrokecolor{currentstroke}%
\pgfsetdash{}{0pt}%
\pgfsys@defobject{currentmarker}{\pgfqpoint{0.000000in}{-0.048611in}}{\pgfqpoint{0.000000in}{0.000000in}}{%
\pgfpathmoveto{\pgfqpoint{0.000000in}{0.000000in}}%
\pgfpathlineto{\pgfqpoint{0.000000in}{-0.048611in}}%
\pgfusepath{stroke,fill}%
}%
\begin{pgfscope}%
\pgfsys@transformshift{1.945677in}{0.528000in}%
\pgfsys@useobject{currentmarker}{}%
\end{pgfscope}%
\end{pgfscope}%
\begin{pgfscope}%
\definecolor{textcolor}{rgb}{0.000000,0.000000,0.000000}%
\pgfsetstrokecolor{textcolor}%
\pgfsetfillcolor{textcolor}%
\pgftext[x=1.945677in,y=0.430778in,,top]{\color{textcolor}\sffamily\fontsize{8.000000}{9.600000}\selectfont 10}%
\end{pgfscope}%
\begin{pgfscope}%
\pgfsetbuttcap%
\pgfsetroundjoin%
\definecolor{currentfill}{rgb}{0.000000,0.000000,0.000000}%
\pgfsetfillcolor{currentfill}%
\pgfsetlinewidth{0.803000pt}%
\definecolor{currentstroke}{rgb}{0.000000,0.000000,0.000000}%
\pgfsetstrokecolor{currentstroke}%
\pgfsetdash{}{0pt}%
\pgfsys@defobject{currentmarker}{\pgfqpoint{0.000000in}{-0.048611in}}{\pgfqpoint{0.000000in}{0.000000in}}{%
\pgfpathmoveto{\pgfqpoint{0.000000in}{0.000000in}}%
\pgfpathlineto{\pgfqpoint{0.000000in}{-0.048611in}}%
\pgfusepath{stroke,fill}%
}%
\begin{pgfscope}%
\pgfsys@transformshift{2.865900in}{0.528000in}%
\pgfsys@useobject{currentmarker}{}%
\end{pgfscope}%
\end{pgfscope}%
\begin{pgfscope}%
\definecolor{textcolor}{rgb}{0.000000,0.000000,0.000000}%
\pgfsetstrokecolor{textcolor}%
\pgfsetfillcolor{textcolor}%
\pgftext[x=2.865900in,y=0.430778in,,top]{\color{textcolor}\sffamily\fontsize{8.000000}{9.600000}\selectfont 20}%
\end{pgfscope}%
\begin{pgfscope}%
\pgfsetbuttcap%
\pgfsetroundjoin%
\definecolor{currentfill}{rgb}{0.000000,0.000000,0.000000}%
\pgfsetfillcolor{currentfill}%
\pgfsetlinewidth{0.803000pt}%
\definecolor{currentstroke}{rgb}{0.000000,0.000000,0.000000}%
\pgfsetstrokecolor{currentstroke}%
\pgfsetdash{}{0pt}%
\pgfsys@defobject{currentmarker}{\pgfqpoint{0.000000in}{-0.048611in}}{\pgfqpoint{0.000000in}{0.000000in}}{%
\pgfpathmoveto{\pgfqpoint{0.000000in}{0.000000in}}%
\pgfpathlineto{\pgfqpoint{0.000000in}{-0.048611in}}%
\pgfusepath{stroke,fill}%
}%
\begin{pgfscope}%
\pgfsys@transformshift{3.786122in}{0.528000in}%
\pgfsys@useobject{currentmarker}{}%
\end{pgfscope}%
\end{pgfscope}%
\begin{pgfscope}%
\definecolor{textcolor}{rgb}{0.000000,0.000000,0.000000}%
\pgfsetstrokecolor{textcolor}%
\pgfsetfillcolor{textcolor}%
\pgftext[x=3.786122in,y=0.430778in,,top]{\color{textcolor}\sffamily\fontsize{8.000000}{9.600000}\selectfont 30}%
\end{pgfscope}%
\begin{pgfscope}%
\pgfsetbuttcap%
\pgfsetroundjoin%
\definecolor{currentfill}{rgb}{0.000000,0.000000,0.000000}%
\pgfsetfillcolor{currentfill}%
\pgfsetlinewidth{0.803000pt}%
\definecolor{currentstroke}{rgb}{0.000000,0.000000,0.000000}%
\pgfsetstrokecolor{currentstroke}%
\pgfsetdash{}{0pt}%
\pgfsys@defobject{currentmarker}{\pgfqpoint{0.000000in}{-0.048611in}}{\pgfqpoint{0.000000in}{0.000000in}}{%
\pgfpathmoveto{\pgfqpoint{0.000000in}{0.000000in}}%
\pgfpathlineto{\pgfqpoint{0.000000in}{-0.048611in}}%
\pgfusepath{stroke,fill}%
}%
\begin{pgfscope}%
\pgfsys@transformshift{4.706345in}{0.528000in}%
\pgfsys@useobject{currentmarker}{}%
\end{pgfscope}%
\end{pgfscope}%
\begin{pgfscope}%
\definecolor{textcolor}{rgb}{0.000000,0.000000,0.000000}%
\pgfsetstrokecolor{textcolor}%
\pgfsetfillcolor{textcolor}%
\pgftext[x=4.706345in,y=0.430778in,,top]{\color{textcolor}\sffamily\fontsize{8.000000}{9.600000}\selectfont 40}%
\end{pgfscope}%
\begin{pgfscope}%
\pgfsetbuttcap%
\pgfsetroundjoin%
\definecolor{currentfill}{rgb}{0.000000,0.000000,0.000000}%
\pgfsetfillcolor{currentfill}%
\pgfsetlinewidth{0.803000pt}%
\definecolor{currentstroke}{rgb}{0.000000,0.000000,0.000000}%
\pgfsetstrokecolor{currentstroke}%
\pgfsetdash{}{0pt}%
\pgfsys@defobject{currentmarker}{\pgfqpoint{0.000000in}{-0.048611in}}{\pgfqpoint{0.000000in}{0.000000in}}{%
\pgfpathmoveto{\pgfqpoint{0.000000in}{0.000000in}}%
\pgfpathlineto{\pgfqpoint{0.000000in}{-0.048611in}}%
\pgfusepath{stroke,fill}%
}%
\begin{pgfscope}%
\pgfsys@transformshift{5.626568in}{0.528000in}%
\pgfsys@useobject{currentmarker}{}%
\end{pgfscope}%
\end{pgfscope}%
\begin{pgfscope}%
\definecolor{textcolor}{rgb}{0.000000,0.000000,0.000000}%
\pgfsetstrokecolor{textcolor}%
\pgfsetfillcolor{textcolor}%
\pgftext[x=5.626568in,y=0.430778in,,top]{\color{textcolor}\sffamily\fontsize{8.000000}{9.600000}\selectfont 50}%
\end{pgfscope}%
\begin{pgfscope}%
\definecolor{textcolor}{rgb}{0.000000,0.000000,0.000000}%
\pgfsetstrokecolor{textcolor}%
\pgfsetfillcolor{textcolor}%
\pgftext[x=3.280000in,y=0.267692in,,top]{\color{textcolor}\sffamily\fontsize{8.000000}{9.600000}\selectfont Época}%
\end{pgfscope}%
\begin{pgfscope}%
\pgfsetbuttcap%
\pgfsetroundjoin%
\definecolor{currentfill}{rgb}{0.000000,0.000000,0.000000}%
\pgfsetfillcolor{currentfill}%
\pgfsetlinewidth{0.803000pt}%
\definecolor{currentstroke}{rgb}{0.000000,0.000000,0.000000}%
\pgfsetstrokecolor{currentstroke}%
\pgfsetdash{}{0pt}%
\pgfsys@defobject{currentmarker}{\pgfqpoint{-0.048611in}{0.000000in}}{\pgfqpoint{-0.000000in}{0.000000in}}{%
\pgfpathmoveto{\pgfqpoint{-0.000000in}{0.000000in}}%
\pgfpathlineto{\pgfqpoint{-0.048611in}{0.000000in}}%
\pgfusepath{stroke,fill}%
}%
\begin{pgfscope}%
\pgfsys@transformshift{0.800000in}{0.803016in}%
\pgfsys@useobject{currentmarker}{}%
\end{pgfscope}%
\end{pgfscope}%
\begin{pgfscope}%
\definecolor{textcolor}{rgb}{0.000000,0.000000,0.000000}%
\pgfsetstrokecolor{textcolor}%
\pgfsetfillcolor{textcolor}%
\pgftext[x=0.526074in, y=0.760806in, left, base]{\color{textcolor}\sffamily\fontsize{8.000000}{9.600000}\selectfont 1.3}%
\end{pgfscope}%
\begin{pgfscope}%
\pgfsetbuttcap%
\pgfsetroundjoin%
\definecolor{currentfill}{rgb}{0.000000,0.000000,0.000000}%
\pgfsetfillcolor{currentfill}%
\pgfsetlinewidth{0.803000pt}%
\definecolor{currentstroke}{rgb}{0.000000,0.000000,0.000000}%
\pgfsetstrokecolor{currentstroke}%
\pgfsetdash{}{0pt}%
\pgfsys@defobject{currentmarker}{\pgfqpoint{-0.048611in}{0.000000in}}{\pgfqpoint{-0.000000in}{0.000000in}}{%
\pgfpathmoveto{\pgfqpoint{-0.000000in}{0.000000in}}%
\pgfpathlineto{\pgfqpoint{-0.048611in}{0.000000in}}%
\pgfusepath{stroke,fill}%
}%
\begin{pgfscope}%
\pgfsys@transformshift{0.800000in}{1.471835in}%
\pgfsys@useobject{currentmarker}{}%
\end{pgfscope}%
\end{pgfscope}%
\begin{pgfscope}%
\definecolor{textcolor}{rgb}{0.000000,0.000000,0.000000}%
\pgfsetstrokecolor{textcolor}%
\pgfsetfillcolor{textcolor}%
\pgftext[x=0.526074in, y=1.429626in, left, base]{\color{textcolor}\sffamily\fontsize{8.000000}{9.600000}\selectfont 1.4}%
\end{pgfscope}%
\begin{pgfscope}%
\pgfsetbuttcap%
\pgfsetroundjoin%
\definecolor{currentfill}{rgb}{0.000000,0.000000,0.000000}%
\pgfsetfillcolor{currentfill}%
\pgfsetlinewidth{0.803000pt}%
\definecolor{currentstroke}{rgb}{0.000000,0.000000,0.000000}%
\pgfsetstrokecolor{currentstroke}%
\pgfsetdash{}{0pt}%
\pgfsys@defobject{currentmarker}{\pgfqpoint{-0.048611in}{0.000000in}}{\pgfqpoint{-0.000000in}{0.000000in}}{%
\pgfpathmoveto{\pgfqpoint{-0.000000in}{0.000000in}}%
\pgfpathlineto{\pgfqpoint{-0.048611in}{0.000000in}}%
\pgfusepath{stroke,fill}%
}%
\begin{pgfscope}%
\pgfsys@transformshift{0.800000in}{2.140654in}%
\pgfsys@useobject{currentmarker}{}%
\end{pgfscope}%
\end{pgfscope}%
\begin{pgfscope}%
\definecolor{textcolor}{rgb}{0.000000,0.000000,0.000000}%
\pgfsetstrokecolor{textcolor}%
\pgfsetfillcolor{textcolor}%
\pgftext[x=0.526074in, y=2.098445in, left, base]{\color{textcolor}\sffamily\fontsize{8.000000}{9.600000}\selectfont 1.5}%
\end{pgfscope}%
\begin{pgfscope}%
\pgfsetbuttcap%
\pgfsetroundjoin%
\definecolor{currentfill}{rgb}{0.000000,0.000000,0.000000}%
\pgfsetfillcolor{currentfill}%
\pgfsetlinewidth{0.803000pt}%
\definecolor{currentstroke}{rgb}{0.000000,0.000000,0.000000}%
\pgfsetstrokecolor{currentstroke}%
\pgfsetdash{}{0pt}%
\pgfsys@defobject{currentmarker}{\pgfqpoint{-0.048611in}{0.000000in}}{\pgfqpoint{-0.000000in}{0.000000in}}{%
\pgfpathmoveto{\pgfqpoint{-0.000000in}{0.000000in}}%
\pgfpathlineto{\pgfqpoint{-0.048611in}{0.000000in}}%
\pgfusepath{stroke,fill}%
}%
\begin{pgfscope}%
\pgfsys@transformshift{0.800000in}{2.809473in}%
\pgfsys@useobject{currentmarker}{}%
\end{pgfscope}%
\end{pgfscope}%
\begin{pgfscope}%
\definecolor{textcolor}{rgb}{0.000000,0.000000,0.000000}%
\pgfsetstrokecolor{textcolor}%
\pgfsetfillcolor{textcolor}%
\pgftext[x=0.526074in, y=2.767264in, left, base]{\color{textcolor}\sffamily\fontsize{8.000000}{9.600000}\selectfont 1.6}%
\end{pgfscope}%
\begin{pgfscope}%
\pgfsetbuttcap%
\pgfsetroundjoin%
\definecolor{currentfill}{rgb}{0.000000,0.000000,0.000000}%
\pgfsetfillcolor{currentfill}%
\pgfsetlinewidth{0.803000pt}%
\definecolor{currentstroke}{rgb}{0.000000,0.000000,0.000000}%
\pgfsetstrokecolor{currentstroke}%
\pgfsetdash{}{0pt}%
\pgfsys@defobject{currentmarker}{\pgfqpoint{-0.048611in}{0.000000in}}{\pgfqpoint{-0.000000in}{0.000000in}}{%
\pgfpathmoveto{\pgfqpoint{-0.000000in}{0.000000in}}%
\pgfpathlineto{\pgfqpoint{-0.048611in}{0.000000in}}%
\pgfusepath{stroke,fill}%
}%
\begin{pgfscope}%
\pgfsys@transformshift{0.800000in}{3.478293in}%
\pgfsys@useobject{currentmarker}{}%
\end{pgfscope}%
\end{pgfscope}%
\begin{pgfscope}%
\definecolor{textcolor}{rgb}{0.000000,0.000000,0.000000}%
\pgfsetstrokecolor{textcolor}%
\pgfsetfillcolor{textcolor}%
\pgftext[x=0.526074in, y=3.436083in, left, base]{\color{textcolor}\sffamily\fontsize{8.000000}{9.600000}\selectfont 1.7}%
\end{pgfscope}%
\begin{pgfscope}%
\pgfsetbuttcap%
\pgfsetroundjoin%
\definecolor{currentfill}{rgb}{0.000000,0.000000,0.000000}%
\pgfsetfillcolor{currentfill}%
\pgfsetlinewidth{0.803000pt}%
\definecolor{currentstroke}{rgb}{0.000000,0.000000,0.000000}%
\pgfsetstrokecolor{currentstroke}%
\pgfsetdash{}{0pt}%
\pgfsys@defobject{currentmarker}{\pgfqpoint{-0.048611in}{0.000000in}}{\pgfqpoint{-0.000000in}{0.000000in}}{%
\pgfpathmoveto{\pgfqpoint{-0.000000in}{0.000000in}}%
\pgfpathlineto{\pgfqpoint{-0.048611in}{0.000000in}}%
\pgfusepath{stroke,fill}%
}%
\begin{pgfscope}%
\pgfsys@transformshift{0.800000in}{4.147112in}%
\pgfsys@useobject{currentmarker}{}%
\end{pgfscope}%
\end{pgfscope}%
\begin{pgfscope}%
\definecolor{textcolor}{rgb}{0.000000,0.000000,0.000000}%
\pgfsetstrokecolor{textcolor}%
\pgfsetfillcolor{textcolor}%
\pgftext[x=0.526074in, y=4.104903in, left, base]{\color{textcolor}\sffamily\fontsize{8.000000}{9.600000}\selectfont 1.8}%
\end{pgfscope}%
\begin{pgfscope}%
\definecolor{textcolor}{rgb}{0.000000,0.000000,0.000000}%
\pgfsetstrokecolor{textcolor}%
\pgfsetfillcolor{textcolor}%
\pgftext[x=0.470519in,y=2.376000in,,bottom,rotate=90.000000]{\color{textcolor}\sffamily\fontsize{8.000000}{9.600000}\selectfont Perda}%
\end{pgfscope}%
\begin{pgfscope}%
\pgfpathrectangle{\pgfqpoint{0.800000in}{0.528000in}}{\pgfqpoint{4.960000in}{3.696000in}}%
\pgfusepath{clip}%
\pgfsetrectcap%
\pgfsetroundjoin%
\pgfsetlinewidth{1.505625pt}%
\definecolor{currentstroke}{rgb}{0.121569,0.466667,0.705882}%
\pgfsetstrokecolor{currentstroke}%
\pgfsetdash{}{0pt}%
\pgfpathmoveto{\pgfqpoint{1.025455in}{4.056000in}}%
\pgfpathlineto{\pgfqpoint{1.117477in}{1.346786in}}%
\pgfpathlineto{\pgfqpoint{1.209499in}{1.216059in}}%
\pgfpathlineto{\pgfqpoint{1.301521in}{1.173435in}}%
\pgfpathlineto{\pgfqpoint{1.393544in}{1.110396in}}%
\pgfpathlineto{\pgfqpoint{1.485566in}{1.144729in}}%
\pgfpathlineto{\pgfqpoint{1.577588in}{1.127592in}}%
\pgfpathlineto{\pgfqpoint{1.669610in}{1.079932in}}%
\pgfpathlineto{\pgfqpoint{1.761633in}{1.110391in}}%
\pgfpathlineto{\pgfqpoint{1.853655in}{1.163441in}}%
\pgfpathlineto{\pgfqpoint{1.945677in}{1.076348in}}%
\pgfpathlineto{\pgfqpoint{2.037699in}{1.104327in}}%
\pgfpathlineto{\pgfqpoint{2.129722in}{0.968121in}}%
\pgfpathlineto{\pgfqpoint{2.221744in}{1.074651in}}%
\pgfpathlineto{\pgfqpoint{2.313766in}{1.107230in}}%
\pgfpathlineto{\pgfqpoint{2.405788in}{1.079717in}}%
\pgfpathlineto{\pgfqpoint{2.497811in}{1.123983in}}%
\pgfpathlineto{\pgfqpoint{2.589833in}{1.059515in}}%
\pgfpathlineto{\pgfqpoint{2.681855in}{1.037445in}}%
\pgfpathlineto{\pgfqpoint{2.773878in}{1.134666in}}%
\pgfpathlineto{\pgfqpoint{2.865900in}{1.029343in}}%
\pgfpathlineto{\pgfqpoint{2.957922in}{1.039000in}}%
\pgfpathlineto{\pgfqpoint{3.049944in}{1.054708in}}%
\pgfpathlineto{\pgfqpoint{3.141967in}{1.026417in}}%
\pgfpathlineto{\pgfqpoint{3.233989in}{1.052423in}}%
\pgfpathlineto{\pgfqpoint{3.326011in}{1.137537in}}%
\pgfpathlineto{\pgfqpoint{3.418033in}{1.085148in}}%
\pgfpathlineto{\pgfqpoint{3.510056in}{1.053308in}}%
\pgfpathlineto{\pgfqpoint{3.602078in}{1.021737in}}%
\pgfpathlineto{\pgfqpoint{3.694100in}{1.066946in}}%
\pgfpathlineto{\pgfqpoint{3.786122in}{1.027243in}}%
\pgfpathlineto{\pgfqpoint{3.878145in}{0.999646in}}%
\pgfpathlineto{\pgfqpoint{3.970167in}{1.002581in}}%
\pgfpathlineto{\pgfqpoint{4.062189in}{1.006442in}}%
\pgfpathlineto{\pgfqpoint{4.154212in}{0.999148in}}%
\pgfpathlineto{\pgfqpoint{4.246234in}{1.024021in}}%
\pgfpathlineto{\pgfqpoint{4.338256in}{1.005839in}}%
\pgfpathlineto{\pgfqpoint{4.430278in}{0.970665in}}%
\pgfpathlineto{\pgfqpoint{4.522301in}{1.005553in}}%
\pgfpathlineto{\pgfqpoint{4.614323in}{0.951031in}}%
\pgfpathlineto{\pgfqpoint{4.706345in}{0.976252in}}%
\pgfpathlineto{\pgfqpoint{4.798367in}{0.955677in}}%
\pgfpathlineto{\pgfqpoint{4.890390in}{0.935400in}}%
\pgfpathlineto{\pgfqpoint{4.982412in}{0.994691in}}%
\pgfpathlineto{\pgfqpoint{5.074434in}{0.959325in}}%
\pgfpathlineto{\pgfqpoint{5.166456in}{0.968347in}}%
\pgfpathlineto{\pgfqpoint{5.258479in}{0.946546in}}%
\pgfpathlineto{\pgfqpoint{5.350501in}{0.962637in}}%
\pgfpathlineto{\pgfqpoint{5.442523in}{0.941497in}}%
\pgfpathlineto{\pgfqpoint{5.534545in}{0.953562in}}%
\pgfusepath{stroke}%
\end{pgfscope}%
\begin{pgfscope}%
\pgfpathrectangle{\pgfqpoint{0.800000in}{0.528000in}}{\pgfqpoint{4.960000in}{3.696000in}}%
\pgfusepath{clip}%
\pgfsetrectcap%
\pgfsetroundjoin%
\pgfsetlinewidth{1.505625pt}%
\definecolor{currentstroke}{rgb}{1.000000,0.498039,0.054902}%
\pgfsetstrokecolor{currentstroke}%
\pgfsetdash{}{0pt}%
\pgfpathmoveto{\pgfqpoint{1.025455in}{0.745068in}}%
\pgfpathlineto{\pgfqpoint{1.117477in}{0.721533in}}%
\pgfpathlineto{\pgfqpoint{1.209499in}{0.756071in}}%
\pgfpathlineto{\pgfqpoint{1.301521in}{1.444013in}}%
\pgfpathlineto{\pgfqpoint{1.393544in}{0.700680in}}%
\pgfpathlineto{\pgfqpoint{1.485566in}{0.697459in}}%
\pgfpathlineto{\pgfqpoint{1.577588in}{0.712024in}}%
\pgfpathlineto{\pgfqpoint{1.669610in}{0.959347in}}%
\pgfpathlineto{\pgfqpoint{1.761633in}{1.453892in}}%
\pgfpathlineto{\pgfqpoint{1.853655in}{2.260214in}}%
\pgfpathlineto{\pgfqpoint{1.945677in}{0.895415in}}%
\pgfpathlineto{\pgfqpoint{2.037699in}{0.864606in}}%
\pgfpathlineto{\pgfqpoint{2.129722in}{1.914387in}}%
\pgfpathlineto{\pgfqpoint{2.221744in}{2.070069in}}%
\pgfpathlineto{\pgfqpoint{2.313766in}{1.991550in}}%
\pgfpathlineto{\pgfqpoint{2.405788in}{1.186500in}}%
\pgfpathlineto{\pgfqpoint{2.497811in}{0.896755in}}%
\pgfpathlineto{\pgfqpoint{2.589833in}{0.705387in}}%
\pgfpathlineto{\pgfqpoint{2.681855in}{0.696000in}}%
\pgfpathlineto{\pgfqpoint{2.773878in}{0.719535in}}%
\pgfpathlineto{\pgfqpoint{2.865900in}{0.738670in}}%
\pgfpathlineto{\pgfqpoint{2.957922in}{0.899668in}}%
\pgfpathlineto{\pgfqpoint{3.049944in}{0.851644in}}%
\pgfpathlineto{\pgfqpoint{3.141967in}{0.717111in}}%
\pgfpathlineto{\pgfqpoint{3.233989in}{0.954934in}}%
\pgfpathlineto{\pgfqpoint{3.326011in}{0.702855in}}%
\pgfpathlineto{\pgfqpoint{3.418033in}{0.723192in}}%
\pgfpathlineto{\pgfqpoint{3.510056in}{0.920652in}}%
\pgfpathlineto{\pgfqpoint{3.602078in}{0.722294in}}%
\pgfpathlineto{\pgfqpoint{3.694100in}{0.722059in}}%
\pgfpathlineto{\pgfqpoint{3.786122in}{1.079228in}}%
\pgfpathlineto{\pgfqpoint{3.878145in}{1.657506in}}%
\pgfpathlineto{\pgfqpoint{3.970167in}{0.828214in}}%
\pgfpathlineto{\pgfqpoint{4.062189in}{0.699215in}}%
\pgfpathlineto{\pgfqpoint{4.154212in}{1.130628in}}%
\pgfpathlineto{\pgfqpoint{4.246234in}{1.136650in}}%
\pgfpathlineto{\pgfqpoint{4.338256in}{0.742698in}}%
\pgfpathlineto{\pgfqpoint{4.430278in}{1.027864in}}%
\pgfpathlineto{\pgfqpoint{4.522301in}{0.768802in}}%
\pgfpathlineto{\pgfqpoint{4.614323in}{0.704317in}}%
\pgfpathlineto{\pgfqpoint{4.706345in}{0.720145in}}%
\pgfpathlineto{\pgfqpoint{4.798367in}{0.800567in}}%
\pgfpathlineto{\pgfqpoint{4.890390in}{0.964613in}}%
\pgfpathlineto{\pgfqpoint{4.982412in}{1.261322in}}%
\pgfpathlineto{\pgfqpoint{5.074434in}{0.895865in}}%
\pgfpathlineto{\pgfqpoint{5.166456in}{0.819267in}}%
\pgfpathlineto{\pgfqpoint{5.258479in}{0.800276in}}%
\pgfpathlineto{\pgfqpoint{5.350501in}{1.246908in}}%
\pgfpathlineto{\pgfqpoint{5.442523in}{0.738489in}}%
\pgfpathlineto{\pgfqpoint{5.534545in}{0.775674in}}%
\pgfusepath{stroke}%
\end{pgfscope}%
\begin{pgfscope}%
\pgfsetrectcap%
\pgfsetmiterjoin%
\pgfsetlinewidth{0.803000pt}%
\definecolor{currentstroke}{rgb}{0.000000,0.000000,0.000000}%
\pgfsetstrokecolor{currentstroke}%
\pgfsetdash{}{0pt}%
\pgfpathmoveto{\pgfqpoint{0.800000in}{0.528000in}}%
\pgfpathlineto{\pgfqpoint{0.800000in}{4.224000in}}%
\pgfusepath{stroke}%
\end{pgfscope}%
\begin{pgfscope}%
\pgfsetrectcap%
\pgfsetmiterjoin%
\pgfsetlinewidth{0.803000pt}%
\definecolor{currentstroke}{rgb}{0.000000,0.000000,0.000000}%
\pgfsetstrokecolor{currentstroke}%
\pgfsetdash{}{0pt}%
\pgfpathmoveto{\pgfqpoint{5.760000in}{0.528000in}}%
\pgfpathlineto{\pgfqpoint{5.760000in}{4.224000in}}%
\pgfusepath{stroke}%
\end{pgfscope}%
\begin{pgfscope}%
\pgfsetrectcap%
\pgfsetmiterjoin%
\pgfsetlinewidth{0.803000pt}%
\definecolor{currentstroke}{rgb}{0.000000,0.000000,0.000000}%
\pgfsetstrokecolor{currentstroke}%
\pgfsetdash{}{0pt}%
\pgfpathmoveto{\pgfqpoint{0.800000in}{0.528000in}}%
\pgfpathlineto{\pgfqpoint{5.760000in}{0.528000in}}%
\pgfusepath{stroke}%
\end{pgfscope}%
\begin{pgfscope}%
\pgfsetrectcap%
\pgfsetmiterjoin%
\pgfsetlinewidth{0.803000pt}%
\definecolor{currentstroke}{rgb}{0.000000,0.000000,0.000000}%
\pgfsetstrokecolor{currentstroke}%
\pgfsetdash{}{0pt}%
\pgfpathmoveto{\pgfqpoint{0.800000in}{4.224000in}}%
\pgfpathlineto{\pgfqpoint{5.760000in}{4.224000in}}%
\pgfusepath{stroke}%
\end{pgfscope}%
\begin{pgfscope}%
\definecolor{textcolor}{rgb}{0.000000,0.000000,0.000000}%
\pgfsetstrokecolor{textcolor}%
\pgfsetfillcolor{textcolor}%
\pgftext[x=3.280000in,y=4.307333in,,base]{\color{textcolor}\sffamily\fontsize{9.600000}{11.520000}\selectfont Função de Perda em Treino (compIV)}%
\end{pgfscope}%
\begin{pgfscope}%
\pgfsetbuttcap%
\pgfsetmiterjoin%
\definecolor{currentfill}{rgb}{1.000000,1.000000,1.000000}%
\pgfsetfillcolor{currentfill}%
\pgfsetfillopacity{0.800000}%
\pgfsetlinewidth{1.003750pt}%
\definecolor{currentstroke}{rgb}{0.800000,0.800000,0.800000}%
\pgfsetstrokecolor{currentstroke}%
\pgfsetstrokeopacity{0.800000}%
\pgfsetdash{}{0pt}%
\pgfpathmoveto{\pgfqpoint{4.635586in}{3.808723in}}%
\pgfpathlineto{\pgfqpoint{5.682222in}{3.808723in}}%
\pgfpathquadraticcurveto{\pgfqpoint{5.704444in}{3.808723in}}{\pgfqpoint{5.704444in}{3.830945in}}%
\pgfpathlineto{\pgfqpoint{5.704444in}{4.146222in}}%
\pgfpathquadraticcurveto{\pgfqpoint{5.704444in}{4.168444in}}{\pgfqpoint{5.682222in}{4.168444in}}%
\pgfpathlineto{\pgfqpoint{4.635586in}{4.168444in}}%
\pgfpathquadraticcurveto{\pgfqpoint{4.613364in}{4.168444in}}{\pgfqpoint{4.613364in}{4.146222in}}%
\pgfpathlineto{\pgfqpoint{4.613364in}{3.830945in}}%
\pgfpathquadraticcurveto{\pgfqpoint{4.613364in}{3.808723in}}{\pgfqpoint{4.635586in}{3.808723in}}%
\pgfpathlineto{\pgfqpoint{4.635586in}{3.808723in}}%
\pgfpathclose%
\pgfusepath{stroke,fill}%
\end{pgfscope}%
\begin{pgfscope}%
\pgfsetrectcap%
\pgfsetroundjoin%
\pgfsetlinewidth{1.505625pt}%
\definecolor{currentstroke}{rgb}{0.121569,0.466667,0.705882}%
\pgfsetstrokecolor{currentstroke}%
\pgfsetdash{}{0pt}%
\pgfpathmoveto{\pgfqpoint{4.657808in}{4.078470in}}%
\pgfpathlineto{\pgfqpoint{4.768919in}{4.078470in}}%
\pgfpathlineto{\pgfqpoint{4.880030in}{4.078470in}}%
\pgfusepath{stroke}%
\end{pgfscope}%
\begin{pgfscope}%
\definecolor{textcolor}{rgb}{0.000000,0.000000,0.000000}%
\pgfsetstrokecolor{textcolor}%
\pgfsetfillcolor{textcolor}%
\pgftext[x=4.968919in,y=4.039582in,left,base]{\color{textcolor}\sffamily\fontsize{8.000000}{9.600000}\selectfont Treinamento}%
\end{pgfscope}%
\begin{pgfscope}%
\pgfsetrectcap%
\pgfsetroundjoin%
\pgfsetlinewidth{1.505625pt}%
\definecolor{currentstroke}{rgb}{1.000000,0.498039,0.054902}%
\pgfsetstrokecolor{currentstroke}%
\pgfsetdash{}{0pt}%
\pgfpathmoveto{\pgfqpoint{4.657808in}{3.915168in}}%
\pgfpathlineto{\pgfqpoint{4.768919in}{3.915168in}}%
\pgfpathlineto{\pgfqpoint{4.880030in}{3.915168in}}%
\pgfusepath{stroke}%
\end{pgfscope}%
\begin{pgfscope}%
\definecolor{textcolor}{rgb}{0.000000,0.000000,0.000000}%
\pgfsetstrokecolor{textcolor}%
\pgfsetfillcolor{textcolor}%
\pgftext[x=4.968919in,y=3.876279in,left,base]{\color{textcolor}\sffamily\fontsize{8.000000}{9.600000}\selectfont Validação}%
\end{pgfscope}%
\end{pgfpicture}%
\makeatother%
\endgroup%
}
    \end{minipage}
    \begin{minipage}{0.45\textwidth}
        \resizebox{\textwidth}{!}{%% Creator: Matplotlib, PGF backend
%%
%% To include the figure in your LaTeX document, write
%%   \input{<filename>.pgf}
%%
%% Make sure the required packages are loaded in your preamble
%%   \usepackage{pgf}
%%
%% Also ensure that all the required font packages are loaded; for instance,
%% the lmodern package is sometimes necessary when using math font.
%%   \usepackage{lmodern}
%%
%% Figures using additional raster images can only be included by \input if
%% they are in the same directory as the main LaTeX file. For loading figures
%% from other directories you can use the `import` package
%%   \usepackage{import}
%%
%% and then include the figures with
%%   \import{<path to file>}{<filename>.pgf}
%%
%% Matplotlib used the following preamble
%%   \def\mathdefault#1{#1}
%%   \everymath=\expandafter{\the\everymath\displaystyle}
%%
%%   \usepackage{fontspec}
%%   \setmainfont{DejaVuSerif.ttf}[Path=\detokenize{/home/josemayer/.local/lib/python3.9/site-packages/matplotlib/mpl-data/fonts/ttf/}]
%%   \setsansfont{DejaVuSans.ttf}[Path=\detokenize{/home/josemayer/.local/lib/python3.9/site-packages/matplotlib/mpl-data/fonts/ttf/}]
%%   \setmonofont{DejaVuSansMono.ttf}[Path=\detokenize{/home/josemayer/.local/lib/python3.9/site-packages/matplotlib/mpl-data/fonts/ttf/}]
%%   \makeatletter\@ifpackageloaded{underscore}{}{\usepackage[strings]{underscore}}\makeatother
%%
\begingroup%
\makeatletter%
\begin{pgfpicture}%
\pgfpathrectangle{\pgfpointorigin}{\pgfqpoint{6.400000in}{4.800000in}}%
\pgfusepath{use as bounding box, clip}%
\begin{pgfscope}%
\pgfsetbuttcap%
\pgfsetmiterjoin%
\definecolor{currentfill}{rgb}{1.000000,1.000000,1.000000}%
\pgfsetfillcolor{currentfill}%
\pgfsetlinewidth{0.000000pt}%
\definecolor{currentstroke}{rgb}{1.000000,1.000000,1.000000}%
\pgfsetstrokecolor{currentstroke}%
\pgfsetdash{}{0pt}%
\pgfpathmoveto{\pgfqpoint{0.000000in}{0.000000in}}%
\pgfpathlineto{\pgfqpoint{6.400000in}{0.000000in}}%
\pgfpathlineto{\pgfqpoint{6.400000in}{4.800000in}}%
\pgfpathlineto{\pgfqpoint{0.000000in}{4.800000in}}%
\pgfpathlineto{\pgfqpoint{0.000000in}{0.000000in}}%
\pgfpathclose%
\pgfusepath{fill}%
\end{pgfscope}%
\begin{pgfscope}%
\pgfsetbuttcap%
\pgfsetmiterjoin%
\definecolor{currentfill}{rgb}{1.000000,1.000000,1.000000}%
\pgfsetfillcolor{currentfill}%
\pgfsetlinewidth{0.000000pt}%
\definecolor{currentstroke}{rgb}{0.000000,0.000000,0.000000}%
\pgfsetstrokecolor{currentstroke}%
\pgfsetstrokeopacity{0.000000}%
\pgfsetdash{}{0pt}%
\pgfpathmoveto{\pgfqpoint{0.800000in}{0.528000in}}%
\pgfpathlineto{\pgfqpoint{5.760000in}{0.528000in}}%
\pgfpathlineto{\pgfqpoint{5.760000in}{4.224000in}}%
\pgfpathlineto{\pgfqpoint{0.800000in}{4.224000in}}%
\pgfpathlineto{\pgfqpoint{0.800000in}{0.528000in}}%
\pgfpathclose%
\pgfusepath{fill}%
\end{pgfscope}%
\begin{pgfscope}%
\pgfsetbuttcap%
\pgfsetroundjoin%
\definecolor{currentfill}{rgb}{0.000000,0.000000,0.000000}%
\pgfsetfillcolor{currentfill}%
\pgfsetlinewidth{0.803000pt}%
\definecolor{currentstroke}{rgb}{0.000000,0.000000,0.000000}%
\pgfsetstrokecolor{currentstroke}%
\pgfsetdash{}{0pt}%
\pgfsys@defobject{currentmarker}{\pgfqpoint{0.000000in}{-0.048611in}}{\pgfqpoint{0.000000in}{0.000000in}}{%
\pgfpathmoveto{\pgfqpoint{0.000000in}{0.000000in}}%
\pgfpathlineto{\pgfqpoint{0.000000in}{-0.048611in}}%
\pgfusepath{stroke,fill}%
}%
\begin{pgfscope}%
\pgfsys@transformshift{1.025455in}{0.528000in}%
\pgfsys@useobject{currentmarker}{}%
\end{pgfscope}%
\end{pgfscope}%
\begin{pgfscope}%
\definecolor{textcolor}{rgb}{0.000000,0.000000,0.000000}%
\pgfsetstrokecolor{textcolor}%
\pgfsetfillcolor{textcolor}%
\pgftext[x=1.025455in,y=0.430778in,,top]{\color{textcolor}{\sffamily\fontsize{8.000000}{9.600000}\selectfont\catcode`\^=\active\def^{\ifmmode\sp\else\^{}\fi}\catcode`\%=\active\def%{\%}0}}%
\end{pgfscope}%
\begin{pgfscope}%
\pgfsetbuttcap%
\pgfsetroundjoin%
\definecolor{currentfill}{rgb}{0.000000,0.000000,0.000000}%
\pgfsetfillcolor{currentfill}%
\pgfsetlinewidth{0.803000pt}%
\definecolor{currentstroke}{rgb}{0.000000,0.000000,0.000000}%
\pgfsetstrokecolor{currentstroke}%
\pgfsetdash{}{0pt}%
\pgfsys@defobject{currentmarker}{\pgfqpoint{0.000000in}{-0.048611in}}{\pgfqpoint{0.000000in}{0.000000in}}{%
\pgfpathmoveto{\pgfqpoint{0.000000in}{0.000000in}}%
\pgfpathlineto{\pgfqpoint{0.000000in}{-0.048611in}}%
\pgfusepath{stroke,fill}%
}%
\begin{pgfscope}%
\pgfsys@transformshift{1.945677in}{0.528000in}%
\pgfsys@useobject{currentmarker}{}%
\end{pgfscope}%
\end{pgfscope}%
\begin{pgfscope}%
\definecolor{textcolor}{rgb}{0.000000,0.000000,0.000000}%
\pgfsetstrokecolor{textcolor}%
\pgfsetfillcolor{textcolor}%
\pgftext[x=1.945677in,y=0.430778in,,top]{\color{textcolor}{\sffamily\fontsize{8.000000}{9.600000}\selectfont\catcode`\^=\active\def^{\ifmmode\sp\else\^{}\fi}\catcode`\%=\active\def%{\%}10}}%
\end{pgfscope}%
\begin{pgfscope}%
\pgfsetbuttcap%
\pgfsetroundjoin%
\definecolor{currentfill}{rgb}{0.000000,0.000000,0.000000}%
\pgfsetfillcolor{currentfill}%
\pgfsetlinewidth{0.803000pt}%
\definecolor{currentstroke}{rgb}{0.000000,0.000000,0.000000}%
\pgfsetstrokecolor{currentstroke}%
\pgfsetdash{}{0pt}%
\pgfsys@defobject{currentmarker}{\pgfqpoint{0.000000in}{-0.048611in}}{\pgfqpoint{0.000000in}{0.000000in}}{%
\pgfpathmoveto{\pgfqpoint{0.000000in}{0.000000in}}%
\pgfpathlineto{\pgfqpoint{0.000000in}{-0.048611in}}%
\pgfusepath{stroke,fill}%
}%
\begin{pgfscope}%
\pgfsys@transformshift{2.865900in}{0.528000in}%
\pgfsys@useobject{currentmarker}{}%
\end{pgfscope}%
\end{pgfscope}%
\begin{pgfscope}%
\definecolor{textcolor}{rgb}{0.000000,0.000000,0.000000}%
\pgfsetstrokecolor{textcolor}%
\pgfsetfillcolor{textcolor}%
\pgftext[x=2.865900in,y=0.430778in,,top]{\color{textcolor}{\sffamily\fontsize{8.000000}{9.600000}\selectfont\catcode`\^=\active\def^{\ifmmode\sp\else\^{}\fi}\catcode`\%=\active\def%{\%}20}}%
\end{pgfscope}%
\begin{pgfscope}%
\pgfsetbuttcap%
\pgfsetroundjoin%
\definecolor{currentfill}{rgb}{0.000000,0.000000,0.000000}%
\pgfsetfillcolor{currentfill}%
\pgfsetlinewidth{0.803000pt}%
\definecolor{currentstroke}{rgb}{0.000000,0.000000,0.000000}%
\pgfsetstrokecolor{currentstroke}%
\pgfsetdash{}{0pt}%
\pgfsys@defobject{currentmarker}{\pgfqpoint{0.000000in}{-0.048611in}}{\pgfqpoint{0.000000in}{0.000000in}}{%
\pgfpathmoveto{\pgfqpoint{0.000000in}{0.000000in}}%
\pgfpathlineto{\pgfqpoint{0.000000in}{-0.048611in}}%
\pgfusepath{stroke,fill}%
}%
\begin{pgfscope}%
\pgfsys@transformshift{3.786122in}{0.528000in}%
\pgfsys@useobject{currentmarker}{}%
\end{pgfscope}%
\end{pgfscope}%
\begin{pgfscope}%
\definecolor{textcolor}{rgb}{0.000000,0.000000,0.000000}%
\pgfsetstrokecolor{textcolor}%
\pgfsetfillcolor{textcolor}%
\pgftext[x=3.786122in,y=0.430778in,,top]{\color{textcolor}{\sffamily\fontsize{8.000000}{9.600000}\selectfont\catcode`\^=\active\def^{\ifmmode\sp\else\^{}\fi}\catcode`\%=\active\def%{\%}30}}%
\end{pgfscope}%
\begin{pgfscope}%
\pgfsetbuttcap%
\pgfsetroundjoin%
\definecolor{currentfill}{rgb}{0.000000,0.000000,0.000000}%
\pgfsetfillcolor{currentfill}%
\pgfsetlinewidth{0.803000pt}%
\definecolor{currentstroke}{rgb}{0.000000,0.000000,0.000000}%
\pgfsetstrokecolor{currentstroke}%
\pgfsetdash{}{0pt}%
\pgfsys@defobject{currentmarker}{\pgfqpoint{0.000000in}{-0.048611in}}{\pgfqpoint{0.000000in}{0.000000in}}{%
\pgfpathmoveto{\pgfqpoint{0.000000in}{0.000000in}}%
\pgfpathlineto{\pgfqpoint{0.000000in}{-0.048611in}}%
\pgfusepath{stroke,fill}%
}%
\begin{pgfscope}%
\pgfsys@transformshift{4.706345in}{0.528000in}%
\pgfsys@useobject{currentmarker}{}%
\end{pgfscope}%
\end{pgfscope}%
\begin{pgfscope}%
\definecolor{textcolor}{rgb}{0.000000,0.000000,0.000000}%
\pgfsetstrokecolor{textcolor}%
\pgfsetfillcolor{textcolor}%
\pgftext[x=4.706345in,y=0.430778in,,top]{\color{textcolor}{\sffamily\fontsize{8.000000}{9.600000}\selectfont\catcode`\^=\active\def^{\ifmmode\sp\else\^{}\fi}\catcode`\%=\active\def%{\%}40}}%
\end{pgfscope}%
\begin{pgfscope}%
\pgfsetbuttcap%
\pgfsetroundjoin%
\definecolor{currentfill}{rgb}{0.000000,0.000000,0.000000}%
\pgfsetfillcolor{currentfill}%
\pgfsetlinewidth{0.803000pt}%
\definecolor{currentstroke}{rgb}{0.000000,0.000000,0.000000}%
\pgfsetstrokecolor{currentstroke}%
\pgfsetdash{}{0pt}%
\pgfsys@defobject{currentmarker}{\pgfqpoint{0.000000in}{-0.048611in}}{\pgfqpoint{0.000000in}{0.000000in}}{%
\pgfpathmoveto{\pgfqpoint{0.000000in}{0.000000in}}%
\pgfpathlineto{\pgfqpoint{0.000000in}{-0.048611in}}%
\pgfusepath{stroke,fill}%
}%
\begin{pgfscope}%
\pgfsys@transformshift{5.626568in}{0.528000in}%
\pgfsys@useobject{currentmarker}{}%
\end{pgfscope}%
\end{pgfscope}%
\begin{pgfscope}%
\definecolor{textcolor}{rgb}{0.000000,0.000000,0.000000}%
\pgfsetstrokecolor{textcolor}%
\pgfsetfillcolor{textcolor}%
\pgftext[x=5.626568in,y=0.430778in,,top]{\color{textcolor}{\sffamily\fontsize{8.000000}{9.600000}\selectfont\catcode`\^=\active\def^{\ifmmode\sp\else\^{}\fi}\catcode`\%=\active\def%{\%}50}}%
\end{pgfscope}%
\begin{pgfscope}%
\definecolor{textcolor}{rgb}{0.000000,0.000000,0.000000}%
\pgfsetstrokecolor{textcolor}%
\pgfsetfillcolor{textcolor}%
\pgftext[x=3.280000in,y=0.267692in,,top]{\color{textcolor}{\sffamily\fontsize{8.000000}{9.600000}\selectfont\catcode`\^=\active\def^{\ifmmode\sp\else\^{}\fi}\catcode`\%=\active\def%{\%}Época}}%
\end{pgfscope}%
\begin{pgfscope}%
\pgfsetbuttcap%
\pgfsetroundjoin%
\definecolor{currentfill}{rgb}{0.000000,0.000000,0.000000}%
\pgfsetfillcolor{currentfill}%
\pgfsetlinewidth{0.803000pt}%
\definecolor{currentstroke}{rgb}{0.000000,0.000000,0.000000}%
\pgfsetstrokecolor{currentstroke}%
\pgfsetdash{}{0pt}%
\pgfsys@defobject{currentmarker}{\pgfqpoint{-0.048611in}{0.000000in}}{\pgfqpoint{-0.000000in}{0.000000in}}{%
\pgfpathmoveto{\pgfqpoint{-0.000000in}{0.000000in}}%
\pgfpathlineto{\pgfqpoint{-0.048611in}{0.000000in}}%
\pgfusepath{stroke,fill}%
}%
\begin{pgfscope}%
\pgfsys@transformshift{0.800000in}{0.612590in}%
\pgfsys@useobject{currentmarker}{}%
\end{pgfscope}%
\end{pgfscope}%
\begin{pgfscope}%
\definecolor{textcolor}{rgb}{0.000000,0.000000,0.000000}%
\pgfsetstrokecolor{textcolor}%
\pgfsetfillcolor{textcolor}%
\pgftext[x=0.455382in, y=0.570380in, left, base]{\color{textcolor}{\sffamily\fontsize{8.000000}{9.600000}\selectfont\catcode`\^=\active\def^{\ifmmode\sp\else\^{}\fi}\catcode`\%=\active\def%{\%}0.00}}%
\end{pgfscope}%
\begin{pgfscope}%
\pgfsetbuttcap%
\pgfsetroundjoin%
\definecolor{currentfill}{rgb}{0.000000,0.000000,0.000000}%
\pgfsetfillcolor{currentfill}%
\pgfsetlinewidth{0.803000pt}%
\definecolor{currentstroke}{rgb}{0.000000,0.000000,0.000000}%
\pgfsetstrokecolor{currentstroke}%
\pgfsetdash{}{0pt}%
\pgfsys@defobject{currentmarker}{\pgfqpoint{-0.048611in}{0.000000in}}{\pgfqpoint{-0.000000in}{0.000000in}}{%
\pgfpathmoveto{\pgfqpoint{-0.000000in}{0.000000in}}%
\pgfpathlineto{\pgfqpoint{-0.048611in}{0.000000in}}%
\pgfusepath{stroke,fill}%
}%
\begin{pgfscope}%
\pgfsys@transformshift{0.800000in}{1.048897in}%
\pgfsys@useobject{currentmarker}{}%
\end{pgfscope}%
\end{pgfscope}%
\begin{pgfscope}%
\definecolor{textcolor}{rgb}{0.000000,0.000000,0.000000}%
\pgfsetstrokecolor{textcolor}%
\pgfsetfillcolor{textcolor}%
\pgftext[x=0.455382in, y=1.006688in, left, base]{\color{textcolor}{\sffamily\fontsize{8.000000}{9.600000}\selectfont\catcode`\^=\active\def^{\ifmmode\sp\else\^{}\fi}\catcode`\%=\active\def%{\%}0.25}}%
\end{pgfscope}%
\begin{pgfscope}%
\pgfsetbuttcap%
\pgfsetroundjoin%
\definecolor{currentfill}{rgb}{0.000000,0.000000,0.000000}%
\pgfsetfillcolor{currentfill}%
\pgfsetlinewidth{0.803000pt}%
\definecolor{currentstroke}{rgb}{0.000000,0.000000,0.000000}%
\pgfsetstrokecolor{currentstroke}%
\pgfsetdash{}{0pt}%
\pgfsys@defobject{currentmarker}{\pgfqpoint{-0.048611in}{0.000000in}}{\pgfqpoint{-0.000000in}{0.000000in}}{%
\pgfpathmoveto{\pgfqpoint{-0.000000in}{0.000000in}}%
\pgfpathlineto{\pgfqpoint{-0.048611in}{0.000000in}}%
\pgfusepath{stroke,fill}%
}%
\begin{pgfscope}%
\pgfsys@transformshift{0.800000in}{1.485204in}%
\pgfsys@useobject{currentmarker}{}%
\end{pgfscope}%
\end{pgfscope}%
\begin{pgfscope}%
\definecolor{textcolor}{rgb}{0.000000,0.000000,0.000000}%
\pgfsetstrokecolor{textcolor}%
\pgfsetfillcolor{textcolor}%
\pgftext[x=0.455382in, y=1.442995in, left, base]{\color{textcolor}{\sffamily\fontsize{8.000000}{9.600000}\selectfont\catcode`\^=\active\def^{\ifmmode\sp\else\^{}\fi}\catcode`\%=\active\def%{\%}0.50}}%
\end{pgfscope}%
\begin{pgfscope}%
\pgfsetbuttcap%
\pgfsetroundjoin%
\definecolor{currentfill}{rgb}{0.000000,0.000000,0.000000}%
\pgfsetfillcolor{currentfill}%
\pgfsetlinewidth{0.803000pt}%
\definecolor{currentstroke}{rgb}{0.000000,0.000000,0.000000}%
\pgfsetstrokecolor{currentstroke}%
\pgfsetdash{}{0pt}%
\pgfsys@defobject{currentmarker}{\pgfqpoint{-0.048611in}{0.000000in}}{\pgfqpoint{-0.000000in}{0.000000in}}{%
\pgfpathmoveto{\pgfqpoint{-0.000000in}{0.000000in}}%
\pgfpathlineto{\pgfqpoint{-0.048611in}{0.000000in}}%
\pgfusepath{stroke,fill}%
}%
\begin{pgfscope}%
\pgfsys@transformshift{0.800000in}{1.921512in}%
\pgfsys@useobject{currentmarker}{}%
\end{pgfscope}%
\end{pgfscope}%
\begin{pgfscope}%
\definecolor{textcolor}{rgb}{0.000000,0.000000,0.000000}%
\pgfsetstrokecolor{textcolor}%
\pgfsetfillcolor{textcolor}%
\pgftext[x=0.455382in, y=1.879302in, left, base]{\color{textcolor}{\sffamily\fontsize{8.000000}{9.600000}\selectfont\catcode`\^=\active\def^{\ifmmode\sp\else\^{}\fi}\catcode`\%=\active\def%{\%}0.75}}%
\end{pgfscope}%
\begin{pgfscope}%
\pgfsetbuttcap%
\pgfsetroundjoin%
\definecolor{currentfill}{rgb}{0.000000,0.000000,0.000000}%
\pgfsetfillcolor{currentfill}%
\pgfsetlinewidth{0.803000pt}%
\definecolor{currentstroke}{rgb}{0.000000,0.000000,0.000000}%
\pgfsetstrokecolor{currentstroke}%
\pgfsetdash{}{0pt}%
\pgfsys@defobject{currentmarker}{\pgfqpoint{-0.048611in}{0.000000in}}{\pgfqpoint{-0.000000in}{0.000000in}}{%
\pgfpathmoveto{\pgfqpoint{-0.000000in}{0.000000in}}%
\pgfpathlineto{\pgfqpoint{-0.048611in}{0.000000in}}%
\pgfusepath{stroke,fill}%
}%
\begin{pgfscope}%
\pgfsys@transformshift{0.800000in}{2.357819in}%
\pgfsys@useobject{currentmarker}{}%
\end{pgfscope}%
\end{pgfscope}%
\begin{pgfscope}%
\definecolor{textcolor}{rgb}{0.000000,0.000000,0.000000}%
\pgfsetstrokecolor{textcolor}%
\pgfsetfillcolor{textcolor}%
\pgftext[x=0.455382in, y=2.315610in, left, base]{\color{textcolor}{\sffamily\fontsize{8.000000}{9.600000}\selectfont\catcode`\^=\active\def^{\ifmmode\sp\else\^{}\fi}\catcode`\%=\active\def%{\%}1.00}}%
\end{pgfscope}%
\begin{pgfscope}%
\pgfsetbuttcap%
\pgfsetroundjoin%
\definecolor{currentfill}{rgb}{0.000000,0.000000,0.000000}%
\pgfsetfillcolor{currentfill}%
\pgfsetlinewidth{0.803000pt}%
\definecolor{currentstroke}{rgb}{0.000000,0.000000,0.000000}%
\pgfsetstrokecolor{currentstroke}%
\pgfsetdash{}{0pt}%
\pgfsys@defobject{currentmarker}{\pgfqpoint{-0.048611in}{0.000000in}}{\pgfqpoint{-0.000000in}{0.000000in}}{%
\pgfpathmoveto{\pgfqpoint{-0.000000in}{0.000000in}}%
\pgfpathlineto{\pgfqpoint{-0.048611in}{0.000000in}}%
\pgfusepath{stroke,fill}%
}%
\begin{pgfscope}%
\pgfsys@transformshift{0.800000in}{2.794126in}%
\pgfsys@useobject{currentmarker}{}%
\end{pgfscope}%
\end{pgfscope}%
\begin{pgfscope}%
\definecolor{textcolor}{rgb}{0.000000,0.000000,0.000000}%
\pgfsetstrokecolor{textcolor}%
\pgfsetfillcolor{textcolor}%
\pgftext[x=0.455382in, y=2.751917in, left, base]{\color{textcolor}{\sffamily\fontsize{8.000000}{9.600000}\selectfont\catcode`\^=\active\def^{\ifmmode\sp\else\^{}\fi}\catcode`\%=\active\def%{\%}1.25}}%
\end{pgfscope}%
\begin{pgfscope}%
\pgfsetbuttcap%
\pgfsetroundjoin%
\definecolor{currentfill}{rgb}{0.000000,0.000000,0.000000}%
\pgfsetfillcolor{currentfill}%
\pgfsetlinewidth{0.803000pt}%
\definecolor{currentstroke}{rgb}{0.000000,0.000000,0.000000}%
\pgfsetstrokecolor{currentstroke}%
\pgfsetdash{}{0pt}%
\pgfsys@defobject{currentmarker}{\pgfqpoint{-0.048611in}{0.000000in}}{\pgfqpoint{-0.000000in}{0.000000in}}{%
\pgfpathmoveto{\pgfqpoint{-0.000000in}{0.000000in}}%
\pgfpathlineto{\pgfqpoint{-0.048611in}{0.000000in}}%
\pgfusepath{stroke,fill}%
}%
\begin{pgfscope}%
\pgfsys@transformshift{0.800000in}{3.230434in}%
\pgfsys@useobject{currentmarker}{}%
\end{pgfscope}%
\end{pgfscope}%
\begin{pgfscope}%
\definecolor{textcolor}{rgb}{0.000000,0.000000,0.000000}%
\pgfsetstrokecolor{textcolor}%
\pgfsetfillcolor{textcolor}%
\pgftext[x=0.455382in, y=3.188225in, left, base]{\color{textcolor}{\sffamily\fontsize{8.000000}{9.600000}\selectfont\catcode`\^=\active\def^{\ifmmode\sp\else\^{}\fi}\catcode`\%=\active\def%{\%}1.50}}%
\end{pgfscope}%
\begin{pgfscope}%
\pgfsetbuttcap%
\pgfsetroundjoin%
\definecolor{currentfill}{rgb}{0.000000,0.000000,0.000000}%
\pgfsetfillcolor{currentfill}%
\pgfsetlinewidth{0.803000pt}%
\definecolor{currentstroke}{rgb}{0.000000,0.000000,0.000000}%
\pgfsetstrokecolor{currentstroke}%
\pgfsetdash{}{0pt}%
\pgfsys@defobject{currentmarker}{\pgfqpoint{-0.048611in}{0.000000in}}{\pgfqpoint{-0.000000in}{0.000000in}}{%
\pgfpathmoveto{\pgfqpoint{-0.000000in}{0.000000in}}%
\pgfpathlineto{\pgfqpoint{-0.048611in}{0.000000in}}%
\pgfusepath{stroke,fill}%
}%
\begin{pgfscope}%
\pgfsys@transformshift{0.800000in}{3.666741in}%
\pgfsys@useobject{currentmarker}{}%
\end{pgfscope}%
\end{pgfscope}%
\begin{pgfscope}%
\definecolor{textcolor}{rgb}{0.000000,0.000000,0.000000}%
\pgfsetstrokecolor{textcolor}%
\pgfsetfillcolor{textcolor}%
\pgftext[x=0.455382in, y=3.624532in, left, base]{\color{textcolor}{\sffamily\fontsize{8.000000}{9.600000}\selectfont\catcode`\^=\active\def^{\ifmmode\sp\else\^{}\fi}\catcode`\%=\active\def%{\%}1.75}}%
\end{pgfscope}%
\begin{pgfscope}%
\pgfsetbuttcap%
\pgfsetroundjoin%
\definecolor{currentfill}{rgb}{0.000000,0.000000,0.000000}%
\pgfsetfillcolor{currentfill}%
\pgfsetlinewidth{0.803000pt}%
\definecolor{currentstroke}{rgb}{0.000000,0.000000,0.000000}%
\pgfsetstrokecolor{currentstroke}%
\pgfsetdash{}{0pt}%
\pgfsys@defobject{currentmarker}{\pgfqpoint{-0.048611in}{0.000000in}}{\pgfqpoint{-0.000000in}{0.000000in}}{%
\pgfpathmoveto{\pgfqpoint{-0.000000in}{0.000000in}}%
\pgfpathlineto{\pgfqpoint{-0.048611in}{0.000000in}}%
\pgfusepath{stroke,fill}%
}%
\begin{pgfscope}%
\pgfsys@transformshift{0.800000in}{4.103048in}%
\pgfsys@useobject{currentmarker}{}%
\end{pgfscope}%
\end{pgfscope}%
\begin{pgfscope}%
\definecolor{textcolor}{rgb}{0.000000,0.000000,0.000000}%
\pgfsetstrokecolor{textcolor}%
\pgfsetfillcolor{textcolor}%
\pgftext[x=0.455382in, y=4.060839in, left, base]{\color{textcolor}{\sffamily\fontsize{8.000000}{9.600000}\selectfont\catcode`\^=\active\def^{\ifmmode\sp\else\^{}\fi}\catcode`\%=\active\def%{\%}2.00}}%
\end{pgfscope}%
\begin{pgfscope}%
\definecolor{textcolor}{rgb}{0.000000,0.000000,0.000000}%
\pgfsetstrokecolor{textcolor}%
\pgfsetfillcolor{textcolor}%
\pgftext[x=0.399826in,y=2.376000in,,bottom,rotate=90.000000]{\color{textcolor}{\sffamily\fontsize{8.000000}{9.600000}\selectfont\catcode`\^=\active\def^{\ifmmode\sp\else\^{}\fi}\catcode`\%=\active\def%{\%}Perda}}%
\end{pgfscope}%
\begin{pgfscope}%
\pgfpathrectangle{\pgfqpoint{0.800000in}{0.528000in}}{\pgfqpoint{4.960000in}{3.696000in}}%
\pgfusepath{clip}%
\pgfsetrectcap%
\pgfsetroundjoin%
\pgfsetlinewidth{1.505625pt}%
\definecolor{currentstroke}{rgb}{0.121569,0.466667,0.705882}%
\pgfsetstrokecolor{currentstroke}%
\pgfsetdash{}{0pt}%
\pgfpathmoveto{\pgfqpoint{1.025455in}{3.119938in}}%
\pgfpathlineto{\pgfqpoint{1.117477in}{2.989744in}}%
\pgfpathlineto{\pgfqpoint{1.209499in}{2.696378in}}%
\pgfpathlineto{\pgfqpoint{1.301521in}{1.884513in}}%
\pgfpathlineto{\pgfqpoint{1.393544in}{1.624205in}}%
\pgfpathlineto{\pgfqpoint{1.485566in}{1.453837in}}%
\pgfpathlineto{\pgfqpoint{1.577588in}{1.352264in}}%
\pgfpathlineto{\pgfqpoint{1.669610in}{1.282472in}}%
\pgfpathlineto{\pgfqpoint{1.761633in}{1.151408in}}%
\pgfpathlineto{\pgfqpoint{1.853655in}{1.071674in}}%
\pgfpathlineto{\pgfqpoint{1.945677in}{0.950429in}}%
\pgfpathlineto{\pgfqpoint{2.037699in}{0.889942in}}%
\pgfpathlineto{\pgfqpoint{2.129722in}{0.881704in}}%
\pgfpathlineto{\pgfqpoint{2.221744in}{0.821739in}}%
\pgfpathlineto{\pgfqpoint{2.313766in}{0.786799in}}%
\pgfpathlineto{\pgfqpoint{2.405788in}{0.783508in}}%
\pgfpathlineto{\pgfqpoint{2.497811in}{0.776452in}}%
\pgfpathlineto{\pgfqpoint{2.589833in}{0.761977in}}%
\pgfpathlineto{\pgfqpoint{2.681855in}{0.747637in}}%
\pgfpathlineto{\pgfqpoint{2.773878in}{0.739926in}}%
\pgfpathlineto{\pgfqpoint{2.865900in}{0.748954in}}%
\pgfpathlineto{\pgfqpoint{2.957922in}{0.888345in}}%
\pgfpathlineto{\pgfqpoint{3.049944in}{0.716032in}}%
\pgfpathlineto{\pgfqpoint{3.141967in}{0.700387in}}%
\pgfpathlineto{\pgfqpoint{3.233989in}{1.591149in}}%
\pgfpathlineto{\pgfqpoint{3.326011in}{1.438249in}}%
\pgfpathlineto{\pgfqpoint{3.418033in}{1.429507in}}%
\pgfpathlineto{\pgfqpoint{3.510056in}{0.891629in}}%
\pgfpathlineto{\pgfqpoint{3.602078in}{0.759246in}}%
\pgfpathlineto{\pgfqpoint{3.694100in}{0.796197in}}%
\pgfpathlineto{\pgfqpoint{3.786122in}{1.113006in}}%
\pgfpathlineto{\pgfqpoint{3.878145in}{0.725077in}}%
\pgfpathlineto{\pgfqpoint{3.970167in}{0.704770in}}%
\pgfpathlineto{\pgfqpoint{4.062189in}{0.697888in}}%
\pgfpathlineto{\pgfqpoint{4.154212in}{0.706787in}}%
\pgfpathlineto{\pgfqpoint{4.246234in}{0.696000in}}%
\pgfpathlineto{\pgfqpoint{4.338256in}{1.078350in}}%
\pgfpathlineto{\pgfqpoint{4.430278in}{0.872986in}}%
\pgfpathlineto{\pgfqpoint{4.522301in}{0.790869in}}%
\pgfpathlineto{\pgfqpoint{4.614323in}{0.788792in}}%
\pgfpathlineto{\pgfqpoint{4.706345in}{0.741345in}}%
\pgfpathlineto{\pgfqpoint{4.798367in}{0.731493in}}%
\pgfpathlineto{\pgfqpoint{4.890390in}{0.707016in}}%
\pgfpathlineto{\pgfqpoint{4.982412in}{0.699439in}}%
\pgfpathlineto{\pgfqpoint{5.074434in}{0.721537in}}%
\pgfpathlineto{\pgfqpoint{5.166456in}{0.705736in}}%
\pgfpathlineto{\pgfqpoint{5.258479in}{1.041506in}}%
\pgfpathlineto{\pgfqpoint{5.350501in}{1.369550in}}%
\pgfpathlineto{\pgfqpoint{5.442523in}{1.226245in}}%
\pgfpathlineto{\pgfqpoint{5.534545in}{1.123301in}}%
\pgfusepath{stroke}%
\end{pgfscope}%
\begin{pgfscope}%
\pgfpathrectangle{\pgfqpoint{0.800000in}{0.528000in}}{\pgfqpoint{4.960000in}{3.696000in}}%
\pgfusepath{clip}%
\pgfsetrectcap%
\pgfsetroundjoin%
\pgfsetlinewidth{1.505625pt}%
\definecolor{currentstroke}{rgb}{1.000000,0.498039,0.054902}%
\pgfsetstrokecolor{currentstroke}%
\pgfsetdash{}{0pt}%
\pgfpathmoveto{\pgfqpoint{1.025455in}{2.428249in}}%
\pgfpathlineto{\pgfqpoint{1.117477in}{4.056000in}}%
\pgfpathlineto{\pgfqpoint{1.209499in}{2.695513in}}%
\pgfpathlineto{\pgfqpoint{1.301521in}{2.193393in}}%
\pgfpathlineto{\pgfqpoint{1.393544in}{2.246843in}}%
\pgfpathlineto{\pgfqpoint{1.485566in}{2.336532in}}%
\pgfpathlineto{\pgfqpoint{1.577588in}{2.218519in}}%
\pgfpathlineto{\pgfqpoint{1.669610in}{2.324577in}}%
\pgfpathlineto{\pgfqpoint{1.761633in}{2.304950in}}%
\pgfpathlineto{\pgfqpoint{1.853655in}{2.343794in}}%
\pgfpathlineto{\pgfqpoint{1.945677in}{2.269573in}}%
\pgfpathlineto{\pgfqpoint{2.037699in}{2.228619in}}%
\pgfpathlineto{\pgfqpoint{2.129722in}{2.274159in}}%
\pgfpathlineto{\pgfqpoint{2.221744in}{2.410140in}}%
\pgfpathlineto{\pgfqpoint{2.313766in}{2.271489in}}%
\pgfpathlineto{\pgfqpoint{2.405788in}{2.271881in}}%
\pgfpathlineto{\pgfqpoint{2.497811in}{2.224478in}}%
\pgfpathlineto{\pgfqpoint{2.589833in}{2.207369in}}%
\pgfpathlineto{\pgfqpoint{2.681855in}{2.306046in}}%
\pgfpathlineto{\pgfqpoint{2.773878in}{2.161939in}}%
\pgfpathlineto{\pgfqpoint{2.865900in}{2.850450in}}%
\pgfpathlineto{\pgfqpoint{2.957922in}{2.190075in}}%
\pgfpathlineto{\pgfqpoint{3.049944in}{2.247146in}}%
\pgfpathlineto{\pgfqpoint{3.141967in}{2.259570in}}%
\pgfpathlineto{\pgfqpoint{3.233989in}{2.370257in}}%
\pgfpathlineto{\pgfqpoint{3.326011in}{3.776304in}}%
\pgfpathlineto{\pgfqpoint{3.418033in}{2.458849in}}%
\pgfpathlineto{\pgfqpoint{3.510056in}{2.232564in}}%
\pgfpathlineto{\pgfqpoint{3.602078in}{2.155588in}}%
\pgfpathlineto{\pgfqpoint{3.694100in}{2.363222in}}%
\pgfpathlineto{\pgfqpoint{3.786122in}{2.255463in}}%
\pgfpathlineto{\pgfqpoint{3.878145in}{2.352552in}}%
\pgfpathlineto{\pgfqpoint{3.970167in}{2.235180in}}%
\pgfpathlineto{\pgfqpoint{4.062189in}{2.363191in}}%
\pgfpathlineto{\pgfqpoint{4.154212in}{2.347507in}}%
\pgfpathlineto{\pgfqpoint{4.246234in}{2.297230in}}%
\pgfpathlineto{\pgfqpoint{4.338256in}{2.230987in}}%
\pgfpathlineto{\pgfqpoint{4.430278in}{2.270502in}}%
\pgfpathlineto{\pgfqpoint{4.522301in}{2.212959in}}%
\pgfpathlineto{\pgfqpoint{4.614323in}{2.146371in}}%
\pgfpathlineto{\pgfqpoint{4.706345in}{2.327056in}}%
\pgfpathlineto{\pgfqpoint{4.798367in}{2.130404in}}%
\pgfpathlineto{\pgfqpoint{4.890390in}{2.306189in}}%
\pgfpathlineto{\pgfqpoint{4.982412in}{2.380925in}}%
\pgfpathlineto{\pgfqpoint{5.074434in}{2.335822in}}%
\pgfpathlineto{\pgfqpoint{5.166456in}{2.216400in}}%
\pgfpathlineto{\pgfqpoint{5.258479in}{2.194965in}}%
\pgfpathlineto{\pgfqpoint{5.350501in}{2.257279in}}%
\pgfpathlineto{\pgfqpoint{5.442523in}{2.370871in}}%
\pgfpathlineto{\pgfqpoint{5.534545in}{2.189017in}}%
\pgfusepath{stroke}%
\end{pgfscope}%
\begin{pgfscope}%
\pgfsetrectcap%
\pgfsetmiterjoin%
\pgfsetlinewidth{0.803000pt}%
\definecolor{currentstroke}{rgb}{0.000000,0.000000,0.000000}%
\pgfsetstrokecolor{currentstroke}%
\pgfsetdash{}{0pt}%
\pgfpathmoveto{\pgfqpoint{0.800000in}{0.528000in}}%
\pgfpathlineto{\pgfqpoint{0.800000in}{4.224000in}}%
\pgfusepath{stroke}%
\end{pgfscope}%
\begin{pgfscope}%
\pgfsetrectcap%
\pgfsetmiterjoin%
\pgfsetlinewidth{0.803000pt}%
\definecolor{currentstroke}{rgb}{0.000000,0.000000,0.000000}%
\pgfsetstrokecolor{currentstroke}%
\pgfsetdash{}{0pt}%
\pgfpathmoveto{\pgfqpoint{5.760000in}{0.528000in}}%
\pgfpathlineto{\pgfqpoint{5.760000in}{4.224000in}}%
\pgfusepath{stroke}%
\end{pgfscope}%
\begin{pgfscope}%
\pgfsetrectcap%
\pgfsetmiterjoin%
\pgfsetlinewidth{0.803000pt}%
\definecolor{currentstroke}{rgb}{0.000000,0.000000,0.000000}%
\pgfsetstrokecolor{currentstroke}%
\pgfsetdash{}{0pt}%
\pgfpathmoveto{\pgfqpoint{0.800000in}{0.528000in}}%
\pgfpathlineto{\pgfqpoint{5.760000in}{0.528000in}}%
\pgfusepath{stroke}%
\end{pgfscope}%
\begin{pgfscope}%
\pgfsetrectcap%
\pgfsetmiterjoin%
\pgfsetlinewidth{0.803000pt}%
\definecolor{currentstroke}{rgb}{0.000000,0.000000,0.000000}%
\pgfsetstrokecolor{currentstroke}%
\pgfsetdash{}{0pt}%
\pgfpathmoveto{\pgfqpoint{0.800000in}{4.224000in}}%
\pgfpathlineto{\pgfqpoint{5.760000in}{4.224000in}}%
\pgfusepath{stroke}%
\end{pgfscope}%
\begin{pgfscope}%
\definecolor{textcolor}{rgb}{0.000000,0.000000,0.000000}%
\pgfsetstrokecolor{textcolor}%
\pgfsetfillcolor{textcolor}%
\pgftext[x=3.280000in,y=4.307333in,,base]{\color{textcolor}{\sffamily\fontsize{9.600000}{11.520000}\selectfont\catcode`\^=\active\def^{\ifmmode\sp\else\^{}\fi}\catcode`\%=\active\def%{\%}Função de Perda em Treino (compIV)}}%
\end{pgfscope}%
\begin{pgfscope}%
\pgfsetbuttcap%
\pgfsetmiterjoin%
\definecolor{currentfill}{rgb}{1.000000,1.000000,1.000000}%
\pgfsetfillcolor{currentfill}%
\pgfsetfillopacity{0.800000}%
\pgfsetlinewidth{1.003750pt}%
\definecolor{currentstroke}{rgb}{0.800000,0.800000,0.800000}%
\pgfsetstrokecolor{currentstroke}%
\pgfsetstrokeopacity{0.800000}%
\pgfsetdash{}{0pt}%
\pgfpathmoveto{\pgfqpoint{4.635586in}{3.808723in}}%
\pgfpathlineto{\pgfqpoint{5.682222in}{3.808723in}}%
\pgfpathquadraticcurveto{\pgfqpoint{5.704444in}{3.808723in}}{\pgfqpoint{5.704444in}{3.830945in}}%
\pgfpathlineto{\pgfqpoint{5.704444in}{4.146222in}}%
\pgfpathquadraticcurveto{\pgfqpoint{5.704444in}{4.168444in}}{\pgfqpoint{5.682222in}{4.168444in}}%
\pgfpathlineto{\pgfqpoint{4.635586in}{4.168444in}}%
\pgfpathquadraticcurveto{\pgfqpoint{4.613364in}{4.168444in}}{\pgfqpoint{4.613364in}{4.146222in}}%
\pgfpathlineto{\pgfqpoint{4.613364in}{3.830945in}}%
\pgfpathquadraticcurveto{\pgfqpoint{4.613364in}{3.808723in}}{\pgfqpoint{4.635586in}{3.808723in}}%
\pgfpathlineto{\pgfqpoint{4.635586in}{3.808723in}}%
\pgfpathclose%
\pgfusepath{stroke,fill}%
\end{pgfscope}%
\begin{pgfscope}%
\pgfsetrectcap%
\pgfsetroundjoin%
\pgfsetlinewidth{1.505625pt}%
\definecolor{currentstroke}{rgb}{0.121569,0.466667,0.705882}%
\pgfsetstrokecolor{currentstroke}%
\pgfsetdash{}{0pt}%
\pgfpathmoveto{\pgfqpoint{4.657808in}{4.078470in}}%
\pgfpathlineto{\pgfqpoint{4.768919in}{4.078470in}}%
\pgfpathlineto{\pgfqpoint{4.880030in}{4.078470in}}%
\pgfusepath{stroke}%
\end{pgfscope}%
\begin{pgfscope}%
\definecolor{textcolor}{rgb}{0.000000,0.000000,0.000000}%
\pgfsetstrokecolor{textcolor}%
\pgfsetfillcolor{textcolor}%
\pgftext[x=4.968919in,y=4.039582in,left,base]{\color{textcolor}{\sffamily\fontsize{8.000000}{9.600000}\selectfont\catcode`\^=\active\def^{\ifmmode\sp\else\^{}\fi}\catcode`\%=\active\def%{\%}Treinamento}}%
\end{pgfscope}%
\begin{pgfscope}%
\pgfsetrectcap%
\pgfsetroundjoin%
\pgfsetlinewidth{1.505625pt}%
\definecolor{currentstroke}{rgb}{1.000000,0.498039,0.054902}%
\pgfsetstrokecolor{currentstroke}%
\pgfsetdash{}{0pt}%
\pgfpathmoveto{\pgfqpoint{4.657808in}{3.915168in}}%
\pgfpathlineto{\pgfqpoint{4.768919in}{3.915168in}}%
\pgfpathlineto{\pgfqpoint{4.880030in}{3.915168in}}%
\pgfusepath{stroke}%
\end{pgfscope}%
\begin{pgfscope}%
\definecolor{textcolor}{rgb}{0.000000,0.000000,0.000000}%
\pgfsetstrokecolor{textcolor}%
\pgfsetfillcolor{textcolor}%
\pgftext[x=4.968919in,y=3.876279in,left,base]{\color{textcolor}{\sffamily\fontsize{8.000000}{9.600000}\selectfont\catcode`\^=\active\def^{\ifmmode\sp\else\^{}\fi}\catcode`\%=\active\def%{\%}Validação}}%
\end{pgfscope}%
\end{pgfpicture}%
\makeatother%
\endgroup%
}
    \end{minipage}

    \caption{Evolução do treinamento com \textit{hypertuning} (à esquerda) e do treinamento com hiperparâmetros fixados (à direita) do sistema especialista da competência IV.}
    \label{fig:exp-fix-c4}
\end{figure}

Pela figura \ref{fig:exp-fix-c4}, é possível notar que o treinamento com hiperparâmetros fixados apresentou uma evolução levemente melhor da função de perda, de modo que, já no início, o valor para a base de validação oscilou em torno de 0,9, com alguns picos no meio do processo. Além disso, a partir da 15ª época, a perda para a base de treino convergiu para aproximadamente 0,1, apresentando 4 aumentos substanciais até o final. No caso do \textit{hypertuning}, a função seguiu registrando números mais altos para a base de treino e de validação. Na primeira circunstância, o valor estagnou em 1,35 ainda nas primeiras iterações, enquanto que, na segunda, a perda oscilou entre 1,3 e 1,5 ao longo do início e do fim do processo. Assim, infere-se que os parâmetros fixados possuem vantagens em relação ao \textit{hypertuning} para a competência IV.

\subsubsection{Competência V}
\label{subsec:exp-fix-c5}

\begin{figure}[H]
    \begin{minipage}{0.45\textwidth}
        \resizebox{\textwidth}{!}{%% Creator: Matplotlib, PGF backend
%%
%% To include the figure in your LaTeX document, write
%%   \input{<filename>.pgf}
%%
%% Make sure the required packages are loaded in your preamble
%%   \usepackage{pgf}
%%
%% Also ensure that all the required font packages are loaded; for instance,
%% the lmodern package is sometimes necessary when using math font.
%%   \usepackage{lmodern}
%%
%% Figures using additional raster images can only be included by \input if
%% they are in the same directory as the main LaTeX file. For loading figures
%% from other directories you can use the `import` package
%%   \usepackage{import}
%%
%% and then include the figures with
%%   \import{<path to file>}{<filename>.pgf}
%%
%% Matplotlib used the following preamble
%%   \def\mathdefault#1{#1}
%%   \everymath=\expandafter{\the\everymath\displaystyle}
%%
%%   \usepackage{fontspec}
%%   \setmainfont{DejaVuSerif.ttf}[Path=\detokenize{/home/josemayer/.local/lib/python3.9/site-packages/matplotlib/mpl-data/fonts/ttf/}]
%%   \setsansfont{DejaVuSans.ttf}[Path=\detokenize{/home/josemayer/.local/lib/python3.9/site-packages/matplotlib/mpl-data/fonts/ttf/}]
%%   \setmonofont{DejaVuSansMono.ttf}[Path=\detokenize{/home/josemayer/.local/lib/python3.9/site-packages/matplotlib/mpl-data/fonts/ttf/}]
%%   \makeatletter\@ifpackageloaded{underscore}{}{\usepackage[strings]{underscore}}\makeatother
%%
\begingroup%
\makeatletter%
\begin{pgfpicture}%
\pgfpathrectangle{\pgfpointorigin}{\pgfqpoint{6.400000in}{4.800000in}}%
\pgfusepath{use as bounding box, clip}%
\begin{pgfscope}%
\pgfsetbuttcap%
\pgfsetmiterjoin%
\definecolor{currentfill}{rgb}{1.000000,1.000000,1.000000}%
\pgfsetfillcolor{currentfill}%
\pgfsetlinewidth{0.000000pt}%
\definecolor{currentstroke}{rgb}{1.000000,1.000000,1.000000}%
\pgfsetstrokecolor{currentstroke}%
\pgfsetdash{}{0pt}%
\pgfpathmoveto{\pgfqpoint{0.000000in}{0.000000in}}%
\pgfpathlineto{\pgfqpoint{6.400000in}{0.000000in}}%
\pgfpathlineto{\pgfqpoint{6.400000in}{4.800000in}}%
\pgfpathlineto{\pgfqpoint{0.000000in}{4.800000in}}%
\pgfpathlineto{\pgfqpoint{0.000000in}{0.000000in}}%
\pgfpathclose%
\pgfusepath{fill}%
\end{pgfscope}%
\begin{pgfscope}%
\pgfsetbuttcap%
\pgfsetmiterjoin%
\definecolor{currentfill}{rgb}{1.000000,1.000000,1.000000}%
\pgfsetfillcolor{currentfill}%
\pgfsetlinewidth{0.000000pt}%
\definecolor{currentstroke}{rgb}{0.000000,0.000000,0.000000}%
\pgfsetstrokecolor{currentstroke}%
\pgfsetstrokeopacity{0.000000}%
\pgfsetdash{}{0pt}%
\pgfpathmoveto{\pgfqpoint{0.800000in}{0.528000in}}%
\pgfpathlineto{\pgfqpoint{5.760000in}{0.528000in}}%
\pgfpathlineto{\pgfqpoint{5.760000in}{4.224000in}}%
\pgfpathlineto{\pgfqpoint{0.800000in}{4.224000in}}%
\pgfpathlineto{\pgfqpoint{0.800000in}{0.528000in}}%
\pgfpathclose%
\pgfusepath{fill}%
\end{pgfscope}%
\begin{pgfscope}%
\pgfsetbuttcap%
\pgfsetroundjoin%
\definecolor{currentfill}{rgb}{0.000000,0.000000,0.000000}%
\pgfsetfillcolor{currentfill}%
\pgfsetlinewidth{0.803000pt}%
\definecolor{currentstroke}{rgb}{0.000000,0.000000,0.000000}%
\pgfsetstrokecolor{currentstroke}%
\pgfsetdash{}{0pt}%
\pgfsys@defobject{currentmarker}{\pgfqpoint{0.000000in}{-0.048611in}}{\pgfqpoint{0.000000in}{0.000000in}}{%
\pgfpathmoveto{\pgfqpoint{0.000000in}{0.000000in}}%
\pgfpathlineto{\pgfqpoint{0.000000in}{-0.048611in}}%
\pgfusepath{stroke,fill}%
}%
\begin{pgfscope}%
\pgfsys@transformshift{1.025455in}{0.528000in}%
\pgfsys@useobject{currentmarker}{}%
\end{pgfscope}%
\end{pgfscope}%
\begin{pgfscope}%
\definecolor{textcolor}{rgb}{0.000000,0.000000,0.000000}%
\pgfsetstrokecolor{textcolor}%
\pgfsetfillcolor{textcolor}%
\pgftext[x=1.025455in,y=0.430778in,,top]{\color{textcolor}{\sffamily\fontsize{8.000000}{9.600000}\selectfont\catcode`\^=\active\def^{\ifmmode\sp\else\^{}\fi}\catcode`\%=\active\def%{\%}0}}%
\end{pgfscope}%
\begin{pgfscope}%
\pgfsetbuttcap%
\pgfsetroundjoin%
\definecolor{currentfill}{rgb}{0.000000,0.000000,0.000000}%
\pgfsetfillcolor{currentfill}%
\pgfsetlinewidth{0.803000pt}%
\definecolor{currentstroke}{rgb}{0.000000,0.000000,0.000000}%
\pgfsetstrokecolor{currentstroke}%
\pgfsetdash{}{0pt}%
\pgfsys@defobject{currentmarker}{\pgfqpoint{0.000000in}{-0.048611in}}{\pgfqpoint{0.000000in}{0.000000in}}{%
\pgfpathmoveto{\pgfqpoint{0.000000in}{0.000000in}}%
\pgfpathlineto{\pgfqpoint{0.000000in}{-0.048611in}}%
\pgfusepath{stroke,fill}%
}%
\begin{pgfscope}%
\pgfsys@transformshift{1.945677in}{0.528000in}%
\pgfsys@useobject{currentmarker}{}%
\end{pgfscope}%
\end{pgfscope}%
\begin{pgfscope}%
\definecolor{textcolor}{rgb}{0.000000,0.000000,0.000000}%
\pgfsetstrokecolor{textcolor}%
\pgfsetfillcolor{textcolor}%
\pgftext[x=1.945677in,y=0.430778in,,top]{\color{textcolor}{\sffamily\fontsize{8.000000}{9.600000}\selectfont\catcode`\^=\active\def^{\ifmmode\sp\else\^{}\fi}\catcode`\%=\active\def%{\%}10}}%
\end{pgfscope}%
\begin{pgfscope}%
\pgfsetbuttcap%
\pgfsetroundjoin%
\definecolor{currentfill}{rgb}{0.000000,0.000000,0.000000}%
\pgfsetfillcolor{currentfill}%
\pgfsetlinewidth{0.803000pt}%
\definecolor{currentstroke}{rgb}{0.000000,0.000000,0.000000}%
\pgfsetstrokecolor{currentstroke}%
\pgfsetdash{}{0pt}%
\pgfsys@defobject{currentmarker}{\pgfqpoint{0.000000in}{-0.048611in}}{\pgfqpoint{0.000000in}{0.000000in}}{%
\pgfpathmoveto{\pgfqpoint{0.000000in}{0.000000in}}%
\pgfpathlineto{\pgfqpoint{0.000000in}{-0.048611in}}%
\pgfusepath{stroke,fill}%
}%
\begin{pgfscope}%
\pgfsys@transformshift{2.865900in}{0.528000in}%
\pgfsys@useobject{currentmarker}{}%
\end{pgfscope}%
\end{pgfscope}%
\begin{pgfscope}%
\definecolor{textcolor}{rgb}{0.000000,0.000000,0.000000}%
\pgfsetstrokecolor{textcolor}%
\pgfsetfillcolor{textcolor}%
\pgftext[x=2.865900in,y=0.430778in,,top]{\color{textcolor}{\sffamily\fontsize{8.000000}{9.600000}\selectfont\catcode`\^=\active\def^{\ifmmode\sp\else\^{}\fi}\catcode`\%=\active\def%{\%}20}}%
\end{pgfscope}%
\begin{pgfscope}%
\pgfsetbuttcap%
\pgfsetroundjoin%
\definecolor{currentfill}{rgb}{0.000000,0.000000,0.000000}%
\pgfsetfillcolor{currentfill}%
\pgfsetlinewidth{0.803000pt}%
\definecolor{currentstroke}{rgb}{0.000000,0.000000,0.000000}%
\pgfsetstrokecolor{currentstroke}%
\pgfsetdash{}{0pt}%
\pgfsys@defobject{currentmarker}{\pgfqpoint{0.000000in}{-0.048611in}}{\pgfqpoint{0.000000in}{0.000000in}}{%
\pgfpathmoveto{\pgfqpoint{0.000000in}{0.000000in}}%
\pgfpathlineto{\pgfqpoint{0.000000in}{-0.048611in}}%
\pgfusepath{stroke,fill}%
}%
\begin{pgfscope}%
\pgfsys@transformshift{3.786122in}{0.528000in}%
\pgfsys@useobject{currentmarker}{}%
\end{pgfscope}%
\end{pgfscope}%
\begin{pgfscope}%
\definecolor{textcolor}{rgb}{0.000000,0.000000,0.000000}%
\pgfsetstrokecolor{textcolor}%
\pgfsetfillcolor{textcolor}%
\pgftext[x=3.786122in,y=0.430778in,,top]{\color{textcolor}{\sffamily\fontsize{8.000000}{9.600000}\selectfont\catcode`\^=\active\def^{\ifmmode\sp\else\^{}\fi}\catcode`\%=\active\def%{\%}30}}%
\end{pgfscope}%
\begin{pgfscope}%
\pgfsetbuttcap%
\pgfsetroundjoin%
\definecolor{currentfill}{rgb}{0.000000,0.000000,0.000000}%
\pgfsetfillcolor{currentfill}%
\pgfsetlinewidth{0.803000pt}%
\definecolor{currentstroke}{rgb}{0.000000,0.000000,0.000000}%
\pgfsetstrokecolor{currentstroke}%
\pgfsetdash{}{0pt}%
\pgfsys@defobject{currentmarker}{\pgfqpoint{0.000000in}{-0.048611in}}{\pgfqpoint{0.000000in}{0.000000in}}{%
\pgfpathmoveto{\pgfqpoint{0.000000in}{0.000000in}}%
\pgfpathlineto{\pgfqpoint{0.000000in}{-0.048611in}}%
\pgfusepath{stroke,fill}%
}%
\begin{pgfscope}%
\pgfsys@transformshift{4.706345in}{0.528000in}%
\pgfsys@useobject{currentmarker}{}%
\end{pgfscope}%
\end{pgfscope}%
\begin{pgfscope}%
\definecolor{textcolor}{rgb}{0.000000,0.000000,0.000000}%
\pgfsetstrokecolor{textcolor}%
\pgfsetfillcolor{textcolor}%
\pgftext[x=4.706345in,y=0.430778in,,top]{\color{textcolor}{\sffamily\fontsize{8.000000}{9.600000}\selectfont\catcode`\^=\active\def^{\ifmmode\sp\else\^{}\fi}\catcode`\%=\active\def%{\%}40}}%
\end{pgfscope}%
\begin{pgfscope}%
\pgfsetbuttcap%
\pgfsetroundjoin%
\definecolor{currentfill}{rgb}{0.000000,0.000000,0.000000}%
\pgfsetfillcolor{currentfill}%
\pgfsetlinewidth{0.803000pt}%
\definecolor{currentstroke}{rgb}{0.000000,0.000000,0.000000}%
\pgfsetstrokecolor{currentstroke}%
\pgfsetdash{}{0pt}%
\pgfsys@defobject{currentmarker}{\pgfqpoint{0.000000in}{-0.048611in}}{\pgfqpoint{0.000000in}{0.000000in}}{%
\pgfpathmoveto{\pgfqpoint{0.000000in}{0.000000in}}%
\pgfpathlineto{\pgfqpoint{0.000000in}{-0.048611in}}%
\pgfusepath{stroke,fill}%
}%
\begin{pgfscope}%
\pgfsys@transformshift{5.626568in}{0.528000in}%
\pgfsys@useobject{currentmarker}{}%
\end{pgfscope}%
\end{pgfscope}%
\begin{pgfscope}%
\definecolor{textcolor}{rgb}{0.000000,0.000000,0.000000}%
\pgfsetstrokecolor{textcolor}%
\pgfsetfillcolor{textcolor}%
\pgftext[x=5.626568in,y=0.430778in,,top]{\color{textcolor}{\sffamily\fontsize{8.000000}{9.600000}\selectfont\catcode`\^=\active\def^{\ifmmode\sp\else\^{}\fi}\catcode`\%=\active\def%{\%}50}}%
\end{pgfscope}%
\begin{pgfscope}%
\definecolor{textcolor}{rgb}{0.000000,0.000000,0.000000}%
\pgfsetstrokecolor{textcolor}%
\pgfsetfillcolor{textcolor}%
\pgftext[x=3.280000in,y=0.267692in,,top]{\color{textcolor}{\sffamily\fontsize{8.000000}{9.600000}\selectfont\catcode`\^=\active\def^{\ifmmode\sp\else\^{}\fi}\catcode`\%=\active\def%{\%}Época}}%
\end{pgfscope}%
\begin{pgfscope}%
\pgfsetbuttcap%
\pgfsetroundjoin%
\definecolor{currentfill}{rgb}{0.000000,0.000000,0.000000}%
\pgfsetfillcolor{currentfill}%
\pgfsetlinewidth{0.803000pt}%
\definecolor{currentstroke}{rgb}{0.000000,0.000000,0.000000}%
\pgfsetstrokecolor{currentstroke}%
\pgfsetdash{}{0pt}%
\pgfsys@defobject{currentmarker}{\pgfqpoint{-0.048611in}{0.000000in}}{\pgfqpoint{-0.000000in}{0.000000in}}{%
\pgfpathmoveto{\pgfqpoint{-0.000000in}{0.000000in}}%
\pgfpathlineto{\pgfqpoint{-0.048611in}{0.000000in}}%
\pgfusepath{stroke,fill}%
}%
\begin{pgfscope}%
\pgfsys@transformshift{0.800000in}{0.877549in}%
\pgfsys@useobject{currentmarker}{}%
\end{pgfscope}%
\end{pgfscope}%
\begin{pgfscope}%
\definecolor{textcolor}{rgb}{0.000000,0.000000,0.000000}%
\pgfsetstrokecolor{textcolor}%
\pgfsetfillcolor{textcolor}%
\pgftext[x=0.526074in, y=0.835340in, left, base]{\color{textcolor}{\sffamily\fontsize{8.000000}{9.600000}\selectfont\catcode`\^=\active\def^{\ifmmode\sp\else\^{}\fi}\catcode`\%=\active\def%{\%}1.7}}%
\end{pgfscope}%
\begin{pgfscope}%
\pgfsetbuttcap%
\pgfsetroundjoin%
\definecolor{currentfill}{rgb}{0.000000,0.000000,0.000000}%
\pgfsetfillcolor{currentfill}%
\pgfsetlinewidth{0.803000pt}%
\definecolor{currentstroke}{rgb}{0.000000,0.000000,0.000000}%
\pgfsetstrokecolor{currentstroke}%
\pgfsetdash{}{0pt}%
\pgfsys@defobject{currentmarker}{\pgfqpoint{-0.048611in}{0.000000in}}{\pgfqpoint{-0.000000in}{0.000000in}}{%
\pgfpathmoveto{\pgfqpoint{-0.000000in}{0.000000in}}%
\pgfpathlineto{\pgfqpoint{-0.048611in}{0.000000in}}%
\pgfusepath{stroke,fill}%
}%
\begin{pgfscope}%
\pgfsys@transformshift{0.800000in}{1.380108in}%
\pgfsys@useobject{currentmarker}{}%
\end{pgfscope}%
\end{pgfscope}%
\begin{pgfscope}%
\definecolor{textcolor}{rgb}{0.000000,0.000000,0.000000}%
\pgfsetstrokecolor{textcolor}%
\pgfsetfillcolor{textcolor}%
\pgftext[x=0.526074in, y=1.337899in, left, base]{\color{textcolor}{\sffamily\fontsize{8.000000}{9.600000}\selectfont\catcode`\^=\active\def^{\ifmmode\sp\else\^{}\fi}\catcode`\%=\active\def%{\%}1.8}}%
\end{pgfscope}%
\begin{pgfscope}%
\pgfsetbuttcap%
\pgfsetroundjoin%
\definecolor{currentfill}{rgb}{0.000000,0.000000,0.000000}%
\pgfsetfillcolor{currentfill}%
\pgfsetlinewidth{0.803000pt}%
\definecolor{currentstroke}{rgb}{0.000000,0.000000,0.000000}%
\pgfsetstrokecolor{currentstroke}%
\pgfsetdash{}{0pt}%
\pgfsys@defobject{currentmarker}{\pgfqpoint{-0.048611in}{0.000000in}}{\pgfqpoint{-0.000000in}{0.000000in}}{%
\pgfpathmoveto{\pgfqpoint{-0.000000in}{0.000000in}}%
\pgfpathlineto{\pgfqpoint{-0.048611in}{0.000000in}}%
\pgfusepath{stroke,fill}%
}%
\begin{pgfscope}%
\pgfsys@transformshift{0.800000in}{1.882667in}%
\pgfsys@useobject{currentmarker}{}%
\end{pgfscope}%
\end{pgfscope}%
\begin{pgfscope}%
\definecolor{textcolor}{rgb}{0.000000,0.000000,0.000000}%
\pgfsetstrokecolor{textcolor}%
\pgfsetfillcolor{textcolor}%
\pgftext[x=0.526074in, y=1.840458in, left, base]{\color{textcolor}{\sffamily\fontsize{8.000000}{9.600000}\selectfont\catcode`\^=\active\def^{\ifmmode\sp\else\^{}\fi}\catcode`\%=\active\def%{\%}1.9}}%
\end{pgfscope}%
\begin{pgfscope}%
\pgfsetbuttcap%
\pgfsetroundjoin%
\definecolor{currentfill}{rgb}{0.000000,0.000000,0.000000}%
\pgfsetfillcolor{currentfill}%
\pgfsetlinewidth{0.803000pt}%
\definecolor{currentstroke}{rgb}{0.000000,0.000000,0.000000}%
\pgfsetstrokecolor{currentstroke}%
\pgfsetdash{}{0pt}%
\pgfsys@defobject{currentmarker}{\pgfqpoint{-0.048611in}{0.000000in}}{\pgfqpoint{-0.000000in}{0.000000in}}{%
\pgfpathmoveto{\pgfqpoint{-0.000000in}{0.000000in}}%
\pgfpathlineto{\pgfqpoint{-0.048611in}{0.000000in}}%
\pgfusepath{stroke,fill}%
}%
\begin{pgfscope}%
\pgfsys@transformshift{0.800000in}{2.385227in}%
\pgfsys@useobject{currentmarker}{}%
\end{pgfscope}%
\end{pgfscope}%
\begin{pgfscope}%
\definecolor{textcolor}{rgb}{0.000000,0.000000,0.000000}%
\pgfsetstrokecolor{textcolor}%
\pgfsetfillcolor{textcolor}%
\pgftext[x=0.526074in, y=2.343017in, left, base]{\color{textcolor}{\sffamily\fontsize{8.000000}{9.600000}\selectfont\catcode`\^=\active\def^{\ifmmode\sp\else\^{}\fi}\catcode`\%=\active\def%{\%}2.0}}%
\end{pgfscope}%
\begin{pgfscope}%
\pgfsetbuttcap%
\pgfsetroundjoin%
\definecolor{currentfill}{rgb}{0.000000,0.000000,0.000000}%
\pgfsetfillcolor{currentfill}%
\pgfsetlinewidth{0.803000pt}%
\definecolor{currentstroke}{rgb}{0.000000,0.000000,0.000000}%
\pgfsetstrokecolor{currentstroke}%
\pgfsetdash{}{0pt}%
\pgfsys@defobject{currentmarker}{\pgfqpoint{-0.048611in}{0.000000in}}{\pgfqpoint{-0.000000in}{0.000000in}}{%
\pgfpathmoveto{\pgfqpoint{-0.000000in}{0.000000in}}%
\pgfpathlineto{\pgfqpoint{-0.048611in}{0.000000in}}%
\pgfusepath{stroke,fill}%
}%
\begin{pgfscope}%
\pgfsys@transformshift{0.800000in}{2.887786in}%
\pgfsys@useobject{currentmarker}{}%
\end{pgfscope}%
\end{pgfscope}%
\begin{pgfscope}%
\definecolor{textcolor}{rgb}{0.000000,0.000000,0.000000}%
\pgfsetstrokecolor{textcolor}%
\pgfsetfillcolor{textcolor}%
\pgftext[x=0.526074in, y=2.845577in, left, base]{\color{textcolor}{\sffamily\fontsize{8.000000}{9.600000}\selectfont\catcode`\^=\active\def^{\ifmmode\sp\else\^{}\fi}\catcode`\%=\active\def%{\%}2.1}}%
\end{pgfscope}%
\begin{pgfscope}%
\pgfsetbuttcap%
\pgfsetroundjoin%
\definecolor{currentfill}{rgb}{0.000000,0.000000,0.000000}%
\pgfsetfillcolor{currentfill}%
\pgfsetlinewidth{0.803000pt}%
\definecolor{currentstroke}{rgb}{0.000000,0.000000,0.000000}%
\pgfsetstrokecolor{currentstroke}%
\pgfsetdash{}{0pt}%
\pgfsys@defobject{currentmarker}{\pgfqpoint{-0.048611in}{0.000000in}}{\pgfqpoint{-0.000000in}{0.000000in}}{%
\pgfpathmoveto{\pgfqpoint{-0.000000in}{0.000000in}}%
\pgfpathlineto{\pgfqpoint{-0.048611in}{0.000000in}}%
\pgfusepath{stroke,fill}%
}%
\begin{pgfscope}%
\pgfsys@transformshift{0.800000in}{3.390345in}%
\pgfsys@useobject{currentmarker}{}%
\end{pgfscope}%
\end{pgfscope}%
\begin{pgfscope}%
\definecolor{textcolor}{rgb}{0.000000,0.000000,0.000000}%
\pgfsetstrokecolor{textcolor}%
\pgfsetfillcolor{textcolor}%
\pgftext[x=0.526074in, y=3.348136in, left, base]{\color{textcolor}{\sffamily\fontsize{8.000000}{9.600000}\selectfont\catcode`\^=\active\def^{\ifmmode\sp\else\^{}\fi}\catcode`\%=\active\def%{\%}2.2}}%
\end{pgfscope}%
\begin{pgfscope}%
\pgfsetbuttcap%
\pgfsetroundjoin%
\definecolor{currentfill}{rgb}{0.000000,0.000000,0.000000}%
\pgfsetfillcolor{currentfill}%
\pgfsetlinewidth{0.803000pt}%
\definecolor{currentstroke}{rgb}{0.000000,0.000000,0.000000}%
\pgfsetstrokecolor{currentstroke}%
\pgfsetdash{}{0pt}%
\pgfsys@defobject{currentmarker}{\pgfqpoint{-0.048611in}{0.000000in}}{\pgfqpoint{-0.000000in}{0.000000in}}{%
\pgfpathmoveto{\pgfqpoint{-0.000000in}{0.000000in}}%
\pgfpathlineto{\pgfqpoint{-0.048611in}{0.000000in}}%
\pgfusepath{stroke,fill}%
}%
\begin{pgfscope}%
\pgfsys@transformshift{0.800000in}{3.892904in}%
\pgfsys@useobject{currentmarker}{}%
\end{pgfscope}%
\end{pgfscope}%
\begin{pgfscope}%
\definecolor{textcolor}{rgb}{0.000000,0.000000,0.000000}%
\pgfsetstrokecolor{textcolor}%
\pgfsetfillcolor{textcolor}%
\pgftext[x=0.526074in, y=3.850695in, left, base]{\color{textcolor}{\sffamily\fontsize{8.000000}{9.600000}\selectfont\catcode`\^=\active\def^{\ifmmode\sp\else\^{}\fi}\catcode`\%=\active\def%{\%}2.3}}%
\end{pgfscope}%
\begin{pgfscope}%
\definecolor{textcolor}{rgb}{0.000000,0.000000,0.000000}%
\pgfsetstrokecolor{textcolor}%
\pgfsetfillcolor{textcolor}%
\pgftext[x=0.470519in,y=2.376000in,,bottom,rotate=90.000000]{\color{textcolor}{\sffamily\fontsize{8.000000}{9.600000}\selectfont\catcode`\^=\active\def^{\ifmmode\sp\else\^{}\fi}\catcode`\%=\active\def%{\%}Perda}}%
\end{pgfscope}%
\begin{pgfscope}%
\pgfpathrectangle{\pgfqpoint{0.800000in}{0.528000in}}{\pgfqpoint{4.960000in}{3.696000in}}%
\pgfusepath{clip}%
\pgfsetrectcap%
\pgfsetroundjoin%
\pgfsetlinewidth{1.505625pt}%
\definecolor{currentstroke}{rgb}{0.121569,0.466667,0.705882}%
\pgfsetstrokecolor{currentstroke}%
\pgfsetdash{}{0pt}%
\pgfpathmoveto{\pgfqpoint{1.025455in}{4.056000in}}%
\pgfpathlineto{\pgfqpoint{1.117477in}{1.051153in}}%
\pgfpathlineto{\pgfqpoint{1.209499in}{0.875857in}}%
\pgfpathlineto{\pgfqpoint{1.301521in}{0.822341in}}%
\pgfpathlineto{\pgfqpoint{1.393544in}{0.959291in}}%
\pgfpathlineto{\pgfqpoint{1.485566in}{0.855731in}}%
\pgfpathlineto{\pgfqpoint{1.577588in}{0.833725in}}%
\pgfpathlineto{\pgfqpoint{1.669610in}{0.921509in}}%
\pgfpathlineto{\pgfqpoint{1.761633in}{0.892232in}}%
\pgfpathlineto{\pgfqpoint{1.853655in}{0.787710in}}%
\pgfpathlineto{\pgfqpoint{1.945677in}{0.826818in}}%
\pgfpathlineto{\pgfqpoint{2.037699in}{0.856124in}}%
\pgfpathlineto{\pgfqpoint{2.129722in}{0.847073in}}%
\pgfpathlineto{\pgfqpoint{2.221744in}{0.872934in}}%
\pgfpathlineto{\pgfqpoint{2.313766in}{0.864710in}}%
\pgfpathlineto{\pgfqpoint{2.405788in}{0.801086in}}%
\pgfpathlineto{\pgfqpoint{2.497811in}{0.796055in}}%
\pgfpathlineto{\pgfqpoint{2.589833in}{0.815475in}}%
\pgfpathlineto{\pgfqpoint{2.681855in}{0.814724in}}%
\pgfpathlineto{\pgfqpoint{2.773878in}{0.793143in}}%
\pgfpathlineto{\pgfqpoint{2.865900in}{0.782973in}}%
\pgfpathlineto{\pgfqpoint{2.957922in}{0.772656in}}%
\pgfpathlineto{\pgfqpoint{3.049944in}{0.752757in}}%
\pgfpathlineto{\pgfqpoint{3.141967in}{0.787379in}}%
\pgfpathlineto{\pgfqpoint{3.233989in}{0.765715in}}%
\pgfpathlineto{\pgfqpoint{3.326011in}{0.817350in}}%
\pgfpathlineto{\pgfqpoint{3.418033in}{0.800443in}}%
\pgfpathlineto{\pgfqpoint{3.510056in}{0.785560in}}%
\pgfpathlineto{\pgfqpoint{3.602078in}{0.809507in}}%
\pgfpathlineto{\pgfqpoint{3.694100in}{0.735437in}}%
\pgfpathlineto{\pgfqpoint{3.786122in}{0.716906in}}%
\pgfpathlineto{\pgfqpoint{3.878145in}{0.747107in}}%
\pgfpathlineto{\pgfqpoint{3.970167in}{0.766626in}}%
\pgfpathlineto{\pgfqpoint{4.062189in}{0.707464in}}%
\pgfpathlineto{\pgfqpoint{4.154212in}{0.785086in}}%
\pgfpathlineto{\pgfqpoint{4.246234in}{0.726012in}}%
\pgfpathlineto{\pgfqpoint{4.338256in}{0.716841in}}%
\pgfpathlineto{\pgfqpoint{4.430278in}{0.766387in}}%
\pgfpathlineto{\pgfqpoint{4.522301in}{0.770049in}}%
\pgfpathlineto{\pgfqpoint{4.614323in}{0.719971in}}%
\pgfpathlineto{\pgfqpoint{4.706345in}{0.696000in}}%
\pgfpathlineto{\pgfqpoint{4.798367in}{0.742352in}}%
\pgfpathlineto{\pgfqpoint{4.890390in}{0.708449in}}%
\pgfpathlineto{\pgfqpoint{4.982412in}{0.754071in}}%
\pgfpathlineto{\pgfqpoint{5.074434in}{0.712405in}}%
\pgfpathlineto{\pgfqpoint{5.166456in}{0.731106in}}%
\pgfpathlineto{\pgfqpoint{5.258479in}{0.714731in}}%
\pgfpathlineto{\pgfqpoint{5.350501in}{0.745118in}}%
\pgfpathlineto{\pgfqpoint{5.442523in}{0.723612in}}%
\pgfpathlineto{\pgfqpoint{5.534545in}{0.714766in}}%
\pgfusepath{stroke}%
\end{pgfscope}%
\begin{pgfscope}%
\pgfpathrectangle{\pgfqpoint{0.800000in}{0.528000in}}{\pgfqpoint{4.960000in}{3.696000in}}%
\pgfusepath{clip}%
\pgfsetrectcap%
\pgfsetroundjoin%
\pgfsetlinewidth{1.505625pt}%
\definecolor{currentstroke}{rgb}{1.000000,0.498039,0.054902}%
\pgfsetstrokecolor{currentstroke}%
\pgfsetdash{}{0pt}%
\pgfpathmoveto{\pgfqpoint{1.025455in}{1.282890in}}%
\pgfpathlineto{\pgfqpoint{1.117477in}{1.332467in}}%
\pgfpathlineto{\pgfqpoint{1.209499in}{1.284460in}}%
\pgfpathlineto{\pgfqpoint{1.301521in}{1.589362in}}%
\pgfpathlineto{\pgfqpoint{1.393544in}{1.303085in}}%
\pgfpathlineto{\pgfqpoint{1.485566in}{1.572772in}}%
\pgfpathlineto{\pgfqpoint{1.577588in}{1.337591in}}%
\pgfpathlineto{\pgfqpoint{1.669610in}{1.584787in}}%
\pgfpathlineto{\pgfqpoint{1.761633in}{1.279833in}}%
\pgfpathlineto{\pgfqpoint{1.853655in}{1.279627in}}%
\pgfpathlineto{\pgfqpoint{1.945677in}{1.307976in}}%
\pgfpathlineto{\pgfqpoint{2.037699in}{1.345399in}}%
\pgfpathlineto{\pgfqpoint{2.129722in}{1.320229in}}%
\pgfpathlineto{\pgfqpoint{2.221744in}{1.305123in}}%
\pgfpathlineto{\pgfqpoint{2.313766in}{1.291703in}}%
\pgfpathlineto{\pgfqpoint{2.405788in}{1.345237in}}%
\pgfpathlineto{\pgfqpoint{2.497811in}{1.366643in}}%
\pgfpathlineto{\pgfqpoint{2.589833in}{1.368297in}}%
\pgfpathlineto{\pgfqpoint{2.681855in}{1.396025in}}%
\pgfpathlineto{\pgfqpoint{2.773878in}{1.355686in}}%
\pgfpathlineto{\pgfqpoint{2.865900in}{1.440373in}}%
\pgfpathlineto{\pgfqpoint{2.957922in}{1.281392in}}%
\pgfpathlineto{\pgfqpoint{3.049944in}{1.288781in}}%
\pgfpathlineto{\pgfqpoint{3.141967in}{1.281516in}}%
\pgfpathlineto{\pgfqpoint{3.233989in}{1.655237in}}%
\pgfpathlineto{\pgfqpoint{3.326011in}{1.513043in}}%
\pgfpathlineto{\pgfqpoint{3.418033in}{1.365133in}}%
\pgfpathlineto{\pgfqpoint{3.510056in}{1.792675in}}%
\pgfpathlineto{\pgfqpoint{3.602078in}{1.383172in}}%
\pgfpathlineto{\pgfqpoint{3.694100in}{1.427667in}}%
\pgfpathlineto{\pgfqpoint{3.786122in}{1.279449in}}%
\pgfpathlineto{\pgfqpoint{3.878145in}{1.291416in}}%
\pgfpathlineto{\pgfqpoint{3.970167in}{1.447832in}}%
\pgfpathlineto{\pgfqpoint{4.062189in}{1.314228in}}%
\pgfpathlineto{\pgfqpoint{4.154212in}{1.282526in}}%
\pgfpathlineto{\pgfqpoint{4.246234in}{1.343416in}}%
\pgfpathlineto{\pgfqpoint{4.338256in}{1.303421in}}%
\pgfpathlineto{\pgfqpoint{4.430278in}{1.280207in}}%
\pgfpathlineto{\pgfqpoint{4.522301in}{1.316911in}}%
\pgfpathlineto{\pgfqpoint{4.614323in}{1.534431in}}%
\pgfpathlineto{\pgfqpoint{4.706345in}{1.326797in}}%
\pgfpathlineto{\pgfqpoint{4.798367in}{1.280487in}}%
\pgfpathlineto{\pgfqpoint{4.890390in}{1.281297in}}%
\pgfpathlineto{\pgfqpoint{4.982412in}{1.292811in}}%
\pgfpathlineto{\pgfqpoint{5.074434in}{1.521579in}}%
\pgfpathlineto{\pgfqpoint{5.166456in}{1.351344in}}%
\pgfpathlineto{\pgfqpoint{5.258479in}{1.313763in}}%
\pgfpathlineto{\pgfqpoint{5.350501in}{1.289974in}}%
\pgfpathlineto{\pgfqpoint{5.442523in}{1.386367in}}%
\pgfpathlineto{\pgfqpoint{5.534545in}{1.339398in}}%
\pgfusepath{stroke}%
\end{pgfscope}%
\begin{pgfscope}%
\pgfsetrectcap%
\pgfsetmiterjoin%
\pgfsetlinewidth{0.803000pt}%
\definecolor{currentstroke}{rgb}{0.000000,0.000000,0.000000}%
\pgfsetstrokecolor{currentstroke}%
\pgfsetdash{}{0pt}%
\pgfpathmoveto{\pgfqpoint{0.800000in}{0.528000in}}%
\pgfpathlineto{\pgfqpoint{0.800000in}{4.224000in}}%
\pgfusepath{stroke}%
\end{pgfscope}%
\begin{pgfscope}%
\pgfsetrectcap%
\pgfsetmiterjoin%
\pgfsetlinewidth{0.803000pt}%
\definecolor{currentstroke}{rgb}{0.000000,0.000000,0.000000}%
\pgfsetstrokecolor{currentstroke}%
\pgfsetdash{}{0pt}%
\pgfpathmoveto{\pgfqpoint{5.760000in}{0.528000in}}%
\pgfpathlineto{\pgfqpoint{5.760000in}{4.224000in}}%
\pgfusepath{stroke}%
\end{pgfscope}%
\begin{pgfscope}%
\pgfsetrectcap%
\pgfsetmiterjoin%
\pgfsetlinewidth{0.803000pt}%
\definecolor{currentstroke}{rgb}{0.000000,0.000000,0.000000}%
\pgfsetstrokecolor{currentstroke}%
\pgfsetdash{}{0pt}%
\pgfpathmoveto{\pgfqpoint{0.800000in}{0.528000in}}%
\pgfpathlineto{\pgfqpoint{5.760000in}{0.528000in}}%
\pgfusepath{stroke}%
\end{pgfscope}%
\begin{pgfscope}%
\pgfsetrectcap%
\pgfsetmiterjoin%
\pgfsetlinewidth{0.803000pt}%
\definecolor{currentstroke}{rgb}{0.000000,0.000000,0.000000}%
\pgfsetstrokecolor{currentstroke}%
\pgfsetdash{}{0pt}%
\pgfpathmoveto{\pgfqpoint{0.800000in}{4.224000in}}%
\pgfpathlineto{\pgfqpoint{5.760000in}{4.224000in}}%
\pgfusepath{stroke}%
\end{pgfscope}%
\begin{pgfscope}%
\definecolor{textcolor}{rgb}{0.000000,0.000000,0.000000}%
\pgfsetstrokecolor{textcolor}%
\pgfsetfillcolor{textcolor}%
\pgftext[x=3.280000in,y=4.307333in,,base]{\color{textcolor}{\sffamily\fontsize{9.600000}{11.520000}\selectfont\catcode`\^=\active\def^{\ifmmode\sp\else\^{}\fi}\catcode`\%=\active\def%{\%}Função de Perda em Treino (compV)}}%
\end{pgfscope}%
\begin{pgfscope}%
\pgfsetbuttcap%
\pgfsetmiterjoin%
\definecolor{currentfill}{rgb}{1.000000,1.000000,1.000000}%
\pgfsetfillcolor{currentfill}%
\pgfsetfillopacity{0.800000}%
\pgfsetlinewidth{1.003750pt}%
\definecolor{currentstroke}{rgb}{0.800000,0.800000,0.800000}%
\pgfsetstrokecolor{currentstroke}%
\pgfsetstrokeopacity{0.800000}%
\pgfsetdash{}{0pt}%
\pgfpathmoveto{\pgfqpoint{4.635586in}{3.808723in}}%
\pgfpathlineto{\pgfqpoint{5.682222in}{3.808723in}}%
\pgfpathquadraticcurveto{\pgfqpoint{5.704444in}{3.808723in}}{\pgfqpoint{5.704444in}{3.830945in}}%
\pgfpathlineto{\pgfqpoint{5.704444in}{4.146222in}}%
\pgfpathquadraticcurveto{\pgfqpoint{5.704444in}{4.168444in}}{\pgfqpoint{5.682222in}{4.168444in}}%
\pgfpathlineto{\pgfqpoint{4.635586in}{4.168444in}}%
\pgfpathquadraticcurveto{\pgfqpoint{4.613364in}{4.168444in}}{\pgfqpoint{4.613364in}{4.146222in}}%
\pgfpathlineto{\pgfqpoint{4.613364in}{3.830945in}}%
\pgfpathquadraticcurveto{\pgfqpoint{4.613364in}{3.808723in}}{\pgfqpoint{4.635586in}{3.808723in}}%
\pgfpathlineto{\pgfqpoint{4.635586in}{3.808723in}}%
\pgfpathclose%
\pgfusepath{stroke,fill}%
\end{pgfscope}%
\begin{pgfscope}%
\pgfsetrectcap%
\pgfsetroundjoin%
\pgfsetlinewidth{1.505625pt}%
\definecolor{currentstroke}{rgb}{0.121569,0.466667,0.705882}%
\pgfsetstrokecolor{currentstroke}%
\pgfsetdash{}{0pt}%
\pgfpathmoveto{\pgfqpoint{4.657808in}{4.078470in}}%
\pgfpathlineto{\pgfqpoint{4.768919in}{4.078470in}}%
\pgfpathlineto{\pgfqpoint{4.880030in}{4.078470in}}%
\pgfusepath{stroke}%
\end{pgfscope}%
\begin{pgfscope}%
\definecolor{textcolor}{rgb}{0.000000,0.000000,0.000000}%
\pgfsetstrokecolor{textcolor}%
\pgfsetfillcolor{textcolor}%
\pgftext[x=4.968919in,y=4.039582in,left,base]{\color{textcolor}{\sffamily\fontsize{8.000000}{9.600000}\selectfont\catcode`\^=\active\def^{\ifmmode\sp\else\^{}\fi}\catcode`\%=\active\def%{\%}Treinamento}}%
\end{pgfscope}%
\begin{pgfscope}%
\pgfsetrectcap%
\pgfsetroundjoin%
\pgfsetlinewidth{1.505625pt}%
\definecolor{currentstroke}{rgb}{1.000000,0.498039,0.054902}%
\pgfsetstrokecolor{currentstroke}%
\pgfsetdash{}{0pt}%
\pgfpathmoveto{\pgfqpoint{4.657808in}{3.915168in}}%
\pgfpathlineto{\pgfqpoint{4.768919in}{3.915168in}}%
\pgfpathlineto{\pgfqpoint{4.880030in}{3.915168in}}%
\pgfusepath{stroke}%
\end{pgfscope}%
\begin{pgfscope}%
\definecolor{textcolor}{rgb}{0.000000,0.000000,0.000000}%
\pgfsetstrokecolor{textcolor}%
\pgfsetfillcolor{textcolor}%
\pgftext[x=4.968919in,y=3.876279in,left,base]{\color{textcolor}{\sffamily\fontsize{8.000000}{9.600000}\selectfont\catcode`\^=\active\def^{\ifmmode\sp\else\^{}\fi}\catcode`\%=\active\def%{\%}Validação}}%
\end{pgfscope}%
\end{pgfpicture}%
\makeatother%
\endgroup%
}
    \end{minipage}
    \begin{minipage}{0.45\textwidth}
        \resizebox{\textwidth}{!}{%% Creator: Matplotlib, PGF backend
%%
%% To include the figure in your LaTeX document, write
%%   \input{<filename>.pgf}
%%
%% Make sure the required packages are loaded in your preamble
%%   \usepackage{pgf}
%%
%% Also ensure that all the required font packages are loaded; for instance,
%% the lmodern package is sometimes necessary when using math font.
%%   \usepackage{lmodern}
%%
%% Figures using additional raster images can only be included by \input if
%% they are in the same directory as the main LaTeX file. For loading figures
%% from other directories you can use the `import` package
%%   \usepackage{import}
%%
%% and then include the figures with
%%   \import{<path to file>}{<filename>.pgf}
%%
%% Matplotlib used the following preamble
%%   \def\mathdefault#1{#1}
%%   \everymath=\expandafter{\the\everymath\displaystyle}
%%
%%   \usepackage{fontspec}
%%   \setmainfont{DejaVuSerif.ttf}[Path=\detokenize{/home/josemayer/.local/lib/python3.9/site-packages/matplotlib/mpl-data/fonts/ttf/}]
%%   \setsansfont{DejaVuSans.ttf}[Path=\detokenize{/home/josemayer/.local/lib/python3.9/site-packages/matplotlib/mpl-data/fonts/ttf/}]
%%   \setmonofont{DejaVuSansMono.ttf}[Path=\detokenize{/home/josemayer/.local/lib/python3.9/site-packages/matplotlib/mpl-data/fonts/ttf/}]
%%   \makeatletter\@ifpackageloaded{underscore}{}{\usepackage[strings]{underscore}}\makeatother
%%
\begingroup%
\makeatletter%
\begin{pgfpicture}%
\pgfpathrectangle{\pgfpointorigin}{\pgfqpoint{6.400000in}{4.800000in}}%
\pgfusepath{use as bounding box, clip}%
\begin{pgfscope}%
\pgfsetbuttcap%
\pgfsetmiterjoin%
\definecolor{currentfill}{rgb}{1.000000,1.000000,1.000000}%
\pgfsetfillcolor{currentfill}%
\pgfsetlinewidth{0.000000pt}%
\definecolor{currentstroke}{rgb}{1.000000,1.000000,1.000000}%
\pgfsetstrokecolor{currentstroke}%
\pgfsetdash{}{0pt}%
\pgfpathmoveto{\pgfqpoint{0.000000in}{0.000000in}}%
\pgfpathlineto{\pgfqpoint{6.400000in}{0.000000in}}%
\pgfpathlineto{\pgfqpoint{6.400000in}{4.800000in}}%
\pgfpathlineto{\pgfqpoint{0.000000in}{4.800000in}}%
\pgfpathlineto{\pgfqpoint{0.000000in}{0.000000in}}%
\pgfpathclose%
\pgfusepath{fill}%
\end{pgfscope}%
\begin{pgfscope}%
\pgfsetbuttcap%
\pgfsetmiterjoin%
\definecolor{currentfill}{rgb}{1.000000,1.000000,1.000000}%
\pgfsetfillcolor{currentfill}%
\pgfsetlinewidth{0.000000pt}%
\definecolor{currentstroke}{rgb}{0.000000,0.000000,0.000000}%
\pgfsetstrokecolor{currentstroke}%
\pgfsetstrokeopacity{0.000000}%
\pgfsetdash{}{0pt}%
\pgfpathmoveto{\pgfqpoint{0.800000in}{0.528000in}}%
\pgfpathlineto{\pgfqpoint{5.760000in}{0.528000in}}%
\pgfpathlineto{\pgfqpoint{5.760000in}{4.224000in}}%
\pgfpathlineto{\pgfqpoint{0.800000in}{4.224000in}}%
\pgfpathlineto{\pgfqpoint{0.800000in}{0.528000in}}%
\pgfpathclose%
\pgfusepath{fill}%
\end{pgfscope}%
\begin{pgfscope}%
\pgfsetbuttcap%
\pgfsetroundjoin%
\definecolor{currentfill}{rgb}{0.000000,0.000000,0.000000}%
\pgfsetfillcolor{currentfill}%
\pgfsetlinewidth{0.803000pt}%
\definecolor{currentstroke}{rgb}{0.000000,0.000000,0.000000}%
\pgfsetstrokecolor{currentstroke}%
\pgfsetdash{}{0pt}%
\pgfsys@defobject{currentmarker}{\pgfqpoint{0.000000in}{-0.048611in}}{\pgfqpoint{0.000000in}{0.000000in}}{%
\pgfpathmoveto{\pgfqpoint{0.000000in}{0.000000in}}%
\pgfpathlineto{\pgfqpoint{0.000000in}{-0.048611in}}%
\pgfusepath{stroke,fill}%
}%
\begin{pgfscope}%
\pgfsys@transformshift{1.025455in}{0.528000in}%
\pgfsys@useobject{currentmarker}{}%
\end{pgfscope}%
\end{pgfscope}%
\begin{pgfscope}%
\definecolor{textcolor}{rgb}{0.000000,0.000000,0.000000}%
\pgfsetstrokecolor{textcolor}%
\pgfsetfillcolor{textcolor}%
\pgftext[x=1.025455in,y=0.430778in,,top]{\color{textcolor}{\sffamily\fontsize{8.000000}{9.600000}\selectfont\catcode`\^=\active\def^{\ifmmode\sp\else\^{}\fi}\catcode`\%=\active\def%{\%}0}}%
\end{pgfscope}%
\begin{pgfscope}%
\pgfsetbuttcap%
\pgfsetroundjoin%
\definecolor{currentfill}{rgb}{0.000000,0.000000,0.000000}%
\pgfsetfillcolor{currentfill}%
\pgfsetlinewidth{0.803000pt}%
\definecolor{currentstroke}{rgb}{0.000000,0.000000,0.000000}%
\pgfsetstrokecolor{currentstroke}%
\pgfsetdash{}{0pt}%
\pgfsys@defobject{currentmarker}{\pgfqpoint{0.000000in}{-0.048611in}}{\pgfqpoint{0.000000in}{0.000000in}}{%
\pgfpathmoveto{\pgfqpoint{0.000000in}{0.000000in}}%
\pgfpathlineto{\pgfqpoint{0.000000in}{-0.048611in}}%
\pgfusepath{stroke,fill}%
}%
\begin{pgfscope}%
\pgfsys@transformshift{1.945677in}{0.528000in}%
\pgfsys@useobject{currentmarker}{}%
\end{pgfscope}%
\end{pgfscope}%
\begin{pgfscope}%
\definecolor{textcolor}{rgb}{0.000000,0.000000,0.000000}%
\pgfsetstrokecolor{textcolor}%
\pgfsetfillcolor{textcolor}%
\pgftext[x=1.945677in,y=0.430778in,,top]{\color{textcolor}{\sffamily\fontsize{8.000000}{9.600000}\selectfont\catcode`\^=\active\def^{\ifmmode\sp\else\^{}\fi}\catcode`\%=\active\def%{\%}10}}%
\end{pgfscope}%
\begin{pgfscope}%
\pgfsetbuttcap%
\pgfsetroundjoin%
\definecolor{currentfill}{rgb}{0.000000,0.000000,0.000000}%
\pgfsetfillcolor{currentfill}%
\pgfsetlinewidth{0.803000pt}%
\definecolor{currentstroke}{rgb}{0.000000,0.000000,0.000000}%
\pgfsetstrokecolor{currentstroke}%
\pgfsetdash{}{0pt}%
\pgfsys@defobject{currentmarker}{\pgfqpoint{0.000000in}{-0.048611in}}{\pgfqpoint{0.000000in}{0.000000in}}{%
\pgfpathmoveto{\pgfqpoint{0.000000in}{0.000000in}}%
\pgfpathlineto{\pgfqpoint{0.000000in}{-0.048611in}}%
\pgfusepath{stroke,fill}%
}%
\begin{pgfscope}%
\pgfsys@transformshift{2.865900in}{0.528000in}%
\pgfsys@useobject{currentmarker}{}%
\end{pgfscope}%
\end{pgfscope}%
\begin{pgfscope}%
\definecolor{textcolor}{rgb}{0.000000,0.000000,0.000000}%
\pgfsetstrokecolor{textcolor}%
\pgfsetfillcolor{textcolor}%
\pgftext[x=2.865900in,y=0.430778in,,top]{\color{textcolor}{\sffamily\fontsize{8.000000}{9.600000}\selectfont\catcode`\^=\active\def^{\ifmmode\sp\else\^{}\fi}\catcode`\%=\active\def%{\%}20}}%
\end{pgfscope}%
\begin{pgfscope}%
\pgfsetbuttcap%
\pgfsetroundjoin%
\definecolor{currentfill}{rgb}{0.000000,0.000000,0.000000}%
\pgfsetfillcolor{currentfill}%
\pgfsetlinewidth{0.803000pt}%
\definecolor{currentstroke}{rgb}{0.000000,0.000000,0.000000}%
\pgfsetstrokecolor{currentstroke}%
\pgfsetdash{}{0pt}%
\pgfsys@defobject{currentmarker}{\pgfqpoint{0.000000in}{-0.048611in}}{\pgfqpoint{0.000000in}{0.000000in}}{%
\pgfpathmoveto{\pgfqpoint{0.000000in}{0.000000in}}%
\pgfpathlineto{\pgfqpoint{0.000000in}{-0.048611in}}%
\pgfusepath{stroke,fill}%
}%
\begin{pgfscope}%
\pgfsys@transformshift{3.786122in}{0.528000in}%
\pgfsys@useobject{currentmarker}{}%
\end{pgfscope}%
\end{pgfscope}%
\begin{pgfscope}%
\definecolor{textcolor}{rgb}{0.000000,0.000000,0.000000}%
\pgfsetstrokecolor{textcolor}%
\pgfsetfillcolor{textcolor}%
\pgftext[x=3.786122in,y=0.430778in,,top]{\color{textcolor}{\sffamily\fontsize{8.000000}{9.600000}\selectfont\catcode`\^=\active\def^{\ifmmode\sp\else\^{}\fi}\catcode`\%=\active\def%{\%}30}}%
\end{pgfscope}%
\begin{pgfscope}%
\pgfsetbuttcap%
\pgfsetroundjoin%
\definecolor{currentfill}{rgb}{0.000000,0.000000,0.000000}%
\pgfsetfillcolor{currentfill}%
\pgfsetlinewidth{0.803000pt}%
\definecolor{currentstroke}{rgb}{0.000000,0.000000,0.000000}%
\pgfsetstrokecolor{currentstroke}%
\pgfsetdash{}{0pt}%
\pgfsys@defobject{currentmarker}{\pgfqpoint{0.000000in}{-0.048611in}}{\pgfqpoint{0.000000in}{0.000000in}}{%
\pgfpathmoveto{\pgfqpoint{0.000000in}{0.000000in}}%
\pgfpathlineto{\pgfqpoint{0.000000in}{-0.048611in}}%
\pgfusepath{stroke,fill}%
}%
\begin{pgfscope}%
\pgfsys@transformshift{4.706345in}{0.528000in}%
\pgfsys@useobject{currentmarker}{}%
\end{pgfscope}%
\end{pgfscope}%
\begin{pgfscope}%
\definecolor{textcolor}{rgb}{0.000000,0.000000,0.000000}%
\pgfsetstrokecolor{textcolor}%
\pgfsetfillcolor{textcolor}%
\pgftext[x=4.706345in,y=0.430778in,,top]{\color{textcolor}{\sffamily\fontsize{8.000000}{9.600000}\selectfont\catcode`\^=\active\def^{\ifmmode\sp\else\^{}\fi}\catcode`\%=\active\def%{\%}40}}%
\end{pgfscope}%
\begin{pgfscope}%
\pgfsetbuttcap%
\pgfsetroundjoin%
\definecolor{currentfill}{rgb}{0.000000,0.000000,0.000000}%
\pgfsetfillcolor{currentfill}%
\pgfsetlinewidth{0.803000pt}%
\definecolor{currentstroke}{rgb}{0.000000,0.000000,0.000000}%
\pgfsetstrokecolor{currentstroke}%
\pgfsetdash{}{0pt}%
\pgfsys@defobject{currentmarker}{\pgfqpoint{0.000000in}{-0.048611in}}{\pgfqpoint{0.000000in}{0.000000in}}{%
\pgfpathmoveto{\pgfqpoint{0.000000in}{0.000000in}}%
\pgfpathlineto{\pgfqpoint{0.000000in}{-0.048611in}}%
\pgfusepath{stroke,fill}%
}%
\begin{pgfscope}%
\pgfsys@transformshift{5.626568in}{0.528000in}%
\pgfsys@useobject{currentmarker}{}%
\end{pgfscope}%
\end{pgfscope}%
\begin{pgfscope}%
\definecolor{textcolor}{rgb}{0.000000,0.000000,0.000000}%
\pgfsetstrokecolor{textcolor}%
\pgfsetfillcolor{textcolor}%
\pgftext[x=5.626568in,y=0.430778in,,top]{\color{textcolor}{\sffamily\fontsize{8.000000}{9.600000}\selectfont\catcode`\^=\active\def^{\ifmmode\sp\else\^{}\fi}\catcode`\%=\active\def%{\%}50}}%
\end{pgfscope}%
\begin{pgfscope}%
\definecolor{textcolor}{rgb}{0.000000,0.000000,0.000000}%
\pgfsetstrokecolor{textcolor}%
\pgfsetfillcolor{textcolor}%
\pgftext[x=3.280000in,y=0.267692in,,top]{\color{textcolor}{\sffamily\fontsize{8.000000}{9.600000}\selectfont\catcode`\^=\active\def^{\ifmmode\sp\else\^{}\fi}\catcode`\%=\active\def%{\%}Época}}%
\end{pgfscope}%
\begin{pgfscope}%
\pgfsetbuttcap%
\pgfsetroundjoin%
\definecolor{currentfill}{rgb}{0.000000,0.000000,0.000000}%
\pgfsetfillcolor{currentfill}%
\pgfsetlinewidth{0.803000pt}%
\definecolor{currentstroke}{rgb}{0.000000,0.000000,0.000000}%
\pgfsetstrokecolor{currentstroke}%
\pgfsetdash{}{0pt}%
\pgfsys@defobject{currentmarker}{\pgfqpoint{-0.048611in}{0.000000in}}{\pgfqpoint{-0.000000in}{0.000000in}}{%
\pgfpathmoveto{\pgfqpoint{-0.000000in}{0.000000in}}%
\pgfpathlineto{\pgfqpoint{-0.048611in}{0.000000in}}%
\pgfusepath{stroke,fill}%
}%
\begin{pgfscope}%
\pgfsys@transformshift{0.800000in}{0.594294in}%
\pgfsys@useobject{currentmarker}{}%
\end{pgfscope}%
\end{pgfscope}%
\begin{pgfscope}%
\definecolor{textcolor}{rgb}{0.000000,0.000000,0.000000}%
\pgfsetstrokecolor{textcolor}%
\pgfsetfillcolor{textcolor}%
\pgftext[x=0.455382in, y=0.552085in, left, base]{\color{textcolor}{\sffamily\fontsize{8.000000}{9.600000}\selectfont\catcode`\^=\active\def^{\ifmmode\sp\else\^{}\fi}\catcode`\%=\active\def%{\%}0.00}}%
\end{pgfscope}%
\begin{pgfscope}%
\pgfsetbuttcap%
\pgfsetroundjoin%
\definecolor{currentfill}{rgb}{0.000000,0.000000,0.000000}%
\pgfsetfillcolor{currentfill}%
\pgfsetlinewidth{0.803000pt}%
\definecolor{currentstroke}{rgb}{0.000000,0.000000,0.000000}%
\pgfsetstrokecolor{currentstroke}%
\pgfsetdash{}{0pt}%
\pgfsys@defobject{currentmarker}{\pgfqpoint{-0.048611in}{0.000000in}}{\pgfqpoint{-0.000000in}{0.000000in}}{%
\pgfpathmoveto{\pgfqpoint{-0.000000in}{0.000000in}}%
\pgfpathlineto{\pgfqpoint{-0.048611in}{0.000000in}}%
\pgfusepath{stroke,fill}%
}%
\begin{pgfscope}%
\pgfsys@transformshift{0.800000in}{1.054421in}%
\pgfsys@useobject{currentmarker}{}%
\end{pgfscope}%
\end{pgfscope}%
\begin{pgfscope}%
\definecolor{textcolor}{rgb}{0.000000,0.000000,0.000000}%
\pgfsetstrokecolor{textcolor}%
\pgfsetfillcolor{textcolor}%
\pgftext[x=0.455382in, y=1.012212in, left, base]{\color{textcolor}{\sffamily\fontsize{8.000000}{9.600000}\selectfont\catcode`\^=\active\def^{\ifmmode\sp\else\^{}\fi}\catcode`\%=\active\def%{\%}0.25}}%
\end{pgfscope}%
\begin{pgfscope}%
\pgfsetbuttcap%
\pgfsetroundjoin%
\definecolor{currentfill}{rgb}{0.000000,0.000000,0.000000}%
\pgfsetfillcolor{currentfill}%
\pgfsetlinewidth{0.803000pt}%
\definecolor{currentstroke}{rgb}{0.000000,0.000000,0.000000}%
\pgfsetstrokecolor{currentstroke}%
\pgfsetdash{}{0pt}%
\pgfsys@defobject{currentmarker}{\pgfqpoint{-0.048611in}{0.000000in}}{\pgfqpoint{-0.000000in}{0.000000in}}{%
\pgfpathmoveto{\pgfqpoint{-0.000000in}{0.000000in}}%
\pgfpathlineto{\pgfqpoint{-0.048611in}{0.000000in}}%
\pgfusepath{stroke,fill}%
}%
\begin{pgfscope}%
\pgfsys@transformshift{0.800000in}{1.514549in}%
\pgfsys@useobject{currentmarker}{}%
\end{pgfscope}%
\end{pgfscope}%
\begin{pgfscope}%
\definecolor{textcolor}{rgb}{0.000000,0.000000,0.000000}%
\pgfsetstrokecolor{textcolor}%
\pgfsetfillcolor{textcolor}%
\pgftext[x=0.455382in, y=1.472339in, left, base]{\color{textcolor}{\sffamily\fontsize{8.000000}{9.600000}\selectfont\catcode`\^=\active\def^{\ifmmode\sp\else\^{}\fi}\catcode`\%=\active\def%{\%}0.50}}%
\end{pgfscope}%
\begin{pgfscope}%
\pgfsetbuttcap%
\pgfsetroundjoin%
\definecolor{currentfill}{rgb}{0.000000,0.000000,0.000000}%
\pgfsetfillcolor{currentfill}%
\pgfsetlinewidth{0.803000pt}%
\definecolor{currentstroke}{rgb}{0.000000,0.000000,0.000000}%
\pgfsetstrokecolor{currentstroke}%
\pgfsetdash{}{0pt}%
\pgfsys@defobject{currentmarker}{\pgfqpoint{-0.048611in}{0.000000in}}{\pgfqpoint{-0.000000in}{0.000000in}}{%
\pgfpathmoveto{\pgfqpoint{-0.000000in}{0.000000in}}%
\pgfpathlineto{\pgfqpoint{-0.048611in}{0.000000in}}%
\pgfusepath{stroke,fill}%
}%
\begin{pgfscope}%
\pgfsys@transformshift{0.800000in}{1.974676in}%
\pgfsys@useobject{currentmarker}{}%
\end{pgfscope}%
\end{pgfscope}%
\begin{pgfscope}%
\definecolor{textcolor}{rgb}{0.000000,0.000000,0.000000}%
\pgfsetstrokecolor{textcolor}%
\pgfsetfillcolor{textcolor}%
\pgftext[x=0.455382in, y=1.932466in, left, base]{\color{textcolor}{\sffamily\fontsize{8.000000}{9.600000}\selectfont\catcode`\^=\active\def^{\ifmmode\sp\else\^{}\fi}\catcode`\%=\active\def%{\%}0.75}}%
\end{pgfscope}%
\begin{pgfscope}%
\pgfsetbuttcap%
\pgfsetroundjoin%
\definecolor{currentfill}{rgb}{0.000000,0.000000,0.000000}%
\pgfsetfillcolor{currentfill}%
\pgfsetlinewidth{0.803000pt}%
\definecolor{currentstroke}{rgb}{0.000000,0.000000,0.000000}%
\pgfsetstrokecolor{currentstroke}%
\pgfsetdash{}{0pt}%
\pgfsys@defobject{currentmarker}{\pgfqpoint{-0.048611in}{0.000000in}}{\pgfqpoint{-0.000000in}{0.000000in}}{%
\pgfpathmoveto{\pgfqpoint{-0.000000in}{0.000000in}}%
\pgfpathlineto{\pgfqpoint{-0.048611in}{0.000000in}}%
\pgfusepath{stroke,fill}%
}%
\begin{pgfscope}%
\pgfsys@transformshift{0.800000in}{2.434803in}%
\pgfsys@useobject{currentmarker}{}%
\end{pgfscope}%
\end{pgfscope}%
\begin{pgfscope}%
\definecolor{textcolor}{rgb}{0.000000,0.000000,0.000000}%
\pgfsetstrokecolor{textcolor}%
\pgfsetfillcolor{textcolor}%
\pgftext[x=0.455382in, y=2.392594in, left, base]{\color{textcolor}{\sffamily\fontsize{8.000000}{9.600000}\selectfont\catcode`\^=\active\def^{\ifmmode\sp\else\^{}\fi}\catcode`\%=\active\def%{\%}1.00}}%
\end{pgfscope}%
\begin{pgfscope}%
\pgfsetbuttcap%
\pgfsetroundjoin%
\definecolor{currentfill}{rgb}{0.000000,0.000000,0.000000}%
\pgfsetfillcolor{currentfill}%
\pgfsetlinewidth{0.803000pt}%
\definecolor{currentstroke}{rgb}{0.000000,0.000000,0.000000}%
\pgfsetstrokecolor{currentstroke}%
\pgfsetdash{}{0pt}%
\pgfsys@defobject{currentmarker}{\pgfqpoint{-0.048611in}{0.000000in}}{\pgfqpoint{-0.000000in}{0.000000in}}{%
\pgfpathmoveto{\pgfqpoint{-0.000000in}{0.000000in}}%
\pgfpathlineto{\pgfqpoint{-0.048611in}{0.000000in}}%
\pgfusepath{stroke,fill}%
}%
\begin{pgfscope}%
\pgfsys@transformshift{0.800000in}{2.894930in}%
\pgfsys@useobject{currentmarker}{}%
\end{pgfscope}%
\end{pgfscope}%
\begin{pgfscope}%
\definecolor{textcolor}{rgb}{0.000000,0.000000,0.000000}%
\pgfsetstrokecolor{textcolor}%
\pgfsetfillcolor{textcolor}%
\pgftext[x=0.455382in, y=2.852721in, left, base]{\color{textcolor}{\sffamily\fontsize{8.000000}{9.600000}\selectfont\catcode`\^=\active\def^{\ifmmode\sp\else\^{}\fi}\catcode`\%=\active\def%{\%}1.25}}%
\end{pgfscope}%
\begin{pgfscope}%
\pgfsetbuttcap%
\pgfsetroundjoin%
\definecolor{currentfill}{rgb}{0.000000,0.000000,0.000000}%
\pgfsetfillcolor{currentfill}%
\pgfsetlinewidth{0.803000pt}%
\definecolor{currentstroke}{rgb}{0.000000,0.000000,0.000000}%
\pgfsetstrokecolor{currentstroke}%
\pgfsetdash{}{0pt}%
\pgfsys@defobject{currentmarker}{\pgfqpoint{-0.048611in}{0.000000in}}{\pgfqpoint{-0.000000in}{0.000000in}}{%
\pgfpathmoveto{\pgfqpoint{-0.000000in}{0.000000in}}%
\pgfpathlineto{\pgfqpoint{-0.048611in}{0.000000in}}%
\pgfusepath{stroke,fill}%
}%
\begin{pgfscope}%
\pgfsys@transformshift{0.800000in}{3.355057in}%
\pgfsys@useobject{currentmarker}{}%
\end{pgfscope}%
\end{pgfscope}%
\begin{pgfscope}%
\definecolor{textcolor}{rgb}{0.000000,0.000000,0.000000}%
\pgfsetstrokecolor{textcolor}%
\pgfsetfillcolor{textcolor}%
\pgftext[x=0.455382in, y=3.312848in, left, base]{\color{textcolor}{\sffamily\fontsize{8.000000}{9.600000}\selectfont\catcode`\^=\active\def^{\ifmmode\sp\else\^{}\fi}\catcode`\%=\active\def%{\%}1.50}}%
\end{pgfscope}%
\begin{pgfscope}%
\pgfsetbuttcap%
\pgfsetroundjoin%
\definecolor{currentfill}{rgb}{0.000000,0.000000,0.000000}%
\pgfsetfillcolor{currentfill}%
\pgfsetlinewidth{0.803000pt}%
\definecolor{currentstroke}{rgb}{0.000000,0.000000,0.000000}%
\pgfsetstrokecolor{currentstroke}%
\pgfsetdash{}{0pt}%
\pgfsys@defobject{currentmarker}{\pgfqpoint{-0.048611in}{0.000000in}}{\pgfqpoint{-0.000000in}{0.000000in}}{%
\pgfpathmoveto{\pgfqpoint{-0.000000in}{0.000000in}}%
\pgfpathlineto{\pgfqpoint{-0.048611in}{0.000000in}}%
\pgfusepath{stroke,fill}%
}%
\begin{pgfscope}%
\pgfsys@transformshift{0.800000in}{3.815184in}%
\pgfsys@useobject{currentmarker}{}%
\end{pgfscope}%
\end{pgfscope}%
\begin{pgfscope}%
\definecolor{textcolor}{rgb}{0.000000,0.000000,0.000000}%
\pgfsetstrokecolor{textcolor}%
\pgfsetfillcolor{textcolor}%
\pgftext[x=0.455382in, y=3.772975in, left, base]{\color{textcolor}{\sffamily\fontsize{8.000000}{9.600000}\selectfont\catcode`\^=\active\def^{\ifmmode\sp\else\^{}\fi}\catcode`\%=\active\def%{\%}1.75}}%
\end{pgfscope}%
\begin{pgfscope}%
\definecolor{textcolor}{rgb}{0.000000,0.000000,0.000000}%
\pgfsetstrokecolor{textcolor}%
\pgfsetfillcolor{textcolor}%
\pgftext[x=0.399826in,y=2.376000in,,bottom,rotate=90.000000]{\color{textcolor}{\sffamily\fontsize{8.000000}{9.600000}\selectfont\catcode`\^=\active\def^{\ifmmode\sp\else\^{}\fi}\catcode`\%=\active\def%{\%}Perda}}%
\end{pgfscope}%
\begin{pgfscope}%
\pgfpathrectangle{\pgfqpoint{0.800000in}{0.528000in}}{\pgfqpoint{4.960000in}{3.696000in}}%
\pgfusepath{clip}%
\pgfsetrectcap%
\pgfsetroundjoin%
\pgfsetlinewidth{1.505625pt}%
\definecolor{currentstroke}{rgb}{0.121569,0.466667,0.705882}%
\pgfsetstrokecolor{currentstroke}%
\pgfsetdash{}{0pt}%
\pgfpathmoveto{\pgfqpoint{1.025455in}{2.981836in}}%
\pgfpathlineto{\pgfqpoint{1.117477in}{2.580913in}}%
\pgfpathlineto{\pgfqpoint{1.209499in}{2.211098in}}%
\pgfpathlineto{\pgfqpoint{1.301521in}{1.953000in}}%
\pgfpathlineto{\pgfqpoint{1.393544in}{2.053127in}}%
\pgfpathlineto{\pgfqpoint{1.485566in}{1.808467in}}%
\pgfpathlineto{\pgfqpoint{1.577588in}{1.803727in}}%
\pgfpathlineto{\pgfqpoint{1.669610in}{1.601293in}}%
\pgfpathlineto{\pgfqpoint{1.761633in}{1.509424in}}%
\pgfpathlineto{\pgfqpoint{1.853655in}{1.890375in}}%
\pgfpathlineto{\pgfqpoint{1.945677in}{1.450369in}}%
\pgfpathlineto{\pgfqpoint{2.037699in}{1.992061in}}%
\pgfpathlineto{\pgfqpoint{2.129722in}{1.206939in}}%
\pgfpathlineto{\pgfqpoint{2.221744in}{1.094144in}}%
\pgfpathlineto{\pgfqpoint{2.313766in}{1.028031in}}%
\pgfpathlineto{\pgfqpoint{2.405788in}{1.063803in}}%
\pgfpathlineto{\pgfqpoint{2.497811in}{0.995924in}}%
\pgfpathlineto{\pgfqpoint{2.589833in}{0.937664in}}%
\pgfpathlineto{\pgfqpoint{2.681855in}{0.910338in}}%
\pgfpathlineto{\pgfqpoint{2.773878in}{0.900486in}}%
\pgfpathlineto{\pgfqpoint{2.865900in}{0.860262in}}%
\pgfpathlineto{\pgfqpoint{2.957922in}{0.839437in}}%
\pgfpathlineto{\pgfqpoint{3.049944in}{0.842728in}}%
\pgfpathlineto{\pgfqpoint{3.141967in}{0.831360in}}%
\pgfpathlineto{\pgfqpoint{3.233989in}{0.822363in}}%
\pgfpathlineto{\pgfqpoint{3.326011in}{0.813532in}}%
\pgfpathlineto{\pgfqpoint{3.418033in}{0.801776in}}%
\pgfpathlineto{\pgfqpoint{3.510056in}{0.772122in}}%
\pgfpathlineto{\pgfqpoint{3.602078in}{0.770939in}}%
\pgfpathlineto{\pgfqpoint{3.694100in}{0.758468in}}%
\pgfpathlineto{\pgfqpoint{3.786122in}{0.756549in}}%
\pgfpathlineto{\pgfqpoint{3.878145in}{0.844282in}}%
\pgfpathlineto{\pgfqpoint{3.970167in}{0.741143in}}%
\pgfpathlineto{\pgfqpoint{4.062189in}{0.802207in}}%
\pgfpathlineto{\pgfqpoint{4.154212in}{1.238536in}}%
\pgfpathlineto{\pgfqpoint{4.246234in}{0.784939in}}%
\pgfpathlineto{\pgfqpoint{4.338256in}{0.755688in}}%
\pgfpathlineto{\pgfqpoint{4.430278in}{0.734274in}}%
\pgfpathlineto{\pgfqpoint{4.522301in}{0.755454in}}%
\pgfpathlineto{\pgfqpoint{4.614323in}{0.767200in}}%
\pgfpathlineto{\pgfqpoint{4.706345in}{0.713537in}}%
\pgfpathlineto{\pgfqpoint{4.798367in}{0.696000in}}%
\pgfpathlineto{\pgfqpoint{4.890390in}{0.705462in}}%
\pgfpathlineto{\pgfqpoint{4.982412in}{0.701331in}}%
\pgfpathlineto{\pgfqpoint{5.074434in}{0.726402in}}%
\pgfpathlineto{\pgfqpoint{5.166456in}{0.747688in}}%
\pgfpathlineto{\pgfqpoint{5.258479in}{0.723636in}}%
\pgfpathlineto{\pgfqpoint{5.350501in}{0.721982in}}%
\pgfpathlineto{\pgfqpoint{5.442523in}{0.773369in}}%
\pgfpathlineto{\pgfqpoint{5.534545in}{0.726089in}}%
\pgfusepath{stroke}%
\end{pgfscope}%
\begin{pgfscope}%
\pgfpathrectangle{\pgfqpoint{0.800000in}{0.528000in}}{\pgfqpoint{4.960000in}{3.696000in}}%
\pgfusepath{clip}%
\pgfsetrectcap%
\pgfsetroundjoin%
\pgfsetlinewidth{1.505625pt}%
\definecolor{currentstroke}{rgb}{1.000000,0.498039,0.054902}%
\pgfsetstrokecolor{currentstroke}%
\pgfsetdash{}{0pt}%
\pgfpathmoveto{\pgfqpoint{1.025455in}{2.870444in}}%
\pgfpathlineto{\pgfqpoint{1.117477in}{3.074301in}}%
\pgfpathlineto{\pgfqpoint{1.209499in}{2.746835in}}%
\pgfpathlineto{\pgfqpoint{1.301521in}{3.090119in}}%
\pgfpathlineto{\pgfqpoint{1.393544in}{3.457218in}}%
\pgfpathlineto{\pgfqpoint{1.485566in}{2.920387in}}%
\pgfpathlineto{\pgfqpoint{1.577588in}{3.086752in}}%
\pgfpathlineto{\pgfqpoint{1.669610in}{2.869943in}}%
\pgfpathlineto{\pgfqpoint{1.761633in}{2.875684in}}%
\pgfpathlineto{\pgfqpoint{1.853655in}{2.943645in}}%
\pgfpathlineto{\pgfqpoint{1.945677in}{3.687376in}}%
\pgfpathlineto{\pgfqpoint{2.037699in}{2.940408in}}%
\pgfpathlineto{\pgfqpoint{2.129722in}{3.758921in}}%
\pgfpathlineto{\pgfqpoint{2.221744in}{3.217788in}}%
\pgfpathlineto{\pgfqpoint{2.313766in}{3.165654in}}%
\pgfpathlineto{\pgfqpoint{2.405788in}{3.741693in}}%
\pgfpathlineto{\pgfqpoint{2.497811in}{3.285033in}}%
\pgfpathlineto{\pgfqpoint{2.589833in}{3.345918in}}%
\pgfpathlineto{\pgfqpoint{2.681855in}{3.501570in}}%
\pgfpathlineto{\pgfqpoint{2.773878in}{3.594545in}}%
\pgfpathlineto{\pgfqpoint{2.865900in}{4.056000in}}%
\pgfpathlineto{\pgfqpoint{2.957922in}{3.475760in}}%
\pgfpathlineto{\pgfqpoint{3.049944in}{2.919331in}}%
\pgfpathlineto{\pgfqpoint{3.141967in}{2.864258in}}%
\pgfpathlineto{\pgfqpoint{3.233989in}{3.117855in}}%
\pgfpathlineto{\pgfqpoint{3.326011in}{3.000843in}}%
\pgfpathlineto{\pgfqpoint{3.418033in}{2.931534in}}%
\pgfpathlineto{\pgfqpoint{3.510056in}{2.969345in}}%
\pgfpathlineto{\pgfqpoint{3.602078in}{2.878694in}}%
\pgfpathlineto{\pgfqpoint{3.694100in}{3.061498in}}%
\pgfpathlineto{\pgfqpoint{3.786122in}{2.815878in}}%
\pgfpathlineto{\pgfqpoint{3.878145in}{3.013225in}}%
\pgfpathlineto{\pgfqpoint{3.970167in}{2.955611in}}%
\pgfpathlineto{\pgfqpoint{4.062189in}{2.919483in}}%
\pgfpathlineto{\pgfqpoint{4.154212in}{2.950114in}}%
\pgfpathlineto{\pgfqpoint{4.246234in}{2.881088in}}%
\pgfpathlineto{\pgfqpoint{4.338256in}{2.936635in}}%
\pgfpathlineto{\pgfqpoint{4.430278in}{2.938387in}}%
\pgfpathlineto{\pgfqpoint{4.522301in}{2.853391in}}%
\pgfpathlineto{\pgfqpoint{4.614323in}{2.830344in}}%
\pgfpathlineto{\pgfqpoint{4.706345in}{2.964292in}}%
\pgfpathlineto{\pgfqpoint{4.798367in}{2.831614in}}%
\pgfpathlineto{\pgfqpoint{4.890390in}{2.902355in}}%
\pgfpathlineto{\pgfqpoint{4.982412in}{2.873895in}}%
\pgfpathlineto{\pgfqpoint{5.074434in}{3.105755in}}%
\pgfpathlineto{\pgfqpoint{5.166456in}{3.148133in}}%
\pgfpathlineto{\pgfqpoint{5.258479in}{2.990375in}}%
\pgfpathlineto{\pgfqpoint{5.350501in}{3.117424in}}%
\pgfpathlineto{\pgfqpoint{5.442523in}{3.524095in}}%
\pgfpathlineto{\pgfqpoint{5.534545in}{2.910098in}}%
\pgfusepath{stroke}%
\end{pgfscope}%
\begin{pgfscope}%
\pgfsetrectcap%
\pgfsetmiterjoin%
\pgfsetlinewidth{0.803000pt}%
\definecolor{currentstroke}{rgb}{0.000000,0.000000,0.000000}%
\pgfsetstrokecolor{currentstroke}%
\pgfsetdash{}{0pt}%
\pgfpathmoveto{\pgfqpoint{0.800000in}{0.528000in}}%
\pgfpathlineto{\pgfqpoint{0.800000in}{4.224000in}}%
\pgfusepath{stroke}%
\end{pgfscope}%
\begin{pgfscope}%
\pgfsetrectcap%
\pgfsetmiterjoin%
\pgfsetlinewidth{0.803000pt}%
\definecolor{currentstroke}{rgb}{0.000000,0.000000,0.000000}%
\pgfsetstrokecolor{currentstroke}%
\pgfsetdash{}{0pt}%
\pgfpathmoveto{\pgfqpoint{5.760000in}{0.528000in}}%
\pgfpathlineto{\pgfqpoint{5.760000in}{4.224000in}}%
\pgfusepath{stroke}%
\end{pgfscope}%
\begin{pgfscope}%
\pgfsetrectcap%
\pgfsetmiterjoin%
\pgfsetlinewidth{0.803000pt}%
\definecolor{currentstroke}{rgb}{0.000000,0.000000,0.000000}%
\pgfsetstrokecolor{currentstroke}%
\pgfsetdash{}{0pt}%
\pgfpathmoveto{\pgfqpoint{0.800000in}{0.528000in}}%
\pgfpathlineto{\pgfqpoint{5.760000in}{0.528000in}}%
\pgfusepath{stroke}%
\end{pgfscope}%
\begin{pgfscope}%
\pgfsetrectcap%
\pgfsetmiterjoin%
\pgfsetlinewidth{0.803000pt}%
\definecolor{currentstroke}{rgb}{0.000000,0.000000,0.000000}%
\pgfsetstrokecolor{currentstroke}%
\pgfsetdash{}{0pt}%
\pgfpathmoveto{\pgfqpoint{0.800000in}{4.224000in}}%
\pgfpathlineto{\pgfqpoint{5.760000in}{4.224000in}}%
\pgfusepath{stroke}%
\end{pgfscope}%
\begin{pgfscope}%
\definecolor{textcolor}{rgb}{0.000000,0.000000,0.000000}%
\pgfsetstrokecolor{textcolor}%
\pgfsetfillcolor{textcolor}%
\pgftext[x=3.280000in,y=4.307333in,,base]{\color{textcolor}{\sffamily\fontsize{9.600000}{11.520000}\selectfont\catcode`\^=\active\def^{\ifmmode\sp\else\^{}\fi}\catcode`\%=\active\def%{\%}Função de Perda em Treino (compV)}}%
\end{pgfscope}%
\begin{pgfscope}%
\pgfsetbuttcap%
\pgfsetmiterjoin%
\definecolor{currentfill}{rgb}{1.000000,1.000000,1.000000}%
\pgfsetfillcolor{currentfill}%
\pgfsetfillopacity{0.800000}%
\pgfsetlinewidth{1.003750pt}%
\definecolor{currentstroke}{rgb}{0.800000,0.800000,0.800000}%
\pgfsetstrokecolor{currentstroke}%
\pgfsetstrokeopacity{0.800000}%
\pgfsetdash{}{0pt}%
\pgfpathmoveto{\pgfqpoint{4.635586in}{3.808723in}}%
\pgfpathlineto{\pgfqpoint{5.682222in}{3.808723in}}%
\pgfpathquadraticcurveto{\pgfqpoint{5.704444in}{3.808723in}}{\pgfqpoint{5.704444in}{3.830945in}}%
\pgfpathlineto{\pgfqpoint{5.704444in}{4.146222in}}%
\pgfpathquadraticcurveto{\pgfqpoint{5.704444in}{4.168444in}}{\pgfqpoint{5.682222in}{4.168444in}}%
\pgfpathlineto{\pgfqpoint{4.635586in}{4.168444in}}%
\pgfpathquadraticcurveto{\pgfqpoint{4.613364in}{4.168444in}}{\pgfqpoint{4.613364in}{4.146222in}}%
\pgfpathlineto{\pgfqpoint{4.613364in}{3.830945in}}%
\pgfpathquadraticcurveto{\pgfqpoint{4.613364in}{3.808723in}}{\pgfqpoint{4.635586in}{3.808723in}}%
\pgfpathlineto{\pgfqpoint{4.635586in}{3.808723in}}%
\pgfpathclose%
\pgfusepath{stroke,fill}%
\end{pgfscope}%
\begin{pgfscope}%
\pgfsetrectcap%
\pgfsetroundjoin%
\pgfsetlinewidth{1.505625pt}%
\definecolor{currentstroke}{rgb}{0.121569,0.466667,0.705882}%
\pgfsetstrokecolor{currentstroke}%
\pgfsetdash{}{0pt}%
\pgfpathmoveto{\pgfqpoint{4.657808in}{4.078470in}}%
\pgfpathlineto{\pgfqpoint{4.768919in}{4.078470in}}%
\pgfpathlineto{\pgfqpoint{4.880030in}{4.078470in}}%
\pgfusepath{stroke}%
\end{pgfscope}%
\begin{pgfscope}%
\definecolor{textcolor}{rgb}{0.000000,0.000000,0.000000}%
\pgfsetstrokecolor{textcolor}%
\pgfsetfillcolor{textcolor}%
\pgftext[x=4.968919in,y=4.039582in,left,base]{\color{textcolor}{\sffamily\fontsize{8.000000}{9.600000}\selectfont\catcode`\^=\active\def^{\ifmmode\sp\else\^{}\fi}\catcode`\%=\active\def%{\%}Treinamento}}%
\end{pgfscope}%
\begin{pgfscope}%
\pgfsetrectcap%
\pgfsetroundjoin%
\pgfsetlinewidth{1.505625pt}%
\definecolor{currentstroke}{rgb}{1.000000,0.498039,0.054902}%
\pgfsetstrokecolor{currentstroke}%
\pgfsetdash{}{0pt}%
\pgfpathmoveto{\pgfqpoint{4.657808in}{3.915168in}}%
\pgfpathlineto{\pgfqpoint{4.768919in}{3.915168in}}%
\pgfpathlineto{\pgfqpoint{4.880030in}{3.915168in}}%
\pgfusepath{stroke}%
\end{pgfscope}%
\begin{pgfscope}%
\definecolor{textcolor}{rgb}{0.000000,0.000000,0.000000}%
\pgfsetstrokecolor{textcolor}%
\pgfsetfillcolor{textcolor}%
\pgftext[x=4.968919in,y=3.876279in,left,base]{\color{textcolor}{\sffamily\fontsize{8.000000}{9.600000}\selectfont\catcode`\^=\active\def^{\ifmmode\sp\else\^{}\fi}\catcode`\%=\active\def%{\%}Validação}}%
\end{pgfscope}%
\end{pgfpicture}%
\makeatother%
\endgroup%
}
    \end{minipage}

    \caption{Evolução do treinamento com \textit{hypertuning} (à esquerda) e do treinamento com hiperparâmetros fixados (à direita) do sistema especialista da competência V.}
    \label{fig:exp-fix-c5}
\end{figure}

Pela figura \ref{fig:exp-fix-c5}, é possível notar que o treinamento com hiperparâmetros fixados apresentou uma evolução mais suave da função de perda, de modo que, a partir da 25ª época, o valor para a base de validação oscilou em torno de 1,3, com alguns pequenos picos no fim do processo, e, a partir da 20ª época, a perda para a base de treino convergiu para aproximadamente 0,1, apresentando apenas um aumento substancial posteriormente. No caso do \textit{hypertuning}, a função registrou números mais altos para a base de treino e de validação. Na primeira circunstância, o valor oscilou por volta de 1,8 em todas as iterações, enquanto que, na segunda, a perda desceu muito rápido para 1,7 nas primeiras épocas e assim permaneceu até o fim. Assim, nota-se que os parâmetros fixados possuem vantagens em relação ao \textit{hypertuning} para a competência V.

\section{Avaliação}
\label{sec:evaluate}

Após o treinamento das redes especialistas, experimentos de avaliação dos modelos criados foram conduzidos visando comparar o desempenho dos sistemas de correção automática. Para isso, utilizamos as bases de dados estendida e simples da Essay-BR, além das arquiteturas com hiperparâmetros otimizados e fixos, de modo que os resultados obtidos pudessem ser comparados.

Analisaremos o desempenho dos modelos utilizando as métricas apresentadas na seção \ref{subsec:methodology-evaluation} --- \textit{Quadratic Weighted Kappa} (\textbf{QWK}), Divergência (\textbf{DIV}), Proporção de Correspondência Exata (\textbf{PCE}) e \textit{Mean Squared Error} (\textbf{MSE}). Quanto maior o QWK e o PCE, melhor a qualidade da rede. De modo análogo, quanto menor a DIV e o MSE, melhor o desempenho do modelo.

Compararemos as versões desenvolvidas na tabela \ref{tab:eval-metrics}, em que \textbf{RTBE} e \textbf{RTBS} representam as redes com \textit{hypertuning} nas bases estendida e simples, respectivamente, e \textbf{RFBE} e \textbf{RFBS} representam as redes com hiperparâmetros fixos nas bases estendida e simples, também nessa ordem.

\begin{table}[H]
    \centering
    \begin{adjustbox}{width=\textwidth,totalheight=\textheight,keepaspectratio}
    \begin{tabular}{c|cccc|cccc|cccc|cccc|cccc}
        \toprule
        & \multicolumn{4}{c|}{Comp. I} & \multicolumn{4}{c|}{Comp. II} & \multicolumn{4}{c|}{Comp. III} & \multicolumn{4}{c|}{Comp. IV} & \multicolumn{4}{c}{Comp. V} \\
        & \textbf{DIV} & \textbf{MSE} & \textbf{PCE} & \textbf{QWK} & \textbf{DIV} & \textbf{MSE} & \textbf{PCE} & \textbf{QWK} & \textbf{DIV} & \textbf{MSE} & \textbf{PCE} & \textbf{QWK} & \textbf{DIV} & \textbf{MSE} & \textbf{PCE} & \textbf{QWK} & \textbf{DIV} & \textbf{MSE} & \textbf{PCE} & \textbf{QWK} \\ \midrule
        \textbf{RTBE} & 0,012 & 0,653 & 0,468 & 0,0 & 0,017 & 0,978 & 0,367 & 0,0 & 0,029 & 0,903 & 0,465 & 0,0 & 0,032 & 1,294 & 0,381 & 0,0 & 0,079 & 1,729 & 0,338 & 0,0 \\
        \textbf{RTBS} & 0,018 & 0,651 & 0,561 & 0,0 & 0,021 & 0,998 & 0,349 & 0,0 & 0,023 & 0,846 & 0,467 & 0,0 & 0,027 & 1,125 & 0,427 & 0,0 & 0,061 & 1,580 & 0,381 & 0,0 \\
        \textbf{RFBE} & \textbf{0,008} & \textbf{0,358} & 0,683 & \textbf{0,595} & 0,017 & 0,702 & 0,554 & 0,468 & 0,013 & 0,623 & 0,546 & 0,501 & 0,022 & 0,805 & \textbf{0,513} & 0,576 & 0,036 & 1,156 & 0,421 & 0,483 \\
        \textbf{RFBS} & 0.013 & 0.417 & \textbf{0.720} & 0.542 & \textbf{0.016} & \textbf{0.640} & \textbf{0.572} & \textbf{0.562} & \textbf{0.010} & \textbf{0.554} & \textbf{0.593} & \textbf{0.539} & \textbf{0.010} & \textbf{0.701} & 0.504 & \textbf{0.621} & \textbf{0.030} & \textbf{0.932} & \textbf{0.494} & \textbf{0.548} \\
        \bottomrule
    \end{tabular}
    \end{adjustbox}
    \caption{Métricas de avaliação dos modelos de correção automática por competência.}
    \label{tab:eval-metrics}
\end{table}

Nas seções \ref{subsec:eval-c1}, \ref{subsec:eval-c2}, \ref{subsec:eval-c3}, \ref{subsec:eval-c4} e \ref{subsec:eval-c5} avaliaremos os modelos construídos para cada competência, com base nas métricas apresentadas na tabela \ref{tab:eval-metrics}, e datalharemos o processo de inferência das notas por parte das melhores redes especialistas.

\subsection{Competência I}
\label{subsec:eval-c1}

A melhor rede especialista para a competência I foi a \textbf{RFBE}, que obteve um QWK de 0,595, uma divergência de de 0,008, um MSE de 0,358 e uma PCE de 0,683. A única métrica em que esse modelo ficou atrás foi a PCE, que foi maior para a \textbf{RFBS}. A figura \ref{fig:eval-c1-confusion-matrix} mostra a matriz de confusão para a rede.

\begin{figure}[H]
    \centering
    \resizebox{0.5\textwidth}{!}{\input{../figuras/matrizes/matrix_c1}}
    \caption{Matriz de confusão do modelo RFBE da competência I.}
    \label{fig:eval-c1-confusion-matrix}
\end{figure}

Em geral, a partir da figura \ref{fig:eval-c1-confusion-matrix}, é possível observar que a rede especialista apresentou um desempenho razoável, de modo que as notas inferidas, na maioria das vezes, estão próximas ou são iguais às notas reais. Nota-se, entretanto, uma dificuldade em atribuir notas extremas: o modelo geralmente erra ao inferir pontuações 0, 1 e 5.

\subsection{Competência II}
\label{subsec:eval-c2}

A melhor rede especialista para a competência II foi a \textbf{RFBS}, que obteve um QWK de 0,562, uma divergência de de 0,016, um MSE de 0,640 e uma PCE de 0,572. A figura \ref{fig:eval-c2-confusion-matrix} mostra a matriz de confusão para a rede.

\begin{figure}[H]
    \centering
    \resizebox{0.5\textwidth}{!}{\input{../figuras/matrizes/matrix_c2}}
    \caption{Matriz de confusão do modelo RFBS da competência II.}
    \label{fig:eval-c2-confusion-matrix}
\end{figure}

Em geral, a partir da figura \ref{fig:eval-c2-confusion-matrix}, é possível observar que a rede especialista apresentou um desempenho razoável, de modo que notas inferidas, também na maior parte das vezes, estão próximas ou são iguais às notas reais. Para pontuações 0, 1 e 2, no entanto, notamos que a distribuição das notas inferidas tem um comportamente análogo ao aleatório. O melhor modelo da competência II também tem dificuldades de generalização, já que não atribuiu nenhuma nota 0 ou 5 para a divisão de testes.

\subsection{Competência III}
\label{subsec:eval-c3}

A melhor rede especialista para a competência III foi a \textbf{RFBS}, que obteve um QWK de 0,539, uma divergência de de 0,010, um MSE de 0,554 e uma PCE de 0,593. A figura \ref{fig:eval-c3-confusion-matrix} mostra a matriz de confusão para a rede.

\begin{figure}[H]
    \centering
    \resizebox{0.5\textwidth}{!}{\input{../figuras/matrizes/matrix_c3}}
    \caption{Matriz de confusão do modelo RFBS da competência III.}
    \label{fig:eval-c3-confusion-matrix}
\end{figure}

Em geral, a partir da figura \ref{fig:eval-c3-confusion-matrix}, é possível observar que a rede especialista apresentou um desempenho razoável, de modo que as notas inferidas, também na maioria das vezes, estão próximas ou são iguais às notas reais. O melhor modelo da competência III teve dificuldades em lidar com notas extremas, já que errou em 100\% dos casos que atribuiu nota 5 para uma competência e acertou apenas 6\% das notas 0.

\subsection{Competência IV}
\label{subsec:eval-c4}

A melhor rede especialista para a competência IV foi a \textbf{RFBS}, que obteve um QWK de 0,621, uma divergência de de 0,010, um MSE de 0,701 e uma PCE de 0,504. O modelo só registrou subdesempenho na métrica de PCE, em relação à versão \textbf{RFBE}. A figura \ref{fig:eval-c4-confusion-matrix} mostra a matriz de confusão para a rede.

\begin{figure}[H]
    \centering
    \resizebox{0.5\textwidth}{!}{\input{../figuras/matrizes/matrix_c4}}
    \caption{Matriz de confusão do modelo RFBS da competência IV.}
    \label{fig:eval-c4-confusion-matrix}
\end{figure}

Em geral, a partir da figura \ref{fig:eval-c4-confusion-matrix}, é possível observar que a rede especialista apresentou um desempenho razoável, de modo que as notas inferidas, em boa parte das vezes, estão próximas ou são iguais às notas reais. O melhor modelo para a competência IV, no entanto, tem dificuldades em reconhecer notas baixas como 0 e 1, atribuindo-as inadequadamente em relação às pontuações reais.

\subsection{Competência V}
\label{subsec:eval-c5}

Por fim, a melhor rede especialista para a competência V foi a \textbf{RFBS}, que obteve um QWK de 0,548, uma divergência de de 0,030, um MSE de 0,932 e uma PCE de 0,494. A figura \ref{fig:eval-c5-confusion-matrix} mostra a matriz de confusão para a rede.

\begin{figure}[H]
    \centering
    \resizebox{0.5\textwidth}{!}{\input{../figuras/matrizes/matrix_c5}}
    \caption{Matriz de confusão do modelo RFBS da competência V.}
    \label{fig:eval-c5-confusion-matrix}
\end{figure}

Em geral, a partir da figura \ref{fig:eval-c5-confusion-matrix}, é possível observar que a rede especialista apresentou um desempenho razoável, de modo que as notas inferidas, para pontuações reais acima de 2, também estão próximas ou são iguais às notas reais. Nota-se que o melhor modelo para a competência V, no entanto, tem dificuldades em reconhecer notas baixas como 0 e 1, atribuindo quantidades significativas de pontuações 2 e 3.

\chapter{Conclusões}

Neste trabalho, foram explorados o desenvolvimento e a avaliação de modelos de correção automática de redações em língua portuguesa, com foco no contexto do ENEM no Brasil. Para isso, foi utilizada a base de dados Essay-BR, que contém diversos textos avaliados por profissionais, na construção de sistemas especializados em atribuir notas a cada competência exigida pelo exame.

A abordagem adotada utilizou o BERT, uma arquitetura baseada em \textit{transformers}, para extrair informações relevantes dos textos dissertativo-argumentativos produzidos pelos alunos, com o propósito de treinar modelos generalistas o suficiente para pontuar novas produções.

A avaliação dos sistemas produzidos também foi realizada com base no conjunto de dados da Essay-BR, considerando instâncias isoladas de teste. Nesse caso, diversas técnicas de treino, como o \textit{hypertuning}, foram empregadas, com o intuito de avaliar a eficácia real da construção desses modelos e responder se, de fato, são viáveis de aplicação frente à correção humana.

Durante o trabalho, um dos desafios enfrentados foi o antagonismo entre o tempo de treino e a convergência das redes no processo de \textit{hypertuning}. Devido ao grande espaço de exploração de hiperparâmetros, tornou-se necessário diminuir o número de épocas no treinamento de cada configuração, o que levou a escolhas genéricas de valores que não necessariamente refletiam o cenário de convergência do processo, mas eram mais sensíveis às inicializações aleatórias das redes. A forma plausível de se solucionar tal problema seria pelo aumento do número de épocas, que geraria, no entanto, períodos de treino mais longos.

Os resultados obtidos, contemplados no capítulo \ref{chap:experiments}, corroboraram essas dificuldades. Eles mostram que a técnica de \textit{hypertuning} não produz efeitos satisfatórios em relação ao aprendizado dos sistemas, já que, a partir dela, são escolhidos como melhores hiperparâmetros os valores que só otimizam as redes na média, mas não as generalizam. Além disso, dadas as frequentes oscilações da função de perda ao longo dos treinos com as melhores configurações, evidenciadas pelos gráficos \ref{fig:exp-hyp-c1}, \ref{fig:exp-hyp-c2}, \ref{fig:exp-hyp-c3}, \ref{fig:exp-hyp-c4} e \ref{fig:exp-hyp-c5}, nota-se que a quantidade de dados pode não ser suficiente para que essas redes aprendam adequadamente a tarefa de avaliação.

Por outro lado, a técnica de \textit{fine-tuning} com hiperparâmetros fixados, que foi utilizada para treinar os melhores modelos obtidos, mostrou-se mais eficaz, já que é possível realizar experimentos que contemplem um número maior de épocas utilizando-se de configurações com maiores chances de convergência. Apesar de ser igualmente difícil encontrar valores razoáveis iniciais, a formulação de possíveis heurísticas no método manual pode tornar o processo de escolha de novos hiperparâmetros mais rápido.

Outra constatação importante foi a da eficácia da Essay-BR básica em relação a sua versão estendida. Apesar de conter mais dados, notou-se, pela tabela \ref{tab:eval-metrics}, que a base com mais instâncias apresentou subdesempenho em relação às métricas do conjunto simplificado. Isso pode ser consequência da introdução das novas redações de uma mesma fonte à Essay-BR, que podem não ter agregado valores de generalização suficientes ao treinamento.

De modo geral, os resultados obtidos com os modelos treinados foram satisfatórios, já que os melhores sistemas foram capazes de pontuar as redações com notas próximas às atribuídas pelos avaliadores humanos. No entanto, a análise dos erros cometidos pelos modelos mostrou que ainda há muito espaço para melhorias, principalmente no que diz respeito à pontuação das competências II e V, que avaliam a adequação ao tema e a capacidade de elaborar uma proposta de intervenção para o problema, respectivamente.

Deduz-se que um dos impasses da avaliação dessas duas competências é a correlação intrínseca que ambas possuem com o tema das redações, informação não contemplada nos textos submetidos a treino. Uma das possíveis soluções seria adicionar esse dado no início da entrada de cada rede especialista, utilizando o \textit{token} especial \texttt{[SEP]} para realizar a divisão entre o tema e a redação.

Como abordagens futuras, sugere-se a construção de um mecanismo de explicabilidade dos modelos, que permita a análise de quais características dos textos são mais relevantes para a atribuição de notas. Além disso, a utilização de técnicas de \textit{data augmentation} pode ser uma alternativa para a melhoria dos resultados, já que a base de dados utilizada é relativamente pequena. Por fim, utilizar novos modelos de linguagem pode ser uma escolha viável para obter resultados melhores, permitindo uma comparação que não se limite ao nível da arquitetura das redes especialistas, mas também explore suas estruturas subjacentes.



%%%%%%%%%%%%%%%%%%%%%%%%%%%% APÊNDICES E ANEXOS %%%%%%%%%%%%%%%%%%%%%%%%%%%%%%%%

% Um apêndice é algum conteúdo adicional de sua autoria que faz parte e
% colabora com a ideia geral do texto mas que, por alguma razão, não precisa
% fazer parte da sequência do discurso; por exemplo, a demonstração de um
% teorema intermediário, as perguntas usadas em uma pesquisa qualitativa etc.
%
% Um anexo é um documento que não faz parte da tese (em geral, nem é de sua
% autoria) mas é relevante para o conteúdo; por exemplo, a especificação do
% padrão técnico ou a legislação que o trabalho discute, um artigo de jornal
% apresentando a percepção do público sobre o tema da tese etc.
%
% Os comandos appendix e annex reiniciam a numeração de capítulos e passam
% a numerá-los com letras. "annex" não faz parte de nenhuma classe padrão,
% foi criado para este modelo. Se o trabalho não tiver apêndices ou anexos,
% remova estas linhas.
%
% Diferentemente de \mainmatter, \backmatter etc., \appendix e \annex não
% forçam o início de uma nova página. Em geral isso não é importante, pois
% o comando seguinte costuma ser "\chapter", mas pode causar problemas com
% a formatação dos cabeçalhos. Assim, vamos forçar uma nova página antes
% de cada um deles.

%%%%%%%%%%%%%%% SEÇÕES FINAIS (BIBLIOGRAFIA E ÍNDICE REMISSIVO) %%%%%%%%%%%%%%%%

% O comando backmatter desabilita a numeração de capítulos.
\backmatter

\pagestyle{backmatter}

% Espaço adicional no sumário antes das referências / índice remissivo
\addtocontents{toc}{\vspace{2\baselineskip plus .5\baselineskip minus .5\baselineskip}}

% A bibliografia é obrigatória

\printbibliography[
  title=\refname\label{bibliografia}, % "Referências", recomendado pela ABNT
  %title=\bibname\label{bibliografia}, % "Bibliografia"
  heading=bibintoc, % Inclui a bibliografia no sumário
]

\end{document}

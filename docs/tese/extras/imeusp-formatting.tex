%%%%%%%%%%%%%%%%%%%%%%%%%%%%%%%%%%%%%%%%%%%%%%%%%%%%%%%%%%%%%%%%%%%%%%%%%%%%%%%%
%%%%%%%%%%%%%%%%%%%%%%% SUMÁRIO, CABEÇALHOS, SEÇÕES %%%%%%%%%%%%%%%%%%%%%%%%%%%%
%%%%%%%%%%%%%%%%%%%%%%%%%%%%%%%%%%%%%%%%%%%%%%%%%%%%%%%%%%%%%%%%%%%%%%%%%%%%%%%%

% Formatação personalizada do sumário, lista de tabelas/figuras etc.
%\usepackage{titletoc}

% titlesec permite definir formatação personalizada de títulos, seções etc.
% Observe que titlesec é incompatível com os comandos refsection
% e refsegment do pacote biblatex!
% Vamos usar titlesec apenas
% para fazer títulos, seções etc. não serem justificados.
\usepackage[raggedright]{titlesec}

% Permite saber o número total de páginas; útil para colocar no
% rodapé algo como "página 3 de 10" com "\thepage\ de \zpageref{LastPage}"
%\usepackage{zref-lastpage,zref-user}

% Permite definir cabeçalhos e rodapés
%\usepackage{fancyhdr}

% Personalização de cabeçalhos e rodapés com o estilo deste modelo
\usepackage{imeusp-headers} % carregado do diretório extras (veja basics.tex)

% biblatex pode ser configurado para inserir a bibliografia no sumário;
% bibtex não oferece essa possibilidade. Com esta package, resolvemos
% esse problema.
\usepackage[nottoc,notlot,notlof]{tocbibind}

% Só olha até o nível 2 (subseções) para gerar o sumário e os
% cabeçalhos, ou seja, não coloca nomes de subsubseções (nível 3)
% no sumário nem nos cabeçalhos.
\setcounter{tocdepth}{2}

% Só numera até o nível 2 (subseções, como 2.3.1), ou seja, não numera
% sub-subseções (como 2.3.1.1). Veja que isso afeta referências
% cruzadas: se você fizer \ref{uma-sub-subsecao} sem que ela seja
% numerada, a referência vai apontar para a seção um nível acima.
\setcounter{secnumdepth}{2}

% Normalmente, o capítulo de introdução não deve ser numerado, mas
% deve aparecer no sumário. Por padrão, LaTeX não oferece uma solução
% para isso, então criamos aqui os comandos \unnumberedchapter,
% \unnumberedsection e \unnumberedsubsection.
\newcommand{\unnumberedchapter}[2][]{
  \ifblank{#1}
    {
      \chapter*{#2}
      \phantomsection
      \addcontentsline{toc}{chapter}{#2}
      \chaptermark{#2}
    }
    {
      \chapter*{#2}
      \phantomsection
      \addcontentsline{toc}{chapter}{#1}
      \chaptermark{#1}
    }
}

\newcommand{\unnumberedsection}[2][]{
  \ifblank{#1}
    {
      \section*{#2}
      \phantomsection
      \addcontentsline{toc}{section}{#2}
      \sectionmark{#2}
    }
    {
      \section*{#2}
      \phantomsection
      \addcontentsline{toc}{section}{#1}
      \sectionmark{#1}
    }
}

\newcommand{\unnumberedsubsection}[2][]{
  \ifblank{#1}
    {
      \subsection*{#2}
      \phantomsection
      \addcontentsline{toc}{subsection}{#2}
    }
    {
      \subsection*{#2}
      \phantomsection
      \addcontentsline{toc}{subsection}{#1}
    }
}


%%%%%%%%%%%%%%%%%%%%%%%%%%%%%%%%%%%%%%%%%%%%%%%%%%%%%%%%%%%%%%%%%%%%%%%%%%%%%%%%
%%%%%%%%%%%%%%%%%%%%%%%%%% ESPAÇAMENTO E ALINHAMENTO %%%%%%%%%%%%%%%%%%%%%%%%%%%
%%%%%%%%%%%%%%%%%%%%%%%%%%%%%%%%%%%%%%%%%%%%%%%%%%%%%%%%%%%%%%%%%%%%%%%%%%%%%%%%

% LaTeX por default segue o estilo americano e não faz a indentação da
% primeira linha do primeiro parágrafo de uma seção; este pacote ativa essa
% indentação, como é o estilo brasileiro
\usepackage{indentfirst}

% A primeira linha de cada parágrafo costuma ter um pequeno recuo para
% tornar mais fácil visualizar onde cada parágrafo começa. Além disso, é
% possível colocar um espaço em branco entre um parágrafo e outro. Esta
% package coloca o espaço em branco e desabilita o recuo; como queremos
% o espaço *e* o recuo, é preciso guardar o valor padrão do recuo e
% redefini-lo depois de carregar a package.
% TODO: depois que ubuntu 18.04 se tornar obsoleta (abril/2023), remover
%       as linhas "oldparindent" e carregar a package com a opção "indent".
\newlength\oldparindent
\setlength\oldparindent\parindent
\usepackage[parfill]{parskip}
\setlength\parindent\oldparindent


%%%%%%%%%%%%%%%%%%%%%%%%%%%%%%%%%%%%%%%%%%%%%%%%%%%%%%%%%%%%%%%%%%%%%%%%%%%%%%%%
%%%%%%%%%%%%%%%%%%%%%%%%%% EPÍGRAFE E NOTAS DE RODAPÉ %%%%%%%%%%%%%%%%%%%%%%%%%%
%%%%%%%%%%%%%%%%%%%%%%%%%%%%%%%%%%%%%%%%%%%%%%%%%%%%%%%%%%%%%%%%%%%%%%%%%%%%%%%%

% O formato padrão do pacote epigraph é bem feinho...
% Outra opção para epígrafes é o pacote quotchap
\usepackage{epigraph}

\setlength\epigraphwidth{.85\textwidth}

% Sem linha entre o texto e o autor
\setlength{\epigraphrule}{0pt}

% Ambiente auxiliar para colocar margem à direita da epígrafe
% (como sempre, o modo mais simples de mudar as margens de um
% pagrágrafo é fazer de conta que é uma lista)
\newenvironment{epShiftLeft}
  {
    \par\begin{list}{}
      {
        \leftmargin 0pt
        \labelwidth 0pt
        \labelsep 0pt
        \itemsep 0pt
        \topsep 0pt
        \partopsep 0pt
        \rightmargin 2em
      }
    \item\FlushRight
  }
  {
    \end{list}
    % O espaço padrão que epigraph coloca entre a citação
    % e o autor é muito pequeno; vamos aumentar um pouco
    \vspace*{.3\baselineskip}
  }

\renewcommand\textflush{epShiftLeft}
\renewcommand\sourceflush{epShiftLeft}

\newcommand{\epigrafe}[2] {%
  \ifstrempty{#2}{
    \epigraph{\itshape #1}{}
  }{
    \epigraph{\itshape #1}{--- #2}
  }
}

% Formato personalizado para as notas de rodapé. Copiado quase
% literalmente do exemplo na documentação das classes-padrão de
% LaTeX2e (texdoc classes). Seria possível fazer algo similar
% usando list{} com um único item usando \@thefnmark como label.

% \footnotesep não é um espaço adicional, mas sim um strut que
% existe no começo de cada nota. É por isso que o valor é "grande"
% (\baselineskip) mas a separação de fato é pequena.
\makeatletter
\renewcommand\@makefntext[1]{%
    \setlength{\footnotesep}{1\baselineskip}%
    \@setpar{%
        \@@par
        \@tempdima = \hsize
        \advance\@tempdima-4pt\relax
        \parshape \@ne 4pt \@tempdima
    }%
    \par
    \parindent 1em\noindent
    \parskip .3\baselineskip
    \hbox to \z@{\hss\@makefnmark\,}#1%
}
\makeatother

% \maketitle redefine as notas de rodapé (\thanks) para usar símbolos
% ao invés de números, mas essa não é a única mudança. \maketitle
% também muda \@makefnmark para que a indicação de nota de rodapé
% não ocupe espaço horizontal (isso é feito com \rlap). Isso é feito
% porque a lista de autores em geral é similar a
% \author{Fulano\thanks{instituição 1}, Ciclano\thanks{instituição 2}}.
% Com essa mudança, a nota aparece acima da vírgula entre os autores.
% Mas isso significa que\maketitle precisa também modificar \@makefntext
% para que esse efeito aconteça apenas na lista de autores e não na
% nota em si. Assim, como criamos um novo formato para as notas de
% rodapé, precisamos mudar o formato em \maketitle também.
\makeatletter
\newcommand\@maketitlemakefntext[1]{%
    \setlength{\footnotesep}{1\baselineskip}%
    \@setpar{%
        \@@par
        \@tempdima = \hsize
        \advance\@tempdima-4pt\relax
        \parshape \@ne 4pt \@tempdima
    }%
    \par
    \parindent 1em\noindent
    \parskip .3\baselineskip
    \hbox to \z@{\hss\@textsuperscript{\normalfont\@thefnmark}\,}#1%
}

\patchcmd\maketitle
  {\long\def\@makefntext}
  {\let\@makefntext\@maketitlemakefntext\long\def\@disabledmakefntext}
  {}{}

\makeatother

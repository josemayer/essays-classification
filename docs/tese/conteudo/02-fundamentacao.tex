\chapter{Fundamentação Teórica}

\section{A Redação no ENEM}

A prova de redação no ENEM desempenha um papel fundamental na qualificação das habilidades dos estudantes, medindo a capacidade de expressarem ideias de maneira clara, coerente e persuasiva. Para avaliar as redações, são estabelecidos critérios objetivos que consideram a estrutura do texto e métricas específicas do exame, baseadas em uma matriz de referência.

\subsection{Estrutura do Texto}

A estrutura da redação no ENEM segue o formato dissertativo-argumentativo, exigindo que os participantes defendam um ponto de vista sobre um tema previamente estabelecido, conforme abordado em \cite{cartilha-redacao}. A composição textual deve respeitar a norma padrão da língua portuguesa e o gênero determinado, mantendo uma introdução, desenvolvimento e conclusão. A clareza na exposição de ideias, a adequação ao tema e a coesão textual são elementos essenciais para uma boa pontuação.

Para os autores \citet{platao-e-fiorin}, a dissertação deve explicitar uma visão concreta da realidade, recorrendo a referências para ilustrar afirmações ou para sustentar argumentos. No contexto do ENEM, esse é o papel exercido pela coesão e articulação da redação, já que o texto deve manter uma progressão lógica e um encadeamento de ideias com o fim de defender o ponto de vista do aluno. Além disso, os autores defendem que a dissertação sempre deve seguir um eixo temático à medida que aborda conceitos do mundo de modo abstrato, sem relações temporais ou espaciais. Nesse sentido, na prova, é imperativo que o texto apresente uma coerência temática, alinhando-se ao tópico proposto.

Ainda ao longo do texto, os estudantes devem articular informações externas para embasar seus discursos, o que é definido, em \cite{platao-e-fiorin}, como argumentos de autoridade e argumentos baseados em provas concretas. Assim, aliado à seleção de elementos coesivos adequados, o propósito da redação deve ser o de convencer o interlocutor em torno do tema abordado.

Por fim, é exigido que os alunos apresentem uma proposta de intervenção na conclusão do texto, que deve conter as ações necessárias para mitigar o problema levantado, os agentes responsáveis, os meios de solução, os possíveis desdobramentos da aplicação da proposta e, por fim, um detalhamento mais refinado da mediação (\cite[p.~20]{cartilha-redacao}).

\subsection{Métricas de Avaliação}

A avaliação das redações no ENEM adota uma abordagem multifacetada, incorporando diversas métricas para assegurar uma análise abrangente, incluindo desde a coesão textual até a capacidade de argumentação e a adequação ao padrão culto da língua. A seleção e organização eficazes de informações também são critérios fundamentais. A combinação desses parâmetros contribui significativamente para a construção de uma avaliação equitativa e holística das habilidades de escrita dos participantes.

Cada redação no ENEM é avaliada com base na matriz de referência do exame, composta por cinco competências, delineadas em suas respectivas habilidades, conforme detalhado em \cite{cartilha-redacao}. A primeira competência concentra-se na avaliação do emprego correto da língua portuguesa, abrangendo aspectos como ortografia, pontuação, concordância, regência verbal, etc. A segunda competência analisa a capacidade de interpretação do participante em relação ao tema proposto. Na terceira competência, são avaliados critérios como seleção e organização de informações, articulação de ideias e construção de argumentos. A quarta competência incide sobre o encadeamento dos argumentos e a progressão temática do texto. Por fim, a quinta competência avalia a habilidade do participante em elaborar uma proposta de intervenção para o problema abordado, considerando a adequação ao tema, a viabilidade prática de aplicação e o respeito aos direitos humanos. As cinco competências podem ser sintetizadas, de acordo com a Cartilha de Redação do ENEM (\cite{cartilha-redacao}), em:

\begin{enumerate}
    \item[\textbf{I.}] Demonstrar domínio da modalidade escrita formal da língua portuguesa.
    \item[\textbf{II.}] Compreender a proposta de redação e aplicar conceitos das várias áreas de conhecimento para desenvolver o tema, dentro dos limites estruturais do texto dissertativo-argumentativo em prosa.
    \item[\textbf{III.}] Selecionar, relacionar, organizar e interpretar informações, fatos, opiniões e argumentos em defesa de um ponto de vista.
    \item[\textbf{IV.}] Demonstrar conhecimento dos mecanismos linguísticos necessários para a construção da argumentação.
    \item[\textbf{V.}] Elaborar proposta de intervenção para o problema abordado, respeitando os direitos humanos.
\end{enumerate}

\subsection{Atribuição de Notas}

A atribuição de notas na redação do ENEM segue um processo rigoroso. Cada uma das cinco competências é avaliada em uma escala de 0 a 200 pontos, variando, nesse intervalo, em valores múltiplos de 40. As redações são corrigidas, inicialmente, por dois especialistas independentes e todo o processo é meticulosamente estruturado para prevenir possíveis divergências nas pontuações finais.

Duas avaliações são consideradas divergentes caso a nota atribuida a qualquer uma das competências difira em mais de 80 pontos ou se a diferença total entre as notas seja superior a 100 pontos. Nesses casos, uma terceira correção é realizada, e a nota final é a média aritmética das duas avaliações mais próximas. Caso todas as avaliações ainda sejam discrepantes entre si, a redação é submetida a uma banca independente, responsável por atribuir a nota final ao texto.

As notas levam em consideração as métricas da matriz de referência do exame, conforme estabelecido em \cite[p.~9-22]{cartilha-redacao}. A pontuação final varia de 0 a 1000 pontos, sendo calculada a partir da soma dos pontos atribuídos a cada competência. Este sistema de avaliação visa garantir uma análise abrangente e justa das redações dos participantes.

A tabela \ref{tab:competencia-1} abaixo, extraída da Cartilha de Redação do ENEM, ilustra a associação entre as pontuações atribuídas à competência I e os níveis de desempenho esperados dos alunos. As demais competências possuem uma associação similar de acordo com seus respectivos campos de avaliação.

\begin{table}[H]
    \centering
    \caption{Níveis de desempenho esperados para a competência I e notas associadas. Tabela extraída de \cite[p.~10]{cartilha-redacao}}
    \label{tab:competencia-1}
    \begin{tabularx}{\textwidth}{|c|X|}
        \hline
        \textbf{Pontuação} & \textbf{Níveis de desempenho} \\
        \hline
        200 pontos & Demonstra excelente domínio da modalidade escrita formal da língua portuguesa e de escolha de registro. Desvios gramaticais ou de convenções da escrita serão aceitos somente como excepcionalidade e quando não caracterizarem reincidência. \\
        \hline
        160 pontos & Demonstra bom domínio da modalidade escrita formal da língua portuguesa e de escolha de registro, com poucos desvios gramaticais e de convenções da escrita. \\
        \hline
        120 pontos & Demonstra domínio mediano da modalidade escrita formal da língua portuguesa e de escolha de registro, com alguns desvios gramaticais e de convenções da escrita. \\
        \hline
        80 pontos & Demonstra domínio insuficiente da modalidade escrita formal da língua portuguesa, com muitos desvios gramaticais, de escolha de registro e de convenções da escrita. \\
        \hline
        40 pontos & Demonstra domínio precário da modalidade escrita formal da língua portuguesa, de forma sistemática, com diversificados e frequentes desvios gramaticais, de escolha de registro e de convenções da escrita. \\
        \hline
        0 pontos & Demonstra desconhecimento da modalidade escrita formal da língua portuguesa. \\
        \hline
    \end{tabularx}
\end{table}

%\section{Avaliação Automática de Redações}

%\subsection{Definição e Importância}

%\subsection{Breve Histórico}

%\section{Representação de Textos para o Computador}

%\subsection{N-gramas}

%\subsection{Word Embeddings}

%\subsubsection{Word2Vec}

%\subsubsection{GloVe}

%\subsubsection{FastText}

%\section{Modelos Iniciais de Avaliação Automática de Redações}

%\subsection{Modelos Baseados em Extração de Características}

%\subsection{Modelos Baseados em Análise de Estruturas}

%\subsection{Modelos Baseados em Regressão}

%\section{Modelos de Linguagem}

%\subsection{Modelos Baseados em N-gramas}

%\subsection{Redes Neurais Recorrentes}

%\subsection{Long Short-Term Memory}

%\subsection{Gated Recurrent Unit}

%\section{Modelos de Linguagem Baseados em Transformadores}

%\subsection{Arquitetura}

%\subsection{Mecanismo de Atenção}

%\section{BERT: Bidirectional Encoder Representations from Transformers}

%\subsection{Arquitetura}

%\subsection{Pré-treinamento}

%\subsection{Fine-tuning}

%\subsection{BERTimbau: BERT para o Português Brasileiro}

%\subsection{O BERT na Avaliação Automática de Redações}

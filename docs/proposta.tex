\documentclass{article}
\usepackage[utf8]{inputenc}
\usepackage[portuguese]{babel}
\usepackage{hyperref}
\usepackage{multirow}
\usepackage{graphicx}

\title{
    Avaliação Automática de Redações Dissertativo-Argumentativas no Modelo do
    ENEM\\
    \vspace{0.5cm}
    \large Trabalho de Formatura Supervisionado (MAC0499)
}
\author{
    \textbf{Aluno}: José Lucas Silva Mayer
    \and
    \textbf{Orientador}: Denis Deratani Mauá
    \and
    \textbf{Co-orientador}: Igor Cataneo Silveira
}
\date{Abril, 2023}

\begin{document}

\maketitle

\section*{\centering Resumo}

O Exame Nacional do Ensino Médio (ENEM) é uma das principais formas de acesso ao
ensino superior no Brasil e abrange diversas áreas do conhecimento. Uma parte
dele é composta pela prova de redação, que é aplicada para avaliar a habilidade
de produção textual dos alunos sobre um tema relevante em âmbito nacional. A
correção das redações, cujo modelo é o dissertativo-argumentativo, é realizada
manualmente por avaliadores, o que torna o processo custoso e demorado. Nesse
sentido, o uso de técnicas de inteligência artificial para automatizar a
correção desses textos pode representar uma alternativa viável para agilizar a
tarefa. Diante disso, este tema proposto tem como objetivo explorar o uso de
técnicas de aprendizado profundo para a classificação automática de redações
dissertativo-argumentativas do ENEM, com base em modelos pré-treinados de
processamento de linguagem natural para a língua portuguesa. Visa-se, com isso,
definir um projeto cujo intuito é investigar arquiteturas de redes neurais para
ajustar os modelos e avaliar os resultados com métricas de precisão conhecidas.
Essa pesquisa poderá ser um avanço para o desenvolvimento de sistemas
automáticos de correção de redações em língua portuguesa, bem como para
aprimorar e facilitar o processo de avaliação dos estudantes no ENEM.

\vspace{0.3cm}

\noindent \textbf{Palavras-chave}: Redes Neurais $\bullet$ Processamento de
Linguagem Natural $\bullet$ Correção Automática $\bullet$ Redações.

\newpage

\section{Introdução}

Uma das principais habilidades exigidas dos estudantes do ensino básico é a
produção textual. O Exame Nacional do Ensino Médio (ENEM), prova aplicada
nacionalmente com o intuito de analisar o conhecimento construído no período de
formação básica dos alunos, exige que os estudantes redijam uma redação no
modelo dissertativo-argumentativo, a partir de uma proposta de tema e de textos
de apoio norteadores. A avaliação desses textos é realizada com base em cinco
competências cujas notas podem valer de 0 a 200, de modo que ao menos dois
professores avaliam independentemente cada redação, conforme a cartilha de
orientação do exame em 2022 \cite{Cartilha}.

O modelo de correção adotado pelo ENEM, no entanto, demanda muito tempo e muitos
recursos, e a quantidade de redações que precisam ser avaliadas por cada
corretor pode comprometer a qualidade do processo. Nesse contexto, a utilização
de técnicas de inteligência artificial tem se mostrado uma alternativa viável
para automatizar a correção de redações.

Um dos principais desafios para a tarefa é a escassez de dados em língua
portuguesa disponíveis para o treinamento de modelos de aprendizado de máquina,
impactando na precisão dos sistemas desenvolvidos. Recentemente, organizou-se
uma coletânea aberta de redações no modelo do ENEM e suas avaliações
correspondentes, o que abre novas possibilidades para a pesquisa na área.

Nesse contexto, esta proposta de trabalho sugere o uso de técnicas de
aprendizado profundo para avaliar redações de acordo com as competências do
ENEM, baseando-se em modelos conhecidos de processamento de linguagem natural,
pré-treinados com foco na língua portuguesa. As próximas seções, em ordem,
detalham as justificativas, os objetivos, a metodologia proposta e, por fim, o
cronograma esperado do projeto.

\section{Justificativa}
A avaliação de textos dissertativo-argumentativos no ENEM é um processo que
demanda muito tempo devido ao modelo de correção utilizado e à falta de
equilíbrio entre o número de redações e o número de corretores disponíveis, como
mencionado por Lesme (2021) \cite{Corretores}. Por conta disso, a classificação
automática de redações no modelo ENEM já foi objeto de estudos anteriores na
área de inteligência artificial, incluindo, dentre outros, o desenvolvimento de
um classificador de Bayes por Bazelato e Amorim (2013) \cite{Bayes} e a extração
e análise automática do volume de aspectos textuais das dissertações por Veloso
e Amorim (2017) \cite{Multi-Aspect}.

Como o ENEM segue diretrizes bem definidas, a tarefa de correção tem padrões
intrínsecos que podem ser detectados e compreendidos com o uso do aprendizado
profundo. No entanto, a maioria dos trabalhos anteriores foi baseada em técnicas
de inteligência artificial mais elementares, que não lidam com grandes volumes
de textos para classificação. Isso ocorre porque a correção automática de
redações em português é uma área que enfrenta o desafio da falta de dados na
linguagem, como mencionado por Veloso e Amorim (2017) \cite{Multi-Aspect}, o que
limita a precisão do treinamento.

Um estudo recente de Marinho \textit{et al}. (2021) \cite{Dataset} organizou uma
coletânea aberta de redações no modelo do ENEM e suas avaliações
correspondentes, atribuídas por profissionais da área de acordo com os critérios
de correção do exame. Assim, com esse maior conjunto de dados, é possível
investigar o uso de técnicas mais avançadas de aprendizado supervisionado na
correção automática e comparar o desempenho de modelos mais robustos com as
abordagens anteriormente realizadas nesse campo.

\section{Objetivos}

Baseando-se na quantidade de dados de redações no modelo do ENEM
disponibilizadas abertamente por Marinho \textit{et al} (2021) \cite{Dataset}.,
assim como nas motivações anteriormente apresentadas, almeja-se, com a
proposição desse trabalho:

\begin{itemize}
    \item Utilizar técnicas de aprendizado profundo para classificar redações em
    notas de acordo com as competências do ENEM.
    \item Investigar arquiteturas de redes neurais para o ajuste fino de modelos
    pré-treinados que possibilitem a avaliação automática de redações.
    \item Avaliar os modelos de correção automática gerados, baseando-se em suas
    medidas de precisão.
    \item Adquirir familiaridade com bibliotecas de aprendizado de máquina e de
    avaliação estatística de dados.
\end{itemize}

\section{Metodologia}

Para realizar a avaliação automática de redações no modelo do ENEM, pretende-se
utilizar uma abordagem baseada em técnicas de aprendizado profundo, a partir do
ajuste fino da arquitetura de redes neurais conhecida como BERT
(\textbf{B}idirectional \textbf{E}ncoder \textbf{R}epresentations from
\textbf{T}ransformers), apresentada pelo artigo de Devlin \textit{et al}. (2018)
\cite{BERT}.

O BERT é um modelo pré-treinado de grande escala, especializado em tarefas
envolvendo o processamento de linguagem natural, que utiliza uma arquitetura
baseada em um \textit{transformer} para capturar relações de dependência entre
palavras em textos. Essa arquitetura é especialmente adequada para lidar com a
complexidade da língua portuguesa, devido à existência de uma ordem livre de
palavras e dependências sintáticas complexas dentro da linguagem.

Para o desenvolvimento do projeto, objetiva-se utilizar o BERTimbau, a versão
pré-treinada do BERT em português desenvolvida por Souza \textit{et al}. (2019)
\cite{BERTimbau}. Esse modelo foi construído com base em uma grande quantidade
de textos em português, permitindo a captura de características específicas da
língua que possam aparecer nas redações a serem avaliadas.

A partir do BERTimbau, pretende-se criar uma arquitetura capaz de refinar o
\textit{transformer} pré-treinado para cumprir a tarefa de correção automática
dos textos de acordo com as cinco competências exigidas pelo ENEM. Essa
arquitetura será treinada no conjunto de dados contendo as redações e suas
respectivas notas, disponibilizado por Marinho \textit{et al}. (2021)
\cite{Dataset}. A avaliação do desempenho da arquitetura será realizada por meio
de métricas como acurácia, \textit{Quadratic Weighted Kappa} (QWK) (Cohen, 1968
\cite{QWK}), F1-\textit{score}, etc.

Para a implementação da arquitetura de redes neurais baseada no BERTimbau,
pretende-se utilizar a biblioteca do Tensorflow, que fornece uma ampla gama de
ferramentas para o treinamento e avaliação de modelos de aprendizado profundo.
Para a etapa de  manipulação do \textit{dataset}, pretende-se utilizar, também,
a biblioteca Pandas, que oferece uma forma fácil de lidar e extrair informações
de um grande conjunto de dados. Além disso, para acelerar o processo de
treinamento, serão utilizadas máquinas com GPU disponibilizadas pela ferramenta
Google Colab, que permitem uma execução mais rápida e eficiente de tarefas
computacionalmente intensivas.

Por fim, a análise dos resultados obtidos será realizada para identificar
arquiteturas que tenham um bom desempenho na classificação dos textos
dissertativo-argumentativos e para avaliar o uso das técnicas de apendizado
profundo em relação a outras abordagens já realizadas.

\section{Cronograma}

\begin{table}[htbp]
  \centering
  \renewcommand{\arraystretch}{1.5}
  \resizebox{\textwidth}{!}{%
    \begin{tabular}{|p{4cm}|*{9}{c|}}
      \hline
      \multirow{2}{*}{Tarefas} & \multicolumn{9}{c|}{Meses} \\ \cline{2-10}
      & Abril & Maio & Junho & Julho & Agosto & Setembro & Outubro & Novembro &
      Dezembro \\ \hline
      Revisão bibliográfica & \scalebox{2}{$\times$} & \scalebox{2}{$\times$} &
      \scalebox{2}{$\times$} & & & & & & \\ \hline
      Estudo dos modelos de linguagem pré-treinados (BERT e BERTimbau) e do
      \textit{dataset} de redações. & \scalebox{2}{$\times$} &
      \scalebox{2}{$\times$} & \scalebox{2}{$\times$} & & & & & & \\ \hline
      Elaboração da arquitetura de ajuste fino para avaliar as competências em
      separado. & & \scalebox{2}{$\times$} & \scalebox{2}{$\times$} &
      \scalebox{2}{$\times$} & \scalebox{2}{$\times$} & & & & \\ \hline
      Aprimoramento da estrutura dos avaliadores para o cálculo da nota final. &
      & & & & \scalebox{2}{$\times$} & \scalebox{2}{$\times$} &
      \scalebox{2}{$\times$} & & \\ \hline
      Avaliação das arquiteturas de aprendizado e análise dos resultados
      obtidos. & & & & & & \scalebox{2}{$\times$} & \scalebox{2}{$\times$} & &
      \\ \hline
      Elaboração da monografia. & & & & & & \scalebox{2}{$\times$} &
      \scalebox{2}{$\times$} & \scalebox{2}{$\times$} & \\ \hline
      Produção do pôster e preparação da apresentação final. & & & & & & & &
      \scalebox{2}{$\times$} & \scalebox{2}{$\times$} \\ \hline
    \end{tabular}
  }
\end{table}



\begin{thebibliography}{5}

  \bibitem{Cartilha}
  BRASIL. \emph{A redação no Enem 2022: cartilha do participante}. Instituto
  Nacional de Estudos e Pesquisas Educacionais Anísio Teixeira (Inep). Brasília,
  2022.

  \bibitem{Corretores}
  LESME, A. \emph{Enem 2021: corretores podem corrigir até 200 redações por
  dia}. Brasil Escola, 2021. Disponível em:
  \url{https://vestibular.brasilescola.uol.com.br/enem/enem-2021-corretores-podem-corrigir-ate-200-redacoes-por-dia/351641.html}.
  Acesso em: 24 de abril de 2023.

  \bibitem{Bayes}
  BAZELATO, B. S.; AMORIM, E. C. F. \emph{A Bayesian Classifier to Automatic
  Correction of Portuguese Essay}. XVIII Conferência Internacional sobre
  Informática na Educação. Porto Alegre, 2013. p. 779-782.

  \bibitem{Multi-Aspect}
  VELOSO, A.; AMORIM, E. C. F.  \emph{A multi-aspect analysis of automatic essay
  scoring for Brazilian Portuguese}. Proceedings of the Student Research
  Workshop at the 15th Conference of the European Chapter of the Association for
  Computational Linguistics. Valencia, 2017. p. 94-102.

  \bibitem{Dataset}
  MARINHO, J. C.; ANCHIÊTA, R. T.; MOURA, R. S. \emph{Essay-BR: a Brazilian
  Corpus of Essays}. Anais do III Dataset Showcase Workshop. Rio de Janeiro,
  2021. Porto Alegre: Sociedade Brasileira de Computação. p. 53-64.

  \bibitem{BERT}
  DEVLIN, J.; LEE, K.; TOUTANOVA, K.; CHANG, M. \emph{BERT: Pre-training of Deep
  Bidirectional Transformers for Language Understanding}. arXiv preprint
  arXiv:1810.04805, 2019.

  \bibitem{BERTimbau}
  SOUZA, F.; LOTUFO, R.; NOGUEIRA, R. \emph{BERTimbau: Pretrained BERT Models
  for Brazilian Portuguese}. Intelligent Systems. Cham, 2020. Springer
  International Publishing. p. 403-417.

  \bibitem{QWK}
  COHEN, J. \emph{Weighted kappa:  nominal scale agreement provision for scaled
  dis-agreement or partial credit}. Psychological bulletin, volume 70,
  \textit{issue} 4. 1968.

\end{thebibliography}

\end{document}

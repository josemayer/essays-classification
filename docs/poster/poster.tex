% Authors: Nelson Lago, Arthur Del Esposte and Eduardo Zambom Santana
% Portions of the example contents: Arthur Del Esposte
% This file is distributed under the MIT Licence

%%%%%%%%%%%%%%%%%%%%%%%%%%%%%%%%%%%%%%%%%%%%%%%%%%%%%%%%%%%%%%%%%%%%%%%%%%%%%%%%
%%%%%%%%%%%%%%%%%%%%%%%%%%%%%%%%% PREÂMBULO %%%%%%%%%%%%%%%%%%%%%%%%%%%%%%%%%%%%
%%%%%%%%%%%%%%%%%%%%%%%%%%%%%%%%%%%%%%%%%%%%%%%%%%%%%%%%%%%%%%%%%%%%%%%%%%%%%%%%

% A língua padrão é a última citada
\documentclass[
  xcolor={hyperref,svgnames,x11names,table},
  hyperref={pdfencoding=unicode,plainpages=false,pdfpagelabels=true,breaklinks=true},
  brazilian,english,
]{beamer}

\usepackage{standalone}
% Vários pacotes e opções de configuração genéricos
\input{extras/basics}
\input{extras/languages}
\input{extras/fonts}
\input{extras/floats}
\input{extras/index}
\input{extras/bibconfig}
\input{extras/hyperlinks}
\hidelinks % desabilita cor/sublinhado dos links (URLs, refs etc.)
\input{extras/source-code}
\input{extras/utils}

% Diretórios onde estão as figuras; com isso, não é preciso colocar o caminho
% completo em \includegraphics (e nem a extensão).
\graphicspath{{figuras/},{logos/}}

% Comandos rápidos para mudar de língua:
% \en -> muda para o inglês
% \br -> muda para o português
% \texten{blah} -> o texto "blah" é em inglês
% \textbr{blah} -> o texto "blah" é em português
\babeltags{br = brazilian, en = english}

% Espaçamento simples
\singlespacing


%%%%%%%%%%%%%%%%%%%%%%%%%%%% APARÊNCIA DO BEAMER %%%%%%%%%%%%%%%%%%%%%%%%%%%%%%%

%% Possible paper sizes: a0, a0b, a1, a2, a3, a4.
%% Possible orientations: portrait, landscape
%% Font sizes can be changed using the scale option.
\usepackage[size=a1,orientation=portrait,scale=1.8]{beamerposter}

\usetheme{imeusp-poster} % carregado do diretório extras (veja basics.tex)

% O padrão usa um tom de vermelho escuro como cor principal; a opção
% "greeny" troca essa cor por um tom de verde; a opção "sandy" usa o
% mesmo tom de verde mas modifica a cor padrão dos blocos para um tom
% amarelado. "bluey" usa as cores do manual de identidade visual do IME.
\usecolortheme[bluey]{imeusp} % carregado do diretório extras (veja basics.tex)

%Remove ícones de navegação
\beamertemplatenavigationsymbolsempty


%%%%%%%%%%%%%%%%%%%%%%%%%% COMANDOS PARA O USUÁRIO %%%%%%%%%%%%%%%%%%%%%%%%%%%%%

% Medidas feitas "a olho"
\newcommand\singlecol{\column{.963\textwidth}}
\newcommand\halfcol{\column{.46\textwidth}}
\newcommand\onethirdcol{\column{.2922\textwidth}}
\newcommand\twothirdscol{\column{.626\textwidth}}
\newcommand\onefourthcol{\column{.2084\textwidth}}

% Blocos de cor personalizada
\newenvironment{coloredblock}[2]%
  {
    \setbeamercolor{block title}{fg=white,bg=#1!80!white}
    \setbeamercolor{block body}{fg=darkgray,bg=#1!20!white}
    \setbeamercolor{local structure}{fg=darkgray,bg=#1!20!white}
    \begin{block}{#2}
  }
  {\end{block}}

% Bibliografia. Apenas estilos bibliográficos derivados de numeric,
% alphabetic, authortitle e authoryear (como beamer-ime) vão funcionar
% bem aqui! Outros estilos, como abnt ou apa, vão gerar problemas de
% layout que você vai precisar ajustar manualmente. Observe que, num
% poster ou apresentação, provavelmente é uma boa ideia usar apenas
% \nocite e não \cite.
\usepackage[
  %style=extras/plainnat-ime, % variante de autor-data, similar a plainnat
  %style=alphabetic, % similar a alpha
  %style=numeric, % comum em artigos
  %style=authoryear-comp, % autor-data "padrão" do biblatex
  %style=apa, % variante de autor-data, muito usado
  %style=abnt,
  style=extras/beamer-ime,
]{biblatex}

% Num poster, a bibliografia pode ficar em tamanho menor
\renewcommand*{\bibfont}{\footnotesize}

%%%%%%%%%%%%%%%%%%%%%%%%%%%%%%%%%%%%%%%%%%%%%%%%%%%%%%%%%%%%%%%%%%%%%%%%%%%%%%%%
%%%%%%%%%%%%%%%%%%%%%%%%%%%%%%%%%% METADADOS %%%%%%%%%%%%%%%%%%%%%%%%%%%%%%%%%%%
%%%%%%%%%%%%%%%%%%%%%%%%%%%%%%%%%%%%%%%%%%%%%%%%%%%%%%%%%%%%%%%%%%%%%%%%%%%%%%%%

% O arquivo com os dados bibliográficos para biblatex; você pode usar
% este comando mais de uma vez para acrescentar múltiplos arquivos
\addbibresource{bibliografia.bib}

% Este comando permite acrescentar itens à lista de referências sem incluir
% uma referência de fato no texto (pode ser usado em qualquer lugar do texto)
%\nocite{bronevetsky02,schmidt03:MSc, FSF:GNU-GPL, CORBA:spec, MenaChalco08}
% Com este comando, todos os itens do arquivo .bib são incluídos na lista
% de referências
%\nocite{*}

\title[CCSL]{\vspace{10mm} Avaliação Automática de Redações no Modelo do \\ ENEM por meio do \textit{fine-tuning} do BERTimbau \vspace{10mm}}

\institute{Departamento de Ciência da Computação --- Universidade de São Paulo}

\author[josemayer@usp.br]{\textbf{José Lucas Silva Mayer} \tiny{(Sob a orientação de Denis Deratani Mauá e Igor Cataneo Silveira)}}
\date{11 de Dezembro, 2023}

% Optional foot image
\footimage{\raisebox{1cm}[0pt][0pt]{\includegraphics[width=10cm]{ccsl-logo} \hspace{5mm}}}

%%%%%%%%%%%%%%%%%%%%%%%%%%%%%%%%%%%%%%%%%%%%%%%%%%%%%%%%%%%%%%%%%%%%%%%%%%%%%%%%
%%%%%%%%%%%%%%%%%%%%%%%%%%%%%%% INÍCIO DO POSTER %%%%%%%%%%%%%%%%%%%%%%%%%%%%%%%
%%%%%%%%%%%%%%%%%%%%%%%%%%%%%%%%%%%%%%%%%%%%%%%%%%%%%%%%%%%%%%%%%%%%%%%%%%%%%%%%

\addtobeamertemplate{block begin}{}{\justifying}

\begin{document}

% Em um poster não há \maketitle

\begin{frame}[fragile]\centering

\vspace{-.5\baselineskip}

\begin{columns}[T]

    \halfcol

    \begin{block}{Introdução}
        O surgimento de modelos de linguagem como o BERT (\cite{bert2018}), baseados em \textit{transformers}, superou técnicas tradicionais de classificação de texto ao utilizar mecanismos que consideram o significado e a relação entre palavras de forma eficiente. A capacidade de ajuste fino de tais modelos pode, ainda, destacá-los na tarefa de avaliação automática de redações, já que permite a eles um treinamento especializado para a assimilação de padrões. Tal abordagem pode ser aplicada a exames como o ENEM, que possui uma cadeia logística complexa e lenta para a realização das correções. Este trabalho visa utilizar conjuntos de dados anotados e abertos para realizar o treinamento de cinco sistemas especialistas em avaliações de redações no modelo do ENEM, cada um com foco em uma competência distinta do exame.
    \end{block}

    \begin{block}{Objetivos}
        \begin{itemize}
            \item Desenvolver modelos de correção automática de redações baseados no BERT.
            \item Comparar desempenho de técnicas de treinamento e arquitetura de sistemas especialistas.
            \item Avaliar precisão da avaliação automática em relação à correção humana.
            \item Discutir vantagens ou desvantagens da abordagem automatizada e documentar resultados obtidos.
        \end{itemize}
    \end{block}

    \begin{block}{Fundamentação Teórica}
        O BERT, um modelo de linguagem baseado em transformadores, supera limitações recorrentes ao utilizar o mecanismo de atenção para modelar dependências bidirecionais. Ele atua principalmente na codificação, usando o \textit{encoder} dos transformadores.
        \begin{figure}[H]
            \centering
            \includestandalone{figuras/transformer}
            \caption{Arquitetura de um transformador (\cite{attention2017}).}
            \label{fig:transformer}
        \end{figure}
    \end{block}

    \halfcol

    \begin{block}{Metodologia}
        Utilizando a base de dados Essay-BR (\cite{marinho-et-al-22}), em suas versões simples e estendida, foram treinados modelos especializados em atribuir notas às cinco competências do ENEM (\cite{cartilha-redacao}). A arquitetura de cada sistema foi elaborada com base na técnica de \textit{fine-tuning} do BERT, usando o modelo pré-treinado BERTimbau para capturar nuances linguísticas em textos em língua portuguesa. Após o treinamento, os modelos foram rigorosamente avaliados com métricas de desempenho de aprendizado de máquina, como o \textit{Quadratic Weighted Kappa} (QWK) (\cite{cohen-1968-qwk}), e métricas do ENEM.
    \end{block}

    \begin{block}{Arquitetura dos Modelos}
        \begin{figure}[H]
            \centering
            \includegraphics{figuras/full_architecture.pdf}
            \caption{Arquitetura dos modelos de correção automática.}
            \label{fig:arquitetura}
        \end{figure}
    \end{block}

    \begin{block}{Resultados}
        Sendo PCE a proporção de correspondência exata e MSE o erro quadrático médio, os melhores resultados dos sistemas avaliadores são mostrados pela tabela abaixo.
        \begin{table}[H]
            \small
            \centering
            \begin{tabular}{llll}
                \toprule
                \textbf{Modelos} & QWK & PCE & MSE \\
                \midrule
                \textbf{Competência I} & 0,595 & 68,3\% & 0,358 \\
                \textbf{Competência II} & 0,562 & 57,2\% & 0,640 \\
                \textbf{Competência III} & 0,539 & 59,3\% & 0,554 \\
                \textbf{Competência IV} & 0,621 & 50,4\% & 0,701 \\
                \textbf{Competência V} & 0,548 & 49,4\% & 0,932 \\
                \bottomrule
            \end{tabular}
        \end{table}
    \end{block}

    \begin{block}{Bibliografia}
        \vspace{.017\baselineskip}
        \printbibliography
        \vspace{.017\baselineskip}
    \end{block}

\end{columns}

\vspace{.5\baselineskip}

\end{frame}

\end{document}
